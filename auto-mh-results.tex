
\section{Flow123d implementation}
Several aspects of the implementation in the software Flow123d is described in this section.
Mainly, the changes in the input file format are addressed,
adaptive integration and refined output mesh are described.

% \subsection{Adaptive quadrature}
The adaptive integration strategy is adopted from the pressure model from Chapter \ref{chap:xfem_pressure}.
Additionally, the quadrature rules are generalized to all the dimensions, which enables
computations on 2d and 3d elements, including integration over their faces which is necessary in the mixed model.

% \subsection{Output Mesh}
% \label{sec:output_mesh}
An approach to the visualization of the results containing enriched non-polynomial approximation  in VTK format is described.
The construction of an adaptively refined output mesh is suggested, in order to display singular enrichments.


\section{Numerical Tests in Flow123d}
\subsection{Test Cases in 1d-2d} \label{sec:num_test_cases_1d2d}
In this section, a~set of numerical tests of the 1d-2d model defined in \ref{sec:coupled_12d} is provided.
Some of the properties of the velocity enrichment are demonstrated, optimal convergence rate in velocity $L_2$ error
is shown. Different sizes of the enrichment zone are compared.

\begin{figure}[!htb]
    \centering
    \subfloat[TC2-a]{
    \includegraphics[width=0.482\textwidth]{\results tc_26_no_source_error_2,0.pdf} }
    \subfloat[TC2-b]{
    \includegraphics[width=0.482\textwidth]{\results tc_26_source_error_2,0.pdf} }
    \caption[Error distribution in 1d-2d.]
    {Test case in 1d-2d. The elementwise $L_2$ error in velocity is displayed at refinement level 5.
    The green circle indicates the enrichment radius $R_w$. }
    \label{fig:mh_tc2_error}
\end{figure}
A selected result is displayed in \fig{fig:mh_tc2_error}.
It includes 2 wells, perpendicular to the aquifer intersecting it near its center.
The enrichment zones overlap each other in the middle, resulting in very small velocity error.
The left subfigure corresponds to a~problem with zero source term. In the right subfigure,
a~sinusoidal source term prescribed, the error of the singular part is inferior.

\subsection{Test Cases in 1d-3d} \label{sec:num_test_cases_1d3d}
In this section, a~set of numerical tests of the 1d-3d model defined in \ref{sec:coupled_13d} is provided.
As in the 1d-2d case, the optimal convergence rate in velocity $L_2$ error is demonstrated,
different sizes of the enrichment zone are compared.

\begin{figure}[!htb]
    \centering
    \subfloat[TC5-a]{
    \includegraphics[width=0.482\textwidth]{\results tc_33_2w_no_source_error_2,0.pdf} }
    \subfloat[TC5-b]{
    \includegraphics[width=0.482\textwidth]{\results tc_33_2w_source_error_2,0.pdf} }
    \caption[Error distribution in 1d-3d.]
    {Test case in 1d-3d. The elementwise $L_2$ error in velocity is displayed at refinement level 5.
    The green cylinders indicate the enrichment zone. }
    \label{fig:mh_tc5_error}
\end{figure}
A selected result is displayed in \fig{fig:mh_tc5_error}.
It has the same setting as the 1d-2d case in \ref{sec:num_test_cases_1d2d},
it is however extruded to 3 dimension along the $z$ axis.
Lower error can be seen in the enriched zone in the left subfigure,
the error of the singular part of the solution is inferior to the error of the regular part
in the right subfigure.

\pagebreak

Finally, a~more complex problem is solved, including 10 wells of different tilt inside of a~rock block.
A~Dirichlet boundary condition is applied on the sides and a~zero normal flux boundary condition is set at the bases.
An analytical solution is not available in this case, the discrete solution can be inspected qualitatively only,
see the solution of velocity \fig{fig:mh_tc7_error}.
The block is meshed regularly, the element size can be noticed along the edges of the lower base.
The refined solution in the vicinity of the wells is due to the refined output mesh.

The water balance over the domain boundaries can be computed.
The relative difference between the flux into and out of the system is below 0.1 \%.
Thus, we can conclude, that the communication between the wells and the block 
is well approximated in terms of the water balance.

\begin{figure}[!htb]
    \centering
    \includegraphics[width=0.8\textwidth]{\results tc_40_10w.pdf}
    \caption[Velocity in 10 wells problem.]
    {Magnitude of velocity in 10 wells problem. Ten wells are represented symbolically by the blue tubes.
    The block is clipped so that we can see inside.}
    \label{fig:mh_tc7_error}
\end{figure}



% The implementation of the mixed-hybrid model using XFEM is a~part of the software Flow123d, version 3.0.x.
% The code is not however included in the official release yet, but it can be found on GitHub in an experimental branch \verb'PE_xfem_master'.
% There are some new features in the master development branch at the moment which prevent a~fast conflictless merge.
% 
% All about Flow123d, including its installation and usage inside Docker, the input file format and running simulations
% can be found through the official web pages~\cite{flow123d}, its GitHub pages and its current documentation~\cite{flow123d_doc_2018}.
% 
% Flow123d uses the YAML file format as the main input format. To solve the model with singularities and XFEM, additional
% keys must be specified. Code \ref{code:flow123d_input} provides an example input file structure. 
% 
% \vspace{0.5cm}
% \begin{code}
% \caption{Flow123d input including XFEM.}
% \label{code:flow123d_input}
% % \begin{verbatim}
% %         problem: !Coupling_Sequential
% %         ...
% %         flow_equation: !Flow_Darcy_MH
% %             ...
% %             output_specific:
% %             compute_errors: true
% %             python_solution: analytical_solution.py
% %             output_linear_system: true
% %             use_xfem:
% %             use_xfem: true
% %             enrich_velocity: true
% %             enr_radius: 2.0
% %             dim: 2
% %             ...
% % \end{verbatim}
% \texttt{
% problem: !Coupling\_Sequential \\
% \mbox{}\phantom{mm} ... \\
% \mbox{}\phantom{mm}    flow\_equation: !Flow\_Darcy\_MH \\
% \mbox{}\phantom{mmmm}        ... \\
% \mbox{}\phantom{mmmm}        output\_specific: \\
% \mbox{}\phantom{mmmmmm}            compute\_errors: true \\
% \mbox{}\phantom{mmmmmm}            python\_solution: analytical\_solution.py \\
% \mbox{}\phantom{mmmmmm}            output\_linear\_system: true \\
% \mbox{}\phantom{mmmmmm}            fields: \\
% \mbox{}\phantom{mmmmmmmm}             - velocity\_enr \\
% \mbox{}\phantom{mmmmmmmm}             - velocity\_reg \\
% \mbox{}\phantom{mmmmmmmm}             - velocity\_diff \\
% \mbox{}\phantom{mmmmmmmm}             - velocity\_exact \\
% \mbox{}\phantom{mmmmmmmm}             - pressure\_diff \\
% \mbox{}\phantom{mmmm}        use\_xfem: \\
% \mbox{}\phantom{mmmmmm}            use\_xfem: true \\
% \mbox{}\phantom{mmmmmm}            enrich\_velocity: true \\
% \mbox{}\phantom{mmmmmm}            enr\_radius: 2.0 \\
% \mbox{}\phantom{mmmmmm}            dim: 2 \\
% \mbox{}\phantom{mmmm}       output\_stream: \\
% \mbox{}\phantom{mmmmmm}         ... \\
% \mbox{}\phantom{mmmmmm}         output\_mesh: \\
% \mbox{}\phantom{mmmmmmmm}        max\_level: 6 \\
% \mbox{}\phantom{mmmmmmmm}        refine\_by\_error: false \\
% \mbox{}\phantom{mmmmmmmm}        error\_control\_field: pressure\_exact \\
% \mbox{}\phantom{mmmm}        ... \\
% \mbox{}\phantom{mm} ...}
% \end{code}
%  
% % \noindent
% In the first 3 lines the specific problem and equation are selected,
% Darcy flow with mixed-hybrid solver in our case. On the lines 6-8, it is set that the $L_2$ norm of the solution is to be computed,
% the analytic solution is passed through the python script and the linear system is set to be output in the Matlab format.
% On the lines 9-14, the fields (quantities) that are to be output are selected, these are in a~row: the enriched (singular) part of velocity,
% the regular part of velocity, the elementwise $L_2$ velocity error, the exact solution of velocity by the Python script and the elementwise $L_2$ pressure error.
% 
% On the lines 15-19, the XFEM is actually switched on. It is set that velocity is to be enriched (the only choice at the moment),
% the enrichment radius is set to 2.0 and the dimension in which the XFEM is applied is selected (is equal 3 in 3d cases).
% Finally on the lines 22-25, the refined output mesh can be defined. The output mesh can be refined to visualize the non-polynomial solution properly
% (described later in Section \ref{sec:output_mesh}). The maximal level of the refinement and possibly the adaptive refinement by error can be selected.
% 
% The rest of the input file is standard as in other Flow123d simulations.
% We use similar settings in our test cases where analytic solution is available.






% The adaptive quadrature implemented in Flow123d is based on the one developed for the quadrilateral meshes
% in Section \ref{sec:integration}. The same adaptive quadrature rules are used, only the distance function $r$
% is adapted to the simplicial elements. The detail of refined subelements and quadrature points
% is displayed in \fig{fig:adapt_refinement_flow123d}.
% %
% \begin{figure}[!htb]
%   \centering    
%   \subfloat[adaptive quadrature refinement]{\label{fig:adapt_refinement_flow123d_a} 
%     \includegraphics[height=6cm]{\figpath adaptive_refinement.pdf} }
%   \hspace{0pt}
%   \subfloat[adaptive quadrature in detail]{\label{fig:adapt_refinement_flow123d_b} 
%     \includegraphics[height=6cm]{\figpath adaptive_refinement_detail.pdf} }
%   \caption[Adaptive quadrature in Flow123d.]
%   {Adaptive quadrature for triangle elements implemented in Flow123d.
%    Black lines denote enriched elements edges, red lines denote adaptive refinement (subelements edges) and the well
%    edge is blue.
%   }
%   \label{fig:adapt_refinement_flow123d}
% \end{figure}
% 
% The adaptive quadrature is generalized for all three dimensions.
% This way we can use it in the computation of both integrals over elements and integrals
% over element faces (e.g. for accurate computation of $z^w_j$ in $\pi^{RT}_T$ interpolant \eqref{eqn:sgfem_interpolant_vel}).
% Such quadrature on a~face of tetrahedron is shown in \fig{fig:adaptive_refinement_ellipse}.
% \begin{figure}[!htb]
%   \centering    
%     \includegraphics[height=6cm]{\figpath adaptive_refinement_ellipse.pdf}
%   \caption[Adaptive quadrature on a~face of tetrahedron.]
%   {Adaptive quadrature on a~face of tetrahedron intersected by a~well.
%    The well is parallel to $z$-axis (the gray and blue circles indicate the cylinder),
%    the tetrahedron face is covered by the red subelements.
%    First the face is mapped to the plane of the blue cross-section, then the subelements are
%    constructed adaptively around the circle and mapped back to the face.
%    Finally, the quadrature points are placed in the subelements.
%    The actual cross-section is an ellipse (green points/purple axes).
%   }
%   \label{fig:adaptive_refinement_ellipse}
% \end{figure}
% 
% 
% 
% In 3 dimensions the distance vector \eqref{eqn:distance_vector_in_3d} is used to determine the distance of nodes and elements from the singularity.
% The subelements are refined using the edge splitting technique which is later described in Section \ref{sec:element_refinement}.



% % Bey's algorithm (red refinement of tetrahedron):
% %p.29 https://www5.in.tum.de/pub/Joshi2016_Thesis.pdf
% %p.108 http://www.bcamath.org/documentos_public/archivos/publicaciones/sergey_book.pdf    - brandts_2011
% %https://www.math.uci.edu/~chenlong/iFEM/doc/html/uniformrefine3doc.html#1
% %J. Bey. Simplicial grid refinement: on Freudenthal's algorithm and the optimal number of congruence classes. Numer. Math. 85(1):1--29, 2000. p11 Algorithm: RedRefinement3D.
% %p.4 http://www.vis.uni-stuttgart.de/uploads/tx_vispublications/vis97-grosso.pdf
% 
% Due to the enrichment, the finite element approximation is not polynomial anymore.
% If we want to look at the discrete solution, e.g. in Paraview, we have to interpolate it into a~space of piecewise
% constant or linear functions which can be viewed in such software. However, the interpolation on the computational mesh
% is quite coarse, so we suggested creating a~finer mesh specially refined for the output.
% 
% Two types of refinement is implemented -- uniform and adaptive. In the first case the mesh is uniformly refined in
% specified number of steps. In the later case the refinement is governed by an~error criterion which bounds 
% the measured elementwise $L_2$ norm of the difference between the (enriched) discrete solution and its constant/linear approximation.
% The refinement criterion reads as follows
% \begin{equation} \label{eqn:output_ref_criterion}
%     \frac{\norm{v_h - \pi_{out}v_h}_{T}}{\norm{v_h}_{T}} < \textrm{tol}
% \end{equation}
% where $\pi_{out}$ is the interpolator to a~space of constant or linear functions on an arbitrary element $T$ of the output mesh.
% The relative tolerance $\textrm{tol}$ is specified by the user.
% 
% The refinement is meant to be used for the enriched discrete solutions in particular, however in Flow123d the output mesh can be refined
% according to any scalar quantity present in the model. Thus for the quantity $v_h$ in \eqref{eqn:output_ref_criterion} one can use
% for example the source term, the hydraulic conductivity or the supplied analytic solution.
% 
% \subsubsection{Element Refinement}
% \label{sec:element_refinement}
% The refinement of elements for the output mesh is done using edge splitting technique (the so called red refinement).
% Since the output mesh is used only for better visualization of non-polynomial quantities, we do not
% care about any hanging nodes present in the refined mesh.
% 
% In 2d case, the refinement of an element is a~straightforward process: find the midpoints of all sides, connect them and generate 4 triangles.
% These triangles are congruent and have equal surface areas.
% %
% \begin{figure}[!htb]
%     \centering    
%     \includegraphics[width=0.6\textwidth]{\results output_refine.pdf} 
%     \caption[Output mesh refinement.]
%   {An example of an adaptive output mesh refinement in 3d.
%   A~singular function $1/r$ is displayed.}
%   \label{fig:output_refinement_flow123d}
% \end{figure}
% %
% On the other hand, the 3d case is more complicated. After splitting the edges, we obtain 4 tetrahedra at the vertices
% of the original one. The octahedron that remains in the middle can be subdivided according to one of its three diagonals.
% Only the choice of the shortest octahedron diagonal leads to a regular tetrahedra decomposition.
% This algorithm originally comes from Bey~\cite{bey_2000}, further e.g. in~\cite{brandts_2011}.
% 
% We demonstrate the refinement in \fig{fig:output_refinement_flow123d} where we used the adaptively refined output mesh to
% visualize a~singular function $1/r$ in a~cube. The domain is actually cut by a~diagonal plane. One can see the mesh edges in
% the left half of the cube.

  
  
% \section{Numerical Tests in Flow123d}
% We present several numerical tests in which we demonstrate the properties and the behavior of the suggested enrichments in the mixed-hybrid method.
% The tests are computed using our implementation in Flow123d.
% The test setting in 2d is similar to the one used in Section \ref{sec:2d_results_single} and \ref{sec:2d_results_multiple}.
% We will again investigate cases on different domains, with a~single well or with multiple wells, with a~source term or without.
% 
% An input file for Flow123d in Yaml format is prepared for each test.
% The pseudo-analytic solution is passed into Flow123d as a~Python script.
% The solution error is evaluated against the~pseudo-analytic solution determined according
% to Section \ref{sec:prim_analytic_solution}. We remind the pressure solution \eqref{eqn:2d_press_sol_mult} from which the velocity solution is derived:
% \begin{eqnarray}
% p_2 &=& p_{sin} + p_{reg} = \sum\limits_{j\in\mathcal{W}} a_j\log r_j + p_{reg}, \label{eqn:2d_press_solution}\\
% \vc u_2 &=& -\vc K_2 \grad p_2 = - \vc K_2 \Bigg[ \sum\limits_{j\in\mathcal{W}} a_j\frac{\vc r_j}{r_j^2} + \grad p_{reg} \Bigg].  \label{eqn:2d_vel_solution}
% \end{eqnarray}
% 
% The meshes used for computations are simplicial and unstructured.
% They are prepared by the software Gmsh using the default "MeshAdapt" algorithm.
% When examining the convergence of the method, we use Linux shell scripts and parameterized ".geo" files for Gmsh
% to generate series of refined meshes.
% We output the results into VTK format and visualize them in Paraview software.

% \subsection{Test Cases in 1d-2d} \label{sec:num_test_cases_1d2d}
% In this section, a~set of numerical tests of the 1d-2d model defined in \ref{sec:coupled_12d} is provided.
% Some of the properties of the velocity enrichment is demonstrated, optimal convergence rate in velocity $L_2$ error
% is shown. Different sizes of the enrichment zone are compared.

% \subsubsection{Test Case 1}
% In the first test case we consider $\Omega_2$ to be a~circular shaped domain of radius 5.0.
% A~well is perpendicular to the domain and is intersecting the domain in its center.
% 
% Since the enrichment in 2d is of our main interest in this test case, we want to minimize any other effects
% influencing the approximation error. The strength of the singularity is determined by the pressure difference
% between the well and the aquifer, so the accuracy of the solution in the vicinity of the well is also dependent
% on the accuracy of pressure in the well, on $p_1(\vc x_w)$ in particular.
% A~simple way to achieve that is simulating constant pressure inside the well by setting very high conductivity $\vc K_1=10^{10}$
% and setting a~constant Dirichlet boundary condition $g_{1D}$ on both ends of the well.
% 
% Let us denote three different settings: 
% \begin{itemize}
%     \item TC1-a: $f=0$,
%     \item TC1-b: $f=U\sin(\omega x)$,
%     \item TC1-c: $f=U\sin(\omega x)$, no singularity (regular case)
% \end{itemize}
% The first setting considers a~zero source term,
% the second sets the sinusoidal source term.
% The last setting is the regular problem without the singularity, which we solve with
% the standard mixed-hybrid finite element method (MHFEM).
% The input parameters are gathered in Table \ref{tab:tc1_data}.
% %
% \begin{table}[!hb]
% \begin{center}
% \begin{tabular}{cccccccc}
% \toprule
% % \multicolumn{2}{c}{Item} \\
% % \cmidrule(r){1-2}
% $\vc K$ & $\bx_w$  & $\rho_w$ & $\sigma_w$ & $R_w$ & $g_{1D}$ & $\omega$ & $U$ \\
% \midrule
% $10^{-3}$ & [3.33,3.33] & 0.03 & 10.0 & 2.0 & 100 & 1.0 & 200\\
% \bottomrule
% \end{tabular}
% \caption{Input data for Test case 1.}
% \label{tab:tc1_data}
% \end{center}
% \end{table}
% 
% % \noindent
% The model is computed on a~series of refined meshes and the convergence of our method is examined.
% The approximation error of velocity is displayed in \fig{fig:mh_tc1_error}. We can see that 
% the error is mainly accumulated outside the edge of the enriched zone in case of the zero source term.
% Considering the source term, the error of the singular part is inferior to the error of the regular part, as it is apparent
% in the right subfigure.
% %
% \begin{figure}[!htb]
%     \centering
%     \subfloat[TC1-a]{
%     \includegraphics[width=0.482\textwidth]{\results tc-23-error.pdf} }
%     %\hspace{5pt}
%     \subfloat[TC1-b]{
%     \includegraphics[width=0.482\textwidth]{\results tc-24-error.pdf} }
%     \caption[Error distribution in Test case 1.]
%     {Results of Test case 1. The elementwise $L_2$ error in velocity is displayed at refinement level 5.
%     The green circle indicates the enrichment radius $R_w$.}
%     \label{fig:mh_tc1_error}
% \end{figure}
% %
% %
% % 24_2d_circle_well_source_only_template.yaml
% %
% \begin{table}[!htb]
% \begin{center}
% \bgroup
% \def\arraystretch{1.2}
% \setlength\tabcolsep{5pt}
% % \begin{tabular}{r|c|c|c|c|c|r|r}
% \begin{tabular}{rc|cc|cc|cc}
% \toprule
% \multicolumn{2}{c|}{} & \multicolumn{2}{c|}{ TC1-a} & \multicolumn{2}{c|}{TC1-b} & \multicolumn{2}{c}{TC1-c}\\ [3pt] %\midrule
% i & h & $\|\vc u-\vc u_h\|_{L_2(\Omega_2)}$ & order & $\|\vc u-\vc u_h\|_{L_2(\Omega_2)}$
%     & order & $\|\vc u-\vc u_h\|_{L_2(\Omega_2)}$ & order \\ [3pt] \midrule
% 1 & 1.368 &  $7.11\cdot10^{-3}$  &  -   &  $2.42\cdot10^{-1}$  &  -   &  $2.42\cdot10^{-1}$ &   -   \\
% 2 & 0.913 &  $5.31\cdot10^{-3}$  & 0.72 &  $1.72\cdot10^{-1}$  & 0.84 &  $1.72\cdot10^{-1}$ &  0.84 \\
% 3 & 0.575 &  $3.30\cdot10^{-3}$  & 1.03 &  $9.99\cdot10^{-2}$  & 1.19 &  $9.99\cdot10^{-2}$ &  1.19 \\
% 4 & 0.358 &  $2.19\cdot10^{-3}$  & 0.87 &  $6.40\cdot10^{-2}$  & 0.94 &  $6.41\cdot10^{-2}$ &  0.94 \\
% 5 & 0.271 &  $1.61\cdot10^{-3}$  & 1.11 &  $4.50\cdot10^{-2}$  & 1.26 &  $4.51\cdot10^{-2}$ &  1.26 \\
% 6 & 0.176 &  $1.08\cdot10^{-3}$  & 0.92 &  $3.03\cdot10^{-2}$  & 0.92 &  $3.02\cdot10^{-2}$ &  0.94 \\
% \bottomrule
% \end{tabular}
% \caption{Convergence table of SGFEM in Test case 1.}
% \label{tab:mh_tc1_convergence}
% \egroup
% \end{center}
% \end{table}
% 
% The convergence results are shown in Table \ref{tab:mh_tc1_convergence}.
% There are three convergence columns in the table corresponding to different settings.
% The one on the left is for the zero source term case,
% the middle one is for the model including the source term,
% the last column shows the convergence of the regular problem without the singularity solved with MHFEM.
% We see that the convergence of velocity is nearly optimal, closing to 1.0, in all three situations.
% Comparing the magnitude of the approximation error between TC1-b and TC1-c,
% we see that the singularity is only minimally affecting the error which is the desired result.
% 
% 
% \subsubsection{Test Case 2}
% In the second test case we consider a~square shaped domain $\Omega_2=[0,10]\times[0,10]$,
% with two wells perpendicular to the domain $\Omega_2$.
% Similarly to the previous case, the source term is set to zero or defined as $f=U\sin(\omega x)$, $U=80$, $\omega=1.0$.
% The pressure inside the wells is fixed to a~constant value in the same way as in the previous case 
% by setting high conductivity $\vc K_1$ and a~constant Dirichlet boundary condition $g_{1D}$ on both ends of the well.
% The parameters of the wells are gathered in Table \ref{tab:tc2_data}.
% %
% \begin{table}[!hb]
% \begin{center}
% \begin{tabular}{ccccccccc}
% \toprule
% % \multicolumn{2}{c}{Item} \\
% % \cmidrule(r){1-2}
% $w$ & $\bx_w$  & $\rho_w$ & $\sigma_w$ & $R_w$ & $g_{1D}$\\
% \midrule
% 1& [4.1,4.3] & 0.03 & 10.0 & 2.0 & 150 \\
% 2& [5.7,5.9] & 0.03 & 10.0 & 2.0 & 100 \\
% \bottomrule
% \end{tabular}
% \caption{Input data for the wells in Test case 2.}
% \label{tab:tc2_data}
% \end{center}
% \end{table}
% 
% We solved the model with two different enrichment radii. Let us denote the following five settings:
% \begin{itemize}
%     \item TC2-a: $f=0$, $R_w=0.6$
%     \item TC2-b: $f=0$, $R_w=2.0$
%     \item TC2-c: $f=U\sin(\omega x)$, $R_w=0.6$
%     \item TC2-d: $f=U\sin(\omega x)$, $R_w=2.0$
%     \item TC2-e: $f=U\sin(\omega x)$, no singularity (regular case)
% \end{itemize}
% Setting the larger enrichment radius leads to enrichment zones overlap.
% % The last situation, the regular problem, is computed only as a~reference as in Test Case 1.
% 
% \begin{figure}[!htb]
%     \centering
%     \subfloat[TC2-a]{
%     \includegraphics[width=0.482\textwidth]{\results tc_26_no_source_error_0,6.pdf} }
%     \subfloat[TC2-b]{
%     \includegraphics[width=0.482\textwidth]{\results tc_26_no_source_error_2,0.pdf} } \\
%     \subfloat[TC2-c]{
%     \includegraphics[width=0.482\textwidth]{\results tc_26_source_error_0,6.pdf} }
%     \subfloat[TC2-d]{
%     \includegraphics[width=0.482\textwidth]{\results tc_26_source_error_2,0.pdf} }
%     \caption[Error distribution in Test case 2.]
%     {Results of Test case 2. The elementwise $L_2$ error in velocity is displayed at refinement level 5.
%     The green circle indicates the enrichment radius $R_w$. }
%     \label{fig:mh_tc2_error}
% \end{figure}
% %
% 
% The approximation error of velocity is displayed in \fig{fig:mh_tc2_error}. We can again observe
% the error mainly accumulating outside the edge of the enriched zone in TC2-a,b,c.
% The error near the first well is little higher since the singularity is stronger (there is higher pressure inside the well).
% In case TC2-b, we see a~very small error in the overlap of the enrichment zones
% since both singular enrichments are active there.
% Comparing TC2-c and TC2-d, we see that the larger enrichment radius makes the error in the singular part insignificant
% in the scale of the error of the regular part.
% % Considering the source term, the error of the singular part is inferior to the error of the regular part, as it is apparent
% % in the right subfigure.
% 
% \begin{table}[!htb]
% \begin{center}
% \bgroup
% \def\arraystretch{1.2}
% \setlength\tabcolsep{5pt}
% % \begin{tabular}{r|c|c|c|c|c|r|r}
% \begin{tabular}{rc|cc|cc|cc}
% \toprule
% \multicolumn{2}{c|}{} & \multicolumn{2}{c|}{ TC2-a } & \multicolumn{2}{c|}{ TC2-c } & \multicolumn{2}{c}{TC2-d}\\ [3pt] %\midrule
% i & h & $\|\vc u-\vc u_h\|_{L_2(\Omega_2)}$ & order & $\|\vc u-\vc u_h\|_{L_2(\Omega_2)}$
%     & order & $\|\vc u-\vc u_h\|_{L_2(\Omega_2)}$ & order \\ [3pt] \midrule
% 1 & 1.368 &  $6.71\cdot10^{-2}$  &  -   &  $1.38\cdot10^{-1}$  &  -   &  $1.10\cdot10^{-1}$ &   -   \\
% 2 & 0.913 &  $3.44\cdot10^{-2}$  & 1.67 &  $7.80\cdot10^{-2}$  & 1.42 &  $6.94\cdot10^{-1}$ &  1.13 \\
% 3 & 0.575 &  $2.64\cdot10^{-2}$  & 0.70 &  $5.50\cdot10^{-2}$  & 0.93 &  $4.59\cdot10^{-2}$ &  1.10 \\
% 4 & 0.358 &  $1.89\cdot10^{-2}$  & 0.81 &  $3.85\cdot10^{-2}$  & 0.87 &  $3.12\cdot10^{-2}$ &  0.93 \\
% 5 & 0.271 &  $1.34\cdot10^{-2}$  & 0.89 &  $2.62\cdot10^{-2}$  & 0.98 &  $2.06\cdot10^{-2}$ &  1.06 \\
% 6 & 0.176 &  $9.24\cdot10^{-3}$  & 0.90 &  $1.79\cdot10^{-2}$  & 0.94 &  $1.37\cdot10^{-2}$ &  0.99 \\
% \bottomrule
% \end{tabular}
% \caption{Convergence table of SGFEM in Test case 2.}
% \label{tab:mh_tc2_convergence}
% \egroup
% \end{center}
% \end{table}
% 
% The convergence results are shown in Table \ref{tab:mh_tc2_convergence}.
% The three columns in the table correspond to three selected settings in Test case 2.
% We see that the convergence order of velocity approximation is closing to the optimum in all situations.
% The overlapping enrichment zones do not corrupt the accuracy of the approximation in TC2-d.
% Comparing the magnitude of the error in TC2-c and TC2-d columns, we see that the larger enrichment radius leads
% to a~slightly increased accuracy while the convergence order is the same.
% 
% The model with TC2-b setting is converging also optimally, but we do not present all the results for the sake of brevity.
% The larger enrichment radius in TC2-b leads to decrease of the magnitude of the overall error,
% approximately with factor of 3.0 to the TC2-a setting.
% Similarly to Test case 1, the regular problem without singularities TC2-e is also computed.
% The convergence results correspond to the setting TC2-d,
% the magnitude of the approximation error and the convergence order are almost identical
% due to the accurate approximation of the singularities.
% 
% % %
% % \begin{figure}[!htb]
% %     \centering
% %     \includegraphics[width=0.85\textwidth]{\results tc_26_inf-sup.pdf}
% %     \caption[numerical verification of inf-sup condition TC2]
% %     {Graph representing numerical verification of the inf-sup condition in Test Case 2.}
% %     \label{fig:mh_tc2_inf_sup}
% % \end{figure}
% % %
% 
% 
% 
% \subsubsection{Test Case 3}
% The following test case copies the setting of Test case 2, however it includes five wells.
% The input parameters for the wells are gathered in Table \ref{tab:tc3_data}.
% The wells 1,2 and 4 can be seen as the pumping wells, the others as the injection wells.
% Let us again denote 4 different settings with two enrichment radii:
% \begin{itemize}
%     \item TC3-a: $f=0$, $R_w=0.8$,
%     \item TC3-b: $f=0$, $R_w=2.0$,
%     \item TC3-c: $f=U\sin(\omega x)$, $R_w=0.8$,
%     \item TC3-d: $f=U\sin(\omega x)$, $R_w=2.0$.
% %     \item TC3-e: $f=U\sin(\omega x)$, no singularity
% \end{itemize}
% %
% \begin{table}[!htb]
% \begin{center}
% \begin{tabular}{ccccccccc}
% \toprule
% % \multicolumn{2}{c}{Item} \\
% % \cmidrule(r){1-2}
% $w$ & $\bx_w$  & $\rho_w$ & $\sigma_w$ & $g_{1D}$\\
% \midrule
% 1& [2.8,2.5] & 0.03 & 20.0 & -150 \\
% 2& [4.9,5.4] & 0.03 & 10.0 & -30 \\
% 3& [2.9,7.4] & 0.03 & 10.0 & 120 \\
% 4& [7.3,7.8] & 0.03 & 10.0 & -50 \\
% 5& [7.4,2.8] & 0.03 & 20.0 & 100 \\
% \bottomrule
% \end{tabular}
% \caption{Input data for the wells in Test case 3.}
% \label{tab:tc3_data}
% \end{center}
% \end{table}
% %
% \begin{figure}[!htb]
%     \centering
%     \subfloat[TC3-a]{
%     \includegraphics[width=0.482\textwidth]{\results tc_51_no_source_error_0,8.pdf} }
%     \subfloat[TC3-b]{
%     \includegraphics[width=0.482\textwidth]{\results tc_51_no_source_error_2,0.pdf} } \\
%     \subfloat[TC3-c]{
%     \includegraphics[width=0.482\textwidth]{\results tc_51_source_error_0,8.pdf} }
%     \subfloat[TC3-d]{
%     \includegraphics[width=0.482\textwidth]{\results tc_51_source_error_2,0.pdf} }
%     \caption[Error distribution in Test case 3.]
%     {Results of Test case 3. The elementwise $L_2$ error in velocity is displayed at refinement level 5.
%     The green circle indicates the enrichment radius $R_w$. }
%     \label{fig:mh_tc3_error}
% \end{figure}
% %
% 
% The distribution of the velocity error is shown in \fig{fig:mh_tc3_error}. 
% We can see in setting TC3-a that the error is again concentrated on the edges of the enrichment zones of
% the strongest singularities. In setting TC3-b, the error is largest in the central area, 
% which is affected by all the singularities, however it is covered only by the enrichment zone of the well 2.
% The solution is well approximated inside the four enrichment zones
% of the wells 1,3,4,5, we see only small effects of the other singularities there.
% 
% In the subfigures for settings TC3-c and TC3-d, we see the same behavior of the error as in previous test case.
% The error of the regular part is significant in TC3-d while the error of the singular part is not apparent.
% 
% The convergence for the selected settings is displayed in Table \ref{tab:mh_tc3_convergence}.
% We see a~nearly optimal convergence order in all three columns. It is again apparent that
% the larger enrichment radius pushes the approximation error down a~bit but it does not effect the
% convergence order.
% 
% \begin{table}[!htb]
% \begin{center}
% \bgroup
% \def\arraystretch{1.2}
% \setlength\tabcolsep{5pt}
% % \begin{tabular}{r|c|c|c|c|c|r|r}
% \begin{tabular}{rc|cc|cc|cc}
% \toprule
% \multicolumn{2}{c|}{} & \multicolumn{2}{c|}{ TC3-a } & \multicolumn{2}{c|}{ TC3-c } & \multicolumn{2}{c}{TC3-d}\\ [3pt] %\midrule
% i & h & $\|\vc u-\vc u_h\|_{L_2(\Omega_2)}$ & order & $\|\vc u-\vc u_h\|_{L_2(\Omega_2)}$
%     & order & $\|\vc u-\vc u_h\|_{L_2(\Omega_2)}$ & order \\ [3pt] \midrule
% 1 & 1.412 & $2.82\cdot10^{-2}$  &  -   & $1.19\cdot10^{-1}$  &  -   & $1.16\cdot10^{-1}$ &   -   \\
% 2 & 0.946 & $2.11\cdot10^{-2}$  & 0.72 & $7.67\cdot10^{-2}$  & 1.10 & $7.33\cdot10^{-2}$ &  1.13 \\
% 3 & 0.650 & $1.43\cdot10^{-2}$  & 1.03 & $5.06\cdot10^{-2}$  & 1.11 & $4.87\cdot10^{-2}$ &  1.09 \\
% 4 & 0.431 & $1.08\cdot10^{-2}$  & 0.68 & $3.50\cdot10^{-2}$  & 0.89 & $3.32\cdot10^{-2}$ &  0.93 \\
% 5 & 0.292 & $7.34\cdot10^{-3}$  & 1.00 & $2.31\cdot10^{-2}$  & 1.06 & $2.19\cdot10^{-2}$ &  1.06 \\
% 6 & 0.193 & $5.05\cdot10^{-3}$  & 0.91 & $1.56\cdot10^{-2}$  & 0.96 & $1.47\cdot10^{-2}$ &  0.98 \\
% \bottomrule
% \end{tabular}
% \caption{Convergence table of SGFEM in Test case 3.}
% \label{tab:mh_tc3_convergence}
% \egroup
% \end{center}
% \end{table}
% 

% \subsection{Test Cases in 1d-3d} \label{sec:num_test_cases_1d3d}
% 
% In this section, a~set of numerical tests of the 1d-3d model defined in \ref{sec:coupled_13d} is provided.
% As in the 1d-2d case, the optimal convergence rate in velocity $L_2$ error is demonstrated,
% different sizes of the enrichment zone are compared.



% We now present several test cases in 3 dimensions. The geometries of the test cases
% are analogical to the previous test cases in 2d, but extruded to the third dimension in $z$ coordinate.

% \subsubsection{Test Case 4}
% In the first 3d test case, the 3d domain $\Omega_3$ is a~cylinder with the bottom base in $xy$ plane
% with the center at $[3.33,3.33,0]$, of height 2.0 and radius 5.0.
% A~well is intersecting the cylinder along its vertical axis.
% Similarly to the 2d case, we set high conductivity $\vc K_1$ inside the well
% and a~constant Dirichlet boundary condition $g_{1D}$ on both ends of the well to have constant pressure there.
% A~homogeneous Neumann boundary condition for normal flux is applied on both bases of the cylinder. The flow is then governed 
% by the well and the Dirichlet boundary condition on the lateral surface of the cylinder.
% Therefore we can still use the pseudo-analytic solution from Section \ref{sec:prim_analytic_solution}.
% 
% Let us have three different settings: 
% \begin{itemize}
%     \item TC4-a: $f=0$,
%     \item TC4-b: $f=U\sin(\omega x)$,
%     \item TC4-c: $f=U\sin(\omega x)$, no singularity (regular case).
% \end{itemize}
% We use the same parameters for the nonzero source terms as before.
% The the input parameters are gathered in Table \ref{tab:tc4_data}.
% %
% \begin{table}[!htb]
% \begin{center}
% \begin{tabular}{cccccccc}
% \toprule
% % \multicolumn{2}{c}{Item} \\
% % \cmidrule(r){1-2}
% $\vc K$ & $\rho_w$ & $\sigma_w$ & $R_w$ & $g_{1D}$ & $\omega$ & $U$ \\
% \midrule
% $10^{-3}$ & 0.03 & 10.0 & 2.0 & 100 & 1.0 & 80\\
% \bottomrule
% \end{tabular}
% \caption{Input data for Test case 4.}
% \label{tab:tc4_data}
% \end{center}
% \end{table}
% %
% \begin{figure}[!htb]
%     \centering
% %     \subfloat[TC4-a]{
% %     \includegraphics[width=0.482\textwidth]{\results tc_25_no_source_error_1,0.pdf} }
%     \subfloat[TC4-b]{
%     \includegraphics[width=0.482\textwidth]{\results tc_25_no_source_error_2,0.pdf} }
% %     \subfloat[TC4-c]{
% %     \includegraphics[width=0.482\textwidth]{\results tc_25_source_error_1,0.pdf} }
%     \subfloat[TC4-d]{
%     \includegraphics[width=0.482\textwidth]{\results tc_25_source_error_2,0.pdf} }
%     \caption[Error distribution in Test case 4.]
%     {Results of Test case 4. The elementwise $L_2$ error in velocity is displayed at refinement level 5.
%     The green cylinder indicates the enrichment zone. }
%     \label{fig:mh_tc4_error}
% \end{figure}
% 
% \begin{table}[!htb]
% \begin{center}
% \bgroup
% \def\arraystretch{1.2}
% \setlength\tabcolsep{5pt}
% % \begin{tabular}{r|c|c|c|c|c|r|r}
% \begin{tabular}{rc|cc|cc|cc}
% \toprule
% \multicolumn{2}{c|}{} & \multicolumn{2}{c|}{ TC4-a } & \multicolumn{2}{c|}{ TC4-b } & \multicolumn{2}{c}{TC4-c}\\ [3pt] %\midrule
% i & h & $\|\vc u-\vc u_h\|_{L_2(\Omega_2)}$ & order & $\|\vc u-\vc u_h\|_{L_2(\Omega_2)}$
%     & order & $\|\vc u-\vc u_h\|_{L_2(\Omega_2)}$ & order \\ [3pt] \midrule
% 1 & 1.348 &  $1.16\cdot10^{-2}$  &  -   &  $1.81\cdot10^{-1}$  &  -   &  $1.81\cdot10^{-1}$ &   -   \\
% 2 & 1.063 &  $1.10\cdot10^{-2}$  & 0.20 &  $1.50\cdot10^{-1}$  & 0.79 &  $1.50\cdot10^{-1}$ &  0.80 \\
% 3 & 0.845 &  $8.89\cdot10^{-3}$  & 0.95 &  $1.12\cdot10^{-1}$  & 1.27 &  $1.12\cdot10^{-2}$ &  1.28 \\
% 4 & 0.632 &  $7.90\cdot10^{-3}$  & 0.40 &  $8.90\cdot10^{-2}$  & 0.79 &  $8.88\cdot10^{-2}$ &  0.79 \\
% 5 & 0.544 &  $7.27\cdot10^{-3}$  & 0.56 &  $7.36\cdot10^{-2}$  & 1.28 &  $7.34\cdot10^{-2}$ &  1.28 \\
% 6 & 0.408 &  $6.67\cdot10^{-3}$  & 0.30 &  $5.45\cdot10^{-2}$  & 1.04 &  $5.47\cdot10^{-2}$ &  1.02 \\
% \bottomrule
% \end{tabular}
% \caption{Convergence table in Test case 4.}
% \label{tab:mh_tc4_convergence}
% \egroup
% \end{center}
% \end{table}
% 
% The approximation error of velocity is displayed in \fig{fig:mh_tc4_error}. We can see that higher
% error is accumulated outside the edge of the enriched zone in TC4-a. However, we observe a~significant error also
% along the edges of the cylinder bases where the two boundaries with the Dirichlet and the homogeneous Neumann boundary condition
% are adjacent. The velocity shape functions on the elements with sides on both types of boundaries are obviously unable to approximate
% the solution correctly there. This error on the boundary also slows down the convergence rate, as it can be seen in Table \ref{tab:mh_tc4_convergence}.
% Considering the source term in TC4-b, the error of the singular part is inferior to the error of the regular part, as it is apparent
% in the right subfigure. The error along the edges of the cylinder bases is negligible in contrast to TC4-a.
% 
% 
% 
% \subsubsection{Test Case 5}
% In the second 3d test case, the domain $\Omega_3$ is a~block with the bottom base in $xy$ plane,
% with height 2.0.
% Two wells intersecting the block are perpendicular to its base.
% The parameters of the wells are specified in Table \ref{tab:tc5_data}.
% 
% \begin{table}[!hb]
% \begin{center}
% \begin{tabular}{cccccccc}
% \toprule
% % \multicolumn{2}{c}{Item} \\
% % \cmidrule(r){1-2}
% $w$ & $\bx_w$  & $\rho_w$ & $\sigma_w$ & $g_{1D}$\\
% \midrule
% 1& [4.1,4.3] & 0.03 & 10.0 & 150 \\
% 2& [5.7,5.9] & 0.03 & 10.0 & 100 \\
% \bottomrule
% \end{tabular}
% \caption{Input data for the wells in Test case 5.}
% \label{tab:tc5_data}
% \end{center}
% \end{table}
% %
% Analogically to the previous test case, the constant pressure inside the wells is enforced
% and zero normal flux is prescribed on both bases of the block..
% We use again the pseudo-analytic solution to determine the approximation error.
% 
% Let us have four different settings: 
% \begin{itemize}
%     \item TC5-a: $f=0$, $R_w=1.0$,
%     \item TC5-b: $f=0$, $R_w=2.0$,
%     \item TC5-c: $f=U\sin(\omega x)$, $R_w=1.0$,
%     \item TC5-d: $f=U\sin(\omega x)$, $R_w=2.0$.
% \end{itemize}
% We use the same parameters for the nonzero source terms as before: $U=80$, $\omega=1.0$.
% %
% \begin{figure}[!htb]
%     \centering
%     \subfloat[TC5-a]{
%     \includegraphics[width=0.482\textwidth]{\results tc_33_2w_no_source_error_1,0.pdf} }
%     \subfloat[TC5-b]{
%     \includegraphics[width=0.482\textwidth]{\results tc_33_2w_no_source_error_2,0.pdf} } \\
%     \subfloat[TC5-c]{
%     \includegraphics[width=0.482\textwidth]{\results tc_33_2w_source_error_1,0.pdf} }
%     \subfloat[TC5-d]{
%     \includegraphics[width=0.482\textwidth]{\results tc_33_2w_source_error_2,0.pdf} }
%     \caption[Error distribution in Test case 5.]
%     {Results of Test case 5. The elementwise $L_2$ error in velocity is displayed at refinement level 5.
%     The green cylinders indicate the enrichment zone. }
%     \label{fig:mh_tc5_error}
% \end{figure}
% %
% \begin{table}[!htb]
% \begin{center}
% \bgroup
% \def\arraystretch{1.2}
% \setlength\tabcolsep{5pt}
% % \begin{tabular}{r|c|c|c|c|c|r|r}
% \begin{tabular}{rc|cc|cc|cc}
% \toprule
% \multicolumn{2}{c|}{} & \multicolumn{2}{c|}{ TC5-a } & \multicolumn{2}{c|}{ TC5-c } & \multicolumn{2}{c}{TC5-d}\\ [3pt] %\midrule
% i & h & $\|\vc u-\vc u_h\|_{L_2(\Omega_2)}$ & order & $\|\vc u-\vc u_h\|_{L_2(\Omega_2)}$
%     & order & $\|\vc u-\vc u_h\|_{L_2(\Omega_2)}$ & order \\ [3pt] \midrule
% 1 & 1.404 & $5.70\cdot10^{-2}$  &  -   & $2.49\cdot10^{-1}$  &  -   & $2.43\cdot10^{-1}$ &   -   \\
% 2 & 1.190 & $5.36\cdot10^{-2}$  & 0.37 & $2.07\cdot10^{-1}$  & 1.12 & $2.04\cdot10^{-1}$ &  1.05 \\
% 3 & 0.878 & $4.29\cdot10^{-2}$  & 0.73 & $1.55\cdot10^{-1}$  & 0.95 & $1.53\cdot10^{-2}$ &  0.93 \\
% 4 & 0.681 & $3.38\cdot10^{-2}$  & 0.95 & $1.19\cdot10^{-1}$  & 1.02 & $1.18\cdot10^{-2}$ &  1.04 \\
% 5 & 0.515 & $2.83\cdot10^{-2}$  & 0.63 & $9.22\cdot10^{-2}$  & 0.92 & $9.03\cdot10^{-2}$ &  0.96 \\
% 6 & 0.416 & $2.37\cdot10^{-2}$  & 0.83 & $7.51\cdot10^{-2}$  & 0.96 & $7.35\cdot10^{-2}$ &  0.96 \\
% \bottomrule
% \end{tabular}
% \caption{Convergence table in Test case 5.}
% \label{tab:mh_tc5_convergence}
% \egroup
% \end{center}
% \end{table}
% 
% We show the distribution of the error in \fig{tab:mh_tc5_convergence} and the convergence results in Table \ref{fig:mh_tc5_error}.
% In case of settings TC5-a and TC5-b, a~significant error can be again observed on the edges of the block bases as in the previous case.
% Due to this error the convergence rate is suboptimal.
% On the other hand for the settings TC5-c and TC5-d, the results are satisfying and the approximation error behaves similarly
% as in the 2 cases. The error of the regular part is dominating as it can be seen in the subfigures. The convergence order 
% is optimal.
% 
% The well 1 causes a~stronger singularity due to the higher pressure difference,
% therefore the error outside the enrichment zone of the well is also higher.
% This is apparent especially in the subfigure TC5-b. We can also observe a~very low error
% inside the enrichment zones overlap.
% 
% \subsubsection{Test Case 6}
% This test case is analogical to the previous test case, however it includes 5 wells.
% The wells are perpendicular to the $xy$ plane and they have the same parameters as in the 2d Test Case 3.
% The following settings are used:
% \begin{itemize}
%     \item TC6-a: $f=0$, $R_w=1.0$,
%     \item TC6-b: $f=0$, $R_w=2.0$,
%     \item TC6-c: $f=U\sin(\omega x)$, $R_w=1.0$,
%     \item TC6-d: $f=U\sin(\omega x)$, $R_w=2.0$,
%     \item TC6-e: $f=U\sin(\omega x)$, no singularity.
% \end{itemize}
% %
% \begin{figure}[!htb]
%     \centering
%     \subfloat[TC6-a]{
%     \includegraphics[width=0.482\textwidth]{\results tc_31_5w_no_source_error_1,0.pdf} }
%     \subfloat[TC6-b]{
%     \includegraphics[width=0.482\textwidth]{\results tc_31_5w_no_source_error_2,0.pdf} } \\
%     \subfloat[TC6-c]{
%     \includegraphics[width=0.482\textwidth]{\results tc_31_5w_source_error_1,0.pdf} }
%     \subfloat[TC6-d]{
%     \includegraphics[width=0.482\textwidth]{\results tc_31_5w_source_error_2,0.pdf} }
%     \caption[Error distribution in Test case 6.]
%     {Results of Test case 6. The elementwise $L_2$ error in velocity is displayed at refinement level 5.
%     The green cylinders indicate the enrichment zone. }
%     \label{fig:mh_tc6_error}
% \end{figure}
% %
% 
% \begin{table}[!htb]
% \begin{center}
% \bgroup
% \def\arraystretch{1.2}
% \setlength\tabcolsep{5pt}
% % \begin{tabular}{r|c|c|c|c|c|r|r}
% \begin{tabular}{rc|cc|cc|cc}
% \toprule
% \multicolumn{2}{c|}{} & \multicolumn{2}{c|}{ TC6-c } & \multicolumn{2}{c|}{ TC6-d } & \multicolumn{2}{c}{TC6-e}\\ [3pt] %\midrule
% i & h & $\|\vc u-\vc u_h\|_{L_2(\Omega_2)}$ & order & $\|\vc u-\vc u_h\|_{L_2(\Omega_2)}$
%     & order & $\|\vc u-\vc u_h\|_{L_2(\Omega_2)}$ & order \\ [3pt] \midrule
% 1 & 1.404 &  $2.52\cdot10^{-1}$  &  -   & $2.51\cdot10^{-1}$ &  -   &  $2.49\cdot10^{-1}$ &   -   \\
% 2 & 1.190 &  $2.15\cdot10^{-1}$  & 0.96 & $2.11\cdot10^{-1}$ & 1.04 &  $2.10\cdot10^{-1}$ &  1.03 \\
% 3 & 0.878 &  $1.62\cdot10^{-1}$  & 0.93 & $1.59\cdot10^{-1}$ & 0.93 &  $1.59\cdot10^{-2}$ &  0.92 \\
% 4 & 0.681 &  $1.25\cdot10^{-1}$  & 1.01 & $1.23\cdot10^{-1}$ & 1.01 &  $1.23\cdot10^{-2}$ &  1.02 \\
% 5 & 0.515 &  $9.55\cdot10^{-2}$  & 0.98 & $9.39\cdot10^{-2}$ & 0.98 &  $9.31\cdot10^{-2}$ &  0.99 \\
% 6 & 0.416 &  $7.75\cdot10^{-2}$  & 0.97 &   -                & -    &  $7.54\cdot10^{-2}$ &  0.98 \\
% \bottomrule
% \end{tabular}
% \caption{Convergence table in Test case 6.}
% \label{tab:mh_tc6_convergence}
% \egroup
% \end{center}
% \end{table}
% 
% The distribution of the error is displayed in \fig{tab:mh_tc6_convergence}
% and the convergence results are summarized in Table \ref{fig:mh_tc6_error}.
% Setting the zero source term in the settings TC6-a,b, the optimal convergence rate is again corrupted
% by the error on the edges of the bases of the block, although in TC6-a it is not that significant.
% Considering the model with the settings TC6-c,d, we see that the singularities are well approximated
% and the dominant error is in the regular part of the solution.
% The optimal convergence is comparable to the regular case TC6-e.
% 
% For the setting TC6-d, the algebraic solver failed due to insufficient memory on the finest mesh. Since most of the elements
% are enriched from at least one well, there are several rows relatively full which causes loss of sparsity and higher memory consumption.
% 
% 
% \subsubsection{Test Case 7}
% In this test case we want to demonstrate a~more general setting including 10 wells of different tilt.
% All the wells are set with the following parameters: $\sigma_w=100,\; \vc K_1=100,\; \rho_w=0.03,\; R_w=0.8$.
% % The pressure is approximately constant in the wells and is set: $g^1_{1D}=150$, $g^2_{1D}=100$, $g^3_{1D}=200$,
% % $g^4_{1D}=100$, $g^5_{1D}=300$, $g^6_{1D}=-50$, $g^7_{1D}=100$, $g^8_{1D}=200$, $g^9_{1D}=-50$ and $g^{10}_{1D}=200$.
% Instead of fixing pressure in the wells, we prescribe fluxes by the means of the inhomogeneous Neumann boundary condition
% at the top of the wells. The fluxes are gathered in the first row of Table \ref{tab:tc7_data}, the second row
% contains the fluxes scaled by the cross-section, i.e. flux density for Flow123d input.
% The rock block conductivity is set to $10^{-2}$, a~Dirichlet boundary condition $g_{3D}=25(3-z)$
% is applied on the sides and a~homogeneous Neumann boundary condition for normal fluxes is set at the bases.
% %
% \begin{table}[!htb]
% \begin{center}
% \begin{tabular}{r|cccccccccc}
% \toprule
% $w$ & 1  & 2 & 3 & 4 & 5 & 6 & 7 & 8 & 9 & 10\\ \midrule
% $g^w_{1N}$ & 1.414 & 0.848 & 0.565 & 1.414 & 1.131 & -1.131 & 0.848 & 0.848 & -1.131 & 0.565\\[8pt]
% $\dfrac{g^w_{1N}}{\delta_1}$ & 500  & 300 & 200 & 500 & 400 & -400 & 300 & 300 & -400 & 200\\[8pt]
% \bottomrule
% \end{tabular}
% \caption{Prescribed fluxes at the top of the wells in Test case 7.}
% \label{tab:tc7_data}
% \end{center}
% \end{table}
% 
% \begin{figure}[!htb]
%     \centering
%     \includegraphics[width=0.8\textwidth]{\results tc_40_10w.pdf}
%     \caption[Error distribution in Test case 7.]
%     {Results of Test case 7. Ten wells are represented symbolically by the blue tubes.
%     The block is clipped so that we can see inside. The magnitude of velocity is displayed.}
%     \label{fig:mh_tc7_error}
% \end{figure}
% 
% An analytical solution is not available in this case, we can only inspect the discrete solution qualitatively,
% see the solution of velocity \fig{fig:mh_tc7_error}.
% The block is meshed regularly, the element size can be noticed along the edges of the lower base.
% The refined solution in the vicinity of the wells is due to the refined output mesh.
% 
% We calculate the water balance over the domain boundaries.
% There is zero flux on the bases of the block and also at the bottom of the wells.
% The sum of the fluxes at the top of the wells is equal 5.372, while the flux over the 
% block sides is -5.376, i.e. the difference between what flows into and out of the system is 0.004.
% Thus, we can conclude, that the communication between the wells and the block 
% is well approximated in terms of the water balance.

% \section{Summary}
% The summary of the chapter is provided.

% In this chapter we developed an XFEM in a~mixed-hybrid form for approximation of singularities in both 2d and 3d problems.
% The model was formulated following the concept of the software Flow123d and using the Lagrange multipliers for dimensional coupling.
% We used what we have learned in Chapter \ref{chap:xfem_pressure} and transferred our experience into creating
% an SGFEM like enrichment of velocity in the vicinity of the singularities to improve the approximation.
% This enrichment was applied in the discretization together with the lowest order Raviart-Thomas FE.
% The properties of such enrichment were described.
% The numerical validation of the inf-sup stability was discussed, however this topic is left open for further research.
% 
% Regarding the reduced dimensional modeling, our approach enables coupling of non-planar 1d-2d domains
% and coupling of co-dimension 2 in 3d. The possibility to compute such problems on incompatible meshes is a~significant advantage,
% as it was demonstrated even in the small artificial test cases.
% 
% The coupled models were implemented as a~part of the software Flow123d.
% Some of the technical aspects of the code extensions were discussed,
% the input file, the adaptive integration and the output mesh refinement in particular.
% The solution of the algebraic linear system was not addressed.
% A direct solver or Krylov subspace iterative methods with some standard preconditioners were used.
% This topic would definitely deserve more attention in future, in order to solve larger problems efficiently.
% 
% At last, an extensive set of numerical tests was performed and convergence results were presented.
% Although pressure converges suboptimally, because its FE approximation is not enriched, velocity error
% reaches optimal order of convergence.
% Solutions with different enrichment radii and overlapping enrichment zones were compared in most of the test cases.
% The last test case simulated a~more complex problem with multiple wells intersecting the rock in different angles.




