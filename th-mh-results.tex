

\section{Flow123d implementation}

\subsection{Adaptive quadrature}
The adaptive quadrature implemented in Flow123d is based on the one developed for the quadrilateral meshes
in Section \ref{sec:integration}. The same adaptive quadrature rules are used, only the distance function $r$
is adapted to the simplicial elements. The quadrature generalized for all three dimensions.
This way we can use it in the computation of both integrals over elements and integrals
over element faces (e.g. for accurate computation of $z^w_j$ in $\pi^{RT}_T$ interpolant \eqref{eqn:sgfem_interpolant_vel}).

\begin{figure}[!htb]
  \centering    
  \subfloat[adaptive quadrature refinement]{\label{fig:adapt_refinement_flow123d_a} 
    \includegraphics[height=6cm]{\figpath adaptive_refinement.pdf} }
  \hspace{0pt}
  \subfloat[adaptive quadrature in detail]{\label{fig:adapt_refinement_flow123d_b} 
    \includegraphics[height=6cm]{\figpath adaptive_refinement_detail.pdf} }
  \caption[Adaptive quadrature in Flow123d]
  {Adaptive quadrature for triangle elements implemented in Flow123d.
   Black lines denote enriched elements edges, red lines denote adaptive refinement (subelements edges) and the well
   edge is blue.
  }
  \label{fig:adapt_refinement_flow123d}
\end{figure}

In 3 dimensions the distance vector \eqref{eqn:distance_vector_in_3d} is used to determine the distance of nodes and elements from the singularity.
The subelements are refined using the edge splitting technique which is later described in Section \ref{sec:element_refinement}.

\subsection{Output Mesh}
\label{sec:output_mesh}
% Bey's algorithm (red refinement of tetrahedron):
%p.29 https://www5.in.tum.de/pub/Joshi2016_Thesis.pdf
%p.108 http://www.bcamath.org/documentos_public/archivos/publicaciones/sergey_book.pdf    - brandts_2011
%https://www.math.uci.edu/~chenlong/iFEM/doc/html/uniformrefine3doc.html#1
%J. Bey. Simplicial grid refinement: on Freudenthal's algorithm and the optimal number of congruence classes. Numer. Math. 85(1):1--29, 2000. p11 Algorithm: RedRefinement3D.
%p.4 http://www.vis.uni-stuttgart.de/uploads/tx_vispublications/vis97-grosso.pdf

Due to the enrichment, the finite element approximation is not polynomial anymore.
If we want to look at the discrete solution, e.g. in Paraview, we have to interpolate it into a~space of piecewise
constant or linear functions which can be viewed in such software. However the interpolation on the computational mesh
is quite coarse, so we suggested creating a~finer mesh specially refined for the output.

Two types of refinement is implemented -- uniform and adaptive. In the first case the mesh is uniformly refined in
specified number of steps. In the later case the refinement is governed by an~error criterion which bounds 
the measured elementwise $L_2$ norm of the difference between the (enriched) discrete solution and its constant/linear approximation.
The refinement criterion reads as follows
\begin{equation} \label{eqn:output_ref_criterion}
    \frac{\norm{v_h - \pi_{out}v_h}_{T}}{\norm{v_h}_{T}} < \textrm{tol}
\end{equation}
where $\pi_{out}$ is the interpolator to a~space of constant or linear functions on an arbitrary element $T$ of the output mesh.
The relative tolerance $\textrm{tol}$ is specified by the user.

The refinement is ment to be used for the enriched discrete solutions in particular, however in Flow123d the output mesh can be refined
according to any scalar quantity present in the model. Thus for the quantity $v_h$ in \eqref{eqn:output_ref_criterion} one can use
for example the source term, the hydraulic conductivity or the supplied analytic solution.

\subsubsection{Element refinement}
\label{sec:element_refinement}
The refinement of elements for the output mesh is done using edge splitting technique (the so called red refinement).
Since the output mesh is used only for better visualization of non-polynomial quantities, we do not
care about any hanging nodes present in the refined mesh.

In 2d case, the refinement of an element is a~straightforward process: find the midpoints of all sides, connect them and generate 4 triangles.
These triangles are congruent and have equal surface areas.
%
\begin{figure}[!htb]
    \centering    
    \includegraphics[width=0.6\textwidth]{\results output_refine.pdf} 
    \caption[output mesh refinement]
  {An example of an adaptive output mesh refinement in 3d.
  A~singular function $1/r$ is displayed.}
  \label{fig:output_refinement_flow123d}
\end{figure}
%
On the other hand, the 3d case is more complicated. After splitting the edges, we obtain 4 tetrahedra at the vertices
of the original one. The octahedron that remains in the middle can be subdivided according to one of its three diagonals.
Only the choice of the shortest octahedron diagonal leads to a regular tetrahedra decomposition.
This algorithm originally comes from Bey~\cite{bey_2000}, further e.g. in~\cite{brandts_2011}.

We demonstrate the refinement in \fig{fig:output_refinement_flow123d} where we used the adaptively refined output mesh to
visualize a~singular function $1/r$ in a~cube. The domain is actually cut by a~diagonal plane. One can see the mesh edges in
the left half of the cube.








% \section{2d singular model}
% 
% In ambient 3d space, consider 2d problem with point singularity where the pressure $P_w$ is given.
%   
%   Mixed form:
%   \begin{eqnarray}
% \vc K_2^{-1}\vc u_2 + \nabla p_2 &=& 0 \qquad \textrm{in } \Omega_2, \label{eqn:2d_darcy_law}\\
% \nabla \cdot \vc u_2 &=& f_2 \qquad \textrm{in } \Omega_2, \label{eqn:2d_continuity}\\
% \avg{\vc u_2 \cdot \vc n}_w &=& \sigma_w (\avg{p_2}_w - P_w) \qquad \textrm{on } \Gamma_w,\label{eqn:2d_well_cond}\\
% p_2 &=& g_{2D} \qquad \textrm{on } \Gamma_{2D}.
%   \end{eqnarray}
%   
%   \paragraph{Geometry:} well center at $\vc x_w=[x_w,y_w]$, domain of different shapes (square, triangle, circle) with characteristic length $D$, Dirichlet BC $g_{2D}$ on the whole boundary.
%   
%   \emph{Settings 1}: $\vc x_w=[3.33,3.33],\, P_w = 100,\, \rho_w=0.03,\, \sigma_w=10,\, \vc K_2=K_2=1,\, f_2=0,\, D=10.$
%   
%   \emph{Settings 2}: same, $f_2=KU\omega^2\sin(\omega x_w),\, U=20,\, \omega=1.$
%   
%   \paragraph{Analytic solution.} The pressure solution in Settings 1 is set in such form 
%   \begin{eqnarray}
%       p_2 &=& a\log r,\label{eqn:2d_press_sol1}\\
%       \vc u_2 &=& -\vc K_2 a \frac{\vc r}{r^2} \label{eqn:2d_vel_sol1}
%   \end{eqnarray} 
%   that it fulfills the condition \eqref{eqn:2d_well_cond} on the well edge
%   \[ -\vc K_2\nabla p_2(\rho_w) \cdot \vc n = K_2a\frac{1}{\rho_w} = \sigma_w(p_2(\rho_w) - P_w),\]
%   which results in setting
%   \[a=\frac{-\sigma_w\rho_w P_w}{K_2 - \sigma_w\rho_w\log r}\]
%   Logarithm is a harmonic function so it holds $\Delta \log(r) = 0$ and the equation in strong form is satisfied.
%   
%   Setting the nonzero source term $f_2$ in Settings 2, we obtain additional regular part of the solution
%   \begin{eqnarray}
%     p_2 &=& a\log r + U\sin(\omega x_w) \label{eqn:2d_press_sol2}\\
%     \vc u_2 &=& -\vc K_2\nabla p_2 = -K_2 a\frac{\vc r}{r^2} -K_2 U\omega\cos(\omega x_w)\vc e_x \label{eqn:2d_vel_sol2}
%   \end{eqnarray}
%   
%   With Settings 1, we can use these analytic values to observe error in our numerical model:
%   \begin{center}
%   \begin{tabular}{rl}
%     flux over $\Gamma_w$: & $\bar{b}_w \doteq 6.5628$ \\ % 6.5627733533
%     pressure $\avg{p}_w$: & $\bar{\lambda}_w \doteq 78.1241$ %78.1240888223
%   \end{tabular}
%   
%   \centering
%   \begin{tabular}{>{\centering\arraybackslash}p{4.5ex}|
%                     >{\centering\arraybackslash}p{8.3ex}|
%                     >{\centering\arraybackslash}p{9ex}|
%                     >{\centering\arraybackslash}p{12.5ex}
%                     >{\centering\arraybackslash}p{4ex}|
%                     >{\centering\arraybackslash}p{12.5ex}
%                     >{\centering\arraybackslash}p{4ex}}
%                     \multicolumn{5}{c}{$\;\;\;$ source $f=0$} & \multicolumn{2}{c}{\textcolor{OrangeRed}{source $f$}} \\
%     ref level  & $\abs{b_w-\bar{b}_w}$ & $\abs{\lambda_w - \bar{\lambda}_w}$ & $\norm{\vc u- \bar{\vc u}}_{L^2(\Omega)}$ & rate & $\norm{\vc u- \bar{\vc u}}_{L^2(\Omega)}$ & rate \\\hline
%     1  &  0.1000  &  0.0528  &  6.73  &  -   &  112.5  &  -   \\
%     2  &  0.1600  &  0.0871  &  2.96  & 1.18 &  46.1   & 1.29 \\
%     3  &  0.0290  &  0.0155  &  1.68  & 0.82 &  21.9   & 1.08 \\
%     4  &  0.0064  &  0.0034  &  0.94  & 0.84 &  10.6   & 1.05 \\
%     5  &  0.0027  &  0.0014  &  0.51  & 0.88 &  5.6    & 0.91 
%   \end{tabular}
%   \end{center}
%   
%   
%   \subsection{Multiple singularity solution}
%   Considering multiple singularities in the domain, denoted using index set $\mathcal{W}=\{1\ldots W\}$, we can obtain the solution using the superposition principle.
%   The solution is searched in the form
%   \begin{eqnarray}
% %     p_2 &=& p_{sin} + p_{reg} = \sum\limits_{j\in\mathcal{W}} a_j\log r_j + p_{reg}, \label{eqn:2d_press_sol_mult}\\
%     \vc u_2 &=& -\vc K_2\nabla p_2 = -K_2 \sum\limits_{j\in\mathcal{W}} a_j\frac{\vc r_j}{r_j^2} - K_2 \grad p_{reg}, \label{eqn:2d_vel_mult}
%   \end{eqnarray}
%   which satisfies equations \eqref{eqn:2d_continuity} and \eqref{eqn:2d_darcy_law}.
%   To find the unknowns $a_j$, we need to solve the system of linear equations given by the conditions \eqref{eqn:2d_well_cond}
%   \begin{equation}
%     \avg{\vc u_2 \cdot \vc n}_{\Gamma_i} = \sigma_i (\avg{p_2}_{\Gamma_i} - P_i) \qquad  \forall i\in\mathcal{W}. \label{eqn:2d_well_cond_mult}
%   \end{equation}
%   For an average $\avg{\cdot}_{\Gamma_i}$, it holds $\avg{u+v}_{\Gamma_i} = \avg{u}_{\Gamma_i} + \avg{v}_{\Gamma_i}$.
%   Substituing \eqref{eqn:2d_press_sol_mult} and \eqref{eqn:2d_vel_mult} into \eqref{eqn:2d_well_cond_mult}, we get
%   the linear system $\vc M\vc a = \vc b$ where
%   \begin{eqnarray}
%     M_{ii} &=& \frac{K_2}{\rho_i} - \sigma_i\log\rho_i \qquad \forall i\in\mathcal{W},\\
%     M_{ij} &=& K_2\avg{\frac{\vc r_j \cdot \vc n_i}{r_j^2}}_{\Gamma_i} - \sigma_i\avg{\log r_j}_{\Gamma_i} \qquad \forall i,j\in\mathcal{W}, i\neq j,\\
%     b_{i} &=& \sigma_i\left(\avg{p_{reg}}_{\Gamma_i} - P_i \right) - \avg{- K_2 \grad p_{reg}\cdot n_i}_{\Gamma_i} \qquad \forall i\in\mathcal{W}.
%   \end{eqnarray}
%   We evaluate all the averages $\avg{\cdot}_{\Gamma_i}$ in the linear system numerically,
%   using simple composite midpoint rule integration with 1000 equidistant intervals on $\Gamma_i$.
%   This way we obtain pseudo-analytic solution of the multi well problem, which is accurate enough
%   be used for measuring the error of our model.

  
  
\section{Numerical Tests in Flow123d}
We present several numerical tests in which we demonstrate the properties and behaviour of the suggested enrichment in the mixed-hybrid method.
The tests are computed using our implementation in Flow123d.
The test setting in 2d similar to the one used in Section \ref{sec:2d_results_single} and \ref{sec:2d_results_multiple}.
We will again investigate cases on different domains, with a~single well or with multiple wells, with a~source term or without.

An input file for Flow123d in Yaml format is prepared for each test.
The pseudo-analytic solution is passed into Flow123d as a~Python script.
The solution error is evaluated against the~pseudo-analytic solution determined according
to Section \ref{sec:prim_analytic_solution}. We remind the pressure solution \eqref{eqn:2d_press_sol_mult} from which the velocity solution is derived:
\begin{eqnarray}
p_2 &=& p_{sin} + p_{reg} = \sum\limits_{j\in\mathcal{W}} a_j\log r_j + p_{reg}, \label{eqn:2d_press_solution}\\
\vc u_2 &=& -\vc K_2 \grad p_2 = - \vc K_2 \Bigg[ \sum\limits_{j\in\mathcal{W}} a_j\frac{\vc r_j}{r_j^2} + \grad p_{reg} \Bigg].  \label{eqn:2d_vel_solution}
\end{eqnarray}

The meshes used for computations are simplicial and unstructured.
They are prepared by Gmsh software using the default "MeshAdapt" algorithm.
% The model is computed on a~serie of refined meshes and the convergence of our method is examined.
When examining the convergence of the method, we use linux shell scripts and parametrized ".geo" files for Gmsh,
to generate series of refined meshes.

We output the results into VTK format and visualize them in Paraview software. 

\subsubsection{Test Case 1}
In the first test case we consider $\Omega_2$ to be a~circular shaped domain of radius 5.0.
A~well is perpendicular to the domain and is cross-secting the domain in its center.

Since the enrichment in 2d is of our main interest in this test case, we want to minimize any other effects
influencing the approximation error. The strength of the singularity is determined by the pressure difference
between the well and aquifer, so the accuracy of the solution in the vicinity of the well is also dependent
on the accuracy of pressure in the well, on $p_1(\vc x_w)$ in particular.
A~simple way to achieve that is simulating constant pressure inside the well by setting very high conductivity $\vc K_1=10^{10}$
and setting constant Dirichlet boundary condition $g_{1D}$ on both ends of the well.

Let us denote three different settings: 
\begin{itemize}
    \item TC1-a: $f=0$,
    \item TC1-b: $f=U\sin(\omega x)$,
    \item TC1-c: $f=U\sin(\omega x)$, no singularity (regular case)
\end{itemize}
The first setting considers a~zero source term,
the second sets the sinusoidal source term.
The last setting is the regular problem without the singularity, which we solve with
the standard mixed-hybrid finite element method (MHFEM).
The input parameters are gathered in Table \ref{tab:tc1_data}.
%
\begin{table}[!hb]
\begin{center}
\begin{tabular}{cccccccc}
\toprule
% \multicolumn{2}{c}{Item} \\
% \cmidrule(r){1-2}
$\vc K$ & $\bx_w$  & $\rho_w$ & $\sigma_w$ & $R_w$ & $g_{1D}$ & $\omega$ & $U$ \\
\midrule
$10^{-3}$ & [3.33,3.33] & 0.03 & 10.0 & 2.0 & 100 & 1.0 & 200\\
\bottomrule
\end{tabular}
\caption{Input data for Test case 1.}
\label{tab:tc1_data}
\end{center}
\end{table}

% \noindent
The model is computed on a~serie of refined meshes and the convergence of our method is examined.
The approximation error of velocity is displayed in \fig{fig:mh_tc1_error}. We can see that 
the error is mainly acummulated outside the edge of the enriched zone in case of zero source term.
Considering the source term, the error of the singular part is inferior to the error of the regular part, as it is apparent
in the right subfigure.
%
\begin{figure}[!htb]
    \centering
    \subfloat[TC1-a]{
    \includegraphics[width=0.498\textwidth]{\results tc-23-error.pdf} }
    %\hspace{5pt}
    \subfloat[TC1-b]{
    \includegraphics[width=0.498\textwidth]{\results tc-24-error.pdf} }
    \caption
    {Results of Test case 1. The distribution of $L_2$ elementwise error in velocity is displayed at refinement level 5.
    The green circle indicates the enrichment radius $R_w$.}
    \label{fig:mh_tc1_error}
\end{figure}
%
%
% 24_2d_circle_well_source_only_template.yaml
%
\begin{table}[!htb]
\begin{center}
\bgroup
\def\arraystretch{1.2}
\setlength\tabcolsep{5pt}
% \begin{tabular}{r|c|c|c|c|c|r|r}
\begin{tabular}{rc|cc|cc|cc}
\toprule
\multicolumn{2}{c|}{} & \multicolumn{2}{c|}{ TC1-a} & \multicolumn{2}{c|}{TC1-b} & \multicolumn{2}{c}{TC1-c}\\ [3pt] %\midrule
i & h & $\|\vc u-\vc u_h\|_{L^2(\Omega_2)}$ & order & $\|\vc u-\vc u_h\|_{L^2(\Omega_2)}$
    & order & $\|\vc u-\vc u_h\|_{L^2(\Omega_2)}$ & order \\ [3pt] \midrule
1 & 1.368 &  $7.11\cdot10^{-3}$  &  -   &  $2.42\cdot10^{-1}$  &  -   &  $2.42\cdot10^{-1}$ &   -   \\
2 & 0.913 &  $5.31\cdot10^{-3}$  & 0.72 &  $1.72\cdot10^{-1}$  & 0.84 &  $1.72\cdot10^{-1}$ &  0.84 \\
3 & 0.575 &  $3.30\cdot10^{-3}$  & 1.03 &  $9.99\cdot10^{-2}$  & 1.19 &  $9.99\cdot10^{-2}$ &  1.19 \\
4 & 0.358 &  $2.19\cdot10^{-3}$  & 0.87 &  $6.40\cdot10^{-2}$  & 0.94 &  $6.41\cdot10^{-2}$ &  0.94 \\
5 & 0.271 &  $1.61\cdot10^{-3}$  & 1.11 &  $4.50\cdot10^{-2}$  & 1.26 &  $4.51\cdot10^{-2}$ &  1.26 \\
6 & 0.176 &  $1.08\cdot10^{-3}$  & 0.92 &  $3.03\cdot10^{-2}$  & 0.92 &  $3.02\cdot10^{-2}$ &  0.94 \\
\bottomrule
\end{tabular}
\caption{Convergence table of SGFEM in Test case 1.}
\label{tab:mh_tc1_convergence}
\egroup
\end{center}
\end{table}

The convergence results are shown in Table \ref{tab:mh_tc1_convergence}.
There are three convergence columns in the table corresponding to different settings.
The one on the left is for the zero source term case,
the middle one is for the model including the source term,
the last column shows the convergence of the regular problem without the singularity solved with MHFEM.
We see that the convergence of velocity is nearly optimal, closing to 1.0, in all three situations.
Comparing the magnitude of the approximation error between TC1-b and TC1-c,
we see that the singularity is only minimally affecting the error.


\subsubsection{Test Case 2}
In the second test case we consider a~square shaped domain $\Omega_2=[0,10]\times[0,10]$,
with two wells perpendicular to the domain $\Omega_2$.
Similarly to the previous case, the source term is set to zero or defined as $f=U\sin(\omega x)$, $U=80$, $\omega=1.0$.
The pressure inside the wells is fixed to a~constant value in the same way as in the previous case 
by setting high conductivity $\vc K_1$ and constant Dirichlet boundary condition $g_{1D}$ on both ends of the well.
The input parameters for the wells are gathered in Table \ref{tab:tc2_data}.
%
\begin{table}[!hb]
\begin{center}
\begin{tabular}{ccccccccc}
\toprule
% \multicolumn{2}{c}{Item} \\
% \cmidrule(r){1-2}
$w$ & $\bx_w$  & $\rho_w$ & $\sigma_w$ & $R_w$ & $g_{1D}$\\
\midrule
1& [4.1,4.3] & 0.03 & 10.0 & 2.0 & 150 \\
2& [5.7,5.9] & 0.03 & 10.0 & 2.0 & 100 \\
\bottomrule
\end{tabular}
\caption{Input data for the wells in Test case 2.}
\label{tab:tc2_data}
\end{center}
\end{table}

We solved the model with two different enrichment radii. Let us denote four different settings:
\begin{itemize}
    \item TC2-a: $f=0$, $R_w=0.6$
    \item TC2-b: $f=0$, $R_w=2.0$
    \item TC2-c: $f=U\sin(\omega x)$, $R_w=0.6$
    \item TC2-d: $f=U\sin(\omega x)$, $R_w=2.0$
    \item TC2-e: $f=U\sin(\omega x)$, no singularity (regular case)
\end{itemize}
Setting the larger enrichment radius leads to enrichment zones overlap.
% The last situation, the regular problem, is computed only as a~reference as in Test Case 1.

The approximation error of velocity is displayed in \fig{fig:mh_tc2_error}. We can again observe
the error mainly acummulating outside the edge of the enriched zone in TC2-a,b,c.
The error near the first well is little higher since the singularity is stronger (there is higher pressure inside the well).
In the case TC2-b, we see a~very small error in the overlap of the enrichment zones
since both singular enrichments are active there.
Comparing TC2-c and TC2-d, we see that the larger enrichment radius makes the error in the singular part insignificant
in the scale of the error of the regular part.
% Considering the source term, the error of the singular part is inferior to the error of the regular part, as it is apparent
% in the right subfigure.

%
\begin{figure}[!htb]
    \centering
    \subfloat[TC2-a]{
    \includegraphics[width=0.498\textwidth]{\results tc_26_no_source_error_0,6.pdf} }
    \subfloat[TC2-b]{
    \includegraphics[width=0.498\textwidth]{\results tc_26_no_source_error_2,0.pdf} } \\
    \subfloat[TC2-c]{
    \includegraphics[width=0.498\textwidth]{\results tc_26_source_error_0,6.pdf} }
    \subfloat[TC2-d]{
    \includegraphics[width=0.498\textwidth]{\results tc_26_source_error_2,0.pdf} }
    \caption
    {Results of Test case 2. The distribution of $L_2$ elementwise error in velocity is displayed at refinement level 5.
    The green circle indicates the enrichment radius $R_w$. }
    \label{fig:mh_tc2_error}
\end{figure}
%

\begin{table}[!htb]
\begin{center}
\bgroup
\def\arraystretch{1.2}
\setlength\tabcolsep{5pt}
% \begin{tabular}{r|c|c|c|c|c|r|r}
\begin{tabular}{rc|cc|cc|cc}
\toprule
\multicolumn{2}{c|}{} & \multicolumn{2}{c|}{ TC2-a } & \multicolumn{2}{c|}{ TC2-c } & \multicolumn{2}{c}{TC2-d}\\ [3pt] %\midrule
i & h & $\|\vc u-\vc u_h\|_{L^2(\Omega_2)}$ & order & $\|\vc u-\vc u_h\|_{L^2(\Omega_2)}$
    & order & $\|\vc u-\vc u_h\|_{L^2(\Omega_2)}$ & order \\ [3pt] \midrule
1 & 1.368 &  $6.71\cdot10^{-2}$  &  -   &  $1.38\cdot10^{-1}$  &  -   &  $1.10\cdot10^{-1}$ &   -   \\
2 & 0.913 &  $3.44\cdot10^{-2}$  & 1.67 &  $7.80\cdot10^{-2}$  & 1.42 &  $6.94\cdot10^{-1}$ &  1.13 \\
3 & 0.575 &  $2.64\cdot10^{-2}$  & 0.70 &  $5.50\cdot10^{-2}$  & 0.93 &  $4.59\cdot10^{-2}$ &  1.10 \\
4 & 0.358 &  $1.89\cdot10^{-2}$  & 0.81 &  $3.85\cdot10^{-2}$  & 0.87 &  $3.12\cdot10^{-2}$ &  0.93 \\
5 & 0.271 &  $1.34\cdot10^{-2}$  & 0.89 &  $2.62\cdot10^{-2}$  & 0.98 &  $2.06\cdot10^{-2}$ &  1.06 \\
6 & 0.176 &  $9.24\cdot10^{-3}$  & 0.90 &  $1.79\cdot10^{-2}$  & 0.94 &  $1.37\cdot10^{-2}$ &  0.99 \\
\bottomrule
\end{tabular}
\caption{Convergence table of SGFEM in Test case 2.}
\label{tab:mh_tc2_convergence}
\egroup
\end{center}
\end{table}

The convergence results are shown in Table \ref{tab:mh_tc2_convergence}.
The three columns in the table correspond to three selected settings in Test case 2.
We see that the optimal convergence order of velocity approximation is reached in all situations.
The overlapping enrichment zones do not corrupt the accuracy of the approximation in TC2-d.
Comparing the magnitude of the error in TC2-c and TC2-d columns, we see that the larger enrichment radius leads
to a~slightly increased accuracy while the convergence order is the same.

The model with TC2-b setting is converging also optimally, but we do not present all the results for the sake of brevity.
The larger enrichment radius in TC2-b leads to decrease of the magnitude of the overall error,
approximately with factor of 3.0 to the TC2-a setting.
Similarly to Test case 1, the regular problem without singularities TC2-e is also computed.
The convergence results correspond to the setting TC2-d,
the magnitude of the approximation error and the convergence order are almost identical.

% %
% \begin{figure}[!htb]
%     \centering
%     \includegraphics[width=0.85\textwidth]{\results tc_26_inf-sup.pdf}
%     \caption[numerical verification of inf-sup condition TC2]
%     {Graph representing numerical verification of the inf-sup condition in Test Case 2.}
%     \label{fig:mh_tc2_inf_sup}
% \end{figure}
% %



\subsubsection{Test Case 3}
This test copies the setting of Test case 2, however it includes five wells.
The input parameters for the wells are gathered in Table \ref{tab:tc3_data}.
The wells 1,2 and 4 can be seen as the pumping wells, the others the injection wells.
Let us again denote 4 different settings:
\begin{itemize}
    \item TC3-a: $f=0$, $R_w=0.8$,
    \item TC3-b: $f=0$, $R_w=2.0$,
    \item TC3-c: $f=U\sin(\omega x)$, $R_w=0.8$,
    \item TC3-d: $f=U\sin(\omega x)$, $R_w=2.0$.
%     \item TC3-e: $f=U\sin(\omega x)$, no singularity
\end{itemize}
%
\begin{table}[!htb]
\begin{center}
\begin{tabular}{ccccccccc}
\toprule
% \multicolumn{2}{c}{Item} \\
% \cmidrule(r){1-2}
$w$ & $\bx_w$  & $\rho_w$ & $\sigma_w$ & $g_{1D}$\\
\midrule
1& [2.8,2.5] & 0.03 & 20.0 & -150 \\
2& [4.9,5.4] & 0.03 & 10.0 & -30 \\
3& [2.9,7.4] & 0.03 & 10.0 & 120 \\
4& [7.3,7.8] & 0.03 & 10.0 & -50 \\
5& [7.4,2.8] & 0.03 & 20.0 & 100 \\
\bottomrule
\end{tabular}
\caption{Input data for the wells in Test case 3.}
\label{tab:tc3_data}
\end{center}
\end{table}
%
\begin{figure}[!htb]
    \centering
    \subfloat[TC3-a]{
    \includegraphics[width=0.498\textwidth]{\results tc_51_no_source_error_0,8.pdf} }
    \subfloat[TC3-b]{
    \includegraphics[width=0.498\textwidth]{\results tc_51_no_source_error_2,0.pdf} } \\
    \subfloat[TC3-c]{
    \includegraphics[width=0.498\textwidth]{\results tc_51_source_error_0,8.pdf} }
    \subfloat[TC3-d]{
    \includegraphics[width=0.498\textwidth]{\results tc_51_source_error_2,0.pdf} }
    \caption
    {Results of Test case 3. The distribution of $L_2$ elementwise error in velocity is displayed at refinement level 5.
    The green circle indicates the enrichment radius $R_w$. }
    \label{fig:mh_tc3_error}
\end{figure}
%

The distribution of the velocity error is shown in \fig{fig:mh_tc3_error}. 
We can see in setting TC3-a that the error is again concentrated on the edges of the enrichment zones of
the strongest singularities. In setting TC3-b, the error is largest in the central area, 
which is affected by all the singularities. The solution is well approximated inside the four enrichment zones
of the wells 1,3,4,5, we see only small effects of the other singularities there.

In the subfigures for settings TC3-c and TC3-d, we see the same behaviour of the error as in previous test case.
The error of the regular part is significant in TC3-d while the error of the singular part is not apparent.

The convergence for the selected settings is displayed in Table \ref{tab:mh_tc3_convergence}.
We see nearly optimal convergence order in all three columns. It is again apparent that
the larger enrichment radius pushes the approximation error down a~bit but it does not effect the
convergence order.

\begin{table}[!htb]
\begin{center}
\bgroup
\def\arraystretch{1.2}
\setlength\tabcolsep{5pt}
% \begin{tabular}{r|c|c|c|c|c|r|r}
\begin{tabular}{rc|cc|cc|cc}
\toprule
\multicolumn{2}{c|}{} & \multicolumn{2}{c|}{ TC3-a } & \multicolumn{2}{c|}{ TC3-c } & \multicolumn{2}{c}{TC3-d}\\ [3pt] %\midrule
i & h & $\|\vc u-\vc u_h\|_{L^2(\Omega_2)}$ & order & $\|\vc u-\vc u_h\|_{L^2(\Omega_2)}$
    & order & $\|\vc u-\vc u_h\|_{L^2(\Omega_2)}$ & order \\ [3pt] \midrule
1 & 1.412 & $2.82\cdot10^{-2}$  &  -   & $1.19\cdot10^{-1}$  &  -   & $1.16\cdot10^{-1}$ &   -   \\
2 & 0.946 & $2.11\cdot10^{-2}$  & 0.72 & $7.67\cdot10^{-2}$  & 1.10 & $7.33\cdot10^{-2}$ &  1.13 \\
3 & 0.650 & $1.43\cdot10^{-2}$  & 1.03 & $5.06\cdot10^{-2}$  & 1.11 & $4.87\cdot10^{-2}$ &  1.09 \\
4 & 0.431 & $1.08\cdot10^{-2}$  & 0.68 & $3.50\cdot10^{-2}$  & 0.89 & $3.32\cdot10^{-2}$ &  0.93 \\
5 & 0.292 & $7.34\cdot10^{-3}$  & 1.00 & $2.31\cdot10^{-2}$  & 1.06 & $2.19\cdot10^{-2}$ &  1.06 \\
6 & 0.193 & $5.05\cdot10^{-3}$  & 0.91 & $1.56\cdot10^{-2}$  & 0.96 & $1.47\cdot10^{-2}$ &  0.98 \\
\bottomrule
\end{tabular}
\caption{Convergence table of SGFEM in Test case 3.}
\label{tab:mh_tc3_convergence}
\egroup
\end{center}
\end{table}

%
% \begin{figure}[!htb]
%     \centering
%     \includegraphics[width=0.85\textwidth]{\results tc_51_inf-sup.pdf}
%     \caption[numerical verification of inf-sup condition TC3]
%     {Graph representing numerical verification of the inf-sup condition in Test Case 3.}
%     \label{fig:mh_tc3_inf_sup}
% \end{figure}
%

\subsubsection{Test Case 4}
We now present several test cases in 3 dimensions. The geometries of the test cases
are analogic to the previous test cases in 2d, but extruded to the third dimension in $z$ coordinate.

In the first 3d test case, the 3d domain $\Omega_3$ is a~cylinder with the bottom base in $xy$ plane
with the center at $[3.33,3.33,0]$, of height 2.0 and radius 5.0.
A~well is cross-secting the cylinder along its vertical axis.
Similarly to the 2d case, we set high conductivity $\vc K_1$ inside the well and constant Dirichlet boundary condition $g_{1D}$ on both ends of the well,
to have constant pressure there.
Homogenous Neumann boundary condition is applied on both bases of the cylinder. The flow is then governed 
by the well and the boundary condition on the lateral surface of the cylinder.
Therefore we can still use the pseudo-analytic solution from Section \ref{sec:prim_analytic_solution}.

Let us have three different settings: 
\begin{itemize}
    \item TC4-a: $f=0$,
    \item TC4-b: $f=U\sin(\omega x)$,
    \item TC4-c: $f=U\sin(\omega x)$, no singularity (regular case).
\end{itemize}
We use the same parameters for the nonzero source terms as before: $U=80$, $\omega=1.0$.
The rest of the input parameters are gathered in Table \ref{tab:tc4_data}.
%
\begin{table}[!htb]
\begin{center}
\begin{tabular}{cccccccc}
\toprule
% \multicolumn{2}{c}{Item} \\
% \cmidrule(r){1-2}
$\vc K$ & $\rho_w$ & $\sigma_w$ & $R_w$ & $g_{1D}$ & $\omega$ & $U$ \\
\midrule
$10^{-3}$ & 0.03 & 10.0 & 2.0 & 100 & 1.0 & 80\\
\bottomrule
\end{tabular}
\caption{Input data for Test case 4.}
\label{tab:tc4_data}
\end{center}
\end{table}
%
\begin{figure}[!htb]
    \centering
%     \subfloat[TC4-a]{
%     \includegraphics[width=0.498\textwidth]{\results tc_25_no_source_error_1,0.pdf} }
    \subfloat[TC4-b]{
    \includegraphics[width=0.498\textwidth]{\results tc_25_no_source_error_2,0.pdf} }
%     \subfloat[TC4-c]{
%     \includegraphics[width=0.498\textwidth]{\results tc_25_source_error_1,0.pdf} }
    \subfloat[TC4-d]{
    \includegraphics[width=0.498\textwidth]{\results tc_25_source_error_2,0.pdf} }
    \caption
    {Results of Test case 4. The distribution of $L_2$ elementwise error in velocity is displayed at refinement level 5.
    The green cylinders indicates the enrichment zone. }
    \label{fig:mh_tc4_error}
\end{figure}

\begin{table}[!htb]
\begin{center}
\bgroup
\def\arraystretch{1.2}
\setlength\tabcolsep{5pt}
% \begin{tabular}{r|c|c|c|c|c|r|r}
\begin{tabular}{rc|cc|cc|cc}
\toprule
\multicolumn{2}{c|}{} & \multicolumn{2}{c|}{ TC4-a } & \multicolumn{2}{c|}{ TC4-b } & \multicolumn{2}{c}{TC4-c}\\ [3pt] %\midrule
i & h & $\|\vc u-\vc u_h\|_{L^2(\Omega_2)}$ & order & $\|\vc u-\vc u_h\|_{L^2(\Omega_2)}$
    & order & $\|\vc u-\vc u_h\|_{L^2(\Omega_2)}$ & order \\ [3pt] \midrule
1 & 1.348 &  $1.16\cdot10^{-2}$  &  -   &  $1.81\cdot10^{-1}$  &  -   &  $1.81\cdot10^{-1}$ &   -   \\
2 & 1.063 &  $1.10\cdot10^{-2}$  & 0.20 &  $1.50\cdot10^{-1}$  & 0.79 &  $1.50\cdot10^{-1}$ &  0.80 \\
3 & 0.845 &  $8.89\cdot10^{-3}$  & 0.95 &  $1.12\cdot10^{-1}$  & 1.27 &  $1.12\cdot10^{-2}$ &  1.28 \\
4 & 0.632 &  $7.90\cdot10^{-3}$  & 0.40 &  $8.90\cdot10^{-2}$  & 0.79 &  $8.88\cdot10^{-2}$ &  0.79 \\
5 & 0.544 &  $7.27\cdot10^{-3}$  & 0.56 &  $7.36\cdot10^{-2}$  & 1.28 &  $7.34\cdot10^{-2}$ &  1.28 \\
6 & 0.408 &  $6.67\cdot10^{-3}$  & 0.30 &  $5.45\cdot10^{-2}$  & 1.04 &  $5.47\cdot10^{-2}$ &  1.02 \\
\bottomrule
\end{tabular}
\caption{Convergence table in Test case 4.}
\label{tab:mh_tc4_convergence}
\egroup
\end{center}
\end{table}

The approximation error of velocity is displayed in \fig{fig:mh_tc4_error}. We can see that higher
error is acummulated outside the edge of the enriched zone in TC4-a. However we observe significant error also
along the edges of the cylinder bases where the two boundaries with the Dirichlet boundary condition and the zero Neumann condition
are adjacent. The velocity shape functions on the elements with sides on both types of boundaries are obviously unable to approximate
the solution correctly there. This error on the boundary also slows down the convergence rate, as it can be seen in Table \ref{tab:mh_tc4_convergence}.
Considering the source term in TC4-b, the error of the singular part is inferior to the error of the regular part, as it is apparent
in the right subfigure. The error along the edges of the cylinder bases is negligible in contrast to TC4-a.



\subsubsection{Test Case 5}
In the second 3d test case, the domain $\Omega_3$ is a~block with the bottom base in $xy$ plane,
with height 2.0.
Two wells cross-secting the block are perpendicular to its base.
The parameters of the wells are specified in Table \ref{tab:tc5_data}.

\begin{table}[!hb]
\begin{center}
\begin{tabular}{cccccccc}
\toprule
% \multicolumn{2}{c}{Item} \\
% \cmidrule(r){1-2}
$w$ & $\bx_w$  & $\rho_w$ & $\sigma_w$ & $g_{1D}$\\
\midrule
1& [4.1,4.3] & 0.03 & 10.0 & 150 \\
2& [5.7,5.9] & 0.03 & 10.0 & 100 \\
\bottomrule
\end{tabular}
\caption{Input data for the wells in Test case 5.}
\label{tab:tc5_data}
\end{center}
\end{table}
%
Analogically to the previous test case, the constant pressure inside the wells is enforced
and zero normal flux is prescribed on both bases of the block..
We use again the pseudo-analytic solution to determine the approximation error.

Let us have four different settings: 
\begin{itemize}
    \item TC5-a: $f=0$, $R_w=1.0$,
    \item TC5-b: $f=0$, $R_w=2.0$,
    \item TC5-c: $f=U\sin(\omega x)$, $R_w=1.0$,
    \item TC5-d: $f=U\sin(\omega x)$, $R_w=2.0$.
\end{itemize}
We use the same parameters for the nonzero source terms as before: $U=80$, $\omega=1.0$.
%
\begin{figure}[!htb]
    \centering
    \subfloat[TC5-a]{
    \includegraphics[width=0.498\textwidth]{\results tc_33_2w_no_source_error_1,0.pdf} }
    \subfloat[TC5-b]{
    \includegraphics[width=0.498\textwidth]{\results tc_33_2w_no_source_error_2,0.pdf} } \\
    \subfloat[TC5-c]{
    \includegraphics[width=0.498\textwidth]{\results tc_33_2w_source_error_1,0.pdf} }
    \subfloat[TC5-d]{
    \includegraphics[width=0.498\textwidth]{\results tc_33_2w_source_error_2,0.pdf} }
    \caption
    {Results of Test case 5. The distribution of $L_2$ elementwise error in velocity is displayed at refinement level 5.
    The green cylinders indicates the enrichment zone. }
    \label{fig:mh_tc5_error}
\end{figure}
%
\begin{table}[!htb]
\begin{center}
\bgroup
\def\arraystretch{1.2}
\setlength\tabcolsep{5pt}
% \begin{tabular}{r|c|c|c|c|c|r|r}
\begin{tabular}{rc|cc|cc|cc}
\toprule
\multicolumn{2}{c|}{} & \multicolumn{2}{c|}{ TC5-a } & \multicolumn{2}{c|}{ TC5-c } & \multicolumn{2}{c}{TC5-d}\\ [3pt] %\midrule
i & h & $\|\vc u-\vc u_h\|_{L^2(\Omega_2)}$ & order & $\|\vc u-\vc u_h\|_{L^2(\Omega_2)}$
    & order & $\|\vc u-\vc u_h\|_{L^2(\Omega_2)}$ & order \\ [3pt] \midrule
1 & 1.404 & $5.70\cdot10^{-2}$  &  -   & $2.49\cdot10^{-1}$  &  -   & $2.43\cdot10^{-1}$ &   -   \\
2 & 1.190 & $5.36\cdot10^{-2}$  & 0.37 & $2.07\cdot10^{-1}$  & 1.12 & $2.04\cdot10^{-1}$ &  1.05 \\
3 & 0.878 & $4.29\cdot10^{-2}$  & 0.73 & $1.55\cdot10^{-1}$  & 0.95 & $1.53\cdot10^{-2}$ &  0.93 \\
4 & 0.681 & $3.38\cdot10^{-2}$  & 0.95 & $1.19\cdot10^{-1}$  & 1.02 & $1.18\cdot10^{-2}$ &  1.04 \\
5 & 0.515 & $2.83\cdot10^{-2}$  & 0.63 & $9.22\cdot10^{-2}$  & 0.92 & $9.03\cdot10^{-2}$ &  0.96 \\
6 & 0.416 & $2.37\cdot10^{-2}$  & 0.83 & $7.51\cdot10^{-2}$  & 0.96 & $7.35\cdot10^{-2}$ &  0.96 \\
\bottomrule
\end{tabular}
\caption{Convergence table in Test case 5.}
\label{tab:mh_tc5_convergence}
\egroup
\end{center}
\end{table}

We show the distribution of the error in \fig{tab:mh_tc5_convergence} and the convergence results in Table \ref{fig:mh_tc5_error}.
In case of settings TC5-a and TC5-b, a~significant error can be again observed on the edges of the block bases as in the previous case.
Due to this error the convergence rate is sub-optimal.
On the other hand for the settings TC5-c and TC5-d, the results are satisfying and the approximation error behaves similarly
as in the 2 cases. The error of the regular part is dominating as it can be seen in the subfigures. The convergence order 
is optimal.

The well 1 causes a~stronger singularity due to the higher pressure difference,
therefore the error outside the enrichment zone of the well if also higher.
This is apparent especially in the subfigure TC5-b. We can also observe very low error
inside the enrichment zones overlap.

% \subsubsection{Test Case 5}
% In the second 3d test case, the domain $\Omega_3$ is a~block with the bottom base in $xy$ plane,
% with height 2.0.
% Two wells cross-secting the block are perpendicular to its base.
% The parameters of the wells are specified in Table \ref{tab:tc5_data}.
% %
% \begin{table}[!hb]
% \begin{center}
% \begin{tabular}{cccccccc}
% \toprule
% % \multicolumn{2}{c}{Item} \\
% % \cmidrule(r){1-2}
% $w$ & $\bx_w$  & $\rho_w$ & $\sigma_w$ & $g_{1D}$\\
% \midrule
% 1& [3.6,4.1] & 0.03 & 100.0 & 150 \\
% 2& [6.9,7.4] & 0.03 & 100.0 & -100 \\
% \bottomrule
% \end{tabular}
% \caption{Input data for the wells in Test case 5.}
% \label{tab:tc5_data}
% \end{center}
% \end{table}
% 
% Analogically to the previous test case, the constant pressure inside the wells is enforced
% and zero normal flux is prescribed on both bases of the block..
% We use again the pseudo-analytic solution to determine the approximation error.
% 
% Let us have three different settings: 
% \begin{itemize}
%     \item TC5-a: $f=0$, $R_w=2.0$,
%     \item TC5-b: $f=U\sin(\omega x)$, $R_w=1.0$,
%     \item TC5-c: $f=U\sin(\omega x)$, $R_w=2.0$.
% \end{itemize}
% We use the same parameters for the nonzero source terms as before: $U=80$, $\omega=1.0$.
% %
% \begin{figure}[!htb]
%     \centering
%     \subfloat[TC5-a]{
%     \includegraphics[width=0.498\textwidth]{\results tc_31_2w_no_source_error_1,0.pdf} }
%     \subfloat[TC5-b]{
%     \includegraphics[width=0.498\textwidth]{\results tc_31_2w_no_source_error_2,0.pdf} } \\
%     \subfloat[TC5-c]{
%     \includegraphics[width=0.498\textwidth]{\results tc_31_2w_source_error_1,0.pdf} }
%     \subfloat[TC5-d]{
%     \includegraphics[width=0.498\textwidth]{\results tc_31_2w_source_error_2,0.pdf} }
%     \caption
%     {Results of Test case 5. The distribution of $L_2$ elementwise error in velocity is displayed at refinement level 5.
%     The green cylinders indicates the enrichment zone. }
%     \label{fig:mh_tc5_error}
% \end{figure}
% %
% \begin{table}[!htb]
% \begin{center}
% \bgroup
% \def\arraystretch{1.2}
% \setlength\tabcolsep{5pt}
% % \begin{tabular}{r|c|c|c|c|c|r|r}
% \begin{tabular}{rc|cc|cc|cc}
% \toprule
% \multicolumn{2}{c|}{} & \multicolumn{2}{c|}{ TC5-a } & \multicolumn{2}{c|}{ TC5-b } & \multicolumn{2}{c}{TC5-c}\\ [3pt] %\midrule
% i & h & $\|\vc u-\vc u_h\|_{L^2(\Omega_2)}$ & order & $\|\vc u-\vc u_h\|_{L^2(\Omega_2)}$
%     & order & $\|\vc u-\vc u_h\|_{L^2(\Omega_2)}$ & order \\ [3pt] \midrule
% 1 & 1.404 &  $1.98\cdot10^{-2}$  &  -   &  $2.50\cdot10^{-1}$  &  -   &  $2.49\cdot10^{-1}$ &   -   \\
% 2 & 1.190 &  $1.84\cdot10^{-2}$  & 0.47 &  $2.10\cdot10^{-1}$  & 1.04 &  $2.10\cdot10^{-1}$ &  1.03 \\
% 3 & 0.878 &  $1.56\cdot10^{-2}$  & 0.54 &  $1.58\cdot10^{-1}$  & 0.93 &  $1.59\cdot10^{-2}$ &  0.92 \\
% 4 & 0.681 &  $1.39\cdot10^{-2}$  & 0.46 &  $1.23\cdot10^{-1}$  & 1.01 &  $1.23\cdot10^{-2}$ &  1.02 \\
% 5 & 0.515 &  $1.23\cdot10^{-2}$  & 0.42 &  $9.32\cdot10^{-2}$  & 0.98 &  $9.31\cdot10^{-2}$ &  0.99 \\
% 6 & 0.416 &  $1.15\cdot10^{-2}$  & 0.34 &  $7.57\cdot10^{-2}$  & 0.97 &  $7.54\cdot10^{-2}$ &  0.98 \\
% \bottomrule
% \end{tabular}
% \caption{Convergence table in Test case 5.}
% \label{tab:mh_tc5_convergence}
% \egroup
% \end{center}
% \end{table}
% 
% We show the distribution of the error in \fig{tab:mh_tc5_convergence} and the convergence results in Table \ref{fig:mh_tc5_error}.
% In case of settings TC5-a and TC5-b, a~significant error can be again observed on the edges of the block bases as in the previous case.
% Due to this error the convergence rate is sub-optimal.
% On the other hand for the settings TC5-c and TC5-d, the results are satisfying and the approximation error behaves similarly
% as in the 2 cases. The error of the regular part is dominating as it can be seen in the subfigures. The convergence order 
% is optimal.

\subsubsection{Test Case 6}
This test case is analogic to the previous test case, however it includes 5 wells.
The wells are perpendicular to the $xy$ plane and they have the same parameters as in the 2d Test Case 3.

The following settings are used:
\begin{itemize}
    \item TC6-a: $f=0$, $R_w=1.0$,
    \item TC6-b: $f=U\sin(\omega x)$, $R_w=1.0$,
    \item TC6-c: $f=U\sin(\omega x)$, no singularity.
\end{itemize}
%
\begin{figure}[!htb]
    \centering
    \subfloat[TC6-a]{
    \includegraphics[width=0.498\textwidth]{\results tc_31_5w_no_source_error_1,0.pdf} }
    \subfloat[TC6-b]{
    \includegraphics[width=0.498\textwidth]{\results tc_31_5w_source_error_1,0.pdf} }
    \caption
    {Results of Test case 6. The distribution of $L_2$ elementwise error in velocity is displayed at refinement level 5.
    The green cylinders indicates the enrichment zone. }
    \label{fig:mh_tc6_error}
\end{figure}
%

\begin{table}[!htb]
\begin{center}
\bgroup
\def\arraystretch{1.2}
\setlength\tabcolsep{5pt}
% \begin{tabular}{r|c|c|c|c|c|r|r}
\begin{tabular}{rc|cc|cc|cc}
\toprule
\multicolumn{2}{c|}{} & \multicolumn{2}{c|}{ TC6-a } & \multicolumn{2}{c|}{ TC6-b } & \multicolumn{2}{c}{TC6-c}\\ [3pt] %\midrule
i & h & $\|\vc u-\vc u_h\|_{L^2(\Omega_2)}$ & order & $\|\vc u-\vc u_h\|_{L^2(\Omega_2)}$
    & order & $\|\vc u-\vc u_h\|_{L^2(\Omega_2)}$ & order \\ [3pt] \midrule
1 & 1.404 &  $3.98\cdot10^{-2}$  &  -   &  $2.52\cdot10^{-1}$  &  -   &  $2.49\cdot10^{-1}$ &   -   \\
2 & 1.190 &  $3.67\cdot10^{-2}$  & 0.49 &  $2.15\cdot10^{-1}$  & 0.96 &  $2.10\cdot10^{-1}$ &  1.03 \\
3 & 0.878 &  $3.00\cdot10^{-2}$  & 0.66 &  $1.62\cdot10^{-1}$  & 0.93 &  $1.59\cdot10^{-2}$ &  0.92 \\
4 & 0.681 &  $2.49\cdot10^{-2}$  & 0.74 &  $1.25\cdot10^{-1}$  & 1.01 &  $1.23\cdot10^{-2}$ &  1.02 \\
5 & 0.515 &  $2.12\cdot10^{-2}$  & 0.57 &  $9.55\cdot10^{-2}$  & 0.98 &  $9.31\cdot10^{-2}$ &  0.99 \\
6 & 0.416 &  $1.85\cdot10^{-2}$  & 0.64 &  $7.75\cdot10^{-2}$  & 0.97 &  $7.54\cdot10^{-2}$ &  0.98 \\
\bottomrule
\end{tabular}
\caption{Convergence table in Test case 6.}
\label{tab:mh_tc6_convergence}
\egroup
\end{center}
\end{table}

The distribution of the error is displayed in \fig{tab:mh_tc6_convergence}
and the convergence results are summarized in Table \ref{fig:mh_tc6_error}.
Setting the zero source term in the setting TC6-a, the optimal convergence rate is again corrupted by the error on the edges of the bases of the block,
although it is not that significant as in previous cases.
Considering the model with the setting TC6-b, we see that the singularities are well approximated
and the dominant error is in the regular part of the solution. The optimal convergence is comparable to the regular case TC6-c.

\section{Summary}
In this chapter we developed a~new XFEM method for approximation of singularities both in 2 and 3 dimensions.
The model was formulated in the mixed-hybrid form following the concept of the software Flow123d.
We used what we have learned in Chapter \ref{chap:xfem_pressure} and transfered our experience into creating
an SGFEM like enrichment of velocity in the vicinity of the singularities to improve the approximation.

Regarding the reduced dimensional modeling, our approach enables coupling of non-coplanar 1d and 2d domains
(lines intersecting planes in a~single point) and coupling of co-dimension 2 in 3d (lines intersecting 3d volume).
The possibility to compute such problems on incompatible meshes is a~huge advantage.

