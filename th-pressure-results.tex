\section{Numerical Tests}
\label{sec:pressure_results}

In this section we suggest several numerical tests and provide comparison of different enrichments
as suggested in Section \ref{sec:enrichment_func}.

The implementation has been done in C++ language using the Deal II~\cite{bangerth_deal.ii_2007}, version 8.0, 
an open source finite element library. This library provides a well-documented code for high range of operations one needs
for solving partial differential equation: mesh generation, quadratures, different scalar and vector finite elements,
linear algebra, $h$ and $p$ adaptivity, error estimators, postprocessing and output to various formats.
However, it does not support any enrichment techniques like XFEM in the version 8.0.0,
and so it does not in the actual version 9.0.0, released in May, 2018, up to our best knowledge.

We use as much as possible of the relevant library code, although the enrichment functions, well intersections,
distribution and handling of the enriched degrees of freedom, adaptive quadrature and some output routines
must have been implemented on our own.
The source is available on GitHub:
\begin{center}\url{https://github.com/Paulie14/xfem_project}.\end{center}


\subsection{Test Cases with Single Well}
\label{sec:2d_results_single}

% We consider a~single well model with the following common input data.
% The domain $\Omega_2$ is a~square $(-2,2)\times(-2,2)$ and the well is characterized by 
% $\bx_w=[0.004,0.004]$,  $\rho_w=0.003$, $g^w_{1D}=9$, $\sigma_w=10^5$ and $c_w=10^{10}$, see geometry in \fig{fig:geometry}. 
% The enrichment radius is set $R_w=0.3$.
% The test cases differ by the regular part $p_{reg}$ and the source term $f_2$:

% \subsubsection{Comparison of Enrichment Methods} \label{sec:res_comparison}

Let us now define four test cases on which we investigate the behavior of the methods, mainly their convergence properties.
We suggest four analytic solutions which differ by the regular part $p_{reg}$ 
and the corresponding source term $f_2$. Thus the quality of the approximation
of both the singular and the regular part can be later inspected to make a conclusion whether
it behaves as expected.
The list of $p_{reg}$ and $f_2$ follows:
%
\begin{itemize}
\item case 1: $p_{reg} = 0$, $f_2 = 0$,
\item case 2: $p_{reg} = U\sin(\omega x)(r_w-\rho_w)^2$,
    \begin{multline*}
        \qquad\;\, f_2 = -K_2 U\bigg[\left(4 - 2 \frac{\rho_w}{r_w} \right) \sin(\omega x)\\      
              + 4(r_w-\rho_w)\frac{\vc{r}\cdot\vc{e}_1}{r_w} \omega\cos(\omega x)
                - (r_w-\rho_w)^2\omega^2\sin(\omega x) \bigg],
    \end{multline*}
\item case 3: $p_{reg} = U r_w^2$, $f_2 = -4 K_2 U$,
\item case 4: $p_{reg} = U\sin(\omega x)$, $f_2 = K_2 U\omega^2\sin(\omega x)$.
\end{itemize}
The source term $f_2$ is visualized in \fig{fig:prim_sources}.
%
\begin{figure}[!htb]
%   \vspace{0pt}
    \centering    
%     \subfloat[test case 1]{\label{fig:source01} 
%         \includegraphics[width=0.46\textwidth]{\results source07_final.pdf} }
% %   \hspace{5pt}
%     \hfill
    \subfloat[test case 2]{\label{fig:source02} 
        \includegraphics[width=0.46\textwidth]{\results source04_final.pdf} }
    \vskip\baselineskip
    \subfloat[test case 3]{\label{fig:source03} 
        \includegraphics[width=0.46\textwidth]{\results source05_final.pdf} }
%   \hspace{5pt}
    \hfill
    \subfloat[test case 4]{\label{fig:source04} 
        \includegraphics[width=0.46\textwidth]{\results source07_final.pdf} }
    
  \caption{Source term $f_2$ visualization.}
  \label{fig:prim_sources}
\end{figure}
%
Test case 1 includes only the singularity and has zero source term, therefore the approximation
of the singularity itself is in the focus. 
% Later we present the original result from \cite{exner_2016} for this test case.
In Test case 2, the part of the searched solution $p_{reg}$ has both its value and gradient zero on the edge of the well $\Gamma_w$.
% we have no effects of the source term directly on $\Gamma_w$.
In Test case 3, $p_{reg}$ has constant value and gradient on $\Gamma_w$, and finally $p_{reg}$ is fluctuating on $\Gamma_w$ in Test case 4.
Thus from test cases 1 to 4 we are going from the simplest to the most complex problem
in the view of how the approximation is constructed in the vicinity of the well.
% Thus the average decomposition $\Gamma_w$ is not necessary as in the previous cases.
% and the average decomposition plays important role to achieve optimal convergence.

We now list the input data common to all test cases.
The permeability between the aquifer and the well is set to $10^5$.
The value of the parameter $c_w$ (well permeability), explained in \eqref{eqn:linear_pressure_in_1d} is set to $10^{10}$,
so the pressure in the well is nearly constant.
The aquifer domain is a~square $\Omega=[-l_\Omega, l_\Omega]\times[-l_\Omega,l_\Omega]$, with $l_
\Omega$ defined for each test case individually.
As we shall see below, the approximation of the singularity by means of the standard FEM shape functions is better 
when a~node of $\mathcal{T}_{2}$ is inside the well cross-section. Therefore the center of the singularity $\bx_w$
is placed such that we do not take advantage of this feature and solve the worse scenario.
See the common geometry of the test cases in \fig{fig:test_cases_geometry}.
% %
% \begin{table}
% \begin{center}
% \begin{tabular}{c|cccc}
% \toprule
% % \multicolumn{2}{c}{Item} \\
% % \cmidrule(r){1-2}
% parameter & case 1 & case 2 & case 3 & case 4 \\
% \midrule
% $l_\Omega$  & 100    & 2      & 2      & 2  \\ 
% $\bx_w$     & [5.43,5.43]  & [0.004,0.004]  & [0.004,0.004]  & [0.004,0.004] \\
% $\rho_w$    & 0.2    & 0.003  & 0.003  & 0.003 \\
% $\sigma_w$  & $10^5$ & $10^5$ & $10^5$ & $10^5$ \\
% $c_w$       & $10^{10}$ & $10^{10}$ & $10^{10}$ & $10^{10}$ \\
% $R_w$       & 50     & 0.3    & 0.3    & 0.3 \\
% $g^w_{1D}$  & 100    & 4      & 4      & 4  \\
% $\omega$    & --     & 6      & --     & 6  \\
% $U$         & --     & 4      & 0.7    & 4  \\
% \bottomrule
% \end{tabular}
% \caption{Input data for the considered test cases.}
% \label{tab:test_cases_data}
% \end{center}
% \end{table}
% %

\begin{figure}[!htb]
    \centering    
    \def\svgwidth{0.35\textwidth}
        \input{\results geometry.pdf_tex}
  \caption[Geometry of the single aquifer domain.]{Geometry of the aquifer domain common to all the test cases.}
  \label{fig:test_cases_geometry}
\end{figure}

% \begin{figure}[!htb]
% %   \vspace{0pt}
%   \centering    
%   \subfloat[geometry of the problem]{\label{fig:geometry} 
% %     \begin{center}         
%       \def\svgwidth{0.325\textwidth}
%       \input{\results geometry.pdf_tex}
% %     \end{center} 
%       }
%   \hspace{5pt}
%   \subfloat[source term $f$]{\label{fig:solution} 
%     \includegraphics[width=0.58\textwidth]{\results source_term.pdf} }
%   \caption[]
%   {Geometry and the prescribed source term. Note that the cylinder representing the well is thicker then the actual well.}
% %   \label{fig:adapt_refinement}
% \end{figure}


We are interested in the behavior of the enrichment in the 2d domain.
Thus we measure the convergence of the methods only inside the aquifer.
% and we do not pay attention to the solution in the well.
The error of the solution is measured in $L^2$ norm, which is evaluated using higher order quadrature
on the unenriched elements and adaptive quadrature in the enriched area, as defined in Section \ref{sec:adaptive_quad_rules}.

\subsubsection{Test Case 1}
Let us start with Test case 1 and present the results achieved in \cite{exner_2016}.
%
\begin{table}[!htb]
\begin{center}
\begin{tabular}{ccccc}
\toprule
% \multicolumn{2}{c}{Item} \\
% \cmidrule(r){1-2}
$l_\Omega$ & $\bx_w$  & $\rho_w$ & $R_w$ & $g^w_{1D}$\\
\midrule
100 & [5.43,5.43] & 0.2 & 50 & 100 \\
\bottomrule
\end{tabular}
\caption{Input data for Test case 1.}
\label{tab:test_case_1_data}
\end{center}
\end{table}
%
The parameters are gathered in Table \ref{tab:test_case_1_data}.
Convergence of all used methods is measured on a~set of uniformly refined meshes, 
see the convergence rate in \gref{graph:convergence01}.
%
\begin{graph}[!htb]
%   \vspace{0pt}
  \centering    
  \includegraphics[width=\textwidth]{\results convergence01.pdf}
  \caption[Convergence graph in Test case 1.]{Convergence of the $L^2$ norm of the approximation error in Test case 1.}
  \label{graph:convergence01}
\end{graph}
%
At first, we solve the problem by the standard FEM using $h$ adaptivity.
Kelly's error estimator from Deal II library is used to determine 30\% of the elements
with the highest error which are refined in the next step.
The element size $h$ for the convergence graph is determined as the size of the smallest element in the 
vicinity of the well.
From the graph of "FEM\_adapt" it is apparent that the convergence is slow until the size of the elements reaches the scale of the
well cross-section and one of the mesh nodes gets inside. 
Therefore the graph is divided into two parts with different convergence orders 0.56 and 1.27.
%We see that a very fine mesh is needed to capture the singularity but the error is still high.

Next we look at the convergence of the enrichment methods in the XFEM.
The standard XFEM pushes the error down by three orders of magnitude. Its convergence rate is nearly optimal with the order closing to 2.0.
On the other hand we have no results on finer meshes due to the conditioning of the linear system
which deteriorates rapidly for $h<2$ and our conjugate gradient (CG) solver does not converge.
The same problem arises using the ramp function XFEM. It deals better with the error on blending elements but the
ill-conditioning of the system matrix still corrupts the computation. We discuss the conditioning of the system 
a bit more in the next subsection \ref{sec:res_conditioning}.

The shifted XFEM and the SGFEM behave nearly the same way and give the best results as expected.
We alert the reader that the convergence graphs of ramp function XFEM, shifted XFEM and SGFEM overlap in \gref{graph:convergence01}.
We only mention that the SGFEM saves small amount of degrees of freedom on blending elements in comparison
with the shifted and the ramp function XFEM. The order of convergence in $L^2$ norm closing to 2.0 is optimal.


\subsubsection{Test Case 2}
Next we present the results in Test case 2, the convergence graph is in \gref{graph:convergence02}.
%
\begin{table}[!htb]
\begin{center}
\begin{tabular}{ccccccc}
\toprule
% \multicolumn{2}{c}{Item} \\
% \cmidrule(r){1-2}
$l_\Omega$ & $\bx_w$  & $\rho_w$ & $R_w$ & $g^w_{1D}$ & $\omega$ & $U$ \\
\midrule
2 & [0.004,0.004] & 0.003 & 0.3 & 4 & 6 & 4\\
\bottomrule
\end{tabular}
\caption{Input data for Test case 2.}
\label{tab:test_case_2_data}
\end{center}
\end{table}
%
The parameters are gathered in Table \ref{tab:test_case_2_data}.
We expect the standard FE shape functions to optimally approximate the regular part of the solution
and the shape functions of the enrichment to capture the singularity.
%
\begin{graph}[!htb]
%   \vspace{0pt}
  \centering    
  \includegraphics[width=\textwidth]{\results convergence02.pdf}
  \caption[Convergence graph in Test case 2.]{Convergence of the $L^2$ norm of the approximation error in Test case 2. The $\textrm{FEM}_{reg}$
  data comes from the problem without the well solved by standard FEM and with optimal convergence order 2.0.}
  \label{graph:convergence02}
\end{graph}
%
Therefore we solve the problem on the aquifer domain only with the source term $f_2$ but omitting the singularity
to have a reference solution -- an~approximation of $p_{reg}$. We plot this solution labeled as $\textrm{FEM}_{reg}$ in the convergence graph.
Since the source term $f_2$ is a~smooth and bounded function, the FEM is supposed to converge optimally with order 2.0
which is confirmed in the graph.
The error in XFEM is then expected to be of the same magnitude as the error $\textrm{FEM}_{reg}$.

Using the standard FEM approximation, see the line FEM in \gref{graph:convergence02}, the convergence order is low 0.5,
because the size of the elements is still far from $\rho_w$ and it cannot capture the singularity at all.

The XFEM behave similarly as in the Test case 1. The error of the enrichment methods is very close to the error $\textrm{FEM}_{reg}$,
except standard XFEM where the error in the blending elements is significant. The ill-conditioning of the linear system again 
disallows computing on finer meshes in case of the standard and ramp function XFEM.
The shifted XFEM and the SGFEM converge optimally and the error is dominated by the error of the regular part.


\subsubsection{Test Case 3}
The application of the standard and ramp function XFEM suffers from the same weaknesses, as we have shown above, in both Test case 3 and 4.
Due to the fact that these two methods do not perform well in our problems, we do not deal with them any further
and we do not present obtained results.
The parameters are gathered in Table \ref{tab:test_case_3_data}.
%
\begin{table}[!htb]
\begin{center}
\begin{tabular}{cccccc}
\toprule
% \multicolumn{2}{c}{Item} \\
% \cmidrule(r){1-2}
$l_\Omega$ & $\bx_w$  & $\rho_w$ & $R_w$ & $g^w_{1D}$ & $U$ \\
\midrule
2 & [0.004,0.004] & 0.003 & 0.3 & 4 & 0.7\\
\bottomrule
\end{tabular}
\caption{Input data for Test case 3.}
\label{tab:test_case_3_data}
\end{center}
\end{table}
%

\begin{table}[!htb]
\begin{center}
\bgroup
\def\arraystretch{1.2}
\setlength\tabcolsep{5pt}
% \begin{tabular}{r|c|c|c|c|c|r|r}
\begin{tabular}{rc|cc|cc|cc}
\toprule
\multicolumn{2}{c|}{} & \multicolumn{2}{c|}{$\textrm{FEM}_{reg}$} & \multicolumn{2}{c|}{shifted XFEM} & \multicolumn{2}{c}{SGFEM}\\ [3pt] %\midrule
i & h & $\|p-p_h\|_{L^2(\Omega_2)}$ & order & $\|p-p_h\|_{L^2(\Omega_2)}$ & order & $\|p-p_h\|_{L^2(\Omega_2)}$ & order \\ [3pt] \midrule
0 & 0.5000 & 2.45e-01 & --   & 2.73e-01 & --   & 2.35e-01 & --   \\ %\hline
1 & 0.2500 & 6.12e-02 & 2.17 & 7.89e-02 & 1.79 & 6.70e-02 & 1.81 \\ %\hline
2 & 0.1250 & 1.53e-02 & 1.95 & 1.60e-02 & 2.30 & 1.60e-02 & 2.07 \\ %\hline
3 & 0.0625 & 3.82e-03 & 1.99 & 4.08e-03 & 1.97 & 4.08e-03 & 1.97 \\ %\hline
4 & 0.0312 & 9.56e-04 & 2.00 & 1.02e-03 & 2.00 & 1.02e-03 & 2.00 \\ %\hline
5 & 0.0156 & 2.39e-04 & 2.00 & 2.58e-04 & 1.98 & 2.58e-04 & 1.98 \\ %\hline
6 & 0.0078 & 5.97e-05 & 2.00 & 6.44e-05 & 2.00 & 6.44e-05 & 2.00 \\ %\hline
7 & 0.0039 & 1.49e-05 & 2.00 & 1.61e-05 & 2.00 & 1.61e-05 & 2.00 \\ %\hline 
\bottomrule
\end{tabular}
\caption[Convergence table in Test case 3.]
{Convergence table of the shifted XFEM and SGFEM in Test case 3.}
\label{tab:convergence_test3}
\egroup
\end{center}
\end{table}
%
Table \ref{tab:convergence_test3} shows the convergence of the shifted XFEM and SGFEM for Test case 3.
In the column $\textrm{FEM}_{reg}$ the optimal convergence with the order 2.0 of the standard FEM on the regular problem is displayed as a~reference.
We see that both shifted XFEM and SGFEM converge also with the optimal convergence order 2.0. The error of both enrichment methods is a bit higher than the error of $\textrm{FEM}_{reg}$, we would further diminish this
difference by enlarging the enrichment radius. Looking at the graph in \fig{fig:error_distribution_test3} where the error distribution is plotted,
we see that the error is concentrated outside the enrichment radius circle. It is caused due to the insufficient approximation of the singularity
by the standard FE shape functions. We also see from this figure that the approximation quality of the quadrilateral FEs on structured mesh is direction dependent.
%
\begin{figure}[!htb]
%   \vspace{0pt}
  \centering    
  \includegraphics[width=0.5\textwidth]{\results error05_final.pdf}
  \caption[Error distribution in Test case 3.]{$L^2$ error distribution in Test case 3 at the refinement level level 4 ($h=0.0312$).
  A green circle represents the enrichment radius $R_w$. There is no visible difference between shifted XFEM and SGFEM, so only SGFEM is plotted.}
  \label{fig:error_distribution_test3}
\end{figure}
%

\subsubsection{Test Case 4}
The last case of our test series demonstrates the necessity of the averaged terms in the model formulation.
The parameters are gathered in Table \ref{tab:test_case_4_data}.
%
\begin{table}[!htb]
\begin{center}
\begin{tabular}{ccccccc}
\toprule
% \multicolumn{2}{c}{Item} \\
% \cmidrule(r){1-2}
$l_\Omega$ & $\bx_w$  & $\rho_w$ & $R_w$ & $g^w_{1D}$ & $\omega$ & $U$ \\
\midrule
2 & [0.004,0.004] & 0.003 & 0.3 & 4 & 6 & 4\\
\bottomrule
\end{tabular}
\caption{Input data for Test case 4.}
\label{tab:test_case_4_data}
\end{center}
\end{table}
%

The convergence of the considered methods is shown in Table \ref{tab:convergence_test4}, in the same manner as in Test case 3.
We see that the convergence order of shifted XFEM and SGFEM is optimal up to the refinement level 6 and the error corresponds to
the error of the regular problem in $\textrm{FEM}_{reg}$ column. 
%
\begin{table}[!htb]
\begin{center}
\bgroup
\def\arraystretch{1.2}
\setlength\tabcolsep{5pt}
% \begin{tabular}{r|c|c|c|c|c|r|r}
\begin{tabular}{rc|cc|cc|cc}
\toprule
\multicolumn{2}{c|}{} & \multicolumn{2}{c|}{$\textrm{FEM}_{reg}$} & \multicolumn{2}{c|}{shifted XFEM} & \multicolumn{2}{c}{SGFEM}\\ [3pt] %\midrule
i & h & $\|p-p_h\|_{L^2(\Omega_2)}$ & order & $\|p-p_h\|_{L^2(\Omega_2)}$ & order & $\|p-p_h\|_{L^2(\Omega_2)}$ & order \\ [3pt] \midrule
0 & 0.5000 & 1.01e+01 & --   & 9.30e+00 & --   & 9.92e+00 & --   \\ %\hline
1 & 0.2500 & 2.25e+00 & 2.17 & 2.21e+00 & 2.07 & 2.24e+00 & 2.15 \\ %\hline
2 & 0.1250 & 5.84e-01 & 1.95 & 5.75e-01 & 1.94 & 5.80e-01 & 1.95 \\ %\hline
3 & 0.0625 & 1.47e-01 & 1.99 & 1.46e-01 & 1.98 & 1.47e-01 & 1.98 \\ %\hline
4 & 0.0312 & 3.70e-02 & 2.00 & 3.66e-02 & 2.00 & 3.67e-02 & 2.00 \\ %\hline
5 & 0.0156 & 9.24e-03 & 2.00 & 9.15e-03 & 2.00 & 9.17e-03 & 2.00 \\ %\hline
6 & 0.0078 & 2.31e-03 & 2.00 & 2.41e-03 & 1.93 & 2.39e-03 & 1.94 \\ %\hline
7 & 0.0039 & 5.78e-04 & 2.00 & 1.05e-03 & 1.20 & 1.01e-03 & 1.24 \\ %\hline 
\bottomrule
\end{tabular}
\egroup
\caption[Convergence table in Test case 4.]
{Convergence table of the shifted XFEM and SGFEM in Test case 4.}
\label{tab:convergence_test4}
\end{center}
\end{table}

Then the error of the enrichment methods does not decrease as expected at the refinement level 7 and partially 6. 
To explain this behavior we plot the error distribution at two different refinement levels in \fig{fig:error_distribution_test4}.
On the left side, the error in the regular part is dominating. On the right side, the error of the singular part dominates in the center
and it spreads in $x$ direction. Closer look at the solution on the well edge shows that the pressure on the well edge is incorrect.
The problem has been solved several times for different $\rho_w$ (smaller and larger) and this error always shows up when there is 
an~element of the mesh with all its nodes inside the well-aquifer intersection.
We have a~suspicion that the elimination of the degrees of freedom of such elements from the linear system is done incorrectly
in our implementation. On the finer meshes where $\rho_w$ is larger than the element size,
neglecting the fluctuation term on the well edge \eqref{eqn:well_edge_fluctuation_neglect} might actually increase the error.
Since our model is not describing the pressure in the aquifer inside the well-aquifer intersection
(the pressure is constant and equal $p_1(\bx_w)$ according to the reduced 1d model of the well),
and we do not aim in reality to compute on such fine meshes, we do not pursue solving this error at the moment.

\begin{figure}[!htb]
%   \vspace{0pt}
  \centering
  \subfloat[refinement level 4]{\label{fig:error_distribution_test4a} 
        \includegraphics[width=0.48\textwidth]{\results error07_04_final.pdf} }
  \hfill
  \subfloat[refinement level 7]{\label{fig:error_distribution_test4b} 
        \includegraphics[width=0.48\textwidth]{\results error07_07_final.pdf} }
  \caption[Error distribution in Test case 4.]{$L^2$ error distribution in Test case 4 at the refinement level level 4 and 7.
  A green circle represents the enrichment radius $R_w$. Pay attention to the logarithmic scale in the right figure.}
  \label{fig:error_distribution_test4}
\end{figure}
%

% \begin{table}[!htb]
% \begin{center}
% \bgroup
% \def\arraystretch{1.2}
% \setlength\tabcolsep{7pt}
% % \begin{tabular}{r|c|c|c|c|c|r|r}
% \begin{tabular}{rc|cc|cc|rr}
% \toprule
% %\hline
% %&&\multicolumn{2}{c|}{}&&\\[-15pt]
% % i & h & \multicolumn{2}{c|}{$\|p-p_h\|_{L^2(\Omega)}$} & $N^{reg}_{dofs}$ & $N^{enr}_{dofs}$\\ [3pt] \hline
% \multicolumn{2}{c|}{} & \multicolumn{2}{c|}{Test case 3} & \multicolumn{2}{c|}{Test case 4} & \multicolumn{2}{c}{}\\ [3pt] %\midrule
% i & h & $\|p-p_h\|_{L^2(\Omega)}$ & rate & $\|p-p_h\|_{L^2(\Omega)}$ & rate & $N^{reg}_{dofs}$ & $N^{enr}_{dofs}$ \\ [3pt] \midrule
% 0 & 0.5000 & 2.73e-01 & -    & 9.92e+00 & -    & 87 & 6         \\ %\hline
% 1 & 0.2500 & 7.89e-02 & 1.79 & 2.24e+00 & 2.15 & 295 & 6        \\ %\hline
% 2 & 0.1250 & 1.60e-02 & 2.30 & 5.80e-01 & 1.95 & 1110 & 21      \\ %\hline
% 3 & 0.0625 & 4.08e-03 & 1.97 & 1.47e-01 & 1.98 & 4294 & 69      \\ %\hline
% 4 & 0.0312 & 1.02e-03 & 2.00 & 3.67e-02 & 2.00 & 16930 & 289    \\ %\hline
% 5 & 0.0156 & 2.58e-04 & 1.98 & 9.17e-03 & 2.00 & 67206 & 1157   \\ %\hline
% 6 & 0.0078 & 6.44e-05 & 2.00 & 2.39e-03 & 1.94 & 267797 & 4628  \\ %\hline
% 7 & 0.0039 & 1.61e-05 & 2.00 & 1.01e-03 & 1.24 & 1069134 & 18509\\ %\hline 
% \bottomrule
% \end{tabular}
% \egroup
% \label{tab:sgfem_convergence_test34}
% \caption{Convergence table of the SGFEM in Test case 3 and 4.}
% \end{center}
% \end{table}
% 
% \paragraph{Test case 4}
% \begin{table}[!htb]
% \begin{center}
% \bgroup
% \def\arraystretch{1.2}
% \setlength\tabcolsep{7pt}
% % \begin{tabular}{r|c|c|c|c|c|r|r}
% \begin{tabular}{rc|cc|cc|rr}
% \toprule
% %\hline
% %&&\multicolumn{2}{c|}{}&&\\[-15pt]
% % i & h & \multicolumn{2}{c|}{$\|p-p_h\|_{L^2(\Omega)}$} & $N^{reg}_{dofs}$ & $N^{enr}_{dofs}$\\ [3pt] \hline
% \multicolumn{2}{c|}{} & \multicolumn{2}{c|}{Test case 3} & \multicolumn{2}{c|}{Test case 4} & \multicolumn{2}{c}{}\\ [3pt] %\midrule
% i & h & $\|p-p_h\|_{L^2(\Omega)}$ & rate & $\|p-p_h\|_{L^2(\Omega)}$ & rate & $N^{reg}_{dofs}$ & $N^{enr}_{dofs}$ \\ [3pt] \midrule
% 0 & 0.5000 & 2.35e-01 & -    & 9.30e+00 & -    & 103 & 22       \\ %\hline
% 1 & 0.2500 & 6.70e-02 & 1.81 & 2.21e+00 & 2.07 & 311 & 22       \\ %\hline
% 2 & 0.1250 & 1.60e-02 & 2.07 & 5.75e-01 & 1.94 & 1134 & 45      \\ %\hline
% 3 & 0.0625 & 4.08e-03 & 1.97 & 1.46e-01 & 1.98 & 4334 & 109     \\ %\hline
% 4 & 0.0312 & 1.02e-03 & 2.00 & 3.66e-02 & 2.00 & 17010 & 369    \\ %\hline
% 5 & 0.0156 & 2.58e-04 & 1.98 & 9.15e-03 & 2.00 & 67362 & 1313   \\ %\hline
% 6 & 0.0078 & 6.44e-05 & 2.00 & 2.41e-03 & 1.93 & 268105 & 4936  \\ %\hline
% 7 & 0.0039 & 1.61e-05 & 2.00 & 1.05e-03 & 1.20 & 1069750 & 19125\\ %\hline
% \bottomrule
% \end{tabular}
% \egroup
% \label{tab:xfem_convergence_test34}
% \caption{Convergence table of the shifted XFEM in Test case 3 and 4.}
% \end{center}
% \end{table}

\subsubsection{Test Cases Summary}
We solved four single well-aquifer problems with different solutions on the well-aquifer intersection and different source terms.
We showed that the enrichment methods can sufficiently approximate the singularity and that the $L^2$ error of the approximation
can be pushed as low as the error of the regular part of the solution. The optimal convergence order 2.0 has been reached for all
the enrichment methods. The standard XFEM displayed significant error on the blending elements in contrast to the ramp function XFEM.
However, both suffered from ill-conditioning of the linear system. The most promising results were obtained using the shifted XFEM
and SGFEM, which behaved very similarly in our test cases.

Regarding the order convergence 1.8 presented by Gracie and Craig \cite{gracie_modelling_2010}, we obtained similar convergence order 
around 1.7-1.8 in our experiments using the original adaptive quadrature. Although, the order could be lower 
depending on the position of the well to the nodes of the mesh. We do not experience this behavior with our adaptive
quadrature and the convergence order is always close to the optimum of 2.0.

% The error of the shifted XFEM and the SGFEM addressed in \cite{exner_2016} for Test case 4 is suppressed due to two improvements. 
% Firstly the analytic solution has been derived more generally, including the effect of the well permeability $c_w$
% and the averaged terms, which better corresponds to the weak form.
% Secondly the code has been improved and debugged heavily since then and there was apparently a~problem in assembling the averaged terms.



\subsection{Conditioning of System Matrix} \label{sec:res_conditioning}
A~problem with ill-conditioning of the linear system coming from the XFEM methods is mentioned in Section \ref{sec:soa_xfem}
and are encountered also in our test cases in the section above. We do not inspect the matrices of the linear system in details, 
but we use some general results on the conjugate gradients method and the Laplace equation to have an insight on this problem.

\begin{graph}[!htb]
%   \vspace{0pt}
  \centering    
%   \includegraphics[width=\textwidth]{\results iterations.pdf}
    \includegraphics[width=\textwidth]{\results conditioning05.pdf}
  \caption[CG iterations count in Test case 3.]{Graph of dependence of the CG iteration count on the 
  number of degrees of freedom. Measured on both problems with no serious distinction observed.}
  \label{graph:conditioning05}
\end{graph}
%

The condition number for matrices resulting from a~conforming FEM applied to Laplace equation is $O(h^{-2})$, so the iteration count 
for the CG without preconditioning is $O(h^{-1})=O(\sqrt{n})$, where $n=1/h^2$ is number of degrees of freedom in case of linear finite elements in 2d. 
With local preconditioning (Jacobi, SOR, ILU) one can usually achieve the number of iterations $O(h^{-0.5})$, cf. \cite{ern_evaluation_2006}.


Let us use the data from Test case 3 and look at the iteration count needed by the CG solver in \gref{graph:conditioning05}.
The number of iterations of the standard FEM is corresponding to the classic results as mentioned in the paragraph above. 

We can see clearly the enormous growth of the number of iterations in case of the standard XFEM and the ramp 
function XFEM. These problems are generally known and are described for example in the overview of the XFEM in
\cite{fries_xfem_overview_2010}. The usage of enrichment functions can make the approximation space almost linearly 
dependent from which the ill-conditioning of the system arises. 

On the other hand, the SGFEM is proven in \cite{babuska_stable_2012} to overcome this. They state in the conclusion that
the conditioning of the SGFEM system is not worse than that of the standard FEM system.
This is exactly what we see in \gref{graph:conditioning05}, where the trends of the SGFEM and the FEM are nearly the same.
The behavior of the shifted XFEM is similarly good. We explain this by the fact that the difference between the enrichment functions
in shifted XFEM and SGFEM in this particular settings is not that significant.


\begin{graph}[!htb]
%   \vspace{0pt}
  \centering    
    \includegraphics[width=\textwidth]{\results moving_xw_error_zero.pdf}
  \caption[Approximation error dependence on singularity position.]{Graph of dependence of the approximation error
   on the position of the singularity respective to a~mesh node.}
  \label{graph:moving_xw_error}
\end{graph}
%
Another test is considered based on Test case 3, in which the source term is set zero. The model is solved for different 
positions of the singularity $\bx_w$ and the effects on the solution and the linear system is observed. The mesh is structured and fixed with $h=0.125$,
centered at point $[0,0]$ which is an~element node. The position of the well (the singularity center) is parameterized by $\bx_w = [t,t]$ with 
\[t\in\{0,\, 0.002,\, 0.004,\, 0.01,\, 0.015,\, 0.03,\, 0.06,\, 0.1,\, 0.11,\, 0.12,\, 0.123,\, 0.125\},\]
so the well moves from one node of the element to the opposite one along the element diagonal. 
We see the results in two graphs \ref{graph:moving_xw_error} and \ref{graph:moving_xw_iterations}. The first one displays
the approximation error dependence on the relative position of the singularity and the element. The trend is apparent,
however the magnitude of the change is not significant. The error of the standard XFEM exhibits the same behavior
but it is by one order of magnitude higher, so it does not fit in the figure.

\begin{graph}[!htb]
%   \vspace{0pt}
  \centering    
    \includegraphics[width=\textwidth]{\results moving_xw_iterations.pdf}
  \caption[CG iteration count dependence on singularity position.]{Graph of dependence of the CG iteration count
   on the position of the singularity respective to a~mesh node.}
  \label{graph:moving_xw_iterations}
\end{graph}
%
The second figure shows the change in the number of iterations of the CG solver. The trend is again apparent:
the CG deals better with the linear system when the singularity closes to an element node, the situation is worse
with the singularity in the middle. We do not understand this behavior completely, although we provide the following explanation.
The approximation of a~singularity by standard FE shape functions is much better when the singularity center is close to a~mesh node
(we observed this in Test case 1) due to the fact that the magnitude of the singularity is captured by the DoF corresponding to the node. 
Therefore when moving the singularity center away from the mesh node, the contribution of the enrichment to the system matrix is larger
(also observed in the experiments).
Since the conditioning (and the number of CG iterations) is sensitive to the enrichment part of the matrix,
we observe this trend in \gref{graph:moving_xw_iterations}.


% \begin{figure}[!htb]
% %   \vspace{0pt}
%   \centering
%   \subfloat[Approximation error]{\label{fig:moving_xw_error} 
%         \includegraphics[width=\textwidth]{\results moving_xw_error.pdf} }
%   \vskip\baselineskip
%   \subfloat[CG iterations]{\label{fig:moving_xw_iterations}
%         \includegraphics[width=\textwidth]{\results moving_xw_iterations.pdf} }
%         
%   \caption[Dependence on position of singularity]{Graph of dependence of the approximation error and the CG iteration count
%   on the position of the singularity respective to a~mesh node.}
%   \label{fig:conditioning05}
% \end{figure}
%

We are satisfied with the results of the shifted XFEM and SGFEM, so we do not investigate the properties of the linear
system any further at this stage. This area of the problem would need deeper investigation, including a~search for a~proper
preconditioner of the system matrix and looking for an~alternative iteration scheme to solve these specific types of linear systems.



\subsection{Enrichment Radius Estimation} \label{sec:enrichemnt_radius}
Up to now, we set the enrichment radius $R_w$ in the experiments to a~fixed value, without providing any explanation of the particular choice.
In this section, we study the dependence of the solution error on the enrichment radius $R_w$.
We look for a~clue for a~particular choice of $R_w$ that is optimal in a~specific sense.
The first part of this section is devoted to a~theoretical analysis, the later part presents a~numerical validation of obtained estimates.

\subsubsection{Derivation of Theoretical Estimate}
Let us consider a~general elliptic problem to find $p\in V$ satisfying
\[
   a(p, q) = l(q), \text{ for } q \in V,
\]
where $a$ is bounded elliptic bilinear form: $\norm{a}\le \alpha_2$, $a(q, q) \ge \alpha_1 \norm{q}_V^2$, $\alpha_1>0$, and $l$ a~bounded linear form, $l\in V'$. 
Suppose, that the problem is to be solved on a~domain $\Omega \subset \Real^2$ with a~single well of radius $\rho_w$ at the origin. 
Let us assume, that the solution can be split into the singular part $p_{sin}(\vc x)= \log |\vc x|$ and the regular part $p_{reg}=p-p_{sin}$.
Let $V^{reg}_h$ be a~polynomial finite element subspace of $V$ on a~regular mesh of elements with a~maximal side length $h$
and let $V_h=V^{reg}_h + V^{enr}_h$ be a~such enriched space that $p_{sin}$ can be approximated exactly on the enriched domain $Z_w$, i.e.
\begin{equation}
   \inf_{q\in V_h} \norm{p_{sin} - q}_V = \inf_{q\in V^{reg}_h} \norm{p_{sin}|_{Z'_w} - q}_V, \quad Z'_w = \Omega\setminus Z_w.
\end{equation}
Using standard error estimate for elliptic PDE (e.g. \cite[Theorem 13.1]{ciarlet_basic_1991}), we get
\begin{equation}
    \label{eq:std_err_estimate}
    \norm{p - p_h}_{V} \le c_a \inf_{q \in V_h} \norm{p - q}_{V} 
    \le c_a \left(\inf_{q \in V^{reg}_h} \norm{p_{reg} - q}_{V} + \inf_{q \in V_h} \norm{p_{sin} - q}_{V} \right)   
\end{equation}
where $c_a=1+\alpha_2/\alpha_1$.
In the following, we consider $V=H^1(\Omega)$, a square grid and $V^{reg}_h$ formed by bilinear finite elements. 
Then \eqref{eq:std_err_estimate} can be further estimated using approximation property of $V^{reg}_h$:
\begin{equation}
    \label{eq:particular_estimate}
    \norm{p - p_h}_{H^1(\Omega)} \le c_a \big(c h \abs{p_{reg}}_{H^2(\Omega)} + \norm{p_{sin} - \pi_h p_{sin}}_{H^1(Z'_w)} \big)
\end{equation}
where $\pi_h p_{sin}$ denotes interpolation of $p_{sin}$ in $V^{reg}_h$. Our next aim is to find tight estimate for the second term.
To this end, we calculate $H^1$ error on a~single square element $S_{h,r}$ with side $h$ and distance $r$ from origin.
Using parametrization $0<s,t<1$,  we get

\begin{align*}
 (p_{sin} - \pi_h p_{sin})(s,t)&=\log\sqrt{(r+hs)^2+(ht)^2} -\Big[(1-s)(1-t)\log r\\
 &\quad+ (1-s)t\log\sqrt{r^2+h^2} + s(1-t) \log(r+h) \\
 &\quad+ st\log\sqrt{(r+h)^2+h^2} \Big]\\
 &=\frac12 \frac{h^2}{r^2}\left(t^2-t - s^2 +s\right) + O\left(\frac{h^3}{r^3}\right)
\end{align*}
and 
\begin{equation}
 \grad(p_{sin} - \pi_h p_{sin})(s,t) = \frac{h}{r^2} \left( \frac12-s, t-\frac12 \right) + O\left(\frac{h^2}{r^2}\right).
\end{equation}
Assuming $h<r$, we can neglect higher order terms. Then, we obtain by direct integration
\begin{align*}
 \norm{p_{sin} - \pi_h p_{sin}}^2_{L^2(S_{h,r})} \approx \frac14 \frac{h^6}{r^4}\int_0^1\int_0^1 \left(t^2-t-s^2+s\right)^2\,\dd s\, \dd t = \frac{1}{360}\frac{h^6}{r^4} 
\end{align*}
and
\begin{equation}
    \label{eq:grad_estimate_on_square}
    \norm{\grad(p_{sin} - \pi_h p_{sin})}^2_{L^2(S_{h,r})} \approx \frac{2h^4}{r^4} \int_0^1 \Big(t-\frac12\Big)^2 \dd t = \frac{1}{6}\frac{h^4}{r^4}.
\end{equation}
Thus for the density of squared error we have
\[
    \frac{1}{\abs{S_{h,r}}} \norm{p_{sin} - \pi_h p_{sin}}^2_{H^1(S_{h,r})} \approx \frac{h^2}{6r^4}
\]
which after integration over the unenriched domain gives final estimate
\begin{equation}
    \label{eq:singular_approx_error}
    \norm{p_{sin} - \pi_h p_{sin}}_{H^1(Z'_w)} \le \left[\int_0^{2\pi} \int_{R_w}^\infty \frac{h^2} {6r^4} r \,\dd r\, \dd \theta\right]^{1/2} = \sqrt\frac{\pi}{6}\frac{h}{R_w}. 
\end{equation}

Recalling the estimate \eqref{eq:std_err_estimate}, we can conclude that optimal choice of the enrichment radius is $h/R_w\approx \norm{p_{reg}-\pi_h p_{reg}}_{H^1(\Omega)}$, 
which balances the error in the regular and the singular part. Larger $R_w$ would not benefit to better overall approximation error and it would lead unnecessarily to
higher amount of enriched DoFs.
In combination with an a~posteriori error analysis, this could give a~rule for an automatic
determination of the enrichment radius.

\subsubsection{Numeric Validation}
Here we provide numerical validation of the results above.
Our aim is twofold: we first validate the estimate \eqref{eq:grad_estimate_on_square}, secondly we simulate 
the dependence of the error in $L^2$ norm on the enrichment radius numerically and compare it with \eqref{eq:singular_approx_error}.

Validity of the estimate \eqref{eq:grad_estimate_on_square} is verified by calculating the ratio
\begin{equation} \label{eqn:log_h1_estimate_ratio}
\frac{h^{3/2} r^{-2} 12^{-1/2}}{\|p_{sin} - \pi_h p_{sin}\|^2_{H^1(T)}},\quad p_{sin}(\vc x) = \log \abs{\vc x}
\end{equation}
on every element $T$ of the sequence of refined meshes using a~$5\times5$ Gaussian quadrature for the estimation of the $H^1$ norm.
%
\begin{table}[!ht]
\centering
\begin{tabular}{crr}
\toprule
% \multicolumn{2}{c}{Item} \\
% \cmidrule(r){1-2}
$h$    & min & max \\
\midrule
$\rfrac{10}{8}$   & 0.97 & 7.1  \\% & 1.38 & 10.0  \\ %& 0.7 & 5.3   \\
$\rfrac{10}{16}$  & 0.99 & 16.4  \\% & 1.40 & 23.1  \\ %& 1.0 & 17.4  \\
$\rfrac{10}{32}$  & 1.00 & 34.4  \\% & 1.41 & 48.7  \\ %& 1.5 & 51.8  \\
$\rfrac{10}{64}$  & 1.00 & 70.3  \\% & 1.41 & 99.5  \\ %& 2.1 & 150   \\
$\rfrac{10}{128}$ & 1.00 & 142.0   \\% & 1.41 & 201   \\ %& 3.0 & 427   \\
\bottomrule
\end{tabular}
\caption[Validation of the enrichment radius estimate.]
{Minimal and maximal values of the ratio \eqref{eqn:log_h1_estimate_ratio} for sequence of refined 
meshes with element size $h$. Validation of the estimate \eqref{eq:grad_estimate_on_square}.}
\label{tab:log_h1_estimate}
\end{table}
%
Table \ref{tab:log_h1_estimate} reports the minimum and the maximum values of the ratio over all elements of every mesh.
The minimum values are close to 1 independently of $h$ which is in perfect agreement with \eqref{eq:grad_estimate_on_square}.
Moreover, the minimum value is attained on the majority of
elements, see \fig{fig:log_estimate_b}. Both parts of \fig{fig:log_estimate} demonstrate also higher convergence rate on diagonal elements
where the nonlinear term of the bilinear finite elements allows better approximation of the saddle shaped logarithmic surface.
%
\begin{figure}[!htb]
%   \vspace{0pt}
  \centering    
  \subfloat[$\|\log \vc x - p_h\|^2_{H^1(T)}$ in log scale]{\label{fig:log_estimate_a} 
    \includegraphics[width=0.47\textwidth]{\results log_estimate_h1.pdf} }
  \hspace{0pt}
  \subfloat[the ratio \eqref{eqn:log_h1_estimate_ratio}]{\label{fig:log_estimate_b} 
    \includegraphics[width=0.47\textwidth]{\results log_estimate_ratio.pdf} }
  \caption[Log error estimate.]
  {
  Results of the numerical validation of the estimate \eqref{eq:grad_estimate_on_square}. The elements are left out 
  in the center where the $\log$ singularity is situated and where the function is cut off.
  }
  \label{fig:log_estimate}
\end{figure}
%

\begin{graph}[!htb]
%   \vspace{0pt}
  \centering    
  \includegraphics[width=0.9\textwidth]{\results radius_convergence_01.pdf}
%   \subfloat[rozdìlený element s vrtem]{\label{fig:adapt_ref_a} 
%     \includegraphics[width=70mm]{\figpath adaptive_ref.pdf} }
%   \hspace{0pt}
%   \subfloat[detail hranice vrtu]{\label{fig:adapt_ref_b} 
%     \includegraphics[width=72mm]{\figpath adaptive_ref_detail.pdf} }
  \caption[Convergence for different enrichment radii.]{Convergence graph for different enrichment radii. The "FEM reg"
  data comes from the problem without the singularity solved by the standard FEM -- it has the optimal convergence order 2.0.}
  \label{graph:radius_conv_1}
\end{graph}
%
Next, we study the influence of the enrichment radius $R_w$ on the global $L^2$ error. To this end, we solve Test case 3
using the SGFEM for different mesh steps and different values of $R_w$.
Let us remind that $O(h^p)$ convergence of the solution in the $H^1$ norm translates to the $O(h^{p+1})$ convergence of the solution in the $L^2$ norm 
for the linear elliptic problems (cf. \cite[Theorem 19.2]{ciarlet_basic_1991}). According to the estimates \eqref{eq:std_err_estimate}
and \eqref{eq:singular_approx_error}, we expect $O(h^2)$ convergence of $L^2$ norm independently of the enrichment radius. This is 
clearly demonstrated in \gref{graph:radius_conv_1}. For comparison, we plot also the error of the regular part $p_{reg}$ of the solution
% \[
%   p_{reg}(x,y) = U(r_w-\rho_w)^2 \sin(\omega x)
% \]
solved by standard FEM showing the $O(h^2)$ convergence.
As predicted, the total error diminishes with $R_w$ but cannot 
drop under the error of $p_{reg}$. We approximate $\norm{p_{reg} - \pi_h p_{reg}}_{H^1(\Omega)}$
using a~fine mesh and then according to \eqref{eq:singular_approx_error},
we get the optimal value of the enrichment radius
\[
    R_o \sim \sqrt{\frac{\pi}{6}} h/\norm{p_{reg} - \pi_h p_{reg}}_{H^1(\Omega)} \sim 0.32
\]
This value roughly matches a~point in the plots of the error as a~function of $R_w$ in
\gref{graph:radius_conv_2}, from which the error is not decreasing any more.



\begin{graph}[!htb]
%   \vspace{0pt}
  \centering    
  \includegraphics[width=0.9\textwidth]{\results radius_convergence_02.pdf}
%   \subfloat[rozdìlený element s vrtem]{\label{fig:adapt_ref_a} 
%     \includegraphics[width=70mm]{\figpath adaptive_ref.pdf} }
%   \hspace{0pt}
%   \subfloat[detail hranice vrtu]{\label{fig:adapt_ref_b} 
%     \includegraphics[width=72mm]{\figpath adaptive_ref_detail.pdf} }
  \caption[Optimal enrichment radius.]{Dependence of the error on the enrichment radius for different
  element sizes $h$.}
  \label{graph:radius_conv_2}
\end{graph}


\subsection{Test Cases with Multiple Wells}
\label{sec:2d_results_multiple}
In this section, more complex test cases are solved, including more than one well intersecting the aquifer.
We test the enrichment methods, shifted XFEM and SGFEM in particular, whether they
still have such good convergence properties when the wells influence each other and when the enrichment zones coincide with each other.

%
\begin{figure}[!htb]
%   \vspace{0pt}
  \centering    
    \includegraphics[width=0.8\textwidth]{\results test_2w_exact.pdf}
  \caption{Solution of Test case 5.}
  \label{fig:test_2w_exact}
\end{figure}
\subsubsection{Test Case 5}
At first we consider two wells intersecting a~square aquifer $\Omega_2 = [0,10]\times[0, 10]$ at points $\bx_1 = [4.1, 4.3]$ and $\bx_2 = [5.7, 5.9]$.
The pressure at the top of the wells is set $g^1_{1D}=150$ and $g^2_{1D}=100$ respectively.
Remaining parameters are set the same for both wells: $\rho_w = 0.003$, $\sigma_w = 100$, $c_w = 10^{10}$ for $w=1,2$.
The hydraulic conductivity of the aquifer is set $K_2=10^{-3}$ and the aquifers thickness is set $\delta_2=1$.
The source term is sinusoidal as in Test case 4, i.e. $f_2 = K_2U\omega^2\sin(\omega x)$, with parameters $\omega=1$, $U=80$.
The solution is shown in \fig{fig:test_2w_exact}.

\begin{figure}[!htb]
%   \vspace{0pt}
  \centering
  \subfloat[enrichment radius $R_w=0.6$]{\label{fig:error_2w_0-6} 
        \includegraphics[width=0.48\textwidth]{\results error_2w_0,6.pdf} }
  \hfill
  \subfloat[enrichment radius $R_w=2.0$]{\label{fig:error_2w_2-0} 
        \includegraphics[width=0.48\textwidth]{\results error_2w_2,0.pdf} }
  \caption[Error distribution in Test case 5.]{$L^2$ error distribution in Test case 5 for two different $R_w$,
  represented by green circles.}
  \label{fig:error_distribution_test5}
\end{figure}
%
At first we solve the problem using the enrichment radius $R_w=0.6$ for both wells. The results are as expected:
optimal convergence order is observed and CG iterations count increases reasonably as in Section \ref{sec:res_conditioning}.
Next we enlarge the enrichment radius $R_w=2.0$ such that the two circular enrichment zones overlap and
the elements are enriched from both singularities there. We plot the error distribution in \fig{fig:error_distribution_test5}, 
comparing the difference between $R_w=0.6$ on the left and $R_w=2.0$ on the right. We see a~higher approximation error 
accumulating outside the smaller enrichment zones, in case of larger enrichment radii only the error of the regular part is apparent.
%
\begin{table}[!htb]
\begin{center}
\bgroup
\def\arraystretch{1.2}
\setlength\tabcolsep{5pt}
\begin{tabular}{rc|cc|cc}
\toprule
\multicolumn{2}{c|}{} & \multicolumn{2}{c|}{shifted XFEM} & \multicolumn{2}{c}{SGFEM}\\ [3pt] %\midrule
i & h & $\|p-p_h\|_{L^2(\Omega_2)}$ & order & $\|p-p_h\|_{L^2(\Omega_2)}$ & order \\ [3pt] \midrule
0 & 0.5000 & 1.01e+02 & --   & 1.06e+02 & --   \\ %\hline
1 & 0.2500 & 1.66e+01 & 2.61 & 1.78e+01 & 2.57 \\ %\hline
2 & 0.1250 & 4.22e+00 & 1.97 & 4.45e+00 & 2.00 \\ %\hline
3 & 0.0625 & 1.07e+00 & 1.99 & 1.11e+00 & 2.00 \\ %\hline
4 & 0.0312 & 2.67e-01 & 1.99 & 2.78e-01 & 2.00 \\ %\hline
5 & 0.0156 & 6.72e-02 & 1.99 & 6.93e-02 & 2.00 \\ %\hline
6 & 0.0078 & --       & --   & 1.72e-02 & 2.01 \\ %\hline
\bottomrule
\end{tabular}
\caption[Convergence table in Test case 5.]
{Convergence table of the shifted XFEM and SGFEM in Test case 5, using enrichment radius $R_w=2.0$.}
\label{tab:convergence_test5}
\egroup
\end{center}
\end{table}

Table \ref{tab:convergence_test5} displays the optimal convergence order of both shifted XFEM and SGFEM for $R_w=2.0$.
The computation on the most refined mesh is failing in case of the shifted XFEM due to the non-converging CG solver.
The number of CG iterations for both enrichment methods and standard FEM is plotted in \gref{graph:test_2w_conditioning}.
The SGFEM performs as expected, on the other hand the CG iterations for shifted XFEM increase rapidly.
This is due to the enrichment of elements by multiple enrichment functions causing worse conditioning of the resulting linear system.
%
\begin{graph}[!htb]
%   \vspace{0pt}
  \centering    
    \includegraphics[width=0.8\textwidth]{\results test_2w_conditioning.pdf}
  \caption[CG iterations count in Test case 3.]{Graph of dependence of the CG iteration count on the 
  number of degrees of freedom. Measured on both problems with no serious distinction observed.}
  \label{graph:test_2w_conditioning}
\end{graph}



\subsubsection{Test Case 6}
In this test case, five wells are intersecting the aquifer $\Omega_2 = [0,10]\times[0, 10]$.
The well specific data are gathered in Table \ref{tab:test_case6_wells_data}.
The wells 3 and 5 can be seen as injection wells with a~positive flux to the aquifer, the others as pumping wells with a~positive flux from the aquifer.
The common well parameters are $\rho_w = 0.003$, $c_w = 10^{7}$.
The hydraulic conductivity of the aquifer is set $K_2=10^{-3}$ and the aquifers thickness is set $\delta_2=1$.
The source term is similar to the one in Test case 5, using parameters $\omega=1$, $U=-8$.
The solution is shown in \fig{fig:test_5w_exact}.
%
\begin{table}[!htb]
\begin{center}
\begin{tabular}{c|ccccc}
\toprule
% \multicolumn{2}{c}{Item} \\
% \cmidrule(r){1-2}
well $w$ & 1 & 2 & 3 & 4 & 5 \\
\midrule
$\bx_w$     & [2.8,2.5]  & [4.9,5.4]  & [2.9,7.4]  & [7.3,7.8] & [7.4, 2.8] \\
$\sigma_w$  & 20   & 10  & 10  & 10  & 20 \\
$g^w_{1D}$  & -150 & -30 & 120 & -50 & 100 \\
\bottomrule
\end{tabular}
\caption{Input data for the wells in Test case 6.}
\label{tab:test_case6_wells_data}
\end{center}
\end{table}

%
\begin{figure}[!htb]
%   \vspace{0pt}
  \centering    
    \includegraphics[width=0.8\textwidth]{\results test_5w_exact.pdf}
  \caption{Solution of Test case 6.}
  \label{fig:test_5w_exact}
\end{figure}
%

We show the convergence results of the SGFEM in Table \ref{tab:convergence_test6}. The problem is solved with
two different enrichment radii $R_w$, the smaller one assures non-overlapping enrichment zones.
The convergence order is still satisfying, although in case of the larger $R_w$ the error did not decreased enough
at the last refinement level for an unknown reason.
\begin{table}[!htb]
\begin{center}
\bgroup
\def\arraystretch{1.2}
\setlength\tabcolsep{5pt}
\begin{tabular}{rc|cc|cc}
\toprule
\multicolumn{2}{c|}{} & \multicolumn{2}{c|}{$R_w=0.8$} & \multicolumn{2}{c}{$R_w=2.0$}\\ [3pt] %\midrule
i & h & $\|p-p_h\|_{L^2(\Omega_2)}$ & order & $\|p-p_h\|_{L^2(\Omega_2)}$ & order \\ [3pt] \midrule
0 & 0.5000 & 5.09e+01 & --   & 5.08e+01 & --   \\ %\hline
1 & 0.2500 & 2.30e+00 & 4.46 & 1.66e+00 & 4.93 \\ %\hline
2 & 0.1250 & 5.66e-01 & 2.03 & 4.05e-01 & 2.04 \\ %\hline
3 & 0.0625 & 1.44e-01 & 1.98 & 1.03e-01 & 1.98 \\ %\hline
4 & 0.0312 & 3.61e-02 & 1.99 & 2.47e-02 & 2.06 \\ %\hline
5 & 0.0156 & 9.22e-03 & 1.97 & 6.36e-03 & 1.96 \\ %\hline
6 & 0.0078 & 2.41e-03 & 1.94 & 1.77e-03 & 1.85 \\ %\hline
\bottomrule
\end{tabular}
\caption[Convergence table in Test case 6.]
{Convergence table of the SGFEM in Test case 6, using two different enrichment radii.}
\label{tab:convergence_test6}
\egroup
\end{center}
\end{table}

We do not mention the results of shifted XFEM explicitly since it performs nearly the same as SGFEM with the smaller enrichment radius.
In case of overlapping enrichment zones, the shifted XFEM suffers with ill-conditioning as in the previous test case.
On the other hand, the iteration count for the SGFEM increases still with the same rate independently of the chosen $R_w$.



\section{Summary}
\label{sec:summary}

% In this first part of our work we provided the research on the XFEM and the adjacent topics relevant to our main theme, in Section \ref{sec:soa_xfem}.
% We described the fundamentals of the usage of the XFEM and got into details in several implementation aspects such as
% the choice of the local enrichment functions, the accurate integration or the enrichment zone size.

This Chapter was devoted to the investigation of four different enrichment methods
-- the XFEM, the Corrected XFEM with the ramp function and shifting, and the SGFEM. 
The solved problem setting was inspired by the work \cite{gracie_modelling_2010,craig_using_2011} 
of R.~Gracie and J.R.~Craig but our effort was aimed more in understanding the details of the enrichment methods and their comparison
rather than in computation of complex problems. Resolving a~point singularity in a~2d domain was addressed.

The well-aquifer model was defined in the beginning and its weak form was derived, including the analysis of the weak solution existence.
In addition to our article \cite{exner_2016}, the model was formulated correctly using the averaged term on the well-aquifer
cross-section. Then the discretization by means of the XFEM was thoroughly described.
The range of the numerical tests was narrowed to problems with a~single aquifer for which the pseudo-analytic solution was derived.

All the implemented methods were compared in \ref{sec:2d_results_single}. Regarding the XFEM, we saw that at 
least the ramp function must be used in order to optimize the error on the blending elements. 
The shifted XFEM and the SGFEM converged optimally and did not show any difference in the solution precision.
For the last two methods we also showed that the precision is not dependent on the relative position of a~singularity to a~node.

The issue of a~suboptimal convergence order reported in \cite{gracie_modelling_2010} was investigated. 
We revealed the problem and we suggested a~better strategy for the adaptive quadrature. 
The improvement was confirmed by the numerical tests in \ref{sec:pressure_results} where we obtained the optimal 
order of convergence in $L^2$ norm.
An alternative quadrature for integration of enriched functions in polar coordinates was also suggested.
% The quadrature was implemented but the numerical results were not satisfying. This would need a~deeper 
% investigation on the parameters that are needed to be set and a~possible bug in a~code is not excluded.

A proper size of the enrichment zone, defined by the enrichment radius, was studied. The error estimate dependent
on the enrichment radius was derived and it was numerically validated. Furthermore, the optimal enrichment radius was predicted 
for the test problem and it corresponded with the computed data.

The ill-conditioning of the system matrix was observed through the increasing iteration count of conjugate gradients solver.
The standard XFEM and the ramp function XFEM suffered from ill-conditioning even in the simplest test cases.
The shifted XFEM performed well when there was only one singular enrichment per element. Otherwise, when the enrichment
zones from multiple singularities were overlapping, only the SGFEM behaved well in terms of system matrix conditioning.


Regarding the three conditions \ref{enum:sgfem_conditions_a}-\ref{enum:sgfem_conditions_c} in Section \ref{sec:stable_gfem}
for an XFEM to be called the SGFEM, we conclude that our implementation of the SGFEM indeed appears to have these properties,
although we do not provide a~theoretical proof. The SGFEM yields the optimal convergence rate and its conditioning is not worse
than that of the FEM in all presented test cases, both the approximation error and the conditioning is robust with respect to the relative position
of the singularity to the mesh nodes.
The local orthogonalization technique has not been implemented, but it might be beneficial in case of overlapping enrichment zones.
Particularly in Test case 6, it is worth of further experimenting to resolve the slightly suboptimal error decrease at the last refinement level.





% \begin{figure}[!htb]
% %   \vspace{0pt}
%   \centering    
%     \includegraphics[width=0.6\textwidth]{\figpath 2_aquifers-5_wells_mesh_whitebg_crop.pdf}
%   \caption{0D-2D coupling example from my previous work. Distribution of pressure in 2 aquifers (horizontal planes) with 5 wells 
%           (vertical lines). The XFEM is used on a~coarse mesh (at the bottom). }
%   \label{fig:aquifers}
% \end{figure}
