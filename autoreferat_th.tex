\documentclass[bibliography=totocnumbered,dvipsnames,FM,Dis, EN]{tulthesis_autoreferat}

\usepackage[a5paper]{geometry}
% \usepackage{etoolbox}
% \makeatletter
% \patchcmd{\chapter}{\if@openright\cleardoublepage\else\clearpage\fi}{}{}{}
% \makeatother

\usepackage{titlesec}
% \titleformat{\chapter}[display]
% {\normalfont\huge\bfseries}{\chaptertitlename~\thechapter}{20pt}{\Huge}
\titlespacing*{\chapter}{0pt}{15pt}{8pt} %{50pt}{40pt}
\titlespacing*{name=\chapter,numberless}{0pt}{0pt}{8pt}

\usepackage[czech, english]{babel}
\usepackage[utf8]{inputenc}

% Adds bibliography to tocbibind
% options:
% none - switch off everything
% numbib - adds number to bibliography
% https://ctan.org/pkg/tocbibind
\usepackage[nottoc]{tocbibind}

% \setlength{\parindent}{3em}
% \setlength{\parskip}{0.15cm}

\DeclareUnicodeCharacter{00A0}{~}

%% The lineno packages adds line numbers. Start line numbering with
%% \begin{linenumbers}, end it with \end{linenumbers}. Or switch it on
%% for the whole article with \linenumbers.
%% \usepackage{lineno}

% grahpics
\usepackage{graphicx}
\usepackage{subfig}
\newcommand{\fig}[1]{Figure \hyperref[#1]{\ref{#1}}}
\newcommand{\gref}[1]{Graph \hyperref[#1]{\ref{#1}}}
\newcommand{\figpath}{figures/}
\newcommand{\results}{results/}

\DeclareCaptionType[name={Graph}]{graph}
% \newcommand{\listofgarphs}{\tocfile{\listfigures}{graph}}

% živé odkazy v PDF
\usepackage{hyperref}
% \hypersetup{colorlinks=true, linkcolor=tul, urlcolor=tul, citecolor=tul}
\hypersetup{colorlinks=false, linkcolor=tul, urlcolor=tul, citecolor=tul}
\hypersetup{pdftitle={Extended Finite Element Methods on Meshes of Combined Dimensions}}

%tables
\usepackage{booktabs}

% enumeration by alphabet
\usepackage[inline]{enumitem}

\usepackage{url}

%\usepackage{float}
%\newfloat{algorithm}{t}{lop}
\usepackage{array}
\newcommand{\plucker}{Pl\"{u}cker }
\newcommand{\nface}{$n$-face }
\newcommand{\nfaces}{$n$-faces }
\newcommand{\ngh}{NGH }
\newcommand{\figpathins}{figures/intersections/}

\pdfsuppresswarningpagegroup=1

% deklarace pro titulní stránku
\TULthesisType{Autoreferát disertační práce}{Extended Abstract of the Doctoral Thesis}
\TULtitle{Rozšířené metody konečných prvků pro aproximaci singularit}{Extended Finite Element Methods for Approximation of Singularities}
% Extended finite elements for mixed-hybrid model of Darcy flow
\TULprogramme{P3901}{Aplikované vědy v~inženýrství}{Applied Sciences in Engineering}
\TULbranch{3901V055}{Aplikované vědy v~inženýrství}{Applied Sciences in Engineering}
\TULauthor{Ing. Pavel Exner}
\TULsupervisor{doc. Mgr. Jan B{\v r}ezina, Ph.D.}
\TULyear{}


% just for our notes
% \usepackage[usenames,dvipsnames]{color}   %colors
\newcommand{\noteJB}[1]{{\color{Blue} \textbf{JB: } \textit{#1}}}
\newcommand{\notePE}[1]{{\color{Orange} \textbf{PE: } \textit{#1}}}

% \newcommand{\chapterheadingspace}{\vspace*{-25pt}}

%
%*************************************************************************************************************
%
%                                                   DOCUMENT
%
%*************************************************************************************************************
%

\begin{document}


% \ThesisStart{male}

\ThesisTitle{EN}
% \ThesisTitle{CZ}


\begin{abstractCZ}
Tato doktorská práce je zaměřena na řešení problému proudění podzemní vody v~porézním prostředí, které
je ovlivněno přítomností vrtů či studní. Model proudění je sestaven na základě konceptu redukce dimenzí,
který je hojně využíván při modelování rozpukaného porézního přostředí, především granitů.
Vrty jsou modelovány jako 1d objekty, které protínají blok horniny. 
Propojení těchto domén v redukovaném modelu způsobuje singularity v~řešení v~okolí vrtů.
Vrty i~porézní médium jsou síťovány nezávisle na sobě, což vede k výpočetním sítím kombinujícím elementy
různých dimenzí.

Jádrem doktorské práce je pak vývoj specializované metody konečných prvků pro výše popsaný model. 
Pro umožnění propojení sítí různých dimenzí a pro zpřesnění aproximace singularit v okolí vrtů je 
použita rozšířená metoda konečných prvků (XFEM), v~rámci níž jsou navrženy nové typy obohacení
konečně-prvkové aproximace.
Metoda XFEM je nejprve aplikována v~modelu pro tlak, dále je navrženo obohacení pro rychlost a metoda
je použita ve smíšeném modelu, jehož řešením jsou rychlost i~tlak.

Doktorská práce se dále detailně věnuje numerickým aspektům v metodě XFEM, a~to především 
adaptivním kvadraturám, volbě velikosti obohacené oblasti nebo podmíněnosti výsledného lineárního systému.
Vlastnosti navržené XFEM metody a~optimální konvergence jsou ověřeny na sérii numerických experimentů.
Praktickým výstupem doktorské práce je implementace metody XFEM jako součásti open-source softwaru Flow123d.

\end{abstractCZ}

\begin{keywordsCZ}
Rozšířená metoda konečných prvků (XFEM), singularita, sítě kombinovaných dimenzí,
Darcyho proudění, rozpukané porézní prostředí
\end{keywordsCZ}

\vspace{2cm}

\begin{abstractEN}
In this doctoral thesis, a~model of groundwater flow in porous media influenced by wells (boreholes, channels) is developed.
The model is motivated by the reduced dimension approach which is being often used in fractured porous media problems, especially in granite rocks.
The wells are modeled as lower dimensional 1d objects and they intersect the surrounding bulk rock domains.
The coupling between the wells and the rock then causes a~singular behavior of the solution in the higher dimensional domains
in the vicinity of the cross-sections. The domains are discretized separately resulting in an~incompatible mesh of combined dimensions.

The core contribution of this work is in the development of a~specialized finite element method for such model.
Different Extended finite element methods (XFEM) are studied and new enrichments are suggested to better
approximate the singularities and to enable the coupling of the wells with the higher dimensional domains.
At first the XFEM is applied in a~pressure model, later an enrichment for velocity
is suggested and the XFEM is used in a~mixed model, solving both velocity and pressure.

Different numerical aspects of the XFEM is studied in details, including an adaptive quadrature strategy,
a~proper choice of the enrichment zone or a~conditioning of the resulting linear system.
The properties of the suggested XFEM are validated on a~set of numerical tests and the optimal convergence
rate is demonstrated. The XFEM is implemented as a~part of the open-source software Flow123d.

\end{abstractEN}

\begin{keywordsEN}
Extended Finite Element Method (XFEM), Singularity, Meshes of combined dimensions,
Darcy flow, Fractured porous media
\end{keywordsEN}

\clearpage



\tableofcontents
\clearpage

%% \linenumbers

% \listoffigures
% \clearpage

% \listofgraphs
% \clearpage
% 
% \listoftables
% \clearpage
% 
% \listofalgorithmes
% \clearpage

% \listoftables
% \clearpage


\begingroup
\let\clearpage\relax
\listoffigures
\listoftables
\listofgraphs
% \listofalgorithmes
\input auto-list-of-symbols.tex
\endgroup
\clearpage

% \input auto-list-of-symbols.tex
% \clearpage

%%%%%%%%%%%%%%%%%%%%%%%%%%%%%%%%%%%%%%%%%%%%%%%%%%%%%%%%%%%%%%%%%%%%%%%%%%%%%%%%%%%%%%%%%%%%%%%%%%%%%%%%%%%%%%

\chapter{Introduction}

%%%%%%%%%%%%%%%%%%%%%%%%%%%%%%%%%%%%%%%%%%%%%%%%%%%%%%%%%%%%%%%%%%%%%%%%%%%%%%%%%%%%%%%%%%%%%%%%%%%%%%%%%%%%%%

\input auto-intro.tex

%%%%%%%%%%%%%%%%%%%%%%%%%%%%%%%%%%%%%%%%%%%%%%%%%%%%%%%%%%%%%%%%%%%%%%%%%%%%%%%%%%%%%%%%%%%%%%%%%%%%%%%%%%%%%%

\chapter{Reduced Dimensional Models} \label{chap:reduced}

%%%%%%%%%%%%%%%%%%%%%%%%%%%%%%%%%%%%%%%%%%%%%%%%%%%%%%%%%%%%%%%%%%%%%%%%%%%%%%%%%%%%%%%%%%%%%%%%%%%%%%%%%%%%%%

\input auto-reduced_dim.tex

%%%%%%%%%%%%%%%%%%%%%%%%%%%%%%%%%%%%%%%%%%%%%%%%%%%%%%%%%%%%%%%%%%%%%%%%%%%%%%%%%%%%%%%%%%%%%%%%%%%%%%%%%%%%%%

\chapter{Extended Finite Element Method} \label{chap:xfem_soa}

% XFEM - state of the art

%%%%%%%%%%%%%%%%%%%%%%%%%%%%%%%%%%%%%%%%%%%%%%%%%%%%%%%%%%%%%%%%%%%%%%%%%%%%%%%%%%%%%%%%%%%%%%%%%%%%%%%%%%%%%%

\input auto-xfem.tex

%%%%%%%%%%%%%%%%%%%%%%%%%%%%%%%%%%%%%%%%%%%%%%%%%%%%%%%%%%%%%%%%%%%%%%%%%%%%%%%%%%%%%%%%%%%%%%%%%%%%%%%%%%%%%%

\chapter{Pressure Model with Singularities} \label{chap:xfem_pressure}

%%%%%%%%%%%%%%%%%%%%%%%%%%%%%%%%%%%%%%%%%%%%%%%%%%%%%%%%%%%%%%%%%%%%%%%%%%%%%%%%%%%%%%%%%%%%%%%%%%%%%%%%%%%%%%

\input auto-pressure-model.tex

\input auto-pressure-results.tex

%%%%%%%%%%%%%%%%%%%%%%%%%%%%%%%%%%%%%%%%%%%%%%%%%%%%%%%%%%%%%%%%%%%%%%%%%%%%%%%%%%%%%%%%%%%%%%%%%%%%%%%%%%%%%%

\chapter{Mixed Model with Singularities} \label{chap:xfem_mh}

%%%%%%%%%%%%%%%%%%%%%%%%%%%%%%%%%%%%%%%%%%%%%%%%%%%%%%%%%%%%%%%%%%%%%%%%%%%%%%%%%%%%%%%%%%%%%%%%%%%%%%%%%%%%%%

\input auto-mh-model.tex

\input auto-mh-results.tex

%%%%%%%%%%%%%%%%%%%%%%%%%%%%%%%%%%%%%%%%%%%%%%%%%%%%%%%%%%%%%%%%%%%%%%%%%%%%%%%%%%%%%%%%%%%%%%%%%%%%%%%%%%%%%%

\chapter{Mesh Intersection Algorithms} \label{chap:intersections}

%%%%%%%%%%%%%%%%%%%%%%%%%%%%%%%%%%%%%%%%%%%%%%%%%%%%%%%%%%%%%%%%%%%%%%%%%%%%%%%%%%%%%%%%%%%%%%%%%%%%%%%%%%%%%%

\input auto-intersections.tex

%%%%%%%%%%%%%%%%%%%%%%%%%%%%%%%%%%%%%%%%%%%%%%%%%%%%%%%%%%%%%%%%%%%%%%%%%%%%%%%%%%%%%%%%%%%%%%%%%%%%%%%%%%%%%%

\chapter{Conclusion} \label{chap:summary}
% closure - what is done, further plan, mention traineeship

%%%%%%%%%%%%%%%%%%%%%%%%%%%%%%%%%%%%%%%%%%%%%%%%%%%%%%%%%%%%%%%%%%%%%%%%%%%%%%%%%%%%%%%%%%%%%%%%%%%%%%%%%%%%%%

Based on the research of the related works and experience gained at the conferences
(in particular MAMERN VI '15, X-DMS '15-17 and CMWR '18),
we are convinced that we dedicated our efforts to a~very interesting and hot topic with wide range of applications.
We are also not aware of any closely connected work to this topic in the Czech Republic
which puts us in a pioneer position, in the Czech scientific environment at least.

In the first part of this work, a~reduced dimension concept is described and a~model for coupling
groundwater flow in non-planar 1d-2d and 1d-3d domains intersecting each other is suggested. 
The drawbacks of several approaches in FE approximation in such models are discussed leading to
the suggested solution: incompatible meshing of the domains and using the XFEM to couple them back together and to improve
FE approximation of arisen singularities by proper discrete space enrichments.

An~extensive study of the currently available XFEM \cite{fries_xfem_overview_2010, babuska_stable_2012} is provided 
in Chapter \ref{chap:xfem_soa} and singular enrichments are addressed in particular. 
Then in Chapter \ref{chap:xfem_pressure}, a~model simulating pressure in a~well-aquifer 1d-2d system,
inspired by \cite{gracie_modelling_2010, craig_using_2011}, is created.
Different types of enrichments are studied and compared in terms of
convergence rate, linear system conditioning and sensitivity to mesh -- singularity alignment.
In the view of the numerical results, the SGFEM is found to be the most promising method
for singularity approximation in the model.
Apart from that, several implementation aspects of the XFEM are addressed: improved adaptive quadrature rules
for an accurate integration on enriched elements is suggested, optimal enrichment zone
for singular enrichments is investigated. 
The model and methods is verified on a~set of numerical test cases.
The content of these two chapters is partially summarized in the article \cite{exner_2016}.


Chapter \ref{chap:xfem_mh} is dedicated to the XFEM application in a~mixed problem
in order to extend the possibilities of the groundwater model in \cite{sistek_bddc_2015, flow123d}.
The mixed-hybrid form is carefully derived for both non-planar 1d-2d and 1d-3d case.
A~new singular SGFEM like enrichment of the standard Raviart-Thomas finite elements is suggested
and applied to the velocity discrete space.
The model is implemented as an~experimental part of the software Flow123d
and a~set of numerical tests is provided.
Since velocity is important in the attached processes (e.g. transport),
a~stress is put on the velocity precision and the velocity convergence rate is traced.
The optimal order of convergence is observed in both 1d-2d and 1d-3d tests,
which included single and multiple wells, overlapping enrichment zones and non-zero source prescribed.

The difficulty of the suggested vector enrichment is that it shares a~single degree of freedom per singularity
over its whole enrichment zone. This is a~source of two problems. First, the system matrix has some non-sparse rows,
which can then lead to loosing the sparsity when applying a~preconditioner based on elimination.
Second, any heterogeneity, e.g. in conductivity, inside the enrichment zone cannot be captured by
a~single singular enrichment function.
We tried to find proper elementwise enrichment functions, similarly to the ones used in the pressure model,
however we struggled with the hybridized form, where such enrichment functions must be accompanied
by some corresponding Lagrange multipliers.
So far we were unsuccessful in performing a~numerical test of the inf-sup stability
of the mixed-hybrid form, this problem had to be left open.
We intend to further study the suggested elementwise enrichment functions and the inf-sup test in the mixed problem without hybridization,
however such model is not available in Flow123d yet.


A~necessary prerequisite for computations on incompatible meshes is the ability to determine
the intersections of the different meshed domains. In Chapter \ref{chap:intersections}
a~fast and robust algorithm for computing such intersections on simplicial meshes is developed.
Although only the non-planar 1d-2d and 1d-3d cases are necessary in the singular models,
the algorithms are extended also to higher dimensions, namely 2d-2d, 2d-3d.
New models in these cases are in focus of our future work and further development of the software Flow123d.
The properties of the \plucker coordinates \cite{platis_fast_2003, joswig_plucker_2013} are exploited
in the element-element intersection algorithms which provide not only the coordinates but also additional topological information.
This is then used in the global mesh intersection algorithms together with other modern techniques
such as BIH of axes aligned bounding boxes and advancing front tracing.
The suggested algorithms are shown to be competitive to other works \cite{moller_fast_1997, haines_fast_1991}.
The global mesh intersection algorithms are tested in Flow123d on an artificial and a~real case benchmarks
and they exhibit linear time complexity. The results are summarized in \cite{brezina_2017}.
Possible further improvements include a~deeper study on the precision of the used geometric predicates,
e.g. regarding the adaptivity in \cite{shewchuk_adaptive_1997}, and thorough code optimization in Flow123d.


The work was also consulted during the author's traineeship at Technical University in Munich at the Department of Numerical Mathematics
lead by Prof. Barbara Wohlmuth. Mainly the theoretical aspects of the work and new ideas were discussed.
We got also familiarized with a~different approach for problems with Dirac delta sources \cite{koppl_tum_2015, koppl_vidotto_2018}
as a~coupling method for inclusions.


The goals of this thesis as set in the introduction were fulfilled to a great extent.
We studied the XFEM intensively and researched its usage in singular problems.
Apart from the created pressure model, we managed to suggest a~new velocity enrichment in the mixed-hybrid form
and implement a~working model in Flow123d. The model was derived and formulated in detail.
A~lot of technical work was done while preparing all the building blocks for the XFEM in the software.
Eventually, we left several open issues which were addressed above.

As we already pointed out, the future work may concern a~study of the vector enrichments in the mixed form.
Extensions of the discretization for 1d objects that are not straight might be of interest.
A~specialized iterative method can be suggested in order to solve the linear algebraic system efficiently,
including a~proper preconditioner.
Finally, some processes attached to the groundwater model may be considered using the velocity solution.
These processes, namely transport of substances, poroelasticity or heat transfer, then may require
similar kind of enrichment for scalar/vector quantities of interest. 

% %%%%%%%%%%%%%%%%%%%%%%%%%%%%%%%%%%%%%%%%%%%%%%%%%%%%%%%%%%%%%%%%%%%%%%%%%%%%%%%%%%%%%%%%%%%%%%%%%%%%%%%%%%%%%%
% 
\newpage
\chapter{Author's Publications} \label{chap:publications}
% 
% %%%%%%%%%%%%%%%%%%%%%%%%%%%%%%%%%%%%%%%%%%%%%%%%%%%%%%%%%%%%%%%%%%%%%%%%%%%%%%%%%%%%%%%%%%%%%%%%%%%%%%%%%%%%%%
{\large\textsc{Publications}}
\begin{itemize}[label={}, leftmargin=*]

{\small
\item
M. Hokr, J. B{\v r}ezina, J. Kr{\' a}lovcov{\' a}, J. {\v R}{\' i}ha, P. Exner, I. Han{\v c}ilov{\' a}, A. Balv{\' i}n,
\href{https://www.onepetro.org/conference-paper/ARMA-DFNE-18-1386}
{Multidimensional Model of Flow and Transport in Fractured Rock for Support of Czech Deep Geological Repository Program},
\emph{Proceedings of 2nd International Discrete Fracture Network Engineering Conference, Seattle} (2018), ARMA-DFNE-18-1386, \\
American Rock Mechanics Association.\\
\url{https://www.onepetro.org/conference-paper/ARMA-DFNE-18-1386}.

\item
J. B{\v r}ezina, P. Exner, \href{http://www.sciencedirect.com/science/article/pii/S0898122117300792}{Fast algorithms for intersection of non-mat\-ching grids using Plücker coordinates},
\emph{Computers and Mathematics with Applications} 74 (2017) pp. 174-187. ISSN 0898-1221. \\
\href{https://doi.org/10.1016/j.camwa.2017.01.028}{\texttt{doi.org/10.1016/j.camwa.2017.01.028}}.

\item
P. Exner, J. B{\v r}ezina, \href{http://www.sciencedirect.com/science/article/pii/S0096300315012862}{Partition of unity methods for approximation of point water sources in porous media},
\emph{Applied Mathematics and Computation} 273 (2016) pp. 21-32. ISSN 0096-3003.\\
\href{http://dx.doi.org/10.1016/j.amc.2015.09.048}{\texttt{doi:10.1016/j.amc.2015.09.048}}.

\item
O. Sever{\' y}n, J. Stebel, P. Exner, Applied mathematics -- methodical handbook for AP course,
project Education For Effective Transfer Of Technology And Knowledge In Science And Engineering (EE2.3.45.0011),
Technical University of Liberec, 2015.

\item
P. Exner, J. B{\v r}ezina, \href{http://www.ugn.cas.cz/actually/event/2015/sna/sna-sbornik.pdf}{Adaptive integration of singularity in partition of unity methods},
in proceedings of \emph{Seminar on Numerical Analysis 2015}, Institute of Geonics AS CR, Ostrava, (2015) pp. 29-32. ISBN 978-80-86407-55-5. \\
\url{http://www.ugn.cas.cz/actually/event/2015/sna/sna-sbornik.pdf}.

\item
P. Exner, \href{http://www.cs.cas.cz/sna2014/sbornik.pdf}{Partition of unity methods for approximation of point water sources in porous media},
in proceedings of \emph{Seminar on Numerical Analysis 2014}, Institute of Computer Science AS CR, Prague, (2014) pp. 29-32. ISBN 978-80-87136-16-4.\\
\url{http://www.cs.cas.cz/sna2014/sbornik.pdf}.
}
\end{itemize}
%
% \vspace{0.5cm}
\pagebreak
%
{\noindent\large\textsc{Software}}
\begin{itemize}[label={}, leftmargin=*]
{\small
\item
J. B{\v r}ezina, J. Stebel, D. Flanderka, P. Exner, J. Hyb{\v s}, Flow123d, 
Technical University of Liberec, 2011--2019, \\
\href{http://flow123d.github.com}{http://flow123d.github.com}.
}
\end{itemize}

\vspace{0.5cm}
%
{\noindent\large\textsc{Conferences}}
%
\begin{itemize}[label={}, leftmargin=*]
{\small
\item DFNE '18, Seattle (US), [presentation] \\ Multidimensional model of flow and transport in fractured
rock for support of Czech DGR program
\item CMWR '18, Saint Malo (FR), [presentation] \\ Darcy Flow on Incompatible Meshes of
Combined Dimensions
\item X-DMS '17, Ume\aa (SE), [presentation] \\ Singular Enrichment of XFEM on Meshes of
Combined Dimensions
\item HPCSE '17, Sol\'a{\v n} (CZ), [poster] \\ Singular Enrichment in XFEM
\item FEM Symposium '16, Chemnitz (DE), [presentation] \\ Extended Finite Element Methods Dealing with Singularities on Meshes of Combined Dimensions
\item X-DMS '15, Ferrara (IT), [presentation, short course on XFEM attendance]\\ eXtended Discretization Methods
\item MAMERN VI '15, Pau (FR), [presentation] \\ The International Conference on Approximation Methods and Numerical Modeling in Environment and Natural Resources
\item SNA '15, Ostrava (CZ), [presentation] \\ Seminar on Numerical Analysis
\item ESCO '14, Plze{\v n} (CZ), [presentation] \\ European Seminar on Computing
\item Stanford, 2014 (US) [attendance only] \\ 40th Stanford Geothermal Workshop
\item SNA '14, Nymburk (CZ), [presentation] \\ Seminar on Numerical Analysis
\item Student conference '13, Liberec (CZ), [poster] \\
Student conference at the Faculty of Mechatronics, Informatics and Interdisciplinary Studies, Technical University of Liberec
}
\end{itemize}


%%%%%%%%%%%%%%%%%%%%%%%%%%%%%%%%%%%%%%%%%%%%%%%%%%%%%%%%%%%%%%%%%%%%%%%%%%%%%%%%%%%%%%%%%%%%%%%%%%%%%%%%%%%%%%


%   \nocite{gracie_modelling_2010, fries_corrected_2008, babuska_stable_2012, bangerth_deal.ii_2007, 
%           arnold_lecture_2009, craig_using_2011, sistek_bddc_2015, brezzi_mixed_1991, schwenck_xfem-based_2015,
%           fumagalli_efficient_2014, cattaneo_numerical_2015, koppl_tum_2015, maryska_mixed-hybrid_1995}
\newpage
{\small
\bibliographystyle{mybibtex}
% \bibliographystyle{csplainnat}
\bibliography{citace,citace_xfem,citace_sgfem,citace_flow_my,citace_intersections}
}



\newpage\null\thispagestyle{empty}\newpage

\end{document}
