Mathematical modelling plays a~very important role in science and also our daily lives throughout many different
fields of knowledge. Using modern finite element methods (FEM), we are able to simulate, investigate and predict
various phenomena of both nature and industrial character. Recent advances in the field of FEM
enable us to solve models of various scales, dimensions, to overcome numerical problems and also 
to couple interfering features together.

A large set of problems with finite element models, that people nowadays deal with, is connected with 
insufficient accuracy in cases where the model includes large and very small scale phenomena at once.
In our case we are interested in modeling of underground processes.
We can imagine a~simulation of groundwater flow in a~large domain (hundreds of meters or even kilometers) which can be significantly
influenced by thin fractures in the porous media or artificial wells and boreholes (several centimeters in diameter).
These disturbances bring discontinuities and singularities into the model which are hard to capture at the geometric level
and even harder to approximate with the standard polynomial finite elements at the discretization level.
There are several ways one can follow to increase the accuracy of the standard FEM in such models. 

Adaptive meshes can be used in such cases, but it can cost a~lot of computational power to build a~very fine mesh,
and then solve the problem with increasing number of degrees of freedom.
It requires very robust meshing techniques when complex geometries are in question.
There can be other constraints on mesh generation in specific applications, which the meshing tool must obey.
To work optimaly, the FEM demands mesh elements of a~particular quality -- there are different criteria
for edges, angles between the edges or radii of inscribed and circumscribed balls.
In some approaches, the presence of hanging nodes might require a~special treatment or might be entirely unwanted.
In the reduced dimension concept, which is described in details later, the so called compatible meshes might be required.
Some methods for modeling time dependent problems, such as an~opening of fractures in mechanics or interfaces in two-phase flows,
require remeshing at each time step which further amplifies the demands on the meshing tools.
All these aspects make the generation of computational meshes hard or even close to impossible.

Alternatively the reduced dimension concept is often used and models combining different dimensions are being developed.
In this approach the geometry is decomposed in objects of different dimensions (e.g. 2d fractures, 1d wells, 0d point sources)
and the meshes of the objects (computational domains) are created independently.
Later the modelled processes must be coupled between the domains of different dimensions.
The coupling concepts are mostly available at the continous level but their implementation
at the discrete level is non-trivial and problems often appear there.
Two types of meshes are used in these models: compatible and incompatible.
A~compatible mesh is the one, where the intersections of the computational domains are aligned to nodes and sides of elements.
On such a~mesh, the standard FEM with proper coupling terms can be applied relatively easily,
however the construction of the compatible mesh might be problematic.
On the other hand, incompatible meshes with arbitrary intersections are easier to construct, 
but bring a~whole new set of problems in the coupling.


Next there are the so called Extended finite element methods (XFEM); or Partition of unity methods (PUM) or Generalized
finite element methods (GFEM). Their names are sometimes used interchangeably if one talks generally about these types of methods. 
These methods enable us to take advantage of an a~priori knowledge of the model solution character
such as discontinuity or singularity of searched quantities.
The key aspect of the XFEM is that it allows to locally incorporate non-polynomial functions, like a~jump or a~singular function,
into the finite element solution in places, where these features are expected to appear.
This way the standard approximation finite element space is extended and it is able to approximate the small scale phenomena
more accurately.
See an example of XFEM usage in 1d-2d coupling with point intersections in \fig{fig:aquifers}.
%\newline{}
\begin{figure}[!htb]
%   \vspace{0pt}
  \centering    
    \includegraphics[width=0.6\textwidth]{\figpath 2_aquifers-5_wells_mesh_whitebg_crop.pdf}
  \caption[XFEM example for well-aquifer model with singularities]
        { An XFEM example in 1d-2d coupling with singularities at intersection points.
            Distribution of pressure in 2 aquifers (horizontal planes) intersected by 5 wells 
            (vertical lines) is displayed. The model is computed on a~coarse mesh without refinement (visible at the bottom). }
  \label{fig:aquifers}
\end{figure}

This thesis is aimed at further development of the XFEM and its usage in reduced dimension models. 
We intend to create a~model upon an incompatible mesh, where the elements of different dimensions intersect
arbitrarily, and then use the XFEM to glue the modeled processes in different dimensions back together. 
Apparently this would be a~very ambitious goal, therefore we narrow our work to non-planar 1d-2d and 1d-3d coupling
and resolving arisen singularities. 

Our research scope is focused at modeling of the underground processes in porous media
and in this work we address ourselves to a~groundwater flow model in particular.
The model fits in the reduced dimension concept in the software Flow123d, which is being developed at the
Technical University in Liberec, and the model is implemented as an additional feature of this software.

Apart from the groundwater flow, there are various applications where our model might be employed or to which it might be extended.
We list a few: transport in fractured porous media, enhanced geothermal systems and extraction of geothermal energy,
oil and gas extraction, CO$_2$ sequestration, safety assesment of an underground nuclear waste repository,
transport of substances in human body.


\section{Flow123d}
\label{sec:soa_flow123d}

The software Flow123d (latest stable version 2.2.1 in 2018, online documentation in \cite{flow123d_doc_2015}) 
is a~simulator of underground water flow, solute and heat transport in fractured porous media. The software is
developed as an open source code under the versioning system Git where the eponymous project can be found, or 
one can find it on the web page
\begin{center}
\url{http://flow123d.github.io/}
\end{center}
It has been developed at the Technical University of Liberec approximately since 2007.
The main feature of the software is a~computation on complex compatible meshes of combined dimensions, described in
previous section. Therefore, the continuum models and discrete fracture network models can be combined.
%
\begin{figure}[!htb]
  \centering
  \includegraphics[width=0.35\textwidth]{\figpath logo-flow-123d-color.pdf}
  \caption{Logo of Flow123d.}
  \label{fig:logo_flow123d}
\end{figure}
%
Current version includes mixed-hybrid solver for steady and unsteady Darcy flow, finite volume model 
and discontinuous Galerkin model for solute transport of several substances and heat transfer model. 
Using operator splitting, models for various local processes are supported including dual porosity, sorption, decays 
and simple reactions.


Development of the software is an important practical part of the thesis. The aim is to implement support 
of incompatible meshes in the Darcy flow model. More details about implementation goals will be given in
Section \ref{chap:aims}.

%%
% The actuality of these themes is demonstrated by the several conferences with attendants from all over the world.
% We mention the X-DMS (eXtended Discretization Methods), are being held since 2013 including the preceding  XFEM conferences.
% The main topics includes the state-of-art methods for discretization of complex problems on complex domains, with a stress
% on XFEM methods.
% The other conference if DFNE focused on everything concerning the discrete fracture network with main interest in geology.


\section{Aims of the Thesis}\label{chap:aims}
% define aims, characterize and briefly explain solution

The research aims of the thesis can be summarized in two major points
\begin{itemize}
  \item Propose suitable enrichments for the pressure and the velocity that are stable (satisfy inf-sup condition). 
        If possible, emphasise a~good approximation of the velocity field which is necessary for the transport equation.
        
  \item Suggest new data structures and algorithms for the realization phase in the software Flow123d. 
\end{itemize}

\subsection{Research aims}

The following types of enrichments can be considered: wells (point source in 2D, line source in 3D), 
fractures (discontinuity in velocity), boundary of 1D and 2D fractures. 
To narrow the ambitious plan, we will focus on the enrichment for 0D-2D and 1D-3D interactions.

The model with the mixed-hybrid formulation as in \cite{brezina_mixed-hybrid_2010, sistek_bddc_2015} will be followed.
The enrichment will be chosen and incorporated in the formulation so that the singularities in 0D-2D and
1D-3D are better approximated and incompatible meshes can be used. 
We shall start by enriching the pressure field with a~logarithmic function like we
do in Section \ref{sec:model_aquifer} and investigate the consequences in the mixed and mixed-hybrid formulation.

Secondly the velocity field will be enriched. We will look for a~proper vectorial enrichment functions.
The accuracy of the velocity field, in particular the flow across the elements edges, is very important for
the transport equation which is in the main focus in most of the applications.

In a consequence of the enrichment, the discrete spaces will be extended. The influence of the enrichment on the inf-sup 
condition will be investigated and the proof of stability for the enriched formulation should be provided.
The aim is to understand the stability condition good enough so that some constraints on the choice of the enrichment 
functions can be defined. 

The realization of the model in the means of code implementation will require several non-standard techniques.
Efficient data structure must be provided for the enrichment implementation in finite element classes.
The intersections of the mesh domains of different dimensions must be computed. A specialized integration
on the intersected elements is needed to assembly the integrals in the system matrix. 
Solving the final linear algebraic system will require an efficient numerical method.
Each of these sub-problems is not trivial and requires a new approach or at least extending the existing one.
The steps of the realization are summarized below in Section \ref{sec:realization}.

\subsection{Realization} \label{sec:realization}
The suggested numerical model will be implemented and tested in the software Flow123d.
The following upgrades and changes are to be considered.

The enrichment functions need to be defined and support for the enrichment in the finite element classes must be implemented. 
A class that handles the degrees of freedom must be then updated to include the additional enriched degrees of freedom
and the structure of the linear system must be reorganized accordingly.

The enriched elements are in 3D domain. A module in Flow123d that is able to lookup the element intersections
of different dimensions is now being finished (see Chapter \ref{chap:intersections}). The enrichment will propagate from the intersected 3D elements.
The choice of the size of the enrichment area is not straightforward. In case of discontinuities, one layer of elements
near the discontinuity would be suitable, but in case of singularity, the area must be large enough so that 
the large gradients in the quantities are well approximated. For simplicity, the area size can be given as
a~constant parameter.

Then the assembly routines of the enriched terms into the linear system must be provided.
Since the enrichment function would not be polynomial, accurate integration on the enriched elements is required. 
The adaptive elements sub-refinement approach can be used to obtain more precise quadrature, 
or quadrature points defined in cylindrical coordinate system in the vicinity of 1D elements intersections
might be suitable in this case.

Finally, a~reasonable output post-processing should be developed, so that the XFEM solution (piecewise non-polynomial solution)
can be visualized properly. A~kind of adaptive refinement of the output mesh would be suitable.


\section{Document Structure} \label{sec:structure}

Chapter \ref{chap:soa} concludes the research on the related works and represents the current state of art
on the problematic. Modelling on meshes of combined dimension is addressed in Section \ref{sec:soa_model_combined},
survey on the XFEM is presented in Section \ref{sec:soa_xfem} and works regarding the usage of the XFEM on
meshes of combined dimensions are commented in Section \ref{sec:soa_xfem_combined}. Then 
the software Flow123d is described briefly in Section \ref{sec:soa_flow123d}.

Chapter \ref{chap:aims} summarizes the aims of the thesis.

The finished and ongoing work is presented in Chapter \ref{chap:results}. Section \ref{sec:model_aquifer} 
covers a~well-aquifer model of groundwater flow with the XFEM resolving a~point singularity in 2D.
Chapter \ref{chap:intersections} introduces a~new approach for inspecting simplicial elements intersections
of different dimensions using Pl{\"u}cker coordinates.

Chapter \ref{chap:publications} contains a~list of author's publications and attended conferences.
The summary of the thesis is in Chapter \ref{chap:summary} and it is followed by the literature.
