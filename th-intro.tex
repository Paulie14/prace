Mathematical modelling plays a~very important role in science and also our daily lives throughout many different
fields of knowledge. Using modern finite element methods (FEM), we are able to simulate, investigate and predict
various phenomena of both nature and industrial character. Recent advances in the field of FEM
enable us to solve models of various scales, dimensions, to overcome numerical problems and also 
to couple interfering features together.

A large set of problems with finite element models, that people nowadays deal with, is connected with 
insufficient accuracy in cases where the model includes large and very small scale phenomena at once.
In our case we are interested in modeling of underground processes.
We can imagine a~simulation of groundwater flow in a~large domain (hundreds of meters or even kilometers) which can be significantly
influenced by thin fractures in the porous media or artificial wells and boreholes (several centimeters in diameter).
These disturbances bring discontinuities and singularities into the model which are hard to capture at the geometric level
and even harder to approximate with the standard polynomial finite elements at the discretization level.
There are several ways one can follow to increase the accuracy of the standard FEM in such models. 

Adaptive meshes can be used in such cases, but it can cost a~lot of computational power to build a~very fine mesh,
and then solve the problem with increasing number of degrees of freedom.
It requires very robust meshing algorithms when complex geometries are in question.
There can be other constraints on mesh generation in specific applications, which the meshing tool must obey.
To work optimaly, the FEM demands mesh elements of a~particular quality -- there are different criteria
for edges, angles between the edges or radii of inscribed and circumscribed balls.
In some approaches, the presence of hanging nodes might require a~special treatment or might be entirely unwanted.
In the reduced dimension concept, which is described in details later, the so called compatible meshes might be required.
Some methods for modeling time dependent problems, such as an~opening of fractures in mechanics or interfaces in two-phase flows,
require remeshing at each time step which further amplifies the demands on the meshing tools.
All these aspects make the generation of computational meshes hard or even close to impossible.

Alternatively the reduced dimension concept is often used and models combining different dimensions are being developed.
In this approach the geometry is decomposed in objects of different dimensions (e.g. 2d fractures, 1d wells, 0d point sources)
and the meshes of the objects (computational domains) are created independently.
Later the modelled processes must be coupled between the domains of different dimensions.
The coupling concepts are mostly available at the continous level but their implementation
at the discrete level is non-trivial and problems often appear there.
Two types of meshes are used in these models: compatible and incompatible.
A~compatible mesh is the one, where the intersections of the computational domains are aligned to nodes and sides of elements.
On such a~mesh, the standard FEM with proper coupling terms can be applied relatively easily,
however the construction of the compatible mesh might be problematic.
On the other hand, incompatible meshes with arbitrary intersections are easier to construct, 
but bring a~whole new set of problems in the coupling.


Finally there are the so called Extended finite element methods (XFEM); or Partition of unity methods (PUM) or Generalized
finite element methods (GFEM). These names are often used interchangeably if one talks generally about these types of methods. 
The XFEM enable us to take advantage of an a~priori knowledge of the model solution character
such as discontinuity or singularity of searched quantities.
The key aspect of the XFEM is that it allows to locally incorporate non-polynomial functions, like a~jump or a~singular function,
into the finite element solution in places, where these features are expected to appear.
This way the standard finite element approximation space is extended (enriched) and it is able to approximate the small scale phenomena
more accurately.
See an example of XFEM usage in 1d-2d coupling with point intersections in \fig{fig:aquifers}.
%\newline{}
\begin{figure}[!htb]
%   \vspace{0pt}
  \centering    
    \includegraphics[width=0.6\textwidth]{\figpath 2_aquifers-5_wells_mesh_whitebg_crop.pdf}
  \caption[XFEM example for well-aquifer model with singularities.]
        { An XFEM example in 1d-2d coupling with singularities at intersection points.
            Distribution of pressure in 2 aquifers (horizontal planes) intersected by 5 wells 
            (vertical lines) is displayed. The model is computed on a~coarse mesh without refinement (visible at the bottom). }
  \label{fig:aquifers}
\end{figure}

This thesis is aimed at further development of the XFEM and its usage in reduced dimension models. 
We intend to create a~model upon an incompatible mesh, where the elements of different dimensions intersect
arbitrarily, and then use the XFEM to glue the modeled processes in different dimensions back together. 
Apparently this would be a~very ambitious goal, therefore we narrow our work to non-planar 1d-2d and 1d-3d coupling
and resolving arisen singularities. 

Our research scope is focused at modeling of the underground processes in porous media
and in this work we address ourselves to a~groundwater flow model in particular.
We look for a~model that fits in the reduced dimension concept in the software Flow123d, which is being developed at the
Technical University of Liberec (TUL), and we put our effort into implementing the model with XFEM as an additional feature of this software.


\section{Flow123d}
\label{sec:soa_flow123d}

The practical outcome of this thesis is a~new model and its solver implemented as a~part of the software Flow123d.
% It is an open-source software which is being actively developed at TUL.
Flow123d is a~simulator of underground water flow, solute and heat transport in fractured porous media. The software is
developed as an open source code under the versioning system Git where the eponymous project can be found, or 
one can reach it via the web page %\url{http://flow123d.github.io/}.
%
\begin{center}
\url{http://flow123d.github.io/}
\end{center}

It has been developed at the TUL approximately since 2007.
The latest stable version 2.2.1 was released in 2018~\cite{flow123d}.
The main feature of the software is the ability to compute on complex compatible meshes of combined dimensions,
where the continuum models and discrete fracture network models can be coupled. Only the coupling of co-dimension 1
is considered, i.e. coplanar 1d-2d and 2d-3d cases.
This thesis extends the groundwater flow model to non-planar 1d-2d and 1d-3d coupling.
%
\begin{figure}[!htb]
  \centering
  \includegraphics[width=0.35\textwidth]{\figpath logo-flow-123d-color.pdf}
  \caption{Logo of Flow123d.}
  \label{fig:logo_flow123d}
\end{figure}

Current version includes mixed-hybrid solver for steady and unsteady Darcy flow, non-linear solver for unsaturated flow (Richards equation),
finite volume method and discontinuous Galerkin method for solute transport of several substances and heat transfer model. 
Using the operator splitting technique, models for various local processes are supported with the solute transport simulation
including dual porosity, sorption, decays and first order reactions.

The experimental branch of Flow123d based on version 3.0.x, which is about to be officialy released,
includes a~module for computation of intersections of incompatible meshes,
a~rock mechanics model and also the XFEM model based on this thesis.



% Development of the software is an important practical part of the thesis. The aim is to implement support 
% of incompatible meshes in the Darcy flow model. More details about implementation goals will be given in
% Section \ref{chap:aims}.

%%
% The actuality of these themes is demonstrated by the several conferences with attendants from all over the world.
% We mention the X-DMS (eXtended Discretization Methods), are being held since 2013 including the preceding  XFEM conferences.
% The main topics includes the state-of-art methods for discretization of complex problems on complex domains, with a stress
% on XFEM methods.
% The other conference if DFNE focused on everything concerning the discrete fracture network with main interest in geology.


\section{Aims of the Thesis}\label{chap:aims}
% define aims, characterize and briefly explain solution

We are motivated by several applications where this approach provides a~new solution of problems with singular behaviour
and where it extends possibilities of currently available models and methods.
Apart from the already mentioned groundwater flow, in which we focus in this work, there are various other applications
where our model might be employed or to which it might be extended.
We list a few that are closely related: enhanced geothermal systems and extraction of geothermal energy,
oil and gas extraction, CO$_2$ sequestration, safety assesment of an underground nuclear waste repository,
transport of substances in human body or transport in fractured porous media in general.
% Of course the motivation for using our model can be also viewed in more direct way,
% since the (groundwater) flow is just one of procceses that is modeled in some of the listed applications.

Considering what we have preceded in the introduction, we see an open space for our research in further development of reduced dimensional FEM models,
especially of co-dimension 2, in combination with the XFEM and incompatible meshes, where the XFEM provides
a~way for coupling the domains of different dimensions together and to significantly improve the finite element approximation accuracy.

The research aims of this work can be summarized in two major points
\begin{itemize}
    \item Propose suitable XFEM enrichments for the pressure and the velocity in Darcy flow model. 
        If possible, emphasise a~good approximation of the velocity field.% which is necessary for the transport equation.
        
    \item Suggest new data structures and algorithms for the realization phase in the software Flow123d. 
\end{itemize}
The accuracy of the velocity field, in particular the fluxes across the elements edges, 
is in the main focus in most applications, e.g. when the flow model is coupled with a~transport equation.


We briefly sketch the steps that we take to successfully reach these goals.
We inspire ourselves by the work of Gracie and Craig~\cite{gracie_modelling_2010,craig_using_2011},
and we start from a~similar 1d-2d model with XFEM resolving point singularities in pressure.
We compare different XFEMs~\cite{fries_corrected_2008, fries_xfem_overview_2010, babuska_stable_2012,gupta_stable_2013},
investigate the properties of the methods and we suggest some improvements.

Then we transform the model to the mixed-hybrid form based on~\cite{brezina_mixed-hybrid_2010, sistek_bddc_2015}.
We suggest an enrichment for velocity and a~discretization with mixed finite elements.
We implement this XFEM enrichment in Flow123d and we run several numerical experiments.
We further extend these results into 1d-3d coupled model.

To be able to apply the XFEM on the incompatible meshes, it is necessary to determine the intersections
between the meshes of different dimensions. We do not restraint us only to 1d-2d and 1d-3d cases, but we suggest
a~robust algorithm for finding intersections of simplicial meshes in arbitrary 3d space.
Considering a~broader scope, this algorithm can be used also for fractures and cracks in other applications.


% In a consequence of the enrichment, the discrete spaces will be extended. The influence of the enrichment on the inf-sup 
% condition will be investigated and the proof of stability for the enriched formulation should be provided.
% The aim is to understand the stability condition good enough so that some constraints on the choice of the enrichment 
% functions can be defined. 

Regarding the implementation and the code design, several non-standard techniques are required
in comparison to standard FEM codes.
Efficient data structure must be provided for the enriched finite elements.
Handling of the additional degrees of freedom of the enrichment must be considered.
% The intersections of the mesh domains of different dimensions must be computed.
An accurate integration on the intersected (and enriched) elements is needed to assembly the integrals correctly. 
Solving the final linear algebraic system requires an efficient numerical method.
A~reasonable output post-processing must be developed, so that the XFEM solution (piecewise non-polynomial solution)
can be visualized properly.
Each of these sub-problems is not trivial and requires a new approach or at least extending the existing one.


% \subsection{Realization} \label{sec:realization}
% The suggested numerical model will be implemented and tested in the software Flow123d.
% The following upgrades and changes are to be considered.
% 
% The enrichment functions need to be defined and support for the enrichment in the finite element classes must be implemented. 
% A class that handles the degrees of freedom must be then updated to include the additional enriched degrees of freedom
% and the structure of the linear system must be reorganized accordingly.
% 
% The enriched elements are in 3D domain. A module in Flow123d that is able to lookup the element intersections
% of different dimensions is now being finished (see Chapter \ref{chap:intersections}). The enrichment will propagate from the intersected 3D elements.
% The choice of the size of the enrichment area is not straightforward. In case of discontinuities, one layer of elements
% near the discontinuity would be suitable, but in case of singularity, the area must be large enough so that 
% the large gradients in the quantities are well approximated. For simplicity, the area size can be given as
% a~constant parameter.

% Then the assembly routines of the enriched terms into the linear system must be provided.
% Since the enrichment function would not be polynomial, accurate integration on the enriched elements is required. 
% The adaptive elements sub-refinement approach can be used to obtain more precise quadrature, 
% or quadrature points defined in cylindrical coordinate system in the vicinity of 1D elements intersections
% might be suitable in this case.

% Finally, a~reasonable output post-processing should be developed, so that the XFEM solution (piecewise non-polynomial solution)
% can be visualized properly. A~kind of adaptive refinement of the output mesh would be suitable.




\section{Document Structure} \label{sec:structure}

The text is divided into five major parts.
In the first chapter~\ref{chap:reduced}, the reduced dimension concept is described in details and the meshes of combined dimensions are defined.
The models of groundwater flow, coupling non-planar 1d-2d and 1d-3d domains, are formulated and put in the context of the software Flow123d 
and other works.


The background research on various XFEMs is provided in the first part of Chapter \ref{chap:xfem_soa}.
It is then followed by Chapter~\ref{chap:xfem_pressure} where a~well aquifer pressure model
in which singular enrichments are applied by the means of the XFEM.
Various aspects and properties of the model and the method are studied, different enrichment strategies are compared. 

%TODO
%restructure the chapter 3
% xfem research
% well aquifer model
% Numerical tests
%   single well tests
%   conditioning
%   enrichment radius
%   Multi-well tests
% summary

Next the mixed model for pressure and velocity is formulated in Chapter~\ref{chap:xfem_mh}.
A~new enrichment for velocity is proposed in the mixed form.
The coupled models both for non-planar 1d-2d and 1d-3d are defined and followed by several
numerical tests demonstrating the properties of the new enrichment in XFEM.

Chapter~\ref{chap:intersections} is dedicated to the intersection algorithms.
The concept of Pl\"ucker coordinates is introduced and algorithms for various intersection cases
of simplicial elements is described. Numerical experiments are provided for demonstration.


% Chapter \ref{chap:soa} concludes the research on the related works and represents the current state of art
% on the problematic. Modelling on meshes of combined dimension is addressed in Section \ref{sec:soa_model_combined},
% survey on the XFEM is presented in Section \ref{sec:soa_xfem} and works regarding the usage of the XFEM on
% meshes of combined dimensions are commented in Section \ref{sec:soa_xfem_combined}. Then 
% the software Flow123d is described briefly in Section \ref{sec:soa_flow123d}.
% 
% Chapter \ref{chap:aims} summarizes the aims of the thesis.
% 
% The finished and ongoing work is presented in Chapter \ref{chap:results}. Section \ref{sec:model_aquifer} 
% covers a~well-aquifer model of groundwater flow with the XFEM resolving a~point singularity in 2D.
% Chapter \ref{chap:intersections} introduces a~new approach for inspecting simplicial elements intersections
% of different dimensions using Pl{\"u}cker coordinates.

% Chapter \ref{chap:publications} contains a~list of author's publications and attended conferences.
The thesis is closed with the conclusions in the last Chapter \ref{chap:summary}
which is followed by the list of figures and tables and the bibliography.













