
\section{Flow123d implementation}

\subsection{Adaptive quadrature}
The adaptive quadrature implemented in Flow123d is based on the one developed for the quadrilateral meshes
in Section \ref{sec:integration}. It is transformed for the simplicial elements and generalized for
all three dimensions. This way we can use it in the computation of both integrals over elements and integrals
over element faces (e.g. for accurate computation of $z^w_j$ in $\pi^{RT}_T$ interpolant \eqref{eqn:sgfem_interpolant_vel}).

\begin{figure}[!htb]
  \centering    
  \subfloat[adaptive quadrature refinement]{\label{fig:adapt_refinement_flow123d_a} 
    \includegraphics[height=6cm]{\figpath adaptive_refinement.pdf} }
  \hspace{0pt}
  \subfloat[adaptive quadrature in detail]{\label{fig:adapt_refinement_flow123d_b} 
    \includegraphics[height=6cm]{\figpath adaptive_refinement_detail.pdf} }
  \caption[Adaptive quadrature in Flow123d]
  {Adaptive quadrature for triangle elements implemented in Flow123d.
   Black lines denote enriched elements edges, red lines denote adaptive refinement (subelements edges) and the well
   edge is blue.
  }
  \label{fig:adapt_refinement_flow123d}
\end{figure}

\subsection{Output Mesh}
\label{sec:output_mesh}
% Bey's algorithm (red refinement of tetrahedron):
%p.29 https://www5.in.tum.de/pub/Joshi2016_Thesis.pdf
%p.108 http://www.bcamath.org/documentos_public/archivos/publicaciones/sergey_book.pdf    - brandts_2011
%https://www.math.uci.edu/~chenlong/iFEM/doc/html/uniformrefine3doc.html#1
%J. Bey. Simplicial grid refinement: on Freudenthal's algorithm and the optimal number of congruence classes. Numer. Math. 85(1):1--29, 2000. p11 Algorithm: RedRefinement3D.
%p.4 http://www.vis.uni-stuttgart.de/uploads/tx_vispublications/vis97-grosso.pdf

The refinement of elements for the output mesh is done using edge splitting technique (so called red refinement).
Since we use this only for better output visualization of non-polynomial solutions, we do not
care for existence of hanging nodes.

In 2D case, it is straightforward process: find the midpoints of all sides, connect them and generate 4 triangles.
These triangles are congruent and have equal surface areas.
%
\begin{figure}[!htb]
    \centering    
    \includegraphics[width=0.6\textwidth]{\results output_refine.pdf} 
    \caption[output mesh refinement]
  {An example of an adaptive output mesh refinement in 3d.
  A~singular function $1/r$ is displayed.}
  \label{fig:output_refinement_flow123d}
\end{figure}
%
On the other hand, the 3D case is more complicated. After splitting the edges, we obtain 4 tetrahedra at the vertices
of the original one. The octahedron that remains in the middle can be subdivided according to one of its three diagonals.
Only the choice of the shortest octahedron diagonal leads to a regular tetrahedra decomposition.
This algorithm originally comes from Bey~\cite{bey_2000}, further e.g. in~\cite{brandts_2011}.





\section{2d singular model}

In ambient 3d space, consider 2d problem with point singularity where the pressure $P_w$ is given.
  
  Mixed form:
  \begin{eqnarray}
\vc K_2^{-1}\vc u_2 + \nabla p_2 &=& 0 \qquad \textrm{in } \Omega_2, \label{eqn:2d_darcy_law}\\
\nabla \cdot \vc u_2 &=& f_2 \qquad \textrm{in } \Omega_2, \label{eqn:2d_continuity}\\
\avg{\vc u_2 \cdot \vc n}_w &=& \sigma_w (\avg{p_2}_w - P_w) \qquad \textrm{on } \Gamma_w,\label{eqn:2d_well_cond}\\
p_2 &=& g_{2D} \qquad \textrm{on } \Gamma_{2D}.
  \end{eqnarray}
  
  \paragraph{Geometry:} well center at $\vc x_w=[x_w,y_w]$, domain of different shapes (square, triangle, circle) with characteristic length $D$, Dirichlet BC $g_{2D}$ on the whole boundary.
  
  \emph{Settings 1}: $\vc x_w=[3.33,3.33],\, P_w = 100,\, \rho_w=0.03,\, \sigma_w=10,\, \vc K_2=K_2=1,\, f_2=0,\, D=10.$
  
  \emph{Settings 2}: same, $f_2=KU\omega^2\sin(\omega x_w),\, U=20,\, \omega=1.$
  
  \paragraph{Analytic solution.} The pressure solution in Settings 1 is set in such form 
  \begin{eqnarray}
      p_2 &=& a\log r,\label{eqn:2d_press_sol1}\\
      \vc u_2 &=& -\vc K_2 a \frac{\vc r}{r^2} \label{eqn:2d_vel_sol1}
  \end{eqnarray} 
  that it fulfills the condition \eqref{eqn:2d_well_cond} on the well edge
  \[ -\vc K_2\nabla p_2(\rho_w) \cdot \vc n = K_2a\frac{1}{\rho_w} = \sigma_w(p_2(\rho_w) - P_w),\]
  which results in setting
  \[a=\frac{-\sigma_w\rho_w P_w}{K_2 - \sigma_w\rho_w\log r}\]
  Logarithm is a harmonic function so it holds $\Delta \log(r) = 0$ and the equation in strong form is satisfied.
  
  Setting the nonzero source term $f_2$ in Settings 2, we obtain additional regular part of the solution
  \begin{eqnarray}
    p_2 &=& a\log r + U\sin(\omega x_w) \label{eqn:2d_press_sol2}\\
    \vc u_2 &=& -\vc K_2\nabla p_2 = -K_2 a\frac{\vc r}{r^2} -K_2 U\omega\cos(\omega x_w)\vc e_x \label{eqn:2d_vel_sol2}
  \end{eqnarray}
  
  With Settings 1, we can use these analytic values to observe error in our numerical model:
  \begin{center}
  \begin{tabular}{rl}
    flux over $\Gamma_w$: & $\bar{b}_w \doteq 6.5628$ \\ % 6.5627733533
    pressure $\avg{p}_w$: & $\bar{\lambda}_w \doteq 78.1241$ %78.1240888223
  \end{tabular}
  
  \centering
  \begin{tabular}{>{\centering\arraybackslash}p{4.5ex}|
                    >{\centering\arraybackslash}p{8.3ex}|
                    >{\centering\arraybackslash}p{9ex}|
                    >{\centering\arraybackslash}p{12.5ex}
                    >{\centering\arraybackslash}p{4ex}|
                    >{\centering\arraybackslash}p{12.5ex}
                    >{\centering\arraybackslash}p{4ex}}
                    \multicolumn{5}{c}{$\;\;\;$ source $f=0$} & \multicolumn{2}{c}{\textcolor{OrangeRed}{source $f$}} \\
    ref level  & $\abs{b_w-\bar{b}_w}$ & $\abs{\lambda_w - \bar{\lambda}_w}$ & $\norm{\vc u- \bar{\vc u}}_{L^2(\Omega)}$ & rate & $\norm{\vc u- \bar{\vc u}}_{L^2(\Omega)}$ & rate \\\hline
    1  &  0.1000  &  0.0528  &  6.73  &  -   &  112.5  &  -   \\
    2  &  0.1600  &  0.0871  &  2.96  & 1.18 &  46.1   & 1.29 \\
    3  &  0.0290  &  0.0155  &  1.68  & 0.82 &  21.9   & 1.08 \\
    4  &  0.0064  &  0.0034  &  0.94  & 0.84 &  10.6   & 1.05 \\
    5  &  0.0027  &  0.0014  &  0.51  & 0.88 &  5.6    & 0.91 
  \end{tabular}
  \end{center}
  
  
  \subsection{Multiple singularity solution}
  Considering multiple singularities in the domain, denoted using index set $\mathcal{W}=\{1\ldots W\}$, we can obtain the solution using the superposition principle.
  The solution is searched in the form
  \begin{eqnarray}
%     p_2 &=& p_{sin} + p_{reg} = \sum\limits_{j\in\mathcal{W}} a_j\log r_j + p_{reg}, \label{eqn:2d_press_sol_mult}\\
    \vc u_2 &=& -\vc K_2\nabla p_2 = -K_2 \sum\limits_{j\in\mathcal{W}} a_j\frac{\vc r_j}{r_j^2} - K_2 \grad p_{reg}, \label{eqn:2d_vel_mult}
  \end{eqnarray}
  which satisfies equations \eqref{eqn:2d_continuity} and \eqref{eqn:2d_darcy_law}.
  To find the unknowns $a_j$, we need to solve the system of linear equations given by the conditions \eqref{eqn:2d_well_cond}
  \begin{equation}
    \avg{\vc u_2 \cdot \vc n}_{\Gamma_i} = \sigma_i (\avg{p_2}_{\Gamma_i} - P_i) \qquad  \forall i\in\mathcal{W}. \label{eqn:2d_well_cond_mult}
  \end{equation}
  For an average $\avg{\cdot}_{\Gamma_i}$, it holds $\avg{u+v}_{\Gamma_i} = \avg{u}_{\Gamma_i} + \avg{v}_{\Gamma_i}$.
  Substituing \eqref{eqn:2d_press_sol_mult} and \eqref{eqn:2d_vel_mult} into \eqref{eqn:2d_well_cond_mult}, we get
  the linear system $\vc M\vc a = \vc b$ where
  \begin{eqnarray}
    M_{ii} &=& \frac{K_2}{\rho_i} - \sigma_i\log\rho_i \qquad \forall i\in\mathcal{W},\\
    M_{ij} &=& K_2\avg{\frac{\vc r_j \cdot \vc n_i}{r_j^2}}_{\Gamma_i} - \sigma_i\avg{\log r_j}_{\Gamma_i} \qquad \forall i,j\in\mathcal{W}, i\neq j,\\
    b_{i} &=& \sigma_i\left(\avg{p_{reg}}_{\Gamma_i} - P_i \right) - \avg{- K_2 \grad p_{reg}\cdot n_i}_{\Gamma_i} \qquad \forall i\in\mathcal{W}.
  \end{eqnarray}
  We evaluate all the averages $\avg{\cdot}_{\Gamma_i}$ in the linear system numerically,
  using simple composite midpoint rule integration with 1000 equidistant intervals on $\Gamma_i$.
  This way we obtain pseudo-analytic solution of the multi well problem, which is accurate enough
  be used for measuring the error of our model.
