
In this chapter, the algorithms for intersections of incompatible meshes, which are necessary to our reduced dimensional model, are discussed.
Not only the cases 1d-2d, 1d-3d, solved in previous chapters, but a~general approach also for cases 2d-3d and 2d-2d is considered.
The motivation comes from the future needs for the software Flow123d, to enable computations on incompatible meshes even for fractures.
This includes a~Mortar like coupling method~\cite{brezina_2012}, further development of the XFEM model defiend in this thesis,
Multilevel Monte Carlo (MLMC)~\cite{giles_mlmc_2015} simulations using random fracture generation or time evolving fracture network.
The prerequisite for any of these applications is a~fast and robust algorithm for calculating intersections of the individual meshes.
The content of this chapter includes the results published in the author's article~\cite{brezina_2017}.


\section{Introduction to Mesh Intersection}

The research on the currently available algorithms is provided in the introduction.
In particular the following are mentioned: the PANG algorithm \cite{gander_algorithm_2013} for 2d-2d and 3d-3d intersections in mesh overlapping problems or
2d-3d intersections algorithm \cite{massing_efficient_2013, gtslib} used in the implementation of the Nitche method.

The suggested approach is based on the \plucker coordinates, further developing the algorithm of Platis and Theoharis \cite{platis_fast_2003}
for ray-tetrahedron intersections. The algorithm is combined with the advancing front method which allows reusing \plucker coordinates
and their products among neighboring elements and to reduce the number of arithmetic operations. 

\section{Element Intersections}
\label{sec:element intersecitons}
In this section, the algorithms are presented for computing the intersection of a~pair of simplicial elements
of a~different dimension in a~3d ambient space. The fundamental idea is to compute intersection of 1d-2d 
simplices using the \plucker coordinates and to reduce all the other cases to this one. 

In general, an intersection can be a~point, a~line segment or a~polygon, called \emph{intersection polygon} (IP) in common.
IP is represented as a~list of points called \emph{intersection corners} (IC).
The information about the \emph{topological position} of IC is determined (IC in a~vertex, side or face).

% \subsection{\plucker Coordinates}
\plucker coordinates represent a~line in a~3d space. The definition, properties and the~more general context from computational 
geometry can be found e.g. in \cite{joswig_plucker_2013}.
It is shown that using \plucker coordinates and their permuted inner product, the relative position of a~line and triangle
in ambient 3d space can be determined.


% \subsection{Intersection Line-Triangle (1d-2d)}
% \label{sec:1d-2d}
Having computed the \plucker coordinates and all possible permuted inner product in the line-triangle case,
this data can be further used to derive the barycentric coordinates of the actual IC.
The formula is derived in detail.
Then the intersection algorithm for 1d-2d case is suggested, providing also the topological position of the intersection.


% \subsection{Intersection Line-Tetrahedron (1d-3d)}
Computation of the line-tetrahedron intersection uses the 1d-2d algorithm for each pair line-face.
The \plucker coordinates and permuted inner products are computed only once and possibly reused.
Each result of 1d-2d algorithm is treated carefully to process the topological positions correctly.
At most two intersection points are returned.


% \subsection{Intersection Triangle-Tetrahedron (2d-3d)}
In the intersection algorithm in triangle-tetrahedron case, 9 \plucker coordinates (3 sides, 6 edges) and 18 permuted inner products
are needed to compute up to 12 side-face intersections and 6 edge-triangle intersections.
The result is an $n$-side intersection polygon (IP), $n\le 7$.
This algorithm is much trickier to obtain correct topological positions of the ICs and to be able to sort them at the end
to form a~polygon. The algorithm is thouroughly described concerning all possible situations.

% \subsection{Intersection Triangle-Triangle (2d-2d)}
The intersection of two triangles uses up to 6 calls of the 1d-2d algorithm for all side-triangle combinations.
The algorithm again carefully propagates the topological data from the 1d-2d cases, so that some computations can be skipped.
Up to 2 ICs may be found as a~result.

\section{Global Mesh Intersection Algorithm}
\label{sec:front_advancing}
A~composed mesh is considered, containing a~3d mesh, that is called a~bulk mesh, and
other lower dimensional meshes are called component meshes. 
At first, all component-bulk mesh intersections are computed, i.e. 1d-3d and 2d-3d.
The algorithm includes two steps: finding the first non-empty element-element intersection (initiation)
and prologation of the intersection using the advancing front tracing.

% \subsection{Initiation}
% \label{sec:initiation}
The search for the initial element-element intersection is accelerated by two techniques.
First, the axes align bounding boxes (AABB) are computed for all elements, to provide fast
comparison method to detect candidate pairs of elements for the intersection.
Second, the bounding interval hiearchy (BIH)~\cite{wachter_instant_2006} can be build upon AABB, to speed up the search.

% \subsection{Advancing Front Method}
% \label{sec:front}
An advancing front algorithm is suggested to extend the initial intersections, 1d-3d or 2d-3d,
looking for candidate pairs among the neighboring component and bulk elements.
There it is taken advantage of the topological positions of ICs and the neighboring information on elements.
This algorithm runs until the component mesh is covered with all possible bulk elements.

% \subsection{Intersections Between Component Meshes}
% \label{sec:components}
Considering a~situation where the component meshes are in the interior of the 3d bulk mesh,
the component-bulk results are used for search for candidate pairs in the component-component intersections.
The storage of the intersection data, which plays an important part here, is described.
On the other hand, component intersections in the exterior of the bulk mesh must be searched via the initiation
techniques.


\section{Benchmarks}
\label{sec:benchmarks}

In this section, the suggested element-element algorithms are theoretically compared with other available algorithms
in terms of floating point operations (FLOPs) count and two numerical benchmarks by the software Flow123d are presented.

\subsection{Theoretical Comparison}
Three algorithms for the line-triangle intersections are compared: \plucker algorithm 
described in Section \ref{sec:element intersecitons}, the algorithm based on the plane clipping due to Haines \cite{haines_fast_1991}, and the minimum storage
algorithm due to M{\" o}ller and Trumbore (MT) \cite{moller_fast_1997}. For the later two algorithms straightforward modifications are considered
to make them return a~qualitatively same output as our algorithms do.

\begin{table}[!htb]
\begin{center}
\bgroup
\def\arraystretch{1.2}
\setlength\tabcolsep{5pt}
\begin{tabular}{l|lll}
\toprule
    algorithm              & 1d-2d   & 1d-3d    & 2d-3d \\ \midrule
    \plucker               & 92      & 198      & 426 \\
    \plucker (edge reuse)  & 45      & 138      & 264 \\
    Haines                 & 51      & 177      & 469 \\
    M\"oller and Trumbore  & 42      & 168      & 756 \\
\bottomrule
\end{tabular}
\caption[Comparison of intersection algorithms by FLOPs.]
    {Raw number of FLOPs used by different intersection algorithms. Second row contains the estimated effective number of FLOPs per intersection,
    accounting for reusing the computed \plucker data over neighboring elements, while assuming data on edges of 2d and 3d elements are used twice (conservative).}
\label{tab:fundamental_flops}
\egroup
\end{center}
\end{table}
%
The estimated numbers of FLOPs for all cases are summarized in Table \ref{tab:fundamental_flops}.
Algorithms based on the \plucker coordinates should be competitive with the state of the art algorithms in case of 1d-3d 
and 2d-3d intersections. The expected performance for the 1d-2d case seems to be poor, however these intersections are computed after 1d-3d and 2d-3d,
so the \plucker coordinates may be reused.  Similarly, better results are expected in the remaining two intersection cases when the \plucker coordinates and their products are reused by neighboring elements.

\subsection{Global Mesh Intersections}
Three variants of the suggested algorithms are compared on 2d-3d benchmarks in this section.
The first variant FS+AF uses a~full search (FS) over the bulk mesh, i.e. it uses only the unordered array of AABB of elements,
to get the initial pair for the advancing front algorithm (AF).
The second variant BIH+AF uses the BIH on top of AABB to accelerate the initiation of the AF algorithm.
The third variant BIH does not use AF at all and relies on the search through BIH only. 

The artificial case considers a~composed mesh consisting of a~cube and two diagonal rectangular 2d meshes.
A sequence of meshes is prepared with an increasing number of elements.
All three variants exhibit almost linear time complexity in both the initiation and the intersection phase,
relatively to the number of bulk and component elements, respectively.
The FS+AF variant is the fastest one, in particular in its initiation phase.
the BIH variant is about two times slower than the BIH+AF variant during the intersection phase. 

\begin{graph}[!htb]
    \centering
    \includegraphics[width=0.8\textwidth]{\figpathins intersections_bedrichov_both_speed.pdf}
    \caption[Comparison of the algorithms on meshes of Bed{\v r}ichov locality.]
    {Comparison of the algorithms on meshes of Bed{\v r}ichov locality -- 
        interior fractures on the left,
        extending fractures on the right.}
    \label{graph:bedrichov_speed}
\end{graph}
%
Next, the performance of the intersection algorithms on a~mesh of a~real locality in Bed{\v r}ichov in the Jizera mountains
is analyzed. Two meshes are considered, one with 28 fractures in the interior of the bulk mesh, 
the other with artificially extended fractures outside the bulk mesh.
he results for both meshes can be seen in \gref{graph:bedrichov_speed}.
In the first case, FS+AF and BIH+AF algorithms are nearly twice as fast as BIH.
Creating the BIH in the BIH+AF variant pays off and the algorithm performs better than the FS+AF variant.
In the second case, a~large blow up of the FS+AF variant is caused by the exterior component elements
The better performance of both BIH and BIH+AF variants is evident in this case.


Based on the benchmark results, the variant BIH+AF is recommended for general usage,
while the FS+AF might be more efficient when there are few component meshes inside a~bulk mesh.
The algorithms are implemented as a~part of the software Flow123d and the results were published in the article~\cite{brezina_2017}.
It is currently used in the experimental Mortar model for flow and in the XFEM models presented in Chapter~\ref{chap:xfem_mh}.


% \section{Summary}
% Algorithms for computing intersections of pairs of simplicial elements using \plucker coordinates are suggested in this chapter.
% They are shown theoretically to be competitive with the others state of the art algorithms by the complexity in terms of FLOPs count estimates.
% Three variants of the global mesh intersection algorithm for a~composed mesh are proposed, using 
% advancing front tracing and BIH techniques. Based on the benchmark results, the variant BIH+AF is recommended for general usage.
% % while the FS+AF might be more efficient when there are few component meshes inside a~bulk mesh.
% The algorithm is a~part of the software Flow123d and the results were published in the article~\cite{brezina_2017}.
% It is currently used in the experimental Mortar model for flow and in the XFEM models presented in Chapter~\ref{chap:xfem_mh}.
