
In this chapter, a~mixed hybrid model for Problems \ref{thm:problem_2d} and \ref{thm:problem_3d} is derived,
an enrichment for velocity is suggested and numerical experiments in Flow123d are presented, demonstrating
the advantages of the approach.
% \section{Mixed Problems in Groundwater Flow}
The strategic literature, which this chapter is build upon, is listed in the beginning.
Foremost the article \cite{sistek_bddc_2015} is mentioned, where the current model 
of fractured porous media solved in mixed-hybrid form and implemented in Flow123d is described.


\section{Mixed Dirichlet Problem}
At first, a~model considering only a~2d aquifer and wells reduced to their cross-sections $\Gamma_w$
with fixed pressure prescribed as a~Dirichlet boundary condition is solved.
Its weak form is derived and the existence of a~unique continuous solution of the resulting saddle-point
problem is shown by the theory in \cite{brezzi_mixed_1991}.

% \subsection{Discretization}
The derived saddle-point problem is discretized using the hybridization and the lowest Raviart-Thomas FE.
The existence and uniqueness of the solution is provided for the unenriched discrete solution.

Then the enrichment of velocity is suggested.
The global enrichment function for velocity is the derivative of logarithmic enrichment function introduced in Section \ref{sec:prim_discretization}
\begin{equation}
    \vc s_w(\bx) = -\frac{1}{S_e} \frac{\vc r_w}{r_w^2}, \qquad S_e = 2\pi\rho_w.
\end{equation}
where $S_e$ is called an effective (lateral) surface.
The properties of $\vc s_w$ are discussed.

An enrichment, where only a~single enrichment function is considered per singularity, is suggested.
On each enriched element $T^i,\, i\in\mathcal{J}^w_E$, an SGFEM like interpolation 
of the global enrichment function, see \eqref{eqn:sgfem_enrich}, is defined
\begin{equation}
    \vc L_{w}(\bx)|_{T^i} = \vc s_w(\bx) - \pi^{RT}_{T^i}(\vc s_w)(\bx).
    \label{eqn:local_enr_vel}
\end{equation}
where $\pi^{RT}_{T^i}$ is the interpolation operator to standard Raviart-Thomas FE space.
The enriched discrete velocity then takes the form
\begin{equation}
    \vc u_h = 
    \sum \limits_{i=1}^{n_F\cdot N_E} a_i \vc \psi_i + 
    \sum \limits_{w\in\mathcal{W}} b_w \vc L_{w}
    \label{eqn:velocity_enriched_approximation}
\end{equation}
where the first part is the standard approximation ($n_F$ is number of faces per element)
and the second part is the enrichment.

% \subsection{Numerical Verification of LBB Condition.}
% \label{sec:numerical_lbb}
A~numerical test for inf-sup stability validation is suggested there,
to check the LBB condition experimentally, since a~theoretical proof is unavailable.
A~corresponding generalized eigenvalue problem is formulated.
However, unclear relation between the mixed and mixed-hybrid form regarding
their generalized eigenvalue problems is reported.

% \subsection{Numerical Test for Dirichlet Problem.}
Finally, the first numerical results of the 2d Dirichlet problem are presented.
Optimal convergence is demonstrated on a series of simple structured simplicial meshes
which motivates for further development of the model.

\section{Coupled 1d-2d Model}
\label{sec:coupled_12d}
The fully coupled mixed-hybrid 1d-2d model is defined.
To this end, the following discretization spaces are considered
\begin{align}
    V_h &= V^{reg}_h(\Omega_1) \times V_h(\Omega_2), \label{eqn:vel_h_space}\\
    Q_h &= Q_h(\Omega_1) \times Q_h(\Omega_2), \label{eqn:press_h_space} \\
    \Lambda_h &= \Lambda_h(\mathcal{F}_1)\times \Lambda_h(\mathcal{F}_2) \times \Lambda^{enr}_h \label{eqn:lambda_h_space}
\end{align}
The velocity space is composed of $V^{reg}_h(\Omega_d)$, $d=1,2$, the standard Raviart-Thomas zero order FE space,
and $V_h(\Omega_2) = V^{reg}_{h}(\Omega_2) \oplus V^{enr}_h$, where $V^{enr}_h$ is the enriched space.
$Q_h(\Omega_d)$, $d=1,2$, is the $L_2$ space for pressure on elements.
$\Lambda_h(\mathcal{F}_d)$, $d=1,2$, is the space of Lagrange multipliers in the hybridization (pressure on element faces $\mathcal{F}_d$)
and $\Lambda^{enr}_h$ is the space of average traces of fluxes normal components on the well cross-sections.
The saddle point problem then reads:
\thmproblem{ \label{thm:prob_saddle_12d}
Find $\vc u_h=[\vc u_d] \in V_h$ and $p_h = [p_d, \lambda_d, \lambda^m_w] \in Q_h \times \Lambda_h$, 
$d=1,2,\, m\in\mathcal{M},\, w\in\mathcal{W}$ which satisfy
\begin{subequations}
\label{eqn:prob_saddle_coupled_12d}
    \begin{align}
    a_h(\vc u_h,\vc v_h) + b_h(\vc v_h, p_h) &= \langle G, \vc v_h\rangle_{V_h'\times V_h} &&
        \forall \vc v_h\in V_h,
        \label{eqn:prob_saddle_coupled_12d_1}\\
    b_h(\vc u_h, q_h) -c_w(p_h,q_h) &= \langle F, q_h \rangle_{Q_h'\times Q_h} &&
        \forall q_h \in Q_h \times \Lambda_h
        \label{eqn:prob_saddle_coupled_12d_2}
    \end{align}
\end{subequations}
where the forms are
\begin{align*}
a_h(\vc u_h, \vc v_h)=& \sum_{\substack{d=1,2 \\ i\in \mathcal I_{dE}}} \int_{T_d^i}
   \delta_d^{-1} \vc u_d \vc K_d^{-1} \vc v_d \dd\bx, \\
%
b_h(\vc v_h, p_h)=&\sum_{\substack{d=1,2 \\ i\in \mathcal I_{dE}}}\left[
        \int_{T_d^i} -p_d\,\div\,\vc v_d \dd\bx, 
        + \int_{\partial T^i_d\setminus \prtl\Omega}
                 \lambda_d (\vc v_d \cdot \vc n) \dd s\right]\\
        &\color{blue}{ + \sum_{\substack{w\in \mathcal{W} \\ m\in\mathcal{M}}}
            \int_{\Gamma^m_w} \lambda^m_w \avg{\vc v_2 \cdot \vc n}^m_w \dd s},\\
%
\color{blue}
 c_w(p_h,q_h) =& \color{blue}{ \sum_{\substack{w\in \mathcal{W} \\ m\in\mathcal{M}}} \int_{\Gamma^m_w} \delta_2(\bx^m_w)\sigma^m_w
    \big(p_1(\bx^m_w)-\lambda_w\big) \big(q_1(\bx^m_w)-\mu_w\big) \dd s},\\
 \langle G, \vc v_h\rangle_{V_h'\times V_h} =& \sum_{d=1,2}\sum_{i\in \mathcal T_d}
        \int_{\prtl T_d^i\cap \Gamma_{dD}}
                 - g_{dD} (\vc v_d \cdot \vc n) \dd s,\\
 \langle F, q_h\rangle_{Q_h'\times Q_h} =& \sum_{d=1,2}\sum_{i\in \mathcal T_d}%\left(
        \int_{T_d^i} - \delta_d f_d q_d \dd\bx,
%         - \int_{\prtl T_d^i\cap \Gamma_{dN}}
%                  g_{dN} \mu_d\right),
\end{align*}
}
The coupling terms are emphasized in blue color.

\section{Coupled 1d-3d Model}
\label{sec:coupled_13d}
The 1d-3d coupled model is built in the same manner as the 1d-2d problem.


% \subsection{Enrichment Function in 3d}

The well distance function $r_w$ in 3d is defined as a~shortest distance
between a~point and the line representing the well.
The enrichment function $\vc L_{w}$  is suggested in the same form as in 2d 
and the its properties are demonstrated.
The choice of the enrichment zone in 3d is discussed.

% \subsection{Saddle Point Problem in 3d}
Using the same notation as in
\eqref{eqn:vel_h_space}-
% {eqn:press_h_space}
\eqref{eqn:lambda_h_space}
for the discrete space
\begin{align}
    V_h &= V^{reg}_h(\Omega_1) \times V_h(\Omega_3), \label{eqn:vel_h_space_3d}\\
    Q_h &= Q_h(\Omega_1) \times Q_h(\Omega_3), \label{eqn:press_h_space_3d} \\
    \Lambda_h &= \Lambda_h(\mathcal{F}_1)\times \Lambda_h(\mathcal{F}_3) \times \Lambda^{enr}_h, \label{eqn:lambda_h_space_3d}
\end{align}
the saddle point problem for 1d-3d model is formulated
\thmproblem{ \label{thm:prob_saddle_13d}
Find $\vc u_h=[\vc u_d] \in V_h$ and $p_h = [p_d, \lambda_d, \lambda_w] \in Q_h \times \Lambda_h$, 
$d=1,3,\, w\in\mathcal{W}$ which satisfy
\begin{subequations}
\label{eqn:prob_saddle_coupled_13d}
    \begin{align}
    a_h(\vc u_h,\vc v_h) + b_h(\vc v_h, p_h) &= \langle G, \vc v_h\rangle_{V_h'\times V_h} &&
        \forall \vc v_h\in V_h,
        \label{eqn:prob_saddle_coupled_13d_1}\\
    b_h(\vc u_h, q_h) -c_w(p_h,q_h) &= \langle F, q_h \rangle_{Q_h'\times Q_h} &&
        \forall q_h \in Q_h \times \Lambda_h
        \label{eqn:prob_saddle_coupled_13d_2}
    \end{align}
\end{subequations}
where the forms are
\begin{align*}
a_h(\vc u_h, \vc v_h)=& \sum_{\substack{d=1,3 \\ i\in \mathcal I_{dE}}} \int_{T_d^i}
   \delta_d^{-1} \vc u_d \vc K_d^{-1} \vc v_d \dd\bx, \\
%
b_h(\vc v_h, p_h)=&\sum_{\substack{d=1,3 \\ i\in \mathcal I_{dE}}}\left[
        \int_{T_d^i} -p_d\,\div\,\vc v_d \dd\bx, 
        + \int_{\partial T^i_d\setminus \prtl\Omega}
                 \lambda_d (\vc v_d \cdot \vc n) \dd s\right]\\
        &\color{blue}{ + \sum_{w\in \mathcal{W}} \Bigg[ 2\pi\rho_w\abs{\Omega^w_1} \sum_{\substack{i\in\mathcal{I}_{1E} \\ T^i_1\subset\Omega^w_1}}
            \int_{T^i_1} \lambda_w(t) \avg{\vc v_3\cdot \vc n}_w(t) \dd t \Bigg]},\\
%             \int_{\Gamma_w} \lambda_w \avg{\vc v_3 \cdot \vc n}_w \dd s\Bigg]},\\
%
\color{blue}
 c_w(p_h,q_h) =& \\
 \color{blue}{ \sum_{w\in \mathcal{W}} \Bigg[2\pi}
        & \color{blue}{\rho_w\abs{\Omega^w_1} \sum_{\substack{i\in\mathcal{I}_{1E} \\ T^i_1\subset\Omega^w_1}}
        \int_{T^i_1} \sigma_w \big(p_1(t)-\lambda_w(t)\big) \big(q_1(t)-\mu_w(t)\big) \dd t} \Bigg],\\
 \langle G, \vc v_h\rangle_{V_h'\times V_h} =& \sum_{d=1,3}\sum_{i\in \mathcal T_d}
        \int_{\prtl T_d^i\cap \Gamma_{dD}}
                 - g_{dD} (\vc v_d \cdot \vc n) \dd s,\\
 \langle F, q_h\rangle_{Q_h'\times Q_h} =& \sum_{d=1,3}\sum_{i\in \mathcal T_d}%\left(
        \int_{T_d^i} - \delta_d f_d q_d \dd\bx.
%         - \int_{\prtl T_d^i\cap \Gamma_{dN}}
%                  g_{dN} \mu_d\right),
\end{align*}
}
The coupling terms are again emphasized in blue color.
