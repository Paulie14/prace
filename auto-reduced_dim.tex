
In this chapter, the concept of combining models of different spatial dimension is introduced.
The thesis is put into the context of the so called \emph{reduced dimensional models} and
the long-term research at the Technical University of Liberec concentrated around the software Flow123d.
Finally, a~well aquifer model which is to be solved in this work is defined, combining 1d-2d and 1d-3d dimensions.
% Also the relationship to the long-term research at the Technical University in Liberec is described.

% The realistic models of underground processes, namely the groundwater flow and transport, geomechanics and thermal processes, have to deal with 
% a complex of geological formations containing various small scale features such as fractures and wells. Although the scale of these features
% can differ from the size of the computational domain by several orders, they influence significantly the global behavior of the system.
% In the past years, two different approaches for incorporating these features into numerical models have been developed.

The equivalent continuum concept includes the fractures and wells in the model by changing the global properties of the modeled volume
(e.g. increased/decreased hydraulic conductivity).
This homogenization approach smears local effects of the disruptions but the model can still provide a~valid approximation for an initial global view of the system.
% For example, instead of having a~rock with a~complex network of conductive fractures, we use an equivalent continuum with increased conductivity,
% and the domain can be simply discretized without taking the fractures into account.
% The concept is also suitable due to the lack of detailed data which people suffer from in most applications.
% The microscopic effects are neglected by applying the representative elementary volume (homogenization per elements of mesh in FVM, FDM or FEM),
% the macroscopic effects (larger fractures) can be modeled by equivalent continuum.
Since the underlying computational mesh does not need to represent the small scale objects, the equivalent continuum models can be computationally cheap,
but they cannot capture the local effects accurately.
The other approach considers the disruptions in a~discrete sense, keeping its sizes and properties explicitly in the model.
This brings more demanding work at the geometric level, where the disruptions have to be represented so that a~suitable computational mesh can be built.
A significant simplification in this concept is to reduce the dimension of the local phenomena, if possible, and obtain a~reduced dimensional model,

% and to represent these as lower dimensional objects. This means at the geometrical level to neglect the thickness of a~thin fracture and consider a~plane, or
% neglect the cross-section of a~narrow channel and consider a~line instead. The quantities are then averaged over the complementary dimensions.
% This approach is referred as \emph{reducing dimensions} or \emph{reduced dimensional model}.
The concept of reduced dimensions is fundamental in the Discrete Fracture Network (DFN) models \cite{jing_fundamentals_dfn_2007}
and in models combining DFN with equivalent continuum. The spacial discretization of such models leads to
meshes composed of elements of different dimension, in the fractured porous media context e.g. in
\cite{martin_modeling_2005, fumagalli_numerical_2011, brezina_analysis_2015}.
There is a~limited number of models for coupling of co-dimension 2: 1d-3d coupling or point intersections in non-planar 1d-2d coupling.
Such models solving flow and transport in vascular systems are e.g. in \cite{dangelo_coupling_2008} and \cite{koppl_tum_2015}.

% 
% The works mentioned before deal with modeling fractures, however they are not concerned neither about 1d-3d coupling
% nor point intersections in non-planar 1d-2d coupling. These topics are also studied extensively by different groups, see
% especially \cite{dangelo_coupling_2008, cattaneo_numerical_2015} and \cite{koppl_tum_2015, koppl_vidotto_2018}.
% Their applications are mainly addressed to the medical problematic and they analyze the transport of medicaments through
% the vascular system and to the surrounding tissue. The veins can be viewed as 1d channels and the tissue is modeled as a~porous media,
% and thus the flow is governed by the same type of equations as in the groundwater simulations.

In the thesis, a reduced dimensional model for fractured porous media being developed by the means of the software Flow123d
is extended by nonplanar 1d-2d and 1d-3d models.

% In this work we build and expanse a reduced dimensional model on top of the existing one being developed at TUL.
% It combines both the equivalent continuum and the DFN approach including the communication fractures with surrounding bulk domains.
% Since the numerical solution is based on FEM, we are interested in how the fractures and bulk are geometrically discretized.
% Therefore we describe the meshes of combined dimensions below.



\section{Mesh of Combined Dimensions} \label{sec:mesh_combined}
As pointed out before, the reduced dimensional models lead to computations on meshes of combined dimensions.
The reduced dimensional concept splits the computational domain into parts of different dimensions
$\Omega_d$, $d=1,2,3$. The domains $\Omega_2$ and $\Omega_1$ are polytopic (i.e. polygonal and piecewise linear, respectively).
Apart from Chapter \ref{chap:xfem_pressure}, the simplicial meshes are considered.
The \emph{compatible} (or conforming, matching) meshes satify the compatibility conditions, i.e.
the $(d-1)$-dimensional elements are either between $d$-dimensional elements and
match their faces or they extend out of $\Omega_d$. 

% The motivation to model the small scale disruptions in the bulk domain as lower dimensional objects has been
% stated in the introduction of the chapter. Now we fix the idea of a~mesh of combined dimensions and we give
% a~more precise description.

% An open set $\Omega_{3} \subset \Real^3$ represents a~continuous approximation of a~porous medium, i.e. the bulk domain.
% In the same manner, a~set of 2d manifolds $\Omega_2 \subset \Real^3$ is considered, representing the 2d fractures
% and a~set of 1d manifolds $\Omega_1 \subset \Real^3$ representing the 1d channels in an arbitrary 3d space.
% See the Figure \ref{fig:multi-dim}.


% The domains $\Omega_2$ and $\Omega_1$ are polytopic (i.e. polygonal and piecewise linear, respectively).
% Following the documentation of Flow123d \cite{flow123d_doc_2018} and its notation,
% for every dimension $d=1,2,3$, a~triangulation $\mathcal{T}_{d}$ of the open set $\Omega_d$
% is introduced that consists of
% \begin{equation}
% \begin{aligned}
%     \textrm{elements } & T_{d}^{i}, & \quad i \in\mathcal{I}_{d,E} = \{1,\dots,N_{d,E} \}, \\
%     \textrm{faces }    & F_{d}^{i}, & \quad i \in\mathcal{I}_{d,F} = \{1,\dots,N_{d,F} \}, \\
%     \textrm{nodes }    & \vc x_{i}, & \quad i \in\mathcal{I}_{d,N} = \{1,\dots,N_{d,N} \}.
% \end{aligned}
% \end{equation}

% Simplicial meshes are used, so the elements are lines, triangles and tetrahedra.
% If not specified, the simplicial meshes are used, so the elements are lines, triangles and tetrahedra.
% By faces $F^i_d$ we understand $(d-1)$-dimensional boundary objects of elements $T_d^i$,
% i.e. end points $F^i_1$ of $T^i_1$, triangle sides $F^i_2$ of $T^i_2$, tetrahedron faces $F^i_3$ of $T^i_3$.
% We note that in Chapter \ref{chap:xfem_pressure}, the quadrilateral elements are considered in contrast to the rest of the work
% due to the use of a~specific FEM library code.

% $\Omega_2\subset\overline\Omega_3$
% $\Omega_1\subset \overline\Omega_2$

% The elements in \emph{a compatible mesh} must satisfy the following compatibility conditions
% \begin{equation}
%         T_{d-1}^i \cap \mathcal{T}_d \subset \mathcal{F}_d,  \qquad \text{where } \mathcal{F}_d = \bigcup_{k\in\mathcal{I}_{d,E}} \partial T_{d}^{k}
% \end{equation}
% and
% \begin{equation}
%         T_{d-1}^i \cap \mathcal{F}_d = 
%         \begin{cases}
%             T_{d-1}^i, \\
%             \emptyset,
%         \end{cases}
% \end{equation}
% for every $\,i\in\mathcal{I}_{d-1,E}$, and $d=2,3$. 
% That is, the $(d-1)$-dimensional elements are either between $d$-dimensional elements and
% match their faces or they poke out of $\Omega_d$. 
% Under these conditions, a compatible 1d line in 3d space is formed only as an intersection of 
% two 2d planes.
% Direct compatibility in case of co-dimension 2, i.e. for 1d and 3d coupling, is not considered.

\emph{An incompatible mesh} (or non-conforming or non-matching), does not have to satisfy the compatibility conditions
and therefore the intersecting domains can be meshed independtly, reducing the requirements on the meshing tools significantly.
Then of course the intersections must be computed and the discrete model must deal with the incompatible coupling.
Therefore Chapter \ref{chap:intersections} is dedicated to a~development of algorithms
for efficient computation of intersections of incompatible meshes.

% A creation of a~compatible mesh can be challenging on the algorithms, due to the compatibility conditions.
% The requirements on the element quality can be also limiting, since the intersection of domains can be very arbitrary.
% Therefore we look for models not requiring the compatible meshes.

% \emph{An incompatible mesh}, also referred as non-conforming or non-matching, does not have to satisfy the compatibility equations above.
% The creation of the mesh is much simpler when it can be done independently per domain.
% On the other hand, there are three main challenges at the next levels of the model solution.
% First, the intersections of the mesh elements must be computed efficiently and the intersection data
% must be represented in such a way that it can be used simply and cheaply in the FEM code.
% Second, an efficient numerical integration must be available on the intersecting polygons.
% Third, stable coupling conditions between dimensions must be provided.
% Part of this work, Chapter \ref{chap:intersections}, is dedicated to a~development of algorithms
% for computation of intersections of incompatible meshes.


\section{Well-Aquifer Model}
% A~simplified underground water flow model for a~well-aquifer system is defined in this section.
% A~well-aquifer model that we solve in our work is defined in this section.
% We follow the reduced dimension concept and introduce a simplified underground water flow model.

Wells are considered as straight narrow tubes (cylinders) $\Omega^w_C$ of radius $\rho_w$,
indexed by $w\in\mathcal{W}=\{1\ldots W\}$. These are reduced to 1d manifolds
$\Omega^w_1$, parameterized using a~mapping $\nu_w(t): [0,1] \rightarrow \Omega^w_1$.
A~lateral surface of a~cylinder $\Omega^w_C$ is denoted $\partial\Omega^w_C$.

% We consider wells (or boreholes, channels) to be straight narrow tubes (cylinders) $\Omega^w_C$ of radius $\rho_w$;
% we use the index $w\in\mathcal{W}=\{1\ldots W\}$ when referring to a~particular well.
% Using the reduced dimensional concept, the wells are modeled as lower dimensional objects,
% i.e. 1d manifolds $\Omega^w_1$ which represent the central lines of cylinders $\Omega^w_C$.
% The 1d manifold $\Omega^w_1$ can be parameterized using a~mapping $\nu_w: [0,1] \rightarrow \Omega^w_1$:
% \begin{equation} \label{eqn:1d_parametrization}
%     \Omega^w_1 = \{\vc x\in\R^3: \vc x = \nu_w(t),\; t\in[0,1] \}. %\quad \nu_w: [0,1] \rightarrow \Omega^w_1
% \end{equation}
% Later in the text we suppose that functions in $\Omega^w_1$ depend on the relevant parameter $t$.
% \begin{equation}
%     \prtl\Omega^w_C = \{\vc x\in\R^3: \vc x = \nu_w(t) + \vc r,\; \vc r=\rho_w\vc n_{\Omega^w_1}(t),\; t\in[0,1] \},
% \end{equation}
% where $\vc n_{\Omega^w_1}(t)$ is the unit normal vector to the line $\Omega^w_1$ at point $t$.

% The domain of a~well $\Omega^w_1$ has a~boundary (obviously consisting of two end points)
% $\Gamma_1 = \Gamma_{1D} \cup \Gamma_{1N}$, where pressure (Dirichlet boundary condition)
% or flux (Neumann boundary condition) is prescribed.
% If possible, and apparent from context, we denote the domains of all wells by
% \begin{equation}
%     \Omega_1 = \bigcup_{w\in\mathcal{W}} \Omega^w_1, \quad \Omega_C = \bigcup_{w\in\mathcal{W}} \Omega^w_C.
% \end{equation}

A~steady groundwater flow governed by Darcy's law is considered in all domains.
% The same is considered also in wells for simplicity:
% we can imagine the wells filled with different porous material, or in some tests we fix pressure in wells to a~constant value.
Without any coupling terms, one writes for every dimension
\begin{eqnarray}
     \frac{1}{\delta_d} \vc K_d^{-1} \vc u_d + \grad p_d =& 0  &\qquad \textrm{ in }\Omega_d\\
      \div\,\vc u_d  =& \delta_d f_d  &\qquad \textrm{ in }\Omega_d
\end{eqnarray}
where $\delta_d\vc u_d\; [ms^{-1}]$ is the unknown Darcian velocity, $p_d \; [m]$ is the unknown pressure.
Parameter $\delta_d$ is the complement measure of the domain: thickness $[m]$ in 2d,
cross-section $[m^2]$ in 1d and $\delta_3=1\; [-]$ for consistency. Since the wells are considered as cylinders,
the cross-section $\delta_1$ is constant per well: $\delta_1(\vc x)=\delta^w_1=\pi\rho^2_w$ for all $\vc x\in \Omega^w_1$.
The quantity $\vc u_d\; [m^{d-4}s^{-1}]$ itself can be seen as flux density, i.e. flux through $\Omega_d$
with complementary dimension $\delta_d=1$.
The conductivity tensor $\vc K_d\; [ms^{-1}]$ is generally $3\times 3$ matrix,
symmetric and positive definite, possibly representing the anisotropy of the media.
In the source term, $f_d\;[s^{-1}]$ denotes the water source density.

% We note that we use the same operators $\div$ and $\grad$ in all dimensions $d=1,2,3$ for the sake of notation simplicity,
% although in 1d we understand these as follows
% \begin{eqnarray}
% \div\,\vc u_1 = \div_t\,\vc u_1 = \frac{\prtl}{\prtl t} \vc u_1(t), \\
% \grad p_1 = \grad_t p_1 = \frac{\prtl}{\prtl t} p_1(t),
% \end{eqnarray}
% with $t$ being the 1d parameter in \eqref{eqn:1d_parametrization}.

% Two models are presented below, the first one supposes further simplifications and includes 1d-2d coupling between dimensions,
% the other one couples 1d and 3d domains together.


\subsection{1d-2d Model}
% We consider a~system of aquifers separated by aquitards
% (based on Gracie and Craig \cite{gracie_modelling_2010,craig_using_2011}).
% The aquifers are underground horizontal layers of permeable rock containing water in its pores.
% These layers often have negligible flow in the vertical direction so they can be modeled
% as 2d planes by means of the reduced dimensional modeling.
% There can be several aquifers on top of each other, separated by layers of very low permeability
% which are called aquitards.
% In contrast to Gracie and Craig, we suppose the aquitards to be impermeable, 
% so the aquifers can communicate between each other only through the wells.
% 
% On the other hand, we add an artificial volume source term on aquifers.
% This allows us to better study the impact of the prescribed source on the solution
% which is better suited to the numerical experiments we are going to present. 
% 
% The model is defined as a~complex multi-aquifer system to follow our implementation and to see the differences
% we made in comparison to Gracie and Craig. However, before solving the complex model,
% we constrain ourselves to a~single aquifer in a~major part of the work.

% If possible, and apparent from context, we denote unions of the domains
% \begin{equation}
%     \Omega_1 = \bigcup_{w\in\mathcal{W}} \Omega^w_1, \quad \Omega_2 = \bigcup_{m\in\mathcal{M}} \Omega^m_2.
% \end{equation}
% 
% The boundary of domains is split $\partial\Omega_d=\Gamma_{dD}\cup\Gamma_{dN}$ and
% two types of boundary conditions are assumed: Dirichlet on $\Gamma_{dD}$ and Neumann on $\Gamma_{dN}$.

A~system of aquifers separated by aquitards is considered (based on Gracie and Craig \cite{gracie_modelling_2010,craig_using_2011}).
The aquifers, denoted $\Omega^m_2$, $m\in\mathcal{M}=\{1\ldots M\}$, are underground horizontal layers of permeable rock containing water.
The aquifers' boundary consists of three parts $\prtl\Omega^m_2 = \Gamma^m_{2D} \cup \Gamma^m_{2N} \cup \bigcup_{\substack{w\in\mathcal{W}}} \Gamma^m_w$,
where Dirichlet and Neumann boundary conditions are prescribed and cross-sections of the wells and aquifers $\Gamma^m_w = \Omega^m_2 \cap \prtl\Omega^w_C$.

If possible, and apparent from context, the domain unions are denoted
\begin{equation}
    \Omega_1 = \bigcup_{w\in\mathcal{W}} \Omega^w_1, \quad \Omega_2 = \bigcup_{m\in\mathcal{M}} \Omega^m_2, \quad \Gamma_{\#} = \bigcup_{m\in\mathcal{M}} \Gamma^m_{\#}
\end{equation}
with $\#$ being $w,\; 2D$ or $2N$.

% where it is further denoted $\Gamma^m_{ext} = \Gamma^m_{2D} \cup \Gamma^m_{2N}$, and $\Gamma^m_{int}=$ consits

% Let $\Omega^m_2$ be the domain of an aquifer with index $m\in\mathcal{M}=\{1\ldots M\}$.
% We denote the aquifers' boundary $\prtl\Omega^m_2 = \Gamma^m_{ext} \cup \Gamma^m_{int}$,
% where we call $\Gamma^m_{ext} = \Gamma^m_{2D} \cup \Gamma^m_{2N}$ the exterior boundary, consisting of two parts where we prescribe
% Dirichlet and Neumann boundary condition, respectively.
% The interior boundary $\Gamma^m_{int}$ is created by an union of all the cross-sections of the aquifer domain $\Omega^m_2$
% and wells lateral surfaces
% \begin{equation}
%     \Gamma^m_{int} = \bigcup_{\substack{w\in\mathcal{W}}} \Gamma^m_w, \quad \Gamma^m_w = \Omega^m_2 \cap \prtl\Omega^w_C.
% \end{equation}
% Where possible, we again use the simplified notation
% \begin{equation}
%     \Omega_2 = \bigcup_{m\in\mathcal{M}} \Omega^w_2, \quad \Gamma_{\#} = \bigcup_{m\in\mathcal{M}} \Gamma^m_{\#}
% \end{equation}
% with $\#$ being $ext,\; int,\; 2D$ or $2N$.

On $\Gamma^m_w$, a~decomposition of an arbitrary function $q\in C(\overbar\Omega^m_2)$ on average and fluctuation parts
is defined, in the same manner as it can be found in \cite{koppl_vidotto_2018},
\begin{equation}\label{eqn:average_decomposition}
    q = \avg{q}^m_w + \fluct{q}^m_w \quad \textrm{on } \Gamma^m_w, \qquad \avg{q}^m_w = \frac{1}{{\abs{\Gamma^m_w}}} \int_{\Gamma^m_w} q \dd s.
\end{equation}
% where
% \begin{equation} \label{eqn:average_definition}
%     
% \end{equation}
% It holds for the fluctuation term from the definition \eqref{eqn:average_definition}
% \begin{equation}
% \int_{\Gamma^m_w} \fluct{q}^m_w \dd s = \int_{\Gamma^m_w} q - \avg{q}^m_w \dd s = 0.
% \end{equation}

% The average term is linear, since it holds 
% \begin{equation}
%     \avg{q_1+q_2}^m_w = \avg{q_1}^m_w + \avg{q_2}^m_w.
% \end{equation}

% The point where the reduced domain $\Omega^w_1$ intersects with aquifer $\Omega^m_2$ is denoted $\vc x^m_w=\nu_w(t^m)$.
% In order to prescribe a~communication term in $\Omega^w_1$, we use a~Dirac measure at the point $t^m$
% \begin{equation}
%     c = 
%     \begin{cases}
%         1 & t = t^m \\
%         0 & t \neq t^m
%     \end{cases}, 
%     \quad \int_0^1 q(t) \delta_t(t - t^m) \dd t = q(t^m), \quad q\in C(\overbar\Omega^w_1).
% \end{equation}

The definition of the 1d-2d well-aquifer problem follows:
\thmproblem{ \label{thm:problem_2d}
Find $[\vc u_1,\,\vc u_2]$ and $[p_1,\,p_2]$ satisfying
\begin{subequations}
\begin{align}
\delta_d^{-1} \vc K_d^{-1} \vc u_d + \grad p_d &= 0 && \textrm{in } \Omega_d,\;d=1,2,  \\
\div\, \vc u_2 &= \delta_2 f_2 && \textrm{in } \Omega_2, \label{eqn:problem_2d_2d_source}\\
\div\, \vc u_1 &= \delta_1 f_1 + {\color{blue}f_w} && \textrm{in } \Omega^w_1,\;\forall w\in\mathcal{W}, \label{eqn:problem_2d_1d_source}\\
%     & + {\color{blue}\sum_{m\in\mathcal{M}}\abs{\Gamma^m_w}\Sigma^m_w\delta_t(t - t^m)} && \textrm{in } \Omega^w_1,\;\forall w\in\mathcal{W}, \label{eqn:problem_2d_1d_source}\\
\color{blue}\avg{-\delta_2 \vc K_2 \grad p_2 \cdot \vc n}^m_w &= \color{blue}\Sigma^m_w && \forall w\in\mathcal{W},\; \forall m\in\mathcal{M}, \label{eqn:problem_2d_flux_avg}\\
\fluct{p_2}^m_w &= g^m_w && \forall w\in\mathcal{W},\; \forall m\in\mathcal{M}, \label{eqn:problem_2d_press_fluct}\\
p_d &= g_{dD} && \textrm{on } \Gamma_{dD},\; d=1,2, \label{eqn:problem_2d_dirichlet}\\
\delta_d \vc K_d \grad p_d \cdot \vc n &= g_{dN} && \textrm{on } \Gamma_{dN},\; d=1,2 \label{eqn:problem_2d_neumann}
\end{align}
\end{subequations}
where
\[ 
{\color{blue}
%     f_w = \sum_{m\in\mathcal{M}}\abs{\Gamma^m_w} \delta_2(\vc x^m_w)\sigma^m_w\big(\avg{p_2}^m_w-p_1(\vc x^m_w)\big) \delta_t(t - t^m).
    \Sigma^m_w = \delta_2(\vc x^m_w)\sigma^m_w\big(\avg{p_2}^m_w-p_1(\vc x^m_w)\big), \quad f_w=\sum_{m\in\mathcal{M}}\abs{\Gamma^m_w} \Sigma^m_w  \delta_t(t - t^m).
} 
\label{eqn:1d_2d_sigma_source}
\]
}
Term $\delta_t(t - t^m)$ is a~Dirac delta measure. %The dimensional coupling terms are highlighted in the blue color.
The average outward flux from the aquifer over $\Gamma_w$ acts as a positive source term in the well (blue colored terms).

% The term $\Sigma^m_w$ is a~constant representing the flux density from the aquifer $\Omega^m_2$ to the well $\Omega^w_1$
% through their intersection $\Gamma^m_w$. 
% The flux is proportional to the pressure difference with the permeability coefficient $\sigma^m_w>0\;[s^{-1}]$ between $\Omega^m_2$ and $\Omega^w_1$.
% If the pressure difference is negative, the flux is in opposite direction.
% We can then read: the average outward flux from the aquifer
% over $\Gamma^m_w$ in \eqref{eqn:problem_2d_flux_avg} acts as a~positive source term in the well in \eqref{eqn:problem_2d_1d_source}.
% 
% The equation \eqref{eqn:problem_2d_press_fluct} prescribes the fluctuation of pressure on $\Gamma^m_w$,
% which is not defined by the coupling terms. In reality, wells are very narrow in comparison to the scale of aquifers
% and so the changes in pressure on $\Gamma^m_w$ are very small. Thus we consider $g^m_w \approx \rho_w$ and
% we neglect the effects of $g^m_w$ as shown later.
% 
% The last two equations are the boundary conditions, the first one \eqref{eqn:problem_2d_dirichlet} prescribes the boundary pressure $g_{dD}$, 
% the later one \eqref{eqn:problem_2d_neumann} sets the inward flux $g_{dN}$. The sign of the flux in \eqref{eqn:problem_2d_neumann}
% is chosen such that it is consistent with the software Flow123d.
% 
% The problematic place of a reduced dimensional model, if done in the simplest way, would be that after the reduction of wells,
% the 1d-2d intersections are degenerated to points. Then the term in the flux coupling condition \eqref{eqn:problem_2d_flux_avg} would rather
% appear as a~point source on the right hand side of \eqref{eqn:problem_2d_2d_source} creating strong singularities in the solution.
% However, keeping the real scale of the intersections in the model by means of \eqref{eqn:problem_2d_flux_avg}, we cut off the singularity peaks
% at finite height and get rid of the singular solution part running to infinity.
% Still there is left the problem of capturing the intersections and adjacent steep gradients in the solution by the numerical discretization.
% This is the same motivation found in models in \cite{gracie_modelling_2010,craig_using_2011,koppl_vidotto_2018} and ours,
% although treated very differently at the numerical level in each of the works.

\subsection{1d-3d Model}
In the 1d-3d model, the aquifer is not reduced to a plane but it is represented as a~3d domain $\Omega_3$,
with a~boundary $\prtl\Omega_3 = \Gamma_{3D} \cup \Gamma_{3N} \cup \bigcup_{\substack{w\in\mathcal{W}}} \Gamma_w$,
where Dirichlet and Neumann boundary conditions are prescribed and cross-sections of the wells and aquifers $\Gamma_w = \Omega_3 \cap \prtl\Omega^w_C$,
analogically to the 1d-2d model.

% In this model the aquifer domain is not reduced to a plane but it is represented as a~3d domain.
% This would be a necessary consideration when the aquifer cannot be properly modeled in layers.
% Let us denote the domain $\Omega_3$, we do not have the index $m$ here.
% The boundary $\prtl\Omega_3 = \Gamma_{ext} \cup \Gamma_{int}$,
% where the exterior boundary consists of two parts $\Gamma_{ext} = \Gamma_{3D} \cup \Gamma_{3N}$,
% analogically to the 1d-2d model.

% The interior boundary $\Gamma_{int}$ is created by an union of all the cross-sections of the domain $\Omega_3$
% and wells lateral surfaces
% \begin{equation}
%     \Gamma_{int} = \bigcup_{\substack{w\in\mathcal{W}}} \Gamma_w, \quad \Gamma_w = \Omega_3 \cap \prtl\Omega^w_C.
% \end{equation}
On $\Gamma_w$ the average decomposition $w = \avg{g}_w + \{g\}_w$ is defined in a~similar manner
as in the 1d-2d case. However, the average is computed over a circle edge, perpendicular to $\Omega^w_1$,
with its center at a~point $\nu_w(t)\in \Omega^w_1$
\begin{equation} \label{eqn:average_definition_3d}
    \avg{g}_w(t) = \frac{1}{2\pi\rho_w} \int_0^{2\pi} g (t + \rho_w \vc n_{\Omega^w_1}(t,\theta)) \rho_w \dd \theta.
\end{equation}
with $\vc n_{\Omega^w_1}(t,\theta)$ being the unit normal vector of $\Omega^w_1$ at point $\nu_w(t)$ in the direction
determined by the angle $\theta$.

The definition of the 1d-3d problem follows:
\thmproblem{ \label{thm:problem_3d}
Find $[\vc u_1,\,\vc u_3]$ and $[p_1,\,p_3]$ satisfying
\begin{subequations}
\begin{align}
\delta_d^{-1} \vc K_d^{-1} \vc u_d + \grad p_d &= 0 && \textrm{in } \Omega_d,\;d=1,3,  \\
\div\, \vc u_3 &= \delta_3 f_3 && \textrm{in } \Omega_3,\\
\div\, \vc u_1 &= \delta_1 f_1 + {\color{blue}\Sigma_w} && \textrm{in } \Omega^w_1,\;\forall w\in\mathcal{W}, \label{eqn:problem_3d_1d_source}\\
\avg{\color{blue}-\delta_3 \vc K_3 \grad p_3 \cdot \vc n}_w &\color{blue} = \Sigma_w && \textrm{in } \Omega^w_1,\;\forall w\in\mathcal{W}, \label{eqn:problem_3d_flux}\\
\fluct{p_3}_w &= g_w && \forall w\in\mathcal{W}, \label{eqn:problem_3d_press_fluct}\\
p_d &= g_{dD} && \textrm{on } \Gamma_{dD},\; d=1,3, \label{eqn:problem_3d_dirichlet}\\
\delta_d \vc K_d \grad p_d \cdot \vc n &= g_{dN} && \textrm{on } \Gamma_{dN},\; d=1,3 \label{eqn:problem_3d_neumann}
\end{align}
\end{subequations}
where
\[ 
{\color{blue}
    \Sigma_w(\vc x) = \delta_3\sigma_w(\vc x)\big(\avg{p_3}_w(\vc x)-p_1(\vc x)\big) \quad \vc x \in \Omega^w_1.
} 
\label{eqn:1d_3d_sigma_source}
\]
}

One can see the dimensional coupling terms highlighted in the blue color.
In contrast to the 1d-2d case, $\Sigma_w(\vc x)$, $\vc x\in\Omega^w_1$, varies through the well domain.
% The term $\Sigma_w$, similarly to 1d-2d case, represents the flux from the aquifer $\Omega_3$ to the well $\Omega^w_1$.
% The flux is proportional to the pressure difference with the (variable) permeability coefficient $\sigma_w(\vc x)>0\;[s^{-1}]$ between $\Omega_3$ and $\Omega^w_1$.
% Analogically we can then read: the average outward flux from the aquifer at point $\vc x\in\Omega^w_1$
% in \eqref{eqn:problem_3d_flux} acts as a~positive source term in the well in \eqref{eqn:problem_3d_1d_source}.
% The last two equations are the boundary conditions, the first one \eqref{eqn:problem_3d_dirichlet} prescribes the boundary pressure $g_{dD}$, 
% the later one \eqref{eqn:problem_3d_neumann} sets the inward flux $g_{dN}$ in the same manner as in 1d-2d case.

