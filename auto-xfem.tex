
A~background research on the XFEM is provided in this chapter,
describing the fundamentals and giving an overview on the evolution of the XFEM.
Different enrichment methods and various types of enrichments are discussed.

\section{Basic Concept} \label{sec:soa_xfem}

The history and the early development of the XFEM is well written
in an overview article by Fries and Belytschko~\cite{fries_xfem_overview_2010}.
The main feature of this method is the extension of a~space of polynomial shape functions of the finite element
method with a~special function, that enables to better approximate some local effects. This is called \emph{enrichment}.
The special function (enrichment function) is often non-polynomial and describes discontinuity or singularity,
where polynomials leak accuracy.

The XFEM is mainly perceived as a~method for local enrichment, which means that the enrichment is applied only
in a~small subdomain -- several elements of the computational mesh close to the local phenomenon.
The XFEM solution with a~single enrichment is sought in the form
\begin{equation} \label{eqn:soa_xfem_standard_form}
  u(\bx) = \sum_{\alpha\in\mathcal{I}}a_\alpha v_\alpha(\bx)
    + \sum_{\alpha\in\mathcal{J}} b_{\alpha} N_\alpha(\bx)L(\bx)
\end{equation}
where $a_\alpha$ are the standard FE degrees of freedom and $b_{\alpha}$ are the degrees of freedom coming from
the enrichment, $v_\alpha$ are the standard FE basis shape functions. We denote the index sets $\mathcal{I}$ and
$\mathcal{J}$ that contain all indices of the standard and enriched degrees of freedom, respectively.
$L$ is the actual enrichment function and $N_\alpha$ is a~linear FE basis functions used as the partition of unity
$\sum_\alpha N_\alpha(\bx) = 1$.

\subsection{Global Enrichment Functions} \label{sec:global_enrichment}
Two different types of types of enrichments are discussed in this section -- discontinuity and singularity.
A~discontinuity of the quantity of interest can be strong or weak, depending on whether its value
or its derivative is discontinuous. The location of the discontinuity is often described by a~level set function,
which itself can be a~part of searched solution.

A~singularity, which is of the main interest in the thesis, is described by a~function with singularity concentrated in a~single point in 2d
or along a~line in 3d. It has symmetric radial character and thus it is often defined in polar (2d) or cylindrical (3d) coordinate system.
Typical global enrichment function are summarized e.g. in the Natarajan's thesis \cite{natarajan_enriched_2011}.


\subsection{Enrichment Zone} \label{sec:glob_enr_zone}
The choice of the enrichment zone, i.e. elements that are enriched, depends on the enrichment type.
In case of discontinuities, only the nodes of cut elements are enriched.

In case of singularity, the enrichment must well approximate the adjacent high gradients.
To this end, the enrichment zone is typically chosen as a~radial area in the vicinity of the singularity
of fixed (enrichment) radius, which is necessary to obtain optimal convergence rate, discussed e.g. in \cite{fries_corrected_2008}.
The optimal size of the enrichment zone is however not addressed, apart from \cite{gupta_enr_zone_2016}, where
an a~priori estimate for the enrichment radius dependent on $h$ is provided, although containing an unknown constant.
This matter is investigated in detail later in the thesis.


 
\section{Enrichment Methods} \label{sec:enrichment_methods}
In this section, different types of available enrichment methods are discussed.
In \cite{fries_corrected_2008, fries_xfem_overview_2010} the Corrected XFEM is introduced
and the so called \emph{blending} elements (not all element nodes are enriched) and \emph{reproducing} elements
(all element nodes enriched) are differentiated. The problem with lack of the partition of unity on the blending elements is addressed
and a~solution is suggested by means of the \emph{ramp function} $G(\bx)$ in \eqref{eqn:xfem_ramp}.
$G(\bx)$ is equal 1 on reproducing elements, is a~ramp in blending elements and diminishes elsewhere.
\begin{eqnarray} 
    L(\bx) &=& G(\bx) s_(\bx), \label{eqn:xfem_ramp}\\
    L_{\alpha}(\bx) &=& G(\bx) \left[s(\bx) - s(\bx_\alpha)\right]. \label{eqn:xfem_shift}
\end{eqnarray}
Further the \emph{shifted} enrichment is suggested leading to local enrichment function in \eqref{eqn:xfem_shift},
$\bx_\alpha$ being element nodes.

Next, the Stable Generalized FEM was developed in \cite{babuska_stable_2012, gupta_stable_2013}
and applied in an elastic fracture model. This method is supposed to solve the problem with ill-conditioning,
which is often met when using enrichments. The local enrichment function on an element $T$ is defined
\begin{equation} \label{eqn:sgfem_enrich}
    L_{\alpha}|_{T}(\bx) = s(\bx) - \pi_T (s)(\bx),\quad \pi_T (s)(\bx)=\sum_{\beta\in\mathcal{I}_N(T)} N_\beta(\bx) s(\bx_\beta). 
\end{equation} 
where the interpolation $\pi_T$ is built using the linear shape functions
associated with element nodes. It was later shown in \cite{zhang_robust_2016} that
further care must be taken for different enrichment types for this method to
yield optimal convergence and not to suffer with ill-conditioning independent of mesh -- discontinuity (or singularity) alignment.


\section{XFEM in Flow Problems on Meshes of Combined Dimensions} \label{sec:soa_xfem_combined}
There is much less to be found on the usage of the XFEM in the field of flow modeling, especially regarding the dimensions coupling,
than in mechanics. Apart from the references on various types of the XFEM in previous sections,
several references dealing with fractured porous media are provided, e.g. \cite{fumagalli_numerical_2012, schwenck_2015}.
However, no model suitable for a~singularity approximation using XFEM in groundwater flow i available.
