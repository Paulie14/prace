
\section{eXtended Finite Element Method (XFEM)} \label{sec:soa_xfem}

The main feature of this method is the extension of a~space of basis polynomial shape functions of the finite element
method with a~special function, that enables to better approximate some local effects. This is called \emph{enrichment}.
The special function (enrichment function) is often non-polynomial and describes discontinuity or singularity,
where polynomials leak accuracy. We can meet two terms: \emph{extrinsic enrichment}, which adds the enrichment
functions to a~basis and \emph{intrinsic enrichment}, which replaces some basis functions with the enrichment ones.
We are interested in mesh-based methods with extrinsic enrichment and we shall not discuss any mesh-free alternatives.

An overview article by Fries and Belytschko~\cite{fries_xfem_overview_2010} concludes the history and early development
of the XFEM. The origins from the Partition of Unity Method (Babu{\v s}ka and Melenk e.g. in~\cite{babuska_partition_1997}) and
the Generalized Finite Element Method (e.g. by Strouboulis in~\cite{strouboulis_generalized_2000}) 
and the early XFEM (e.g. by Belytschko and Mo{\"e}s in \cite{moes_finite_1999}) are commented in details there.

The most frequent application of the XFEM is in mechanics, where people take advantage of the XFEM especially in
modelling cracks growth without requiring remeshing the computational mesh in time. 
Discontinuous functions (Heaviside function, signum) are used to capture the jump in stress or strain at the crack,
singular functions are applied in the vicinity of the crack tips. The exact function for the selected enrichment
comes either directly from the nature of the phenomenon, or can be obtained as the solution of a local auxiliary
problem. This function is most often reffered as the \emph{global enrichment function}.

The XFEM is mainly perceived as a method for local enrichment, which means that the enrichment is applied only
in a~small subdomain -- several elements of the computational mesh close to the local phenomenon.
The XFEM solution with a~single enrichment is sought in the form
\begin{equation} \label{eqn:soa_xfem_standard_form}
  u(x) = \sum_{\alpha\in\mathcal{I}}a_\alpha v_\alpha(x)
    + \sum_{\alpha\in\mathcal{I}^e} b_{\alpha} \phi_{\alpha}(x),
\end{equation}
where $a_\alpha$ are the standard FE degrees of freedom and $b_{\alpha}$ are the degrees of freedom coming from
the enrichment. $v_\alpha(x)$ are the standard FE basis shape functions. We denote the index sets $\mathcal{I}$ and
$\mathcal{I}^e$ that contain all indices of the standard and enriched degrees of freedom, respectively.
The \emph{local enrichment function} $\phi_{\alpha}$ in~\eqref{eqn:soa_xfem_standard_form} is typically defined as
\begin{equation} \label{eqn:soa_xfem_enrich}
    \phi_{\alpha} = N_\alpha(x)L(x), \quad \alpha\in\mathcal{I}^e,
\end{equation}
where $L$ is the actual enrichment function and $N_\alpha$ is a function of the partition of unity
$\sum_\alpha N_\alpha(x) = 1$, creating the localised degrees of freedom for the enrichment.
Typically, $N_\alpha$ are linear FE basis functions, having support points at the mesh nodes, $N_\alpha(x_\alpha)=1$,
so one is often talking about enriching nodes and calling these \emph{enriched nodes}.
Different functions which make the partition of unity might be used, but linear functions are the most common.

% The index set $\mathcal{I}^e$ includes 
% all enriched nodes, i.e. the nodes that are in the vicinity of the effect, due to which we are applying the enrichment.

The most straightforward choice of $L(x)$ is the global enrichment function itself but it is not optimal. It can suffer with a lack of convergence, a large approximation error at some elements or an ill-conditioning of the linear system.
All these problems were intensively studied and lead to different XFEM methods:
the so called Corrected XFEM~\cite{fries_corrected_2008}
and the Stable Generalized Finite Element Method (SGFEM)~\cite{babuska_stable_2012, gupta_stable_2013}, which 
both became standard for XFEM in general. These are discussed in more details later on.

% sgfem - scaled condition number
% orthogonalization processes

Considering more than one enrichment is straightforward. The approximation will then change into the following form
\begin{equation} \label{eqn:soa_xfem_standard_form_mult1}
    u(x) = \sum_{\alpha\in\mathcal{I}}a_\alpha v_\alpha(x)
        + \sum_{e\in\mathcal{E}}\sum_{\alpha\in\mathcal{I}^e} b_{e\alpha} \phi_{e\alpha}(x)
\end{equation}
\begin{equation} \label{eqn:soa_xfem_standard_form_mult2}
    \phi_{e\alpha}(x) = N_\alpha(x)L_e(x), \quad e\in\mathcal{E},\; \alpha\in\mathcal{I}^e
\end{equation}
The enrichment $e$ is enriching all the nodes in the index set $\mathcal I^e$ with degrees of freedom $b_{e\alpha}$
corresponding to the local enrichment functions $\phi_{e\alpha}$. Note that there can be multiple functions $\phi_{e\alpha}$ with support on the same element. If these functions are close to each other in some measure in the approximation space, then the respective degrees of freedom can become linearly dependent which results in ill-conditioning of the linear system (SGFEM is mainly addressed to this issue).


% There are several types of enrichment, depending on the choice and form of $L_{\alpha}(x)$. We shall later
% discuss so called Corrected XFEM~\cite{fries_corrected_2008} and Stable Generalized Finite Element Method (SGFEM)~\cite{babuska_stable_2012, gupta_stable_2013}.
% The approximation can suffer large errors on the interface between enriched and unenriched
% elements (the enriched element on the interface are called blending elements), the linear system.


\par\noindent\rule{\textwidth}{1pt}

The details on the enrichment application and actual formulation of the methods will be described in detail in Section
\ref{sec:discretization}. There, a~model for a~quasi-3D aquifers-wells problem according to Gracie and Craig~\cite{gracie_modelling_2010,craig_using_2011}
is developed and the XFEM is used to deal with the singularity in pressure at the wells (0D-2D coupling).

Usage of the XFEM is not of course hassle free and gratis. The presence of mostly non-polynomial enrichment functions
has a~huge impact on the system matrix assembly, where these functions must be integrated accurately enough.
Adaptive integration strategies are being developed for a~given class of enrichment functions (e.g. \cite{ventura_fast_2009}), 
but no general, robust and accurate solution is available. We will discuss the integration in detail in Section \ref{sec:integration}.
Another effect on the system matrix is the increase of its condition number because the local enrichment functions can
become nearly linearly dependent. 
% It was thought that the SGFEM is the answer to that problem but it has been
% reported that it is still not a~general solution.

Another problem is the stability of the discretization. In contrast to mechanics, the PDEs describing flow
need to satisfy a~strict discrete inf-sup condition (inequation 3.14, p. 75 in \cite{brezzi_mixed_1991} for mixed formulation).
This inequation is of course affected by the enrichment, however it is discussed in the literature minimally.


\section{XFEM on Meshes of Combined Dimensions} \label{sec:soa_xfem_combined}
There is much less to be found on the usage of the XFEM in the field of flow modelling, especially regarding the dimensions coupling,
than in mechanics.
% Many researchers are using some kind of the extended finite element method, but the applications are mainly aimed
% at problems in mechanics. 
% One of the reasons is that the stability of the discrete mixed form is more sensitive 
% on the choice of the finite elements used in flow problems. 
Apart from the references on various types of the XFEM in previous paragraph, we can name some works regarding flow and the XFEM.
Extensive work has been done in the area of modelling 1D-2D fractured domains with mixed finite 
elements by D'Angelo, Fumagalli and Scotti~\cite{fumagalli_numerical_2012, dangelo_mixed_2012, fumagalli_efficient_2014}. 
In there, the XFEM is used to incorporate additional degrees of freedom of zero order Raviart-Thomas basis 
functions on the intersected elements to allow the discontinuity in velocity. The trick is, that the same
basis functions are used but with different support. The model is formulated in 1D-2D at the moment.

Another work regarding 1D-2D fractures is by Schwenck~\cite{schwenck_xfem-based_2015} (also in \cite{schwenck_2015}), 
where the primary formulation is used and among others, the fracture tip and fractures intersection is dealt with.

Next, there are several publications on multi-phase fluid flow governed by Navier-Stokes equations using the XFEM for
approximation on the interfaces in 2D (\cite{diez_stable_2013,sauerland_stable_2013}). However, these are
little bit further from our application.

A 1D-3D model for investigation of the transport of substances in a~human body has been developed by D'Angelo 
and Zunino et al. (e.g in~\cite{dangelo_coupling_2008,cattaneo_numerical_2015}), without usage of the XFEM.
Further, 1D-3D coupling for blood flow and mass transport in vascularized human tissue is
investigated by K{\" o}ppl in\cite{koppl_tum_2015}.
