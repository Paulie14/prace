\section{Numerical Tests}
\label{sec:pressure_results}

The model as described in this chapter was implemented in C++ language using the Deal II library \cite{bangerth_deal.ii_2007}, , version 8.0.
The model implementation and all the XFEM extensions to the standard FE code were suggested by the author.
The source is available on GitHub: \url{https://github.com/Paulie14/xfem_project}.

% In this section we suggest several numerical tests and provide comparison of different enrichments
% as suggested in Section \ref{sec:enrichment_func}.
% 
% The implementation has been done in C++ language using the Deal II~\cite{bangerth_deal.ii_2007}, version 8.0, 
% an open source finite element library. This library provides a well-documented code for high range of operations one needs
% for solving partial differential equation: mesh generation, quadratures, different scalar and vector finite elements,
% linear algebra, $h$ and $p$ adaptivity, error estimators, postprocessing and output to various formats.
% However, it does not support any enrichment techniques like XFEM in the version 8.0.0,
% and so it does not in the actual version 9.0.0, released in May, 2018, up to our best knowledge.
% 
% We use as much as possible of the relevant library code, although the enrichment functions, well intersections,
% distribution and handling of the enriched degrees of freedom, adaptive quadrature and some output routines
% must have been implemented on our own.
% The source is available on GitHub:
% \begin{center}\url{https://github.com/Paulie14/xfem_project}.\end{center}


% \subsection{Test Cases with Single Well} \label{sec:2d_results_single}
% \subsection{Conditioning of System Matrix} \label{sec:res_conditioning}
% \subsection{Enrichment Radius Estimation} \label{sec:enrichemnt_radius}
% \subsection{Test Cases with Multiple Wells} \label{sec:2d_results_multiple}

Two selected results are displayed in Graphs \ref{graph:convergence02}-\ref{graph:radius_conv_2} and \fig{fig:error_distribution_test5}.
In \gref{graph:convergence02}, different enrichment methods defined in \ref{sec:prim_discretization} are compared
on a~model including a~single well and nonzero source term.
Optimal convergence order is reached in all XFEMs.
\begin{graph}[!htb]
  \centering    
  \includegraphics[width=0.9\textwidth]{\results convergence02.pdf}
  \caption[Convergence comparison graph.]{Convergence of the $L_2$ norm of the approximation error.}
  \label{graph:convergence02}
\end{graph}
%
\begin{graph}[!htb]
  \centering    
  \includegraphics[width=0.8\textwidth]{\results radius_convergence_02.pdf}
  \caption[Optimal enrichment radius.]{Dependence of the error on the enrichment radius for different
  element sizes $h$.}
  \label{graph:radius_conv_2}
\end{graph}
The "FEM reg" data comes from the problem without the singularity solved by standard FEM and with optimal convergence order 2.0 for a~reference.
The standard XFEM displays larger error on blending elements as expected.
However, it together with the ramp function XFEM suffers ill-conditioning.
The standard FEM converge slowly with order 0.5.

The second \gref{graph:radius_conv_2} displays solution error depending on the enrichment radius
for different mesh refinements. It shows that the enrichment radius is worth expanding as long as
the error of the singular part of the solution is significant in comparison to the error if the regular part.
Further increasing of $R_w$ then does not bring increased accuracy. $R_o$ is the optimal enrichment radius
where these two error parts are well balanced -- theoretical estimate agrees with the numerical result.

\begin{figure}[!htb]
  \centering
  \subfloat[enrichment radius $R_w=0.6$]{\label{fig:error_2w_0-6} 
        \includegraphics[width=0.48\textwidth]{\results error_2w_0,6.pdf} }
  \subfloat[enrichment radius $R_w=2.0$]{\label{fig:error_2w_2-0} 
        \includegraphics[width=0.48\textwidth]{\results error_2w_2,0.pdf} }
  \caption[Error distribution in Test case 5.]{$L_2$ error distribution in Test case 5 for two different $R_w$,
  represented by green circles.}
  \label{fig:error_distribution_test5}
\end{figure}
In \fig{fig:error_distribution_test5} the approximation error is displayed
in case of 2 wells and sinusoidal source term along $x$ axis considered. The error of the regular part of the solution is apparent, 
the error of the singular part is inferior in case of larger enrichment radius.
In this problem, the ill-conditioning for shifted XFEM is reported, due
to multiple enrichment in the enrichment zone overlap.
