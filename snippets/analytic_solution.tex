

\documentclass{article}

\usepackage{hyperref}
\hypersetup{
  colorlinks   = true, %Colours links instead of ugly boxes
  urlcolor     = blue, %Colour for external hyperlinks
  linkcolor    = blue, %Colour of internal links
  citecolor   = red %Colour of citations
}

%% For including figures, graphicx.sty has been loaded in
%% elsarticle.cls. If you prefer to use the old commands
%% please give \usepackage{epsfig}
\usepackage{subfig}

\usepackage{graphicx}

%tables
%\usepackage{booktabs}

%% The amssymb package provides various useful mathematical symbols
\usepackage{amssymb}
\usepackage{amsmath}
%  \usepackage{amsfonts}
\usepackage{esint}


%% The amsthm package provides extended theorem environments
\usepackage{amsthm}
\newtheorem{theorem}{Theorem}
\newtheorem{thmproblem}{Problem}
\newtheorem{thmdef}{Definition}

%% The lineno packages adds line numbers. Start line numbering with
%% \begin{linenumbers}, end it with \end{linenumbers}. Or switch it on
%% for the whole article with \linenumbers.
%% \usepackage{lineno}


\newcommand{\fig}[1]{\hyperref[#1]{Figure \ref{#1}}}
\newcommand{\figpath}{../graphics/}


%\numberwithin{equation}{document}
%
\def\div{{\rm div}}
\def\Lapl{\Delta}
\def\grad{\nabla}
\def\supp{{\rm supp}}
\def\dist{{\rm dist}}
%\def\chset{\mathbbm{1}}
\def\chset{1}
%
\def\Tr{{\rm Tr}}
\def\to{\rightarrow}
\def\weakto{\rightharpoonup}
\def\imbed{\hookrightarrow}
\def\cimbed{\subset\subset}
\def\range{{\mathcal R}}
\def\leprox{\lesssim}
\def\argdot{{\hspace{0.18em}\cdot\hspace{0.18em}}}
\def\Distr{{\mathcal D}}
\def\calK{{\mathcal K}}
\def\FromTo{|\rightarrow}
\def\convol{\star}
\def\impl{\Rightarrow}
\DeclareMathOperator*{\esslim}{esslim}
\DeclareMathOperator*{\esssup}{ess\,supp}
\DeclareMathOperator{\ess}{ess}
\DeclareMathOperator{\osc}{osc}
\DeclareMathOperator{\curl}{curl}
\DeclareMathOperator{\cotg}{cotg}

%math:
\def\vc#1{\mathbf{\boldsymbol{#1}}}     % vector
\def\abs#1{\left|#1\right|}
\def\avg#1{\langle#1\rangle}
\def\d{\mathrm{d}}
\def\norm#1{\| #1 \|}
\def\abs#1{| #1 |}
\def\prtl{\partial}

\newcommand{\dd}{\; \mathrm{d}}
\newcommand{\R}{\mathbf{R}}
\newcommand{\bx}{\vc{x}}
\newcommand*\rfrac[2]{{}^{#1}\!/_{#2}}

\def\ol{\overline}


% just for our notes
\usepackage[usenames,dvipsnames]{color}   %colors
\newcommand{\noteJB}[1]{{\color{Blue} \textbf{JB: } \textit{#1}}}
\newcommand{\notePE}[1]{{\color{Orange} \textbf{PE: } \textit{#1}}}

\begin{document}2

\section{Analytic solution for an aquifer-well problem with a source term}

Let us have the following model:
\begin{thmproblem} \label{thm:problem} Aquifer-well model with a source term.
\begin{eqnarray}
-T \Lapl h &=& f \qquad \textrm{ in } \Omega \label{eqn:poisson}\\
h &=& h_D \qquad \textrm{ on } \Gamma_D, \label{eqn:dirichlet} \\
-T\grad h \cdot \vc{n}&=& \langle -T\grad h \cdot \vc{n}\rangle, \qquad \textrm{ on } \partial B_w, \label{eqn:well_edge_bc1}\\
  \langle -T\grad h \cdot \vc{n}\rangle &=& \sigma_w\left( \langle h \rangle - H_w \right) 
          \qquad \textrm{ on } \partial B_w, \label{eqn:well_edge_bc2} \\
-\int \limits_{\partial B_w} \sigma_w\left(\langle h\rangle - H_w\right) &=& c_w(H^1_w - H_w) \label{eqn:well_flow}\\
H^1_w &=& const. \label{eqn:well_top_bc}    
\end{eqnarray}
Domain $\Omega$ is a ring with an exterior boundary $\Gamma_D$ of radius $R$ and an interior boundary $\partial B_w$ 
made by a circular well $B_w=B(\rho_w,\vc{x}_w)$ of radius $\rho_w$ in its center, $\Omega = B(R,\vc{x}_w) - B_w$.

The Poisson equation \eqref{eqn:poisson} and the boundary conditions \eqref{eqn:dirichlet}-\eqref{eqn:well_edge_bc2}
govern the pressure inside the aquifer. The flow balance equation \eqref{eqn:well_flow} and the boundary condition
\eqref{eqn:well_top_bc} govern the pressure inside the well ($H^1_w$ is given).
\end{thmproblem}

We will now look for the analytic solution $[h, H_w]$.
Since we introduced the boundary condition \eqref{eqn:well_edge_bc1} on the well boundary,
we require the solution gradient to be a constant there. 
The constant is then is given by the difference of the average pressure along the well boundary 
and the pressure inside the well in \eqref{eqn:well_edge_bc2}.

Define a function
\begin{equation} \label{eqn:solution}
  h(\vc{x}) = a\log(r) + b + (r-\rho_w)^2 \sin(x),
\end{equation}
where $r = |\vc{x}-\vc{x}_w|$ and $\vc{x}=(x,y)$,
%(resp. $\vc{x_w}=(x_w,y_w)$) 
and let us require, that it satisfies equations \eqref{eqn:poisson}-\eqref{eqn:well_edge_bc1}.
To find the corresponding right hand side $f$, we need to apply the Laplace operator on \eqref{eqn:solution}.

Laplace of the first logarithmic part can be easily proved to be equal zero. 
Now, look at the second part. Since for the Laplace operator, it holds $\forall f,g\in C^2(\Omega)$ that
\footnote{
\begin{multline} \label{eqn:laplace_of_product_derivation}
  \Lapl(fg) = \div\cdot\grad(fg) = \div\cdot\begin{pmatrix} \frac{\partial f}{\partial x} g + f\frac{\partial g}{\partial x} \\ 
      \frac{\partial f}{\partial y} g + f\frac{\partial g}{\partial y}  \end{pmatrix} 
      = \frac{\partial^2 f}{\partial x^2} g + 2\frac{\partial f}{\partial x} \frac{\partial g}{\partial x}
        + f\frac{\partial^2 g}{\partial x^2} + \\
        + \frac{\partial^2 f}{\partial y^2}g + 2\frac{\partial f}{\partial y} \frac{\partial g}{\partial y}
        + \frac{\partial^2 g}{\partial y^2}
        = g\Lapl{f} + 2\grad f\grad g + f\Lapl{g}, \nonumber
\end{multline}
}
\begin{equation} \label{eqn:laplace_of_product}
  \Lapl(fg) = g\Lapl{f} + 2\grad f\grad g + f\Lapl{g},
\end{equation}
we need to compute the following:
%
\begin{eqnarray*}
  \grad (r-\rho_w)^2 &=& 2(r-\rho_w)\frac{\vc{r}}{r}, \\
  \grad (\sin(\omega x)) &=& \omega\cos(\omega x) \vc{e}_1, \\
  \div \cdot\left(\omega\cos(\omega x) \vc{e}_1\right) &=& -\omega^2\sin(\omega x),\\
  \div\cdot\left( 2(r-\rho_w)\frac{\vc{r}}{r} \right) &=& \frac{2}{r}(r-\rho_w) + 2,
\end{eqnarray*}
where $\vc{e}_1$ is the standard euclidean basis vector.
%
Another way to get the last term is using the Laplace operator in radial coordinates:
\begin{equation} \label{eqn:laplace_radial}
  \Lapl f(r) = \frac{1}{r}\frac{\partial}{\partial r}\left( r \frac{\partial f}{\partial r}\right)
    + \frac{1}{r^2}\frac{\partial^2 f}{\partial \theta^2} \nonumber
\end{equation}
which gives us for $f=(r-\rho_w)^2$, independent of $\theta$,
\begin{equation} \label{eqn:laplace_aa}
  \Lapl\big((r-\rho_w)^2\big) = \frac{2}{r}(r-\rho_w) + 2. \nonumber
\end{equation}
%
Using all derivatives derived above and substituing them in \eqref{eqn:laplace_of_product}, we obtain
\begin{equation} 
  \Lapl h(\vc{x}) = \left(4 - 2 \frac{\rho_w}{r} \right) \sin(\omega x)      
                    + 4(r-\rho_w)\frac{\vc{r}\cdot\vc{e}_1}{r} \omega\cos(\omega x)
                    - (r-\rho_w)^2\omega^2\sin(\omega x)
\end{equation}
%

Our next goal is to find constants $a,b\in\R$ ($a<0$ for an injection well and $a>0$ for a pumping well) 
using the following conditions, coming from \eqref{eqn:solution} and \eqref{eqn:well_flow},
\begin{eqnarray}
  a\log \rho_w + b &=& h  \qquad \textrm{ on the well edge,} \label{eqn:p_solution}\\
  a\log R + b &=& h_D  \qquad \textrm{ on } \Gamma_D, \label{eqn:p_dirichlet}\\
  \int \limits_{\partial B_w} \sigma_w\left(\langle h\rangle - H_w\right) &=& c_w(H_w - H^1_w) \qquad \textrm{ the flow into the aquifer.} \label{eqn:p_well_bc}
\end{eqnarray}
Further, we have on the well edge ($r=\rho_w$):
\begin{multline}
-\grad h \cdot \vc{n} = -\grad (a\log(r) + b) \cdot \vc{n} = - \frac{a}{r}\frac{\vc{r}\cdot\vc{n}}{r} = \frac{a}{\rho_w},
\end{multline}
where $\vc{r}\cdot\vc{n}=-r$, because these vectors are just opposite. Putting this in the equation
\eqref{eqn:well_edge_bc2} and using $\langle h \rangle=\langle a\log \rho_w + b \rangle = a\log \rho_w + b$,
and substituing for $b$ from \eqref{eqn:p_dirichlet}, we obtain
\begin{eqnarray}
  \frac{a}{\rho_w} &=& \sigma_w\left( a\log \rho_w + h_D -a\log R - H_w \right), \nonumber\\
  a &=& \frac{\rho_w\sigma_w(h_D-H_w)}{1-t},  \label{eqn:a_const}
\end{eqnarray}
where we denote an auxiliary constant $t=\rho_w\sigma_w\log\left(\frac{\rho_w}{R}\right)$.

Next, we combine \eqref{eqn:p_solution}-\eqref{eqn:p_well_bc} to derive $H_w$:
\begin{equation}
  \bar{\sigma}_w\left( a\log \rho_w + h_D -a\log R - H_w \right) = c_w(H_w-H^1_w), \nonumber
\end{equation}
where $\bar{\sigma}_w=|\partial B_w|\sigma_w$. Substituing \eqref{eqn:a_const} for $a$, we obtain
\begin{eqnarray}
  \left(\bar{\sigma}_w + c_w + \bar{\sigma}_w\frac{t}{1-t}\right)H_w &=& c_w H^1_w + \bar{\sigma}_w h_D + \bar{\sigma}_w h_D\frac{t}{1-t}, \nonumber\\
  H_w &=& \frac{\frac{c_w}{\bar{\sigma}_w}H^1_w + \frac{1}{1-t}h_D}{\frac{c_w}{\bar{\sigma}_w} + \frac{1}{1-t}}. \label{eqn:well_pressure}
\end{eqnarray}

Finally, putting \eqref{eqn:a_const} and \eqref{eqn:well_pressure} together, we get the constant $a$:
\begin{eqnarray}
  a &=& \frac{\rho_w\sigma_w}{1-t}\left[ \frac{\frac{c_w}{\bar{\sigma}_w}(h_D-H^1_w)}{\frac{c_w}{\bar{\sigma}_w}+\frac{1}{1-t}} \right], \nonumber\\
  a &=& \frac{\rho_w\sigma_w c_w (h_D-H^1_w)}{c_w(1-t) + |\partial B_w|\sigma_w}. \label{eqn:a_final}
\end{eqnarray}


\begin{thmdef}
The solution $[h,H_w]$ of the problem \ref{thm:problem}
with the source term
\begin{equation} 
  f = -\left(4 - 2 \frac{\rho_w}{r} \right) \sin(\omega x)      
                    - 4(r-\rho_w)\frac{\vc{r}\cdot\vc{e}_1}{r} \omega\cos(\omega x)
                    + (r-\rho_w)^2\omega^2\sin(\omega x)
\end{equation}
on an aquifer of radius $R$ with a single well of radius $\rho_w$ placed in the center, 
satisfying equations \eqref{eqn:poisson}-\eqref{eqn:well_top_bc}, is
\begin{eqnarray}
  h(\vc{x}) &=& a\log(r) + b + (r-\rho_w)^2 \sin(x), \\
  H_w &=& \frac{c_w(1-t) H^1_w + \bar{\sigma}_w h_D}{c_w(1-t) + \bar{\sigma}_w},
\end{eqnarray}
with
\begin{eqnarray}
  a &=& \frac{\rho_w\sigma_w c_w (h_D-H^1_w)}{c_w(1-t) + |\partial B_w|\sigma_w}, \nonumber \\
  b &=& h_D - a\log R, \nonumber \\
  t &=& \rho_w\sigma_w\log\left(\frac{\rho_w}{R}\right). \nonumber
\end{eqnarray}
\end{thmdef}



 \bibliographystyle{abbrv} 
 \bibliography{../citace.bib}
\end{document}
