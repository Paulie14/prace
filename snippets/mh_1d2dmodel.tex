

\documentclass{article}

\usepackage{hyperref}
\hypersetup{
  colorlinks   = true, %Colours links instead of ugly boxes
  urlcolor     = blue, %Colour for external hyperlinks
  linkcolor    = blue, %Colour of internal links
  citecolor   = red %Colour of citations
}

%% For including figures, graphicx.sty has been loaded in
%% elsarticle.cls. If you prefer to use the old commands
%% please give \usepackage{epsfig}
\usepackage{subfig}

\usepackage{graphicx}

%tables
%\usepackage{booktabs}

%% The amssymb package provides various useful mathematical symbols
\usepackage{amssymb}
\usepackage{amsmath}
%  \usepackage{amsfonts}
\usepackage{esint}


%% The amsthm package provides extended theorem environments
\usepackage{amsthm}
\newtheorem{theorem}{Theorem}
\newtheorem{thmproblem}{Problem}
\newtheorem{thmdef}{Definition}

%% The lineno packages adds line numbers. Start line numbering with
%% \begin{linenumbers}, end it with \end{linenumbers}. Or switch it on
%% for the whole article with \linenumbers.
%% \usepackage{lineno}


\newcommand{\fig}[1]{\hyperref[#1]{Figure \ref{#1}}}
\newcommand{\figpath}{../graphics/}


%\numberwithin{equation}{document}
%
\def\div{{\rm div}}
\def\Lapl{\Delta}
\def\grad{\nabla}
\def\supp{{\rm supp}}
\def\dist{{\rm dist}}
%\def\chset{\mathbbm{1}}
\def\chset{1}
%
\def\Tr{{\rm Tr}}
\def\to{\rightarrow}
\def\weakto{\rightharpoonup}
\def\imbed{\hookrightarrow}
\def\cimbed{\subset\subset}
\def\range{{\mathcal R}}
\def\leprox{\lesssim}
\def\argdot{{\hspace{0.18em}\cdot\hspace{0.18em}}}
\def\Distr{{\mathcal D}}
\def\calK{{\mathcal K}}
\def\FromTo{|\rightarrow}
\def\convol{\star}
\def\impl{\Rightarrow}
\DeclareMathOperator*{\esslim}{esslim}
\DeclareMathOperator*{\esssup}{ess\,supp}
\DeclareMathOperator{\ess}{ess}
\DeclareMathOperator{\osc}{osc}
\DeclareMathOperator{\curl}{curl}
\DeclareMathOperator{\cotg}{cotg}

%math:
\def\vc#1{\mathbf{\boldsymbol{#1}}}     % vector
\def\abs#1{\left|#1\right|}
\def\avg#1{\langle#1\rangle}
\def\d{\mathrm{d}}
\def\norm#1{\| #1 \|}
\def\abs#1{| #1 |}
\def\prtl{\partial}

\newcommand{\dd}{\; \mathrm{d}}
\newcommand{\R}{\mathbf{R}}
\newcommand{\bx}{\vc{x}}
\newcommand*\rfrac[2]{{}^{#1}\!/_{#2}}

\def\ol{\overline}


% just for our notes
\usepackage[usenames,dvipsnames]{color}   %colors
\newcommand{\noteJB}[1]{{\color{Blue} \textbf{JB: } \textit{#1}}}
\newcommand{\notePE}[1]{{\color{Orange} \textbf{PE: } \textit{#1}}}
\usepackage{enumitem}

\begin{document}

\section{Coupled 1d-2d model}

In ambient 3d space, induces point sources (0d-2d problem).
  
  Mixed form:
  \begin{eqnarray}
\vc K_d^{-1}\vc u_d + \nabla p_d &=& 0 \qquad \textrm{in } \Omega_d,\; d=1,2  \\
\nabla \cdot \vc u_2 &=& f_2 \qquad \textrm{in } \Omega_2\\
\nabla \cdot \vc u_1 &=& f_1 + {\color{blue}\sigma_w(\avg{p_2}_w-p_1)}\qquad \textrm{in } \Omega_1\\
\color{blue}\avg{\vc u_2 \cdot \vc n}_w &=& \color{blue}\sigma_w (\avg{p_2}_w - p_{1}) \qquad \textrm{on } \Gamma_w\\
p_d &=& g_{dD} \qquad \textrm{on } \Gamma_{dD},\; d=1,2 \\
\vc u_d \cdot \vc n &=& g_{dN} \qquad \textrm{on } \Gamma_{dN},\; d=1,2
  \end{eqnarray}
  
  
  We shall use the following FE discrete spaces:
  $\vc u_d$ (velocity) - discontinuous $RT_0$ elements + \textcolor{blue}{\bf singular enrichment}\\
  $p_d$ (pressure head)- piecewise constant, $P_0$ on elements \\
  $\lambda_d$ (pressure head traces) - piecewise constant, $P_0$ on edges \\
  \textcolor{blue}{$\lambda_w$ (pressure head traces) - piecewise constant, $P_0$ on $\Gamma_w$}
  
  
  Mixed-Hybrid form (saddle point problem): 
\begin{align}
        \label{Saddle1}
 a(\vc{u},\vc \psi) + b(\vc\psi, p) &= \langle G, \vc \psi\rangle &&\forall 
\vc\psi\in V,\\
        \label{Saddle2}
 b(\vc{u}, \phi) - c(p,\phi) &= \langle F, \phi \rangle &&\forall \phi \in Q,
\end{align}
where bilinear forms on the left-hand side are
\begin{align*}
 a(\vc u, \vc \psi)=&\sum_{d=1,2}\sum_{i\in \mathcal T_d} \int_{T_d^i}
   \frac{1}{\delta_d} \vc u_d \vc K_d^{-1} \vc \psi_d,\\
%
 b(\vc \psi, p)=&\sum_{d=1,2}\sum_{i\in \mathcal T_d} 
        \left(
        \int_{T_d^i} -p_d\, \div\,\vc \psi_d
        +\int_{\prtl T_d^i}
                 \lambda_d (\vc \psi_d \cdot \vc n)
        \right) \\
        &+\color{blue}{\int_{\Gamma_w}
                 \lambda_w \avg{\vc \psi_2 \cdot \vc n}_w},\\
%
\color{blue}
 c(p,\phi) =& \color{blue}
          \int_{\Gamma_w}
               \sigma_w(p_{1}-\lambda_w)(\phi_{1}-\mu_w),
\end{align*}
where $p=(p_d,\lambda_d, \lambda_w)$, $\phi=(\phi_d,\mu_d, \mu_w)$.

In the Dirichlet problem (setting constant pressure in 1d and supposing only 2d mesh) we have $c(p,\phi)=0$.

We use SGFEM like approach to enrich velocity.
The enriched velocity has the form:
\[\sum \limits_{w\in\mathcal{W}} b_w \vc L_w, \]
where every well $w$ is enriched exactly by one DoF $b_w$.
\[
  \vc L_w(\vc x) = \textcolor{red}{\vc s_w(\vc x)} - \sum \limits_{j=1}^3 \textcolor{ForestGreen}{z_j} \vc\psi_j(\vc x),
  \quad \textcolor{ForestGreen}{z_j=\int_{E_j} \vc s_w(\vc x)\cdot \vc n_j},
  \quad \textcolor{red}{\vc s_w(\vc x) = -\frac{1}{2\pi} \frac{\vc r_w}{r_w^2}}
\]

Accurate integration on sides is needed to compute for coefficients $\color{ForestGreen}z_j$ of the $RT_0$ interpolation of $s_w$.
    
The local velocity enrichment function has the following important properties:
  \begin{enumerate}[label=\alph*)]
%     \setlength{\itemsep}{5pt}
    \item
      $\displaystyle \int_T \div\, \vc L_w = \begin{cases}
        0 & \forall T, T \cap \Gamma_w = \emptyset \\
        -\sum \limits_{j=1}^3 z_j, & \forall T, T \cap \Gamma_w \neq \emptyset
    \end{cases}$
    \item $\displaystyle \int_{E_i} \vc L_w \cdot \vc n_i = 0, \quad \int_{\Gamma_w} \vc \psi \cdot \vc n_w = 0$
    \item $\displaystyle \int_{\Gamma_w} \vc L_w \cdot \vc n_w = 1$
  \end{enumerate}

\begin{equation}
  \begin{pmatrix}
     u_1 & u_2 & u_3 & b_w & p & \lambda & \lambda_w
  \end{pmatrix}
\end{equation}

\section{Branch PE xfem innovations}

\subsection{Output Mesh and quadrature refinement}
The refinement of elements for the output mesh is done using edge splitting technique (so called red refinement).
Since we use this only for better output visualization of non-polynomial solutions, we do not
care for existence of hanging nodes.

In 2D case, it is straightforward process: find the midpoints of all sides, connect them and generate 4 triangles.
These triangles are congruent and have equal surface areas.

On the other hand, the 3D case is more complicated. After splitting the edges, we obtain 4 tetrahedra at the vertices
of the original one. The octahedron that remains in the middle can be subdivided according to one of its three diagonals.
Only the choice of the shortest octahedron diagonal leads to a regular tetrahedra decomposition.
This algorithm originally comes from Bey~\cite{}.

% Bey's algorithm (red refinement of tetrahedron):
%p.29 https://www5.in.tum.de/pub/Joshi2016_Thesis.pdf
%p.108 http://www.bcamath.org/documentos_public/archivos/publicaciones/sergey_book.pdf
%https://www.math.uci.edu/~chenlong/iFEM/doc/html/uniformrefine3doc.html#1
%J. Bey. Simplicial grid refinement: on Freudenthal's algorithm and the optimal number of congruence classes. Numer. Math. 85(1):1--29, 2000. p11 Algorithm: RedRefinement3D.
%p.4 http://www.vis.uni-stuttgart.de/uploads/tx_vispublications/vis97-grosso.pdf


\subsection{SGFEM enrichment}
The SGFEM enriched shape function on an element $T$ has the form
\[
  \vc L_w(\vc x) = \textcolor{red}{\vc s_w(\vc x)} - \sum \limits_{j=1}^{n_E} \textcolor{ForestGreen}{z_j} \vc\psi_j(\vc x),
  \quad \textcolor{ForestGreen}{z_j=\int_{E_j} \vc s_w(\vc x)\cdot \vc n_j},
  \quad \textcolor{red}{\vc s_w(\vc x) = -\frac{1}{D_T} \frac{\vc r_w}{r_w^2}},
\]
where $n_E$ is the number of sides in 2D, or faces in 3D, $z_j$ is the flux of the global enrichment function $\vc s_w$ over side or face $E_j$ of $T$
and $D_T$ is the so called effective surface. $D_T$ is a portion of the surface of the singularity (circle $2\pi\rho_w$, cylinder $2\pi\rho_w v$)
that communicates with the element $T$. This area is approximated using the quadrature points on the singularity surface belonging to $T$.

The flux $z_j$ is computed using adaptive quadrature on sides and faces. The quadrature rules are analogical to the rules
for adaptive quadrature in elements. These values are computed only once and saved in a vector for further usage (assembly, error computation, output).


\end{document}
