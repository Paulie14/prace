\begin{abstractCZ}
Tato doktorská práce je zaměřena na řešení problému proudění podzemní vody v~porézním prostředí, které
je ovlivněno přítomností vrtů či studní. Model proudění je sestaven na základě konceptu redukce dimenzí,
který je hojně využíván při modelování rozpukaného porézního přostředí, především granitů.
Vrty jsou modelovány jako 1d objekty, které protínají blok horniny. 
Propojení těchto domén v redukovaném modelu způsobuje singularity v~řešení v~okolí vrtů.
Vrty i~porézní médium jsou síťovány nezávisle na sobě což vede k výpočetním sítím kombinujícím elementy
různých dimenzí.

Jádrem doktorské práce je pak vývoj specializované metody konečných prvků pro výše popsaný model. 
Pro umožnění propojení sítí různých dimenzí a pro zpřesnění aproximace singularit v okolí vrtů je 
použita rozšířená metoda konečných prvků (XFEM), v~rámci níž jsou navrženy nové typy obohacení
konečně-prvkové aproximace.
Metoda XFEM je nejprve aplikována v~modelu pro tlak, dále je navrženo obohacení pro rychlost a metoda
je použita ve smíšeném modelu, jehož řešením jsou rychlost i~tlak.

Doktorská práce se dále detailně věnuje numerickým aspektům v metodě XFEM, a~to především 
adaptivním kvadraturám, volbě velikosti obohacené oblasti nebo podmíněnosti výsledného lineárního systému.
Vlastnosti navržené XFEM metody a~optimální konvergence jsou ověřeny na sérii numerických experimentů.
Praktickým výstupem doktorské práce je implementace metody XFEM jako součásti open-source softwaru Flow123d.

\end{abstractCZ}

\begin{keywordsCZ}
Rozšířená metoda konečných prvků (XFEM), singularita, sítě kombinovaných dimenzí,
Darcyho proudění, rozpukané porézní prostředí
\end{keywordsCZ}

\vspace{2cm}

\begin{abstractEN}
In this doctoral thesis, a~model of groundwater flow in porous media intersected with wells (boreholes, channels) is developed.
The model is motivated by the reduced dimension approach which is being often used in fractured porous media problems, especially in granite rocks.
The wells are modeled as lower dimensional 1d objects and they intersect the surrounding bulk rock domains.
The coupling between the wells and the rock then causes a~singular behaviour of the solution in the higher dimensional domains
in the vicinity of the cross-sections. The domains are discretized separately resulting in an~incompatible mesh of combined dimensions.

The core contribution of this work is in the developement of a~specialized finite element method for such model.
Different Extended finite element methods (XFEM) are studied and new enrichments are suggested to better
approximate the singularities and to enable the coupling of the wells with the higher dimensional domains.
At first the XFEM is applied in a~pressure model, later an enrichment for velocity
is suggested and the XFEM is used in a~mixed model, solving both velocity and pressure.

Different numerical aspects of the XFEM is studied in details, including an adaptive quadrature strategy,
a~proper choice of the enrichment zone or a~conditioning of the resulting linear system.
The properties of the suggested XFEM are validated on a~set of numerical tests and the optimal convergence
rate is demonstrated. The XFEM is implemented as a~part of the open-source software Flow123d.


% Darcy’s law
% reduced dimension approach
% meshes of combined dimensions
% singularity
% Extended finite element method (XFEM)
% mixed form
% implemented in software Flow123d

\end{abstractEN}

\begin{keywordsEN}
Extended Finite Element Method (XFEM), Singularity, Meshes of combined dimensions,
Darcy flow, Fractured porous media
\end{keywordsEN}
