In this chapter, the XFEM with different enrichment techniques is applied on a~pressure well-aquifer model
in 1d-2d case. The problem for the primary unknown is defined, its weak form is derived and
the existence of a~unique discrete solution is proved.
Several important numerical aspects are inspected, mainly an adaptive quadrature and
a~choice of an enrichment radius with respect to the convergence of the methods.

\section{Coupled 1d-2d Model (Primary Weak Form)}
\label{sec:primary_form}

Problem \ref{thm:problem_2d} is rearranged, velocity $\vc u_d$ is substituted from the Darcy's law and the problem is solved
with pressure $p_d$ as the primary unknown quantity.

Considering standard Sobolev space $H^1_0(\Omega_d)=\big\{q_d\in H^1(\Omega_d); q_d|_{\Gamma_{dD}}=0\big\}$,
and taking the Dirichlet boundary condition into account,
the trial space $V$ and the test space $V_0$ are defined
\begin{eqnarray}
  V &=& H^1(\Omega_1)\times H^1(\Omega_2), \label{eqn:prim_weak_space_V}\\
  V_0 &=& H^1_0(\Omega_1)\times V^1_0(\Omega^1_2) \times\cdots\times V^M_0(\Omega^M_2), \label{eqn:prim_weak_space_V0}
\end{eqnarray}
with
% \textrm{with }
\begin{equation}
  V^m_0(\Omega^m_2) = \big\{q_2\in H^1_0(\Omega^m_2):\, \fluct{q_2}^m_w=0,\,\forall w\in\mathcal{W}\big\}
  \quad \forall m\in\mathcal{M}. \label{eqn:prim_weak_space_Vm0}
\end{equation}
%
% The spaces $V,\,V_0$ are equipped with the norm
% \begin{equation}
%     \norm{q}_V = \big( \norm{q_1}^2_{H^1(\Omega_1)} + \norm{q_2}^2_{H^1(\Omega_2)} \big)^{1/2}.
% \end{equation}
The weak solution $p=[p_1,p_2]\in V$ and the test functions $q=[q_1,q_2]\in V_0$ are denoted
and the weak problem is defined
\thmproblem{ \label{thm:problem_2d_prim_weak}
Find $p=p_0 + p_w + p_D \;\in V$ such that
\begin{align}
a(p_0, q) &= l(q) - a(p_w, q) - a(p_D, q) && \forall q\in V_0
\end{align}
where
\begin{multline}
  a(p,q) =
  \int_{\Omega_2} \delta_2 \vc K_2 \grad p_2 \cdot \grad q_2 \dd\bx
  + \int_{\Omega_1} \delta_1 \vc K_1 \grad p_1 \cdot \grad q_1 \dd\bx \\
  + \sum_{\substack{w\in \mathcal{W} \\ m\in \mathcal{M}}} \abs{\Gamma^m_w} \delta_2 \sigma^m_w (\avg{p_2}^m_w - p_1(\vc x^m_w)) (\avg{q_2}^m_w - q_1(\vc x^m_w))
\end{multline}
\vspace{-10pt}
\begin{multline}
  l(q) = \int_{\Omega_2} \delta_2 f_2 q_2 \dd\bx + \int_{\Omega_1} \delta_1 f_1 q_1 \dd\bx
  + \int_{\Gamma_{2N}} g_{2N}q_2 \dd s + \int_{\Gamma_{1N}} g_{1N}q_1 \dd s
\end{multline}
for $p_0\in V_0$ and given $f_d\in L_2(\Omega_d)$ and $g_{dN}\in L_2(\Gamma_{dN})$, $d=1,2$,
while $p_w, p_D\in V$ are functions chosen such that they satisfy Dirichlet boundary conditions.
\vspace{5pt}
}

For the solution $p$ of Problem \ref{thm:problem_2d_prim_weak} to be unique it is shown,
that it is enough to fix the pressure at the top of a single well (e.g. a pumping well
where the pressure can be measured), possibly with Neumann boundary condition
at the rest of the boundary.


\section{Discretization}
\label{sec:prim_discretization}

Problem \ref{thm:problem_2d_prim_weak} is discretized using the linear FE and the XFEM enrichment
in the aquifers domains in the vicinity of the singularities.
The 2d part of the discrete solution is searched in the form
\begin{equation} \label{eqn:prim_xfem_standard_form}
  p_{2h}(\bx) = \sum_{\alpha\in\mathcal{I}_{2N}}a_\alpha N_{2\alpha}(\bx)
    + \sum_{w\in\mathcal{W}} \sum_{\alpha\in\mathcal{J}^w_{2N}} b_{\alpha w} N_{2\alpha}(\bx) L_{\alpha w}(\bx)
\end{equation}
where $\mathcal{J}^w_{2N}\subset\mathcal{I}_{2N}$ denotes the indices of enriched nodes in $\mathcal{T}_2$ 
by the well $w$, and the functions $N_{2\alpha}L_{\alpha w}$ are the local enrichment functions,
while $N_{2\alpha}$, the linear FE shape functions, are playing the role of PU.

The local enrichment is chosen according to the particular choice of the enrichment type in Section \ref{sec:enrichment_methods}.
The singular global enrichment function is defined below.

% \subsection{Enrichment Function}
% \label{sec:enrichment_func}
 The global enrichment function can be obtained from the solution of a~local problem on the neighborhood of the well.
 In this case, it is a~logarithmic function radially symmetric around the well cross-section $\bx_w=[x_w, y_w]$:
\begin{equation} \label{eqn:enrich_func}
s_w(\bx) = 
  \begin{cases}
  \log(r_w(\bx)) & r_w > \rho_w,\\
  \log(\rho_w) & r_w \le \rho_w,\\
  \end{cases}
\end{equation}
where $r_w(\bx) = \|\bx - \bx_w\|= \sqrt{(x-x_w)^2+(y-y_w)^2}$ is the distance function.

The XFEM and the SGFEM apply the enrichment functions only locally. 
Due to the radial character, it is natural to consider a~circular enriched domain $Z_w = B_{R_w}(\vc x_w)$
of the well $w$ given by the enrichment radius $R_w$.
Four different enrichments are defined according to Section Section \ref{sec:enrichment_methods}:
\emph{standard XFEM}, \emph{ramp function XFEM}, \emph{shifted XFEM} and \emph{SGFEM}; which are later compared.


% \subsection{Integration on Enriched Elements}
% \label{sec:integration}
In order to compute the entries of the system matrix, %\eqref{eqn:s_entry} and \eqref{eqn:r_entry} 
the expressions containing the enrichment functions have to be integrated accurately. 
% These of course can be non-polynomial, like they are 
% in our case. The standard quadrature rules are not appropriate any more, for they are constructed to integrate 
% precisely only polynomials up to a~given degree. The higher requirements on the integration precision
% are the price for using enrichment functions and a~coarse mesh.
% 
There are two aspects which the integration must handle properly:
\begin{itemize}
  \item the steep gradient of enrichment shape functions in the vicinity of the singularity,
  \item the singularity cut-off edge geometry.
\end{itemize}

The instability in the integration in \cite{gracie_modelling_2010} is investigated.
An asymptotic analysis of the integration error is presented and new adaptive quadrature rules are defined.
Additionally, an~adaptive quadrature in polar coordinates is suggested.

\section{Single Aquifer Analytic Solution} \label{sec:prim_analytic_solution}
A~pseudo-analytic solution is defined in this section. Considering multiple singularities in the domain,
the solution can be obtained using the superposition principle.
The solution is split into singular and regular part and searched in the form
\begin{equation}
p_2 = p_{sin} + p_{reg} = \sum\limits_{j\in\mathcal{W}} a_j\log r_j + p_{reg}. \label{eqn:2d_press_sol_mult}\\
\end{equation}
The averages $\avg{\cdot}_w$ are evaluated numerically.
