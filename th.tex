\documentclass[bibliography=totocnumbered,dvipsnames,FM,Dis,EN]{tulthesis}

% \setcounter{secnumdepth}{3}

\usepackage[czech, english]{babel}
\usepackage[utf8]{inputenc}

% Adds bibliography to tocbibind
% options:
% none - switch off everything
% numbib - adds number to bibliography
% https://ctan.org/pkg/tocbibind
\usepackage[nottoc]{tocbibind}

% \setlength{\parindent}{3em}
% \setlength{\parskip}{0.15cm}

\DeclareUnicodeCharacter{00A0}{~}

%% The lineno packages adds line numbers. Start line numbering with
%% \begin{linenumbers}, end it with \end{linenumbers}. Or switch it on
%% for the whole article with \linenumbers.
%% \usepackage{lineno}

% grahpics
\usepackage{graphicx}
\usepackage{subfig}
\newcommand{\fig}[1]{Figure \hyperref[#1]{\ref{#1}}}
\newcommand{\gref}[1]{Graph \hyperref[#1]{\ref{#1}}}
\newcommand{\figpath}{figures/}
\newcommand{\results}{results/}

\DeclareCaptionType[name={Graph}]{graph}
% \newcommand{\listofgarphs}{\tocfile{\listfigures}{graph}}

% živé odkazy v PDF
% \usepackage{hyperref}
% \hypersetup{colorlinks=true, linkcolor=tul, urlcolor=tul, citecolor=tul}
\hypersetup{colorlinks=false, linkcolor=tul, urlcolor=tul, citecolor=tul}
\hypersetup{pdftitle={Extended Finite Element Methods on Meshes of Combined Dimensions}}

%tables
\usepackage{booktabs}

% enumeration by alphabet
\usepackage[inline]{enumitem}

\usepackage{url}

% intersections chapter
% algorithms
\usepackage[vlined, linesnumbered, ruled, algochapter]{algorithm2e}
\newcommand{\listofalgorithmes}{\tocfile{\listalgorithmcfname}{loa}}
% \newcommand{\algo}[1]{\hyperref[#1]{Algorithm \ref{#1}}}
\newcommand{\algo}[1]{Algorithm \ref{#1}}
% \newcommand{\mykwsty}[1]{\textcolor{blue}{\emph{#1}}}
\newcommand{\myfuncsty}[1]{\textbf{\texttt{#1}}}
% \newcommand{\myvarsty}[1]{\textcolor{GoogleGreen}{\texttt{#1}}}
% \SetKwSty{mykwsty}
% \SetArgSty{textbf}
\SetFuncSty{myfuncsty}
% \SetDataSty{myvarsty}
\SetKwFunction{setlinks}{set links}
\SetKwFunction{vertexfaces}{vertex faces}
\SetKwFunction{edgefaces}{edge faces}
\SetKwFunction{intersectionA}{intersection12}
\SetKwFunction{intersectionB}{intersection13}
\SetKw{and}{and}
\SetKw{continue}{continue}
\SetKw{break}{break}
% \SetKwFunction{cost}{cost}
% \SetKw{Of}{of}
% \SetKwData{neighbors}{neighbors}
% \SetKwData{edge}{edge}


% for codes we use different labeling
\newenvironment{code}[1][htb]
  {\renewcommand{\algorithmcfname}{Code}% Update algorithm name
   \begin{algorithm}[#1]%
  }{\end{algorithm}}
  
%\usepackage{float}
%\newfloat{algorithm}{t}{lop}
\usepackage{array}
\newcommand{\plucker}{Pl\"{u}cker }
\newcommand{\nface}{$n$-face }
\newcommand{\nfaces}{$n$-faces }
\newcommand{\ngh}{NGH }
\newcommand{\figpathins}{figures/intersections/}

\pdfsuppresswarningpagegroup=1

% deklarace pro titulní stránku
\TULthesisType{Disertační práce}{Doctoral Thesis}
\TULtitle{Rozšířené metody konečných prvků pro aproximaci singularit}{Extended Finite Element Methods for Approximation of Singularities}
% Extended finite elements for mixed-hybrid model of Darcy flow
\TULprogramme{P3901}{Aplikované vědy v~inženýrství}{Applied Sciences in Engineering}
\TULbranch{3901V055}{Aplikované vědy v~inženýrství}{Scientific Engineering (Mathematical Modelling)}
\TULauthor{Ing. Pavel Exner}
\TULsupervisor{Mgr. Jan B{\v r}ezina, Ph.D.}
\TULyear{}


% just for our notes
% \usepackage[usenames,dvipsnames]{color}   %colors
\newcommand{\noteJB}[1]{{\color{Blue} \textbf{JB: } \textit{#1}}}
\newcommand{\notePE}[1]{{\color{Orange} \textbf{PE: } \textit{#1}}}

%
%*************************************************************************************************************
%
%                                                   DOCUMENT
%
%*************************************************************************************************************
%
% Title Page
% Copyright Page
% Abstract
% Dedication, Acknowledgements, and Preface (each optional)
% Table of Contents, with page numbers
% List of Tables, List of Figures, or List of Illustrations, with titles and page numbers (if applicable)
% List of Abbreviations (if applicable)
% List of Symbols (if applicable)
% Chapters, including:
% Introduction, if any
% Main body, with consistent subheadings as appropriate
% Appendices (if applicable)
% Endnotes (if applicable)
% References (see section on References for options)

\begin{document}


\ThesisStart{male}
% \ThesisTitle{CZ}
% \ThesisTitle{EN}

\input th-abstract.tex
\clearpage


\thispagestyle{empty}
% \vspace*{\fill}
\vspace*{2in}
\begin{center}
% \begin{flushright}
\emph{
% Dedicated to the memory of my late father, \\
% Zden{\v e}k Exner. \\
Dedicated to the memory of my late father, Zden{\v e}k Exner,\\
who I still turn to for a guidance through my life,\\
even though we cannot share the days in this world anymore.
}
% \end{flushright}
\end{center}
\vspace*{\fill}
\clearpage


\begin{acknowledgement}
\thispagestyle{empty}
This work woud not have been created without the great support of my colleagues, family and friends
and therefore I would like to express the deepest appreciation to all of them.

% First and foremost, I 

I also greatly appreciate the support from my colleague Mgr. Jan Stebel, Ph.D.
who never refused to answer to any of my questions on various subjects.


% I want to thank someone
% I would (particularly) like to thank someone
% I would like to express my gratitude to someone
% I would like to express the deepest appreciation to someone
% My deepest [heartfelt] appreciation goes to someone
% I would like to show my greatest appreciation to someone
% Special thanks (also) to someone
% I would like to offer my special thanks to someone
% I owe a very important debt to someone
% I owe my deepest gratitude to someone
% 
% Acknowledging comments and support
% Discussions with A and B have been illuminating [insightful]
% I appreciate the feedback offered by someone
% I have greatly benefited from someone
% I have had the support and encouragement of someone
% I received generous support from someone
% My intellectual debt is to someone
% someone gives insightful comments and suggestions
% someone has been (greatly, extraordinarily) tolerant and supportive ...
% someone gives me constructive comments and warm encouragement
% someone made enormous contribution to ...
% someone's support [encouragement, suggestions, comments] were invaluable
% someone's meticulous comments were an enormous help to me
% Advice and comments given by someone has been a great help in...
% I am particularly grateful for the assistance given by someone
% 
% Without his/her guidance and persistent help this paper [dissertation, thesis] would not have been possible.
% Without his/her encouragement [help, guidance], this paper [dissertation, project] would not have materialized.
% 
% I thank someone for permission to use [include] ...
% I would also like to express my gratitude to someone for their financial support.

\end{acknowledgement}

\tableofcontents
\clearpage

%% \linenumbers

\listoffigures
\clearpage

\listofgraphs
\clearpage

\listoftables
\clearpage

\listofalgorithmes
\clearpage

\input th-list-of-symbols.tex
\clearpage


% TODO:
% Fig. vs Figure, ?? the ?? Figure
% use of subequations ??
% definition of seminorm in enrichment radius estimate section

%%%%%%%%%%%%%%%%%%%%%%%%%%%%%%%%%%%%%%%%%%%%%%%%%%%%%%%%%%%%%%%%%%%%%%%%%%%%%%%%%%%%%%%%%%%%%%%%%%%%%%%%%%%%%%

\chapter{Introduction}

% short introduction
% define problematics and open questions
% description of the document structure

%%%%%%%%%%%%%%%%%%%%%%%%%%%%%%%%%%%%%%%%%%%%%%%%%%%%%%%%%%%%%%%%%%%%%%%%%%%%%%%%%%%%%%%%%%%%%%%%%%%%%%%%%%%%%%

\input th-intro.tex


%%%%%%%%%%%%%%%%%%%%%%%%%%%%%%%%%%%%%%%%%%%%%%%%%%%%%%%%%%%%%%%%%%%%%%%%%%%%%%%%%%%%%%%%%%%%%%%%%%%%%%%%%%%%%%

\chapter{Reduced Dimensional Models} \label{chap:reduced}

% Mesh of Combined Dimensions
% Dirac sources models - 1d-2d, 1d-3d (Schwenk, Koeppl, Zunino, D'Andelo)
% xfem concept of Fumagalli, Scotti, D'Angelo ...

%%%%%%%%%%%%%%%%%%%%%%%%%%%%%%%%%%%%%%%%%%%%%%%%%%%%%%%%%%%%%%%%%%%%%%%%%%%%%%%%%%%%%%%%%%%%%%%%%%%%%%%%%%%%%%

\input th-reduced_dim.tex


%%%%%%%%%%%%%%%%%%%%%%%%%%%%%%%%%%%%%%%%%%%%%%%%%%%%%%%%%%%%%%%%%%%%%%%%%%%%%%%%%%%%%%%%%%%%%%%%%%%%%%%%%%%%%%

\chapter{Extended Finite Element Method} \label{chap:xfem_soa}

% XFEM - state of the art

%%%%%%%%%%%%%%%%%%%%%%%%%%%%%%%%%%%%%%%%%%%%%%%%%%%%%%%%%%%%%%%%%%%%%%%%%%%%%%%%%%%%%%%%%%%%%%%%%%%%%%%%%%%%%%

\input th-xfem.tex

%%%%%%%%%%%%%%%%%%%%%%%%%%%%%%%%%%%%%%%%%%%%%%%%%%%%%%%%%%%%%%%%%%%%%%%%%%%%%%%%%%%%%%%%%%%%%%%%%%%%%%%%%%%%%%

\chapter{XFEM in Pressure Model with Singularities} \label{chap:xfem_pressure}

% well aquifer model
% singular enrichment
% numerical results
% aspects - enr radius, adaptive qudrature, conditioning

%%%%%%%%%%%%%%%%%%%%%%%%%%%%%%%%%%%%%%%%%%%%%%%%%%%%%%%%%%%%%%%%%%%%%%%%%%%%%%%%%%%%%%%%%%%%%%%%%%%%%%%%%%%%%%

\input th-pressure-model.tex

\input th-pressure-results.tex

%%%%%%%%%%%%%%%%%%%%%%%%%%%%%%%%%%%%%%%%%%%%%%%%%%%%%%%%%%%%%%%%%%%%%%%%%%%%%%%%%%%%%%%%%%%%%%%%%%%%%%%%%%%%%%

\chapter{Singular Enrichment for Velocity} \label{chap:xfem_mh}

% Flow123d concept
% Mixed-hybrid formulation
% current research on the problematics MH + XFEM

%%%%%%%%%%%%%%%%%%%%%%%%%%%%%%%%%%%%%%%%%%%%%%%%%%%%%%%%%%%%%%%%%%%%%%%%%%%%%%%%%%%%%%%%%%%%%%%%%%%%%%%%%%%%%%

\input th-mh-model.tex

\input th-mh-results.tex

%%%%%%%%%%%%%%%%%%%%%%%%%%%%%%%%%%%%%%%%%%%%%%%%%%%%%%%%%%%%%%%%%%%%%%%%%%%%%%%%%%%%%%%%%%%%%%%%%%%%%%%%%%%%%%

% \chapter{Achieved Results} \label{chap:results}

%%%%%%%%%%%%%%%%%%%%%%%%%%%%%%%%%%%%%%%%%%%%%%%%%%%%%%%%%%%%%%%%%%%%%%%%%%%%%%%%%%%%%%%%%%%%%%%%%%%%%%%%%%%%%%

% In this chapter, we shall describe the work that has already been done. First part (Section \ref{sec:model_aquifer}) 
% is regarding a~model of a~well-aquifer system where the XFEM is applied to resolve a~point singularity in 2D domain. 
% The second part (Section \ref{sec:elements_intersections}) is aimed at a~geometrical problem of finding 
% intersections of incompatible meshes of different dimensions.




%%%%%%%%%%%%%%%%%%%%%%%%%%%%%%%%%%%%%%%%%%%%%%%%%%%%%%%%%%%%%%%%%%%%%%%%%%%%%%%%%%%%%%%%%%%%%%%%%%%%%%%%%%%%%%

\chapter{Mesh Intersection Algorithms} \label{chap:intersections}

%%%%%%%%%%%%%%%%%%%%%%%%%%%%%%%%%%%%%%%%%%%%%%%%%%%%%%%%%%%%%%%%%%%%%%%%%%%%%%%%%%%%%%%%%%%%%%%%%%%%%%%%%%%%%%

\input th-intersections.tex



%%%%%%%%%%%%%%%%%%%%%%%%%%%%%%%%%%%%%%%%%%%%%%%%%%%%%%%%%%%%%%%%%%%%%%%%%%%%%%%%%%%%%%%%%%%%%%%%%%%%%%%%%%%%%%

\chapter{Conclusion} \label{chap:summary}
% closure - what is done, further plan, mention traineeship

%%%%%%%%%%%%%%%%%%%%%%%%%%%%%%%%%%%%%%%%%%%%%%%%%%%%%%%%%%%%%%%%%%%%%%%%%%%%%%%%%%%%%%%%%%%%%%%%%%%%%%%%%%%%%%

Based on the research of the related works and the experience gained at the conferences
(in particular MAMERN VI '15, X-DMS '15-17 and CMWR '18),
we are convinced that we dedicated our efforts to a~very interesting and hot topic with wide range of applications.
We are not aware of any closely connected work to this topic in the Czech Republic
which puts us in a pioneer position at least in the Czech scientific environment.

% The aims of the thesis are stated in the Chapter \ref{chap:aims} and these are both of theoretic and practical nature.
% There is a lot work to do in the code implementation part but some of the work has been done already 
% (Chapter \ref{chap:intersections}). The rest will be done, possibly with some support of the Flow123d development team. 


The work was also consulted during the traineeship at Technical University in Munich at the Department of Numerical Mathematics
lead by Prof. Barbara Wholmuth. Mainly the theoretical aspects of the work and new ideas were discussed.
We got also familiarized with a~different approach for modeling singularities without using XFEM, which is being developed there.


% Multi-level Monte Carlo methods - computation on coarse meshes - can include non-planar 1d-2d or 1d-3d intersections 
% on coarse meshes using the XFEM.

% So far, we studied the well-aquifer model (Section \ref{sec:model_aquifer}). We compared several different XFEM methods, 
% dealt with the integration accuracy and investigated some other aspects of the XFEM usage. The results
% of our work were presented at several conferences and concluded in an article (list in Chapter \ref{chap:publications}).

A 6 months long traineeship is planned for the second half of the year 2016. A cooperation with the team lead
by professor Barbara Wohlmuth at the Technical University in Munich is to be established. They are experts
in the field of porous media modelling and they are also interested in models on meshes with combined dimensions
(see e.g. \cite{schwenck_2015} where fracture intersections in 2D is solved). The aim of the traineeship is to
work together both on the theory and implementation of the XFEM in 0D-2D and 1D-3D flow model.

%%%%%%%%%%%%%%%%%%%%%%%%%%%%%%%%%%%%%%%%%%%%%%%%%%%%%%%%%%%%%%%%%%%%%%%%%%%%%%%%%%%%%%%%%%%%%%%%%%%%%%%%%%%%%%


%   \nocite{gracie_modelling_2010, fries_corrected_2008, babuska_stable_2012, bangerth_deal.ii_2007, 
%           arnold_lecture_2009, craig_using_2011, sistek_bddc_2015, brezzi_mixed_1991, schwenck_xfem-based_2015,
%           fumagalli_efficient_2014, cattaneo_numerical_2015, koppl_tum_2015, maryska_mixed-hybrid_1995}
{\small
\bibliographystyle{mybibtex}
% \bibliographystyle{csplainnat}
\bibliography{citace,citace_xfem,citace_sgfem,citace_flow_my,citace_intersections}
}


\end{document}
