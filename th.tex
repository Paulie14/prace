\documentclass[bibliography=totocnumbered,dvipsnames,FM,Dis]{tulthesis}

\usepackage[czech, english]{babel}
\usepackage[utf8]{inputenc}

% Adds bibliography to tocbibind
% options:
% none - switch off everything
% numbib - adds number to bibliography
% https://ctan.org/pkg/tocbibind
\usepackage[nottoc]{tocbibind}

% \setlength{\parindent}{3em}
% \setlength{\parskip}{0.15cm}

\DeclareUnicodeCharacter{00A0}{~}

%% The lineno packages adds line numbers. Start line numbering with
%% \begin{linenumbers}, end it with \end{linenumbers}. Or switch it on
%% for the whole article with \linenumbers.
%% \usepackage{lineno}

% grahpics
\usepackage{graphicx}
\usepackage{subfig}
\newcommand{\fig}[1]{Figure \hyperref[#1]{\ref{#1}}}
\newcommand{\figpath}{figures/}
\newcommand{\results}{results/}

% živé odkazy v PDF
% \usepackage{hyperref}
% \hypersetup{colorlinks=true, linkcolor=tul, urlcolor=tul, citecolor=tul}
\hypersetup{colorlinks=false, linkcolor=tul, urlcolor=tul, citecolor=tul}
\hypersetup{pdftitle={Extended Finite Element Methods on Meshes of Combined Dimensions}}

%tables
\usepackage{booktabs}

% enumeration by alphabet
\usepackage[inline]{enumitem}

\usepackage{url}

% intersections chapter
% algorithms
\usepackage[vlined, linesnumbered, ruled]{algorithm2e}
%\usepackage{float}
%\newfloat{algorithm}{t}{lop}
\usepackage{array}
\newcommand{\plucker}{Pl\"{u}cker }
\newcommand{\nface}{$n$-face }
\newcommand{\nfaces}{$n$-faces }
\newcommand{\ngh}{NGH }
\newcommand{\algo}[1]{\hyperref[#1]{Algorithm \ref{#1}}}
\newcommand{\figpathins}{figures/intersections/}

\pdfsuppresswarningpagegroup=1

% deklarace pro titulní stránku
\TULthesisType{Disertační práce}{Doctoral Thesis}
\TULtitle{Rozšířené metody konečných prvků pro aproximaci singularit}{Extended Finite Element Methods for Approximation of Singularities}
% Extended finite elements for mixed-hybrid model of Darcy flow
\TULprogramme{P3901}{Aplikované vědy v~inženýrství}{Applied Sciences in Engineering}
\TULbranch{3901V055}{Aplikované vědy v~inženýrství}{Scientific Engineering (Mathematical Modelling)}
\TULauthor{Ing. Pavel Exner}
\TULsupervisor{Mgr. Jan B{\v r}ezina, Ph.D.}
\TULyear{}


% just for our notes
% \usepackage[usenames,dvipsnames]{color}   %colors
\newcommand{\noteJB}[1]{{\color{Blue} \textbf{JB: } \textit{#1}}}
\newcommand{\notePE}[1]{{\color{Orange} \textbf{PE: } \textit{#1}}}

%
%*************************************************************************************************************
%
%                                                   DOCUMENT
%
%*************************************************************************************************************
%

\begin{document}


\ThesisStart{male}
% \ThesisTitle{CZ}
% \ThesisTitle{EN}

\input th-abstract.tex
\clearpage

\begin{acknowledgement}

\end{acknowledgement}

\tableofcontents
\clearpage

%% \linenumbers

% \begin{abbrList}
\section*{Seznam zkratek}\addcontentsline{toc}{section}{Seznam zkratek}

\noindent\emph{Acronyms}
\vspace{0.5cm}

\begin{tabularx}{\linewidth}{@{}lX@{}}
DFN & Discrete Fracture Network \\
FVM & Finite Volume Method \\
FDM & Finite Difference Method \\
FEM & Finite Element Method \\
MHFEM & Mixed Hybrid Finite Element Method \\
PUM & Partition of Unity Method \\
XFEM & Extended Finite Element Method \\
GFEM & Generalized Finite Element Method \\
SGFEM & Stable Generalized Finite Element Method \\
TUL & Technical University in Liberec
\end{tabularx}
\vspace{1cm}

\noindent\emph{Mathematical notation}
\vspace{0.5cm}

\begin{tabularx}{\linewidth}{@{}lX@{}}
$d$ & dimension: 1d, 2d, 3d \\
$\Omega_d$ & domain of dimension $d$ \\
$\Gamma_{dN}$ & part of boundary of $\Omega_d$ with Neumann boundary condition \\
$\Gamma_{dD}$ & part of boundary of $\Omega_d$ with Dirichlet boundary condition \\
$\mathcal T_d$ & mesh consisting of elements of dimension $d$ \\
$T^i_d, F^i_d$  & element and face of $\mathcal T_d$ \\
$\vc x_i$ & node \\
$\mathcal I_{d,E}, \mathcal I_{d,F}, \mathcal I_{d,N}$ & indices of elements faces and nodes of element of dimension $d$ \\
$\Omega^w_C$ & domain (cylinder) of a well $w$ \\
$\Omega^w_1$ & reduced domain (line) of a well $w$ \\
$\mathcal W$ & index set of wells \\
$\mathcal M$ & index set of aquifers \\
$p_d$ & pressure in $d$-dimensional domain \\
$\vc u_d$ & velocity in $d$-dimensional domain \\
$r_w, \rho_w, R_w$ & distance function, radius and enrichment radius of a~well $w$ \\
$s_w, \vc s_w$ & scalar and vector global enrichment function\\
$a(\cdot, \cdot)$ & bilinear form \\
$l(\cdot)$ & linear form \\
$\avg{\cdot}$ & average operator \\
$\fluct{\cdot}$ & fluctuation operator \\
\end{tabularx}
\vspace{1cm}

\noindent\emph{Spaces}
\vspace{0.5cm}

\begin{tabularx}{\linewidth}{@{}lX@{}}
$\R$ & space of real numbers  \\
$\mathbb N$ & space of unsigned integers \\
$\mathbb P^k$ & space of polynomials of order $k$  \\
$\mathbb RT^k$ & space of Raviart-Thomas shape functions of order $k$  \\
$L_2$ & Lebesque space \\
$H^k, H(\div)$ & Hilbert spaces \\
\end{tabularx}

% TODO:
% FVM, FDM, FEM
% XFEM, GFEM, SGFEM
% Fig. vs Figure, ?? the ?? Figure
% use of subequations ??
% definition of seminorm in enrichment radius estimate section

% \textbf{TUL} & Technická univerzita v~Liberci \\
% \textbf{FM} & Fakulta mechatroniky, informatiky a mezioborových studií
% Technické univerzity v~Liberci
% \end{abbrList}

%%%%%%%%%%%%%%%%%%%%%%%%%%%%%%%%%%%%%%%%%%%%%%%%%%%%%%%%%%%%%%%%%%%%%%%%%%%%%%%%%%%%%%%%%%%%%%%%%%%%%%%%%%%%%%

\chapter{Introduction}

% short introduction
% define problematics and open questions
% description of the document structure

%%%%%%%%%%%%%%%%%%%%%%%%%%%%%%%%%%%%%%%%%%%%%%%%%%%%%%%%%%%%%%%%%%%%%%%%%%%%%%%%%%%%%%%%%%%%%%%%%%%%%%%%%%%%%%

\input th-intro.tex


%%%%%%%%%%%%%%%%%%%%%%%%%%%%%%%%%%%%%%%%%%%%%%%%%%%%%%%%%%%%%%%%%%%%%%%%%%%%%%%%%%%%%%%%%%%%%%%%%%%%%%%%%%%%%%

\chapter{Reduced Dimensional Models in Flow Problems} \label{chap:reduced}

% Mesh of Combined Dimensions
% Dirac sources models - 1d-2d, 1d-3d (Schwenk, Koeppl, Zunino, D'Andelo)
% xfem concept of Fumagalli, Scotti, D'Angelo ...

%%%%%%%%%%%%%%%%%%%%%%%%%%%%%%%%%%%%%%%%%%%%%%%%%%%%%%%%%%%%%%%%%%%%%%%%%%%%%%%%%%%%%%%%%%%%%%%%%%%%%%%%%%%%%%

\input th-reduced_dim.tex


%%%%%%%%%%%%%%%%%%%%%%%%%%%%%%%%%%%%%%%%%%%%%%%%%%%%%%%%%%%%%%%%%%%%%%%%%%%%%%%%%%%%%%%%%%%%%%%%%%%%%%%%%%%%%%

\chapter{XFEM in Pressure Model with Singularities} \label{chap:xfem_pressure}

% XFEM - state of the art
% well aquifer model
% singular enrichment
% numerical results
% aspects - enr radius, adaptive qudrature, conditioning

%%%%%%%%%%%%%%%%%%%%%%%%%%%%%%%%%%%%%%%%%%%%%%%%%%%%%%%%%%%%%%%%%%%%%%%%%%%%%%%%%%%%%%%%%%%%%%%%%%%%%%%%%%%%%%

\input th-xfem.tex

\input th-result-pressure.tex

%%%%%%%%%%%%%%%%%%%%%%%%%%%%%%%%%%%%%%%%%%%%%%%%%%%%%%%%%%%%%%%%%%%%%%%%%%%%%%%%%%%%%%%%%%%%%%%%%%%%%%%%%%%%%%

\chapter{Singular Enrichment for Velocity} \label{chap:xfem_mh}

% Flow123d concept
% Mixed-hybrid formulation
% current research on the problematics MH + XFEM

%%%%%%%%%%%%%%%%%%%%%%%%%%%%%%%%%%%%%%%%%%%%%%%%%%%%%%%%%%%%%%%%%%%%%%%%%%%%%%%%%%%%%%%%%%%%%%%%%%%%%%%%%%%%%%

\input th-model.tex

\input th-results-flow123d.tex

%%%%%%%%%%%%%%%%%%%%%%%%%%%%%%%%%%%%%%%%%%%%%%%%%%%%%%%%%%%%%%%%%%%%%%%%%%%%%%%%%%%%%%%%%%%%%%%%%%%%%%%%%%%%%%

% \chapter{Achieved Results} \label{chap:results}

%%%%%%%%%%%%%%%%%%%%%%%%%%%%%%%%%%%%%%%%%%%%%%%%%%%%%%%%%%%%%%%%%%%%%%%%%%%%%%%%%%%%%%%%%%%%%%%%%%%%%%%%%%%%%%

% In this chapter, we shall describe the work that has already been done. First part (Section \ref{sec:model_aquifer}) 
% is regarding a~model of a~well-aquifer system where the XFEM is applied to resolve a~point singularity in 2D domain. 
% The second part (Section \ref{sec:elements_intersections}) is aimed at a~geometrical problem of finding 
% intersections of incompatible meshes of different dimensions.




%%%%%%%%%%%%%%%%%%%%%%%%%%%%%%%%%%%%%%%%%%%%%%%%%%%%%%%%%%%%%%%%%%%%%%%%%%%%%%%%%%%%%%%%%%%%%%%%%%%%%%%%%%%%%%

\chapter{Mesh Intersection Algorithms} \label{chap:intersections}

%%%%%%%%%%%%%%%%%%%%%%%%%%%%%%%%%%%%%%%%%%%%%%%%%%%%%%%%%%%%%%%%%%%%%%%%%%%%%%%%%%%%%%%%%%%%%%%%%%%%%%%%%%%%%%

\input th-intersections.tex



%%%%%%%%%%%%%%%%%%%%%%%%%%%%%%%%%%%%%%%%%%%%%%%%%%%%%%%%%%%%%%%%%%%%%%%%%%%%%%%%%%%%%%%%%%%%%%%%%%%%%%%%%%%%%%

\chapter{Conclusion} \label{chap:summary}
% closure - what is done, further plan, mention traineeship

%%%%%%%%%%%%%%%%%%%%%%%%%%%%%%%%%%%%%%%%%%%%%%%%%%%%%%%%%%%%%%%%%%%%%%%%%%%%%%%%%%%%%%%%%%%%%%%%%%%%%%%%%%%%%%

Based on the research of the related works and the experience gained at the conferences
(in particular MAMERN VI '15, X-DMS '15-17 and CMWR '18),
we are convinced that we dedicated our efforts to a~very interesting and hot topic with wide range of applications.
We are not aware of any closely connected work to this topic in the Czech Republic
which puts us in a pioneer position at least in the Czech scientific environment.

% The aims of the thesis are stated in the Chapter \ref{chap:aims} and these are both of theoretic and practical nature.
% There is a lot work to do in the code implementation part but some of the work has been done already 
% (Chapter \ref{chap:intersections}). The rest will be done, possibly with some support of the Flow123d development team. 


The work was also consulted during the traineeship at Technical University in Munich at the Department of Numerical Mathematics
lead by Prof. Barbara Wholmuth. Mainly the theoretical aspects of the work and new ideas were discussed.
We got also familiarized with a~different approach for modeling singularities without using XFEM, which is being developed there.


% Multi-level Monte Carlo methods - computation on coarse meshes - can include non-planar 1d-2d or 1d-3d intersections 
% on coarse meshes using the XFEM.

% So far, we studied the well-aquifer model (Section \ref{sec:model_aquifer}). We compared several different XFEM methods, 
% dealt with the integration accuracy and investigated some other aspects of the XFEM usage. The results
% of our work were presented at several conferences and concluded in an article (list in Chapter \ref{chap:publications}).

A 6 months long traineeship is planned for the second half of the year 2016. A cooperation with the team lead
by professor Barbara Wohlmuth at the Technical University in Munich is to be established. They are experts
in the field of porous media modelling and they are also interested in models on meshes with combined dimensions
(see e.g. \cite{schwenck_2015} where fracture intersections in 2D is solved). The aim of the traineeship is to
work together both on the theory and implementation of the XFEM in 0D-2D and 1D-3D flow model.

%%%%%%%%%%%%%%%%%%%%%%%%%%%%%%%%%%%%%%%%%%%%%%%%%%%%%%%%%%%%%%%%%%%%%%%%%%%%%%%%%%%%%%%%%%%%%%%%%%%%%%%%%%%%%%


% \listoffigures
% \clearpage
% 
% \listoftables
% \clearpage


%   \nocite{gracie_modelling_2010, fries_corrected_2008, babuska_stable_2012, bangerth_deal.ii_2007, 
%           arnold_lecture_2009, craig_using_2011, sistek_bddc_2015, brezzi_mixed_1991, schwenck_xfem-based_2015,
%           fumagalli_efficient_2014, cattaneo_numerical_2015, koppl_tum_2015, maryska_mixed-hybrid_1995}
{\small
\bibliographystyle{mybibtex}
% \bibliographystyle{csplainnat}
\bibliography{citace,citace_xfem,citace_sgfem,citace_flow_my,citace_intersections}
}


\end{document}
