\documentclass[bibliography=totocnumbered,dvipsnames,FM,Dis]{tulthesis}

\usepackage[czech, english]{babel}
\usepackage[utf8]{inputenc}

% Adds bibliography to tocbibind
% options:
% none - switch off everything
% numbib - adds number to bibliography
% https://ctan.org/pkg/tocbibind
\usepackage[nottoc]{tocbibind}

\DeclareUnicodeCharacter{00A0}{~}

\usepackage{amsmath}
\usepackage{amsfonts}
\usepackage{amssymb}
\usepackage{amsthm}
\usepackage{esint}
%
\newtheorem{theorem}{Theorem}[section]
\newtheorem{proposition}[theorem]{Proposition}
\newtheorem{definition}[theorem]{Definition}
\newtheorem{remark}[theorem]{Remark}
\newtheorem{lemma}[theorem]{Lemma}
\newtheorem{corollary}[theorem]{Corollary}
\newtheorem{thmproblem}{Problem}
%\newtheorem{exercise}[theorem]{Cvičení}

%\numberwithin{equation}{document}
%
\def\div{{\rm div}}
\def\Lapl{\Delta}
\def\grad{\nabla}
\def\supp{{\rm supp}}
\def\dist{{\rm dist}}
%\def\chset{\mathbbm{1}}
\def\chset{1}
%
\def\Tr{{\rm Tr}}
\def\to{\rightarrow}
\def\weakto{\rightharpoonup}
\def\imbed{\hookrightarrow}
\def\cimbed{\subset\subset}
\def\range{{\mathcal R}}
\def\leprox{\lesssim}
\def\argdot{{\hspace{0.18em}\cdot\hspace{0.18em}}}
\def\Distr{{\mathcal D}}
\def\calK{{\mathcal K}}
\def\FromTo{|\rightarrow}
\def\convol{\star}
\def\impl{\Rightarrow}
\DeclareMathOperator*{\esslim}{esslim}
\DeclareMathOperator*{\esssup}{ess\,supp}
\DeclareMathOperator{\ess}{ess}
\DeclareMathOperator{\osc}{osc}
\DeclareMathOperator{\curl}{curl}
\DeclareMathOperator{\cotg}{cotg}

%
%\def\Ess{{\rm ess}}
%\def\Exp{{\rm exp}}
%\def\Implies{\Longrightarrow}
%\def\Equiv{\Longleftrightarrow}
% ****************************************** GENERAL MATH NOTATION
\def\Real{{\rm\bf R}}
\def\C{{\rm\bf C}}
\def\Rd{{{\rm\bf R}^{\rm 3}}}
\def\RN{{{\rm\bf R}^N}}
\def\D{{\mathbb D}}
\def\Nnum{{\rm\bf N}}
\def\Qnum{{\rm\bf Q}}
\def\Measures{{\mathcal M}}
\def\dd{\,{\rm d}}               % differential
\def\sdodt{\genfrac{}{}{}{1}{\rm d}{{\rm d}t}}
\def\dodt{\genfrac{}{}{}{}{\rm d}{{\rm d}t}}
%
\def\vc#1{\mathbf{\boldsymbol{#1}}}     % vector
\def\bx{\mathbf{\boldsymbol{x}}}     % vector x
\def\tn#1{{\mathbb{#1}}}    % tensor
\def\abs#1{\lvert#1\rvert}
\def\Abs#1{\bigl\lvert#1\bigr\rvert}
\def\bigabs#1{\bigl\lvert#1\bigr\rvert}
\def\Bigabs#1{\Big\lvert#1\Big\rvert}
\def\ABS#1{\left\lvert#1\right\rvert}
\def\norm#1{\bigl\Vert#1\bigr\Vert} %norm
\def\metr#1#2{\d\bigl(#1,#2\bigr)}          %metric
\def\close#1{\overline{#1}}
\def\inter#1{#1^\circ}
\def\eqdef{\mathrel{\mathop:}=}     % defining equivalence
\def\where{\,|\,}                    % "where" separator in set's defs
\def\timeD#1{\dot{\overline{{#1}}}}
\def\rfrac#1#2{{}^{#1}\!/_{#2}}
%
% ******************************************* USEFULL MACROS
\def\RomanEnum{\renewcommand{\labelenumi}{\rm (\roman{enumi})}}   % enumerate by roman numbers
\def\rf#1{(\ref{#1})}                                             % ref. shortcut
\def\prtl{\partial}                                        % partial deriv.
\def\Names#1{{\scshape #1}}
\def\rem#1{{\parskip=0cm\par!! {\sl\small #1} !!}}
\def\vysl#1{\par$[$ #1 $]$}

%% The lineno packages adds line numbers. Start line numbering with
%% \begin{linenumbers}, end it with \end{linenumbers}. Or switch it on
%% for the whole article with \linenumbers.
%% \usepackage{lineno}

% grahpics
\usepackage{graphicx}
\usepackage{subfig}
\newcommand{\fig}[1]{\hyperref[#1]{Figure \ref{#1}}}
\newcommand{\figpath}{figures/}
\newcommand{\results}{results/}

% živé odkazy v PDF
\usepackage{hyperref}
% \hypersetup{colorlinks=true, linkcolor=tul, urlcolor=tul, citecolor=tul}
\hypersetup{colorlinks=false, linkcolor=tul, urlcolor=tul, citecolor=tul}
\hypersetup{pdftitle={Extended Finite Element Methods on Meshes of Combined Dimensions}}

%tables
\usepackage{booktabs}

% enumeration by alphabet
\usepackage[inline]{enumitem}

\usepackage{url}

% intersections chapter
% algorithms
\usepackage[vlined, linesnumbered, ruled]{algorithm2e}
%\usepackage{float}
%\newfloat{algorithm}{t}{lop}
\usepackage{array}
\newcommand{\plucker}{Pl\"{u}cker }
\newcommand{\nface}{$n$-face }
\newcommand{\nfaces}{$n$-faces }
\newcommand{\ngh}{NGH }
\newcommand{\algo}[1]{\hyperref[#1]{Algorithm \ref{#1}}}
\newcommand{\figpathins}{figures/intersections/}

\pdfsuppresswarningpagegroup=1

% deklarace pro titulní stránku
\TULthesisType{Disertační práce}{Thesis}
\TULtitle{Rozšířené metody konečných prvků na sítích kombinovaných dimenzích}{Extended Finite Element Methods on Meshes of Combined Dimensions}
% Extended finite elements for mixed-hybrid model of Darcy flow
\TULprogramme{P3901}{Aplikované vědy v~inženýrství}{Applied Sciences in Engineering}
\TULbranch{3901V055}{Aplikované vědy v~inženýrství}{Scientific Engineering (Mathematical Modelling)}
\TULauthor{Ing. Pavel Exner}
\TULsupervisor{Mgr. Jan B{\v r}ezina, Ph.D.}
\TULyear{}


% just for our notes
% \usepackage[usenames,dvipsnames]{color}   %colors
\newcommand{\noteJB}[1]{{\color{Blue} \textbf{JB: } \textit{#1}}}
\newcommand{\notePE}[1]{{\color{Orange} \textbf{PE: } \textit{#1}}}

%
%*************************************************************************************************************
%
%                                                   DOCUMENT
%
%*************************************************************************************************************
%

\begin{document}


%\ThesisStart{male}
\ThesisTitle{CZ}
\ThesisTitle{EN}


\begin{abstractCZ}
Doktorská práce je zaměřena na modelování proudění podzemní vody pomocí metody konečných prvků na sítích, 
které obsahují {ele\-men\-ty} různých dimenzí. 
V takových modelech je často celková zájmová oblast rozsáhlá a~aproximace konečnými prvky není schopná dostatečně přesně zachytit
lokální jevy. 
Použití některé z technik adaptivního zjemňování sítě vede v takovém případě na systémy s enormním počtem neznámých, obzvláště ve 3D.
Oproti tomu jsou dnes moderní tzv. rozšířené metody konečných (XFEM) prvků. Jejich aplikace do modelů se sítěmi obsahujícími
elementy různých dimenzí je však v raném vývoji a~je zde mnoho otevřených problémů.
Hlavním cílem této práce je návrh a~implementace vhodného obohacení pro metodu XFEM v modelu podzemního proudění,
založeném na Darcyho zákonu, a~umožnění výpočtu na tzv. nekompatibilních sítích.
\end{abstractCZ}

\vspace{2cm}

\begin{abstractEN}
This doctoral thesis is aimed at modelling of groundwater flow with the finite element method on meshes of combined dimensions. 
A~general problem of such models is impossibility to capture small-scale 
phenomena in large domains. Instead of using adaptive meshing approach to obtain better approximation, 
extended finite element methods (XFEM) are very popular today. However, their usage in models with combined dimensions
is still developing and has many open questions. The main goal is to propose and possibly 
implement a~proper XFEM enrichment in a~model of groundwater flow, governed by Darcy's law, to enable 
coupling between incompatible meshes of combined dimensions.
\end{abstractEN}

\clearpage

%\begin{acknowledgement}
% Rád bych poděkoval všem, kteří přispěli ke vzniku tohoto dílka.
%\end{acknowledgement}

\tableofcontents

\clearpage

%% \linenumbers

% \begin{abbrList}
% \textbf{TUL} & Technická univerzita v~Liberci \\
% \textbf{FM} & Fakulta mechatroniky, informatiky a mezioborových studií
% Technické univerzity v~Liberci
% \end{abbrList}

%%%%%%%%%%%%%%%%%%%%%%%%%%%%%%%%%%%%%%%%%%%%%%%%%%%%%%%%%%%%%%%%%%%%%%%%%%%%%%%%%%%%%%%%%%%%%%%%%%%%%%%%%%%%%%

\chapter{Introduction}

% short introduction
% define problematics and open questions
% description of the document structure

%%%%%%%%%%%%%%%%%%%%%%%%%%%%%%%%%%%%%%%%%%%%%%%%%%%%%%%%%%%%%%%%%%%%%%%%%%%%%%%%%%%%%%%%%%%%%%%%%%%%%%%%%%%%%%



Mathematical modelling plays a~very important role in science and also our daily lives throughout many different
fields of knowledge. Using modern finite element methods, we are able to simulate, investigate and predict
various phenomena of both nature and industrial character. Recent advances in the field of finite element methods
enable us to solve models of various scales, dimensions, to overcome numerical problems and also 
to couple interfering features together.

A large set of problems with finite element (FE) models, that people nowadays deal with, is connected with 
insufficient accuracy in cases where the model includes large and very small scale phenomena at once.
We can imagine a~simulation of groundwater flow in a~large domain (hundreds of meters) which can be significantly
influenced by thin fractures in the porous media or artificial wells and boreholes (several centimeters in diameter).
These disturbances bring discontinuities and singularities into the model and the finite elements are
unable to capture them accurately enough.

Adaptive meshes can be used in such cases, but it can cost a~lot of computational power to build a~very fine mesh,
and solve the problem with increasing number of degrees of freedom.
It also requires very robust meshing techniques when complex geometries are in question.

Alternatively, models combining different dimensions are being developed. These decompose the geometry
in objects of different dimension (e.g. 2D fractures, 1D wells, 0D point sources), create meshes of the objects independently,
and then they must deal with coupling of the modelled processes among the dimensions. 
Compatible meshes, where the intersections are at nodes and sides of elements, are easier to treat, but harder
or nearly impossible to construct in a~general case. On the other hand, incompatible meshes with arbitrary intersections are easier to construct, 
but bring a~whole new set of problems in the coupling.

Next, there are the so called extended finite element methods (XFEM); or partition of unity methods (PUM) or generalized
finite element methods (GFEM), which names are sometimes used interchangeably. 
These methods enable us to take advantage of an a~priori knowledge of the model solution character.
Then, we are able to locally incorporate non-polynomial functions, like a~jump or a~singularity, into the finite element solution
 in places, where we expect these features to appear. See an example of XFEM usage in 0D-2D coupling in \fig{fig:aquifers}.
%\newline{}
\begin{figure}[!htb]
%   \vspace{0pt}
  \centering    
    \includegraphics[width=0.6\textwidth]{\figpath 2_aquifers-5_wells_mesh_whitebg_crop.pdf}
  \caption{0D-2D coupling example from my previous work. Distribution of pressure in 2 aquifers (horizontal planes) with 5 wells 
          (vertical lines). The XFEM is used on a~coarse mesh (at the bottom). }
  \label{fig:aquifers}
\end{figure}

The thesis is aimed at further development of the XFEM and its usage in multidimensional models. 
We intend to create a~model with incompatible mesh, where elements of different dimensions intersect
arbitrarily, and then use the XFEM to glue the processes in different dimensions back together. 
This of course is a~very ambitious idea, therefore we narrow our plan to 0D-2D and 1D-3D coupling. 

Our research domain is groundwater modelling, therefore we apply the method on such model.
There are, of course, various other applications for such models apart from groundwater flow and transport in fractured porous media:
geothermal problematic and gathering geothermal energy; oil and gas extraction and also gas storage;
construction of safe underground nuclear waste deposits; investigation of transport of substances in human body
and more.


\section{Document Structure} \label{sec:structure}

Chapter \ref{chap:soa} concludes the research on the related works and represents the current state of art
on the problematic. Modelling on meshes of combined dimension is addressed in Section \ref{sec:soa_model_combined},
survey on the XFEM is presented in Section \ref{sec:soa_xfem} and works regarding the usage of the XFEM on
meshes of combined dimensions are commented in Section \ref{sec:soa_xfem_combined}. Then 
the software Flow123d is described briefly in Section \ref{sec:soa_flow123d}.

Chapter \ref{chap:aims} summarizes the aims of the thesis.

The finished and ongoing work is presented in Chapter \ref{chap:results}. Section \ref{sec:model_aquifer} 
covers a~well-aquifer model of groundwater flow with the XFEM resolving a~point singularity in 2D.
Chapter \ref{chap:intersections} introduces a~new approach for inspecting simplicial elements intersections
of different dimensions using Pl{\"u}cker coordinates.

Chapter \ref{chap:publications} contains a~list of author's publications and attended conferences.
The summary of the thesis is in Chapter \ref{chap:summary} and it is followed by the literature.




%%%%%%%%%%%%%%%%%%%%%%%%%%%%%%%%%%%%%%%%%%%%%%%%%%%%%%%%%%%%%%%%%%%%%%%%%%%%%%%%%%%%%%%%%%%%%%%%%%%%%%%%%%%%%%

\chapter{Aims of the Thesis}\label{chap:aims}
% define aims, characterize and briefly explain solution

%%%%%%%%%%%%%%%%%%%%%%%%%%%%%%%%%%%%%%%%%%%%%%%%%%%%%%%%%%%%%%%%%%%%%%%%%%%%%%%%%%%%%%%%%%%%%%%%%%%%%%%%%%%%%%

The research aims of the thesis can be summarized in two major points
\begin{itemize}
  \item Propose suitable enrichments for the pressure and the velocity that are stable (satisfy inf-sup condition). 
        If possible, emphasise a~good approximation of the velocity field which is necessary for the transport equation.
        
  \item Suggest new data structures and algorithms for the realization phase in the software Flow123d. 
\end{itemize}

\section{Research aims}

The following types of enrichments can be considered: wells (point source in 2D, line source in 3D), 
fractures (discontinuity in velocity), boundary of 1D and 2D fractures. 
To narrow the ambitious plan, we will focus on the enrichment for 0D-2D and 1D-3D interactions.

The model with the mixed-hybrid formulation as in \cite{brezina_mixed-hybrid_2010, sistek_bddc_2015} will be followed.
The enrichment will be chosen and incorporated in the formulation so that the singularities in 0D-2D and
1D-3D are better approximated and incompatible meshes can be used. 
We shall start by enriching the pressure field with a~logarithmic function like we
do in Section \ref{sec:model_aquifer} and investigate the consequences in the mixed and mixed-hybrid formulation.

Secondly the velocity field will be enriched. We will look for a~proper vectorial enrichment functions.
The accuracy of the velocity field, in particular the flow across the elements edges, is very important for
the transport equation which is in the main focus in most of the applications.

In a consequence of the enrichment, the discrete spaces will be extended. The influence of the enrichment on the inf-sup 
condition will be investigated and the proof of stability for the enriched formulation should be provided.
The aim is to understand the stability condition good enough so that some constraints on the choice of the enrichment 
functions can be defined. 

The realization of the model in the means of code implementation will require several non-standard techniques.
Efficient data structure must be provided for the enrichment implementation in finite element classes.
The intersections of the mesh domains of different dimensions must be computed. A specialized integration
on the intersected elements is needed to assembly the integrals in the system matrix. 
Solving the final linear algebraic system will require an efficient numerical method.
Each of these sub-problems is not trivial and requires a new approach or at least extending the existing one.
The steps of the realization are summarized below in Section \ref{sec:realization}.

\section{Realization} \label{sec:realization}
The suggested numerical model will be implemented and tested in the software Flow123d.
The following upgrades and changes are to be considered.

The enrichment functions need to be defined and support for the enrichment in the finite element classes must be implemented. 
A class that handles the degrees of freedom must be then updated to include the additional enriched degrees of freedom
and the structure of the linear system must be reorganized accordingly.

The enriched elements are in 3D domain. A module in Flow123d that is able to lookup the element intersections
of different dimensions is now being finished (see Chapter \ref{chap:intersections}). The enrichment will propagate from the intersected 3D elements.
The choice of the size of the enrichment area is not straightforward. In case of discontinuities, one layer of elements
near the discontinuity would be suitable, but in case of singularity, the area must be large enough so that 
the large gradients in the quantities are well approximated. For simplicity, the area size can be given as
a~constant parameter.

Then the assembly routines of the enriched terms into the linear system must be provided.
Since the enrichment function would not be polynomial, accurate integration on the enriched elements is required. 
The adaptive elements sub-refinement approach can be used to obtain more precise quadrature, 
or quadrature points defined in cylindrical coordinate system in the vicinity of 1D elements intersections
might be suitable in this case.

Finally, a~reasonable output post-processing should be developed, so that the XFEM solution (piecewise non-polynomial solution)
can be visualized properly. A~kind of adaptive refinement of the output mesh would be suitable.


%%%%%%%%%%%%%%%%%%%%%%%%%%%%%%%%%%%%%%%%%%%%%%%%%%%%%%%%%%%%%%%%%%%%%%%%%%%%%%%%%%%%%%%%%%%%%%%%%%%%%%%%%%%%%%

\chapter{XFEM -- State of the Art} \label{chap:soa}

% current research on the problematics
% modelling on combined dimensions
% model of flow on combined dimensions
% XFEM
% theory??

%%%%%%%%%%%%%%%%%%%%%%%%%%%%%%%%%%%%%%%%%%%%%%%%%%%%%%%%%%%%%%%%%%%%%%%%%%%%%%%%%%%%%%%%%%%%%%%%%%%%%%%%%%%%%%

\input th-xfem.tex

\section{Model of Flow on Mesh of Combined Dimensions} \label{sec:soa_model_combined}

We are interested in the model of groundwater flow (Darcy flow considered) in fractured porous 
media with a mesh of combined dimensions, therefore we shall often use the related terminology.
We shall refer to the approach that has been developed at the Technical
University in Liberec. 

\subsection{Mesh of Combined Dimensions}
% define terms, problematics, some solutions, refer further to State of the Art
Development of FE models that cover phenomena of different scales led to an idea of combining meshes of
different dimensions to simplify the computation on a~resulting complex computational mesh.

From the general point of view the problem can be seen as a~strongly coupled model where the coupling is not
among physical fields but among domains of different dimensions. The coupling is then performed via boundary
conditions and source terms of the fields equations. The term strongly coupled is used because the coupling 
is apparently bidirectional, the system can hardly be decoupled (meaning solving the PDEs separately). 

\begin{figure}[h]
\centering
\includegraphics[width=10cm]{\figpath ground_fractures}
\caption{
    \label{fig:multi-dim}
    Scheme of a~problem with domains of multiple dimensions, (used from Flow123d documentation~\cite{flow123d_doc_2015}).
}
\end{figure}

An open set $\Omega_{3} \subset \Real^3$ represents a~continuous approximation of porous and fractured medium.
Similarly, we consider a~set of 2D manifolds $\Omega_2\subset\overline\Omega_3$, representing the 2D fractures and a~set of 1D manifolds $\Omega_1\subset \overline\Omega_2$ 
representing the 1D channels (see Fig \ref{fig:multi-dim}).
The domains $\Omega_2$ and $\Omega_1$ are polytopic (i.e. polygonal and piecewise linear, respectively).
For every dimension $d=1,2,3$, a~triangulation $\mathcal{T}_{d}$ of the open set $\Omega_d$
is introduced that consists of finite elements $T_{d}^{i},$\ $i = 1,\dots,N_{E}^{d}$.
The elements are simplices in our case, i.e. lines, triangles and tetrahedra, respectively 
(based on the documentation of Flow123d \cite{flow123d_doc_2015}).

The elements in \emph{a compatible mesh} must satisfy the following compatibility conditions
\begin{equation}
        T_{d-1}^i \cap T_d \subset \mathcal{F}_d,   \qquad \text{where } \mathcal{F}_d = \bigcup_{k} \partial T_{d}^{k}
\end{equation}
and
\begin{equation}
        T_{d-1}^i \cap \mathcal{F}_d    \text{ is either $T_{d-1}^i$ or $\emptyset$}    
\end{equation}
for every $i\in\{1,\dots, N_{E}^{d-1}\}$, $j\in\{1,\dots,N_{E}^{d}\}$,  and $d=2,3$. 
That is, the $(d-1)$-dimensional elements are either between $d$-dimensional elements and
match their sides or they poke out of $\Omega_d$. 

A creation of a~compatible mesh can be challenging, due to the compatibility conditions and also the
element quality requirement. Therefore the aim is to avoid these meshes.

\emph{An incompatible mesh} does not have to satisfy the compatibility equations above. 
General intersections of elements can be considered but these must be then carefully taken care of in the model.


\subsection{Mixed-hybrid formulation}
The theory of mixed and mixed-hybrid finite element method is described in general in the well known book by 
Brezzi and Fortin~\cite{brezzi_mixed_1991}. 
The formulation, its stability and error estimates are studied there in abstract forms.

The actual application of the mixed-hybrid method in the model of flow in fractured porous 
media with combined mesh dimensions is studied in several articles.
In \cite{brezina_mixed-hybrid_2010} the weak formulation and its discretization using the lowest-order Raviart-Thomas finite 
elements is written. The linear system is then reduced using the Schur complement (original idea in~\cite{maryska_mixed-hybrid_1995})
and solved efficiently with preconditioned conjugate gradients method.
In \cite{sistek_bddc_2015} the theoretic part includes the weak and the discrete formulation and also shows
the uniqueness of the discrete solution. Further, the authors discuss application and results of BDDC 
(Balancing Domain Decomposition by Constraints) method used for solution of the linear system.

In \cite{brezina_2012} a~mortar-like method is used to deal with the discrete coupling between equations on incompatible 1D-2D meshes 
of different dimensions. The drawback of this approach is that it uses continuous approximation of velocity, so
it cannot approximate the discontinuity of the velocity over the fractures accurately enough.



\section{Flow123d} \label{sec:soa_flow123d}

The software Flow123d (latest stable version 1.8.2 in 2015, online documentation in \cite{flow123d_doc_2015}) 
is a~simulator of underground water flow, solute and heat transport in fractured porous media. The software is
developed as an open source code under the versioning system Git where the eponymous project can be found, or 
one can find it on the web page
\begin{center}
\url{http://flow123d.github.io/}
\end{center}
It has been developed at the Technical University of Liberec approximately since 2007.
The main feature of the software is a computation on complex compatible meshes of combined dimensions, described in
previous section. Therefore, the continuum models and discrete fracture network models can be combined.
%
\begin{figure}[!htb]
  \centering
  \includegraphics[width=0.35\textwidth]{\figpath logo-flow-123d-color.pdf}
  \caption{Flow balance in the well.}
  \label{fig:logo_flow123d}
\end{figure}
%
Current version includes mixed-hybrid solver for steady and unsteady Darcy flow, finite volume model 
and discontinuous Galerkin model for solute transport of several substances and heat transfer model. 
Using operator splitting, models for various local processes are supported including dual porosity, sorption, decays 
and simple reactions.


Development of the software is an important practical part of the thesis. The aim is to implement support 
of incompatible meshes in the Darcy flow model. More details about implementation goals will be given in
Section \ref{chap:aims}.



%%%%%%%%%%%%%%%%%%%%%%%%%%%%%%%%%%%%%%%%%%%%%%%%%%%%%%%%%%%%%%%%%%%%%%%%%%%%%%%%%%%%%%%%%%%%%%%%%%%%%%%%%%%%%%

\chapter{Model Formulation} \label{chap:model}

%%%%%%%%%%%%%%%%%%%%%%%%%%%%%%%%%%%%%%%%%%%%%%%%%%%%%%%%%%%%%%%%%%%%%%%%%%%%%%%%%%%%%%%%%%%%%%%%%%%%%%%%%%%%%%

\input th-model.tex


%%%%%%%%%%%%%%%%%%%%%%%%%%%%%%%%%%%%%%%%%%%%%%%%%%%%%%%%%%%%%%%%%%%%%%%%%%%%%%%%%%%%%%%%%%%%%%%%%%%%%%%%%%%%%%

\chapter{Achieved Results} \label{chap:results}

%%%%%%%%%%%%%%%%%%%%%%%%%%%%%%%%%%%%%%%%%%%%%%%%%%%%%%%%%%%%%%%%%%%%%%%%%%%%%%%%%%%%%%%%%%%%%%%%%%%%%%%%%%%%%%

% In this chapter, we shall describe the work that has already been done. First part (Section \ref{sec:model_aquifer}) 
% is regarding a~model of a~well-aquifer system where the XFEM is applied to resolve a~point singularity in 2D domain. 
% The second part (Section \ref{sec:elements_intersections}) is aimed at a~geometrical problem of finding 
% intersections of incompatible meshes of different dimensions.

\input th-result-pressure.tex


%%%%%%%%%%%%%%%%%%%%%%%%%%%%%%%%%%%%%%%%%%%%%%%%%%%%%%%%%%%%%%%%%%%%%%%%%%%%%%%%%%%%%%%%%%%%%%%%%%%%%%%%%%%%%%

\chapter{Mesh Intersection Algorithms} \label{chap:intersections}

%%%%%%%%%%%%%%%%%%%%%%%%%%%%%%%%%%%%%%%%%%%%%%%%%%%%%%%%%%%%%%%%%%%%%%%%%%%%%%%%%%%%%%%%%%%%%%%%%%%%%%%%%%%%%%

\input th-intersections.tex


%%%%%%%%%%%%%%%%%%%%%%%%%%%%%%%%%%%%%%%%%%%%%%%%%%%%%%%%%%%%%%%%%%%%%%%%%%%%%%%%%%%%%%%%%%%%%%%%%%%%%%%%%%%%%%

%%%%%%%%%%%%%%%%%%%%%%%%%%%%%%%%%%%%%%%%%%%%%%%%%%%%%%%%%%%%%%%%%%%%%%%%%%%%%%%%%%%%%%%%%%%%%%%%%%%%%%%%%%%%%%

\chapter{Author's Publication Activity} \label{chap:publications}
% presentations at conferences, sborniky z nich
% article
% Flow123d

%%%%%%%%%%%%%%%%%%%%%%%%%%%%%%%%%%%%%%%%%%%%%%%%%%%%%%%%%%%%%%%%%%%%%%%%%%%%%%%%%%%%%%%%%%%%%%%%%%%%%%%%%%%%%%
\large\textsc{Publications}
\begin{itemize}[label={}]

\item
J. B{\v r}ezina, P. Exner, \href{http://www.sciencedirect.com/science/article/pii/S0898122117300792}{Fast algorithms for intersection of non-matching grids using Plücker coordinates},
\emph{Computers and Mathematics with Applications}, Volume 74, July 2017, pp. 174-187, ISSN 0898-1221, \href{https://doi.org/10.1016/j.camwa.2017.01.028}{doi.org/10.1016/j.camwa.2017.01.028}.

\item
P. Exner, J. B{\v r}ezina, \href{http://www.sciencedirect.com/science/article/pii/S0096300315012862}{Partition of unity methods for approximation of point water sources in porous media},
\emph{Applied Mathematics and Computation}, Volume 273, January 2016, pp. 21-32, ISSN 0096-3003, \href{http://dx.doi.org/10.1016/j.amc.2015.09.048}{doi:10.1016/j.amc.2015.09.048}.

\item
P. Exner, J. B{\v r}ezina, \href{http://www.ugn.cas.cz/actually/event/2015/sna/sna-sbornik.pdf}{Adaptive integration of singularity in partition of unity methods},
in proceedings of \emph{Seminar on Numerical Analysis 2015}, Institute of Geonics AS CR, Ostrava, 2015, pp. 29-32, ISBN 978-80-86407-55-5, \\
\href{http://www.ugn.cas.cz/actually/event/2015/sna/sna-sbornik.pdf}{http://www.ugn.cas.cz/actually/event/2015/sna/sna-sbornik.pdf}.

\item
P. Exner, \href{http://www.cs.cas.cz/sna2014/sbornik.pdf}{Partition of unity methods for approximation of point water sources in porous media},
in proceedings of \emph{Seminar on Numerical Analysis 2014}, Institute of Computer Science AS CR, Prague, 2014, pp. 29-32, ISBN 978-80-87136-16-4, \\
\href{http://www.cs.cas.cz/sna2014/sbornik.pdf}{http://www.cs.cas.cz/sna2014/sbornik.pdf}.


\end{itemize}
%
\vspace{0.5cm}
%
\large\textsc{Conferences}
%
\begin{itemize}[label={}]
\item DFNE '18, Seattle (US) \\ Multidimensional model of flow and transport in fractured
rock for support of Czech DGR program (presentation)
\item CMWR '18, Saint Malo (FR) \\ Darcy Flow on Incompatible Meshes of
Combined Dimensions (presentation)
\item X-DMS '17, Ume{\o a} (SE) \\ Singular Enrichment of XFEM on Meshes of
Combined Dimensions (presentation)
\item FEM Symposium '16, Chemnitz (DE) \\ Extended Finite Element Methods Dealing with Singularities on Meshes of Combined Dimensions (presentation)
\item X-DMS '15, Ferrara, (IT) \\ eXtended Discretization Methods (presentation, short course attendance)
\item MAMERN VI '15, Pau, (FR) \\ The International Conference on Approximation Methods and Numerical Modelling in Environment and Natural Resources (presentation)
\item SNA '15, Ostrava, (CZ) \\ Seminar on Numerical Analysis (presentation)
\item ESCO '14, Plze{\v n}, (CZ) \\ European Seminar on Computing (presentation)
\item Stanford, (US) \\ 40th Stanford Geothermal Workshop (attendance only)
\item SNA '14, Nymburk, CZ \\ Seminar on Numerical Analysis (presentation)
\item Student conference 2013, Liberec, CZ \\ Student conference at the Faculty of Mechatronics, Informatics and Interdisciplinary Studies, TUL (poster)
\end{itemize}


%%%%%%%%%%%%%%%%%%%%%%%%%%%%%%%%%%%%%%%%%%%%%%%%%%%%%%%%%%%%%%%%%%%%%%%%%%%%%%%%%%%%%%%%%%%%%%%%%%%%%%%%%%%%%%

\chapter{Summary} \label{chap:summary}
% closure - what is done, further plan, mention traineeship

Based on the research of the related works in Chapter \ref{chap:soa} and the experience gained at the conferences
(in particular X-DMS '15 and MAMERN VI), we are convinced that this is an interesting hot topic which deserves
our attention. On the other hand, we are not aware of any closely connected work to this topic in the Czech Republic
which puts us in a pioneer position at least in the Czech scientific environment.

The aims of the thesis are stated in the Chapter \ref{chap:aims} and these are both of theoretic and practical nature.
There is a lot work to do in the code implementation part but some of the work has been done already 
(Chapter \ref{chap:intersections}). The rest will be done, possibly with some support of the Flow123d development team. 

So far, we studied the well-aquifer model (Section \ref{sec:model_aquifer}). We compared several different XFEM methods, 
dealt with the integration accuracy and investigated some other aspects of the XFEM usage. The results
of our work were presented at several conferences and concluded in an article (list in Chapter \ref{chap:publications}).

A 6 months long traineeship is planned for the second half of the year 2016. A cooperation with the team lead
by professor Barbara Wohlmuth at the Technical University in Munich is to be established. They are experts
in the field of porous media modelling and they are also interested in models on meshes with combined dimensions
(see e.g. \cite{schwenck_2015} where fracture intersections in 2D is solved). The aim of the traineeship is to
work together both on the theory and implementation of the XFEM in 0D-2D and 1D-3D flow model.

%%%%%%%%%%%%%%%%%%%%%%%%%%%%%%%%%%%%%%%%%%%%%%%%%%%%%%%%%%%%%%%%%%%%%%%%%%%%%%%%%%%%%%%%%%%%%%%%%%%%%%%%%%%%%%


%   \nocite{gracie_modelling_2010, fries_corrected_2008, babuska_stable_2012, bangerth_deal.ii_2007, 
%           arnold_lecture_2009, craig_using_2011, sistek_bddc_2015, brezzi_mixed_1991, schwenck_xfem-based_2015,
%           fumagalli_efficient_2014, cattaneo_numerical_2015, koppl_tum_2015, maryska_mixed-hybrid_1995}
\bibliographystyle{mybibtex}
% \bibliographystyle{csplainnat}
\bibliography{citace,citace_xfem,citace_sgfem,citace_flow_my,citace_intersections}

\end{document}
