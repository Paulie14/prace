\documentclass[bibliography=totocnumbered,dvipsnames,FM,Dis]{tulthesis}

\usepackage[czech, english]{babel}
\usepackage[utf8]{inputenc}

% Adds bibliography to tocbibind
% options:
% none - switch off everything
% numbib - adds number to bibliography
% https://ctan.org/pkg/tocbibind
\usepackage[nottoc]{tocbibind}

% \setlength{\parindent}{3em}
% \setlength{\parskip}{0.15cm}

\DeclareUnicodeCharacter{00A0}{~}

\usepackage{amsmath}
\usepackage{amsfonts}
\usepackage{amssymb}
\usepackage{amsthm}
\usepackage{esint}
%
\newtheorem{theorem}{Theorem}[section]
\newtheorem{proposition}[theorem]{Proposition}
\newtheorem{definition}[theorem]{Definition}
\newtheorem{remark}[theorem]{Remark}
\newtheorem{lemma}[theorem]{Lemma}
\newtheorem{corollary}[theorem]{Corollary}
\newtheorem{thmproblem}{Problem}
%\newtheorem{exercise}[theorem]{Cvičení}

%\numberwithin{equation}{document}
%
\def\div{{\rm div}}
\def\Lapl{\Delta}
\def\grad{\nabla}
\def\supp{{\rm supp}}
\def\dist{{\rm dist}}
%\def\chset{\mathbbm{1}}
\def\chset{1}
%
\def\Tr{{\rm Tr}}
\def\to{\rightarrow}
\def\weakto{\rightharpoonup}
\def\imbed{\hookrightarrow}
\def\cimbed{\subset\subset}
\def\range{{\mathcal R}}
\def\leprox{\lesssim}
\def\argdot{{\hspace{0.18em}\cdot\hspace{0.18em}}}
\def\Distr{{\mathcal D}}
\def\calK{{\mathcal K}}
\def\FromTo{|\rightarrow}
\def\convol{\star}
\def\impl{\Rightarrow}
\DeclareMathOperator*{\esslim}{esslim}
\DeclareMathOperator*{\esssup}{ess\,supp}
\DeclareMathOperator{\ess}{ess}
\DeclareMathOperator{\osc}{osc}
\DeclareMathOperator{\curl}{curl}
\DeclareMathOperator{\cotg}{cotg}

%
%\def\Ess{{\rm ess}}
%\def\Exp{{\rm exp}}
%\def\Implies{\Longrightarrow}
%\def\Equiv{\Longleftrightarrow}
% ****************************************** GENERAL MATH NOTATION
\def\Real{{\rm\bf R}}
\def\C{{\rm\bf C}}
\def\Rd{{{\rm\bf R}^{\rm 3}}}
\def\RN{{{\rm\bf R}^N}}
\def\D{{\mathbb D}}
\def\Nnum{{\rm\bf N}}
\def\Qnum{{\rm\bf Q}}
\def\Measures{{\mathcal M}}
\def\dd{\,{\rm d}}               % differential
\def\sdodt{\genfrac{}{}{}{1}{\rm d}{{\rm d}t}}
\def\dodt{\genfrac{}{}{}{}{\rm d}{{\rm d}t}}
%
\def\vc#1{\mathbf{\boldsymbol{#1}}}     % vector
\def\bx{\mathbf{\boldsymbol{x}}}     % vector x
\def\tn#1{{\mathbb{#1}}}    % tensor
\def\abs#1{\lvert#1\rvert}
\def\Abs#1{\bigl\lvert#1\bigr\rvert}
\def\bigabs#1{\bigl\lvert#1\bigr\rvert}
\def\Bigabs#1{\Big\lvert#1\Big\rvert}
\def\ABS#1{\left\lvert#1\right\rvert}
\def\norm#1{\bigl\Vert#1\bigr\Vert} %norm
\def\metr#1#2{\d\bigl(#1,#2\bigr)}          %metric
\def\close#1{\overline{#1}}
\def\inter#1{#1^\circ}
\def\eqdef{\mathrel{\mathop:}=}     % defining equivalence
\def\where{\,|\,}                    % "where" separator in set's defs
\def\timeD#1{\dot{\overline{{#1}}}}
\def\rfrac#1#2{{}^{#1}\!/_{#2}}
%
% ******************************************* USEFULL MACROS
\def\RomanEnum{\renewcommand{\labelenumi}{\rm (\roman{enumi})}}   % enumerate by roman numbers
\def\rf#1{(\ref{#1})}                                             % ref. shortcut
\def\prtl{\partial}                                        % partial deriv.
\def\Names#1{{\scshape #1}}
\def\rem#1{{\parskip=0cm\par!! {\sl\small #1} !!}}
\def\vysl#1{\par$[$ #1 $]$}

%% The lineno packages adds line numbers. Start line numbering with
%% \begin{linenumbers}, end it with \end{linenumbers}. Or switch it on
%% for the whole article with \linenumbers.
%% \usepackage{lineno}

% grahpics
\usepackage{graphicx}
\usepackage{subfig}
\newcommand{\fig}[1]{Figure \hyperref[#1]{\ref{#1}}}
\newcommand{\figpath}{figures/}
\newcommand{\results}{results/}

% živé odkazy v PDF
\usepackage{hyperref}
% \hypersetup{colorlinks=true, linkcolor=tul, urlcolor=tul, citecolor=tul}
\hypersetup{colorlinks=false, linkcolor=tul, urlcolor=tul, citecolor=tul}
\hypersetup{pdftitle={Extended Finite Element Methods on Meshes of Combined Dimensions}}

%tables
\usepackage{booktabs}

% enumeration by alphabet
\usepackage[inline]{enumitem}

\usepackage{url}

% intersections chapter
% algorithms
\usepackage[vlined, linesnumbered, ruled]{algorithm2e}
%\usepackage{float}
%\newfloat{algorithm}{t}{lop}
\usepackage{array}
\newcommand{\plucker}{Pl\"{u}cker }
\newcommand{\nface}{$n$-face }
\newcommand{\nfaces}{$n$-faces }
\newcommand{\ngh}{NGH }
\newcommand{\algo}[1]{\hyperref[#1]{Algorithm \ref{#1}}}
\newcommand{\figpathins}{figures/intersections/}

\pdfsuppresswarningpagegroup=1

% deklarace pro titulní stránku
\TULthesisType{Disertační práce}{Doctoral Thesis}
\TULtitle{Rozšířené metody konečných prvků pro aproximaci singularit}{Extended Finite Element Methods for Approximation of Singularities}
% Extended finite elements for mixed-hybrid model of Darcy flow
\TULprogramme{P3901}{Aplikované vědy v~inženýrství}{Applied Sciences in Engineering}
\TULbranch{3901V055}{Aplikované vědy v~inženýrství}{Scientific Engineering (Mathematical Modelling)}
\TULauthor{Ing. Pavel Exner}
\TULsupervisor{Mgr. Jan B{\v r}ezina, Ph.D.}
\TULyear{}


% just for our notes
% \usepackage[usenames,dvipsnames]{color}   %colors
\newcommand{\noteJB}[1]{{\color{Blue} \textbf{JB: } \textit{#1}}}
\newcommand{\notePE}[1]{{\color{Orange} \textbf{PE: } \textit{#1}}}

%
%*************************************************************************************************************
%
%                                                   DOCUMENT
%
%*************************************************************************************************************
%

\begin{document}


\ThesisStart{male}
% \ThesisTitle{CZ}
% \ThesisTitle{EN}


\begin{abstractCZ}
Tato doktorská práce je zaměřena na řešení problému proudění podzemní vody v~porézním prostředí, které
je ovlivněno přítomností vrtů či studní. Model proudění je sestaven na základě konceptu redukce dimenzí,
který je hojně využíván při modelování rozpukaného porézního přostředí, především granitů.
Vrty jsou modelovány jako 1d objekty, které protínají blok horniny. 
Propojení těchto domén v redukovaném modelu způsobuje singularity v~řešení v~okolí vrtů.
Vrty i~porézní médium jsou síťovány nezávisle na sobě což vede k výpočetním sítím kombinujícím elementy
různých dimenzí.

Jádrem doktorské práce je pak vývoj specializované metody konečných prvků pro výše popsaný model. 
Pro umožnění propojení sítí různých dimenzí a pro zpřesnění aproximace singularit v okolí vrtů je 
použita rozšířená metoda konečných prvků (XFEM), v~rámci níž jsou navrženy nové typy obohacení
konečně-prvkové aproximace.
Metoda XFEM je nejprve aplikována v~modelu pro tlak, dále je navrženo obohacení pro rychlost a metoda
je použita ve smíšeném modelu, jehož řešením jsou rychlost i~tlak.

Doktorská práce se dále detailně věnuje numerickým aspektům v metodě XFEM, a~to především 
adaptivním kvadraturám, volbě velikosti obohacené oblasti nebo podmíněnosti výsledného lineárního systému.
Vlastnosti navržené XFEM metody a~optimální konvergence jsou ověřeny na sérii numerických experimentů.
Praktickým výstupem doktorské práce je implementace metody XFEM jako součásti open-source softwaru Flow123d.

\end{abstractCZ}

\begin{keywordsCZ}
Rozšířená metoda konečných prvků (XFEM), singularita, sítě kombinovaných dimenzí,
Darcyho proudění, rozpukané porézní prostředí
\end{keywordsCZ}

\vspace{2cm}

\begin{abstractEN}
In this doctoral thesis, a~model of groundwater flow in porous media intersected with wells (boreholes, channels) is developed.
The model is motivated by the reduced dimension approach which is being often used in fractured porous media problems, especially in granite rocks.
The wells are modeled as lower dimensional 1d objects and they intersect the surrounding bulk rock domains.
The coupling between the wells and the rock then causes a~singular behaviour of the solution in the higher dimensional domains
in the vicinity of the cross-sections. The domains are discretized separately resulting in an~incompatible mesh of combined dimensions.

The core contribution of this work is in the developement of a~specialized finite element method for such model.
Different Extended finite element methods (XFEM) are studied and new enrichments are suggested to better
approximate the singularities and to enable the coupling of the wells with the higher dimensional domains.
At first the XFEM is applied in a~pressure model, later an enrichment for velocity
is suggested and the XFEM is used in a~mixed model, solving both velocity and pressure.

Different numerical aspects of the XFEM is studied in details, including an adaptive quadrature strategy,
a~proper choice of the enrichment zone or a~conditioning of the resulting linear system.
The properties of the suggested XFEM are validated on a~set of numerical tests and the optimal convergence
rate is demonstrated. The XFEM is implemented as a~part of the open-source software Flow123d.


% Darcy’s law
% reduced dimension approach
% meshes of combined dimensions
% singularity
% Extended finite element method (XFEM)
% mixed form
% implemented in software Flow123d

\end{abstractEN}

\begin{keywordsEN}
Extended Finite Element Method (XFEM), Singularity, Meshes of combined dimensions,
Darcy flow, Fractured porous media
\end{keywordsEN}

\clearpage

\begin{acknowledgement}

\end{acknowledgement}

\tableofcontents
\clearpage

%% \linenumbers

\begin{abbrList}
$d$ & dimension: 1d, 2d, 3d \\
FEM & Finite Element Method \\
MHFEM & Mixed Hybrid Finite Element Method \\
PUM & Partition of Unity Method \\
XFEM & Extended Finite Element Method \\
GFEM & Generalized Finite Element Method \\
SGFEM & Stable Generalized Finite Element Method \\
TUL & Technical University in Liberec

% TODO:
% FVM, FDM, FEM
% XFEM, GFEM, SGFEM
% Fig. vs Figure, ?? the ?? Figure
% use of subequations ??
% definition of seminorm in enrichment radius estimate section

% \textbf{TUL} & Technická univerzita v~Liberci \\
% \textbf{FM} & Fakulta mechatroniky, informatiky a mezioborových studií
% Technické univerzity v~Liberci
\end{abbrList}

%%%%%%%%%%%%%%%%%%%%%%%%%%%%%%%%%%%%%%%%%%%%%%%%%%%%%%%%%%%%%%%%%%%%%%%%%%%%%%%%%%%%%%%%%%%%%%%%%%%%%%%%%%%%%%

\chapter{Introduction}

% short introduction
% define problematics and open questions
% description of the document structure

%%%%%%%%%%%%%%%%%%%%%%%%%%%%%%%%%%%%%%%%%%%%%%%%%%%%%%%%%%%%%%%%%%%%%%%%%%%%%%%%%%%%%%%%%%%%%%%%%%%%%%%%%%%%%%

\input th-intro.tex


%%%%%%%%%%%%%%%%%%%%%%%%%%%%%%%%%%%%%%%%%%%%%%%%%%%%%%%%%%%%%%%%%%%%%%%%%%%%%%%%%%%%%%%%%%%%%%%%%%%%%%%%%%%%%%

\chapter{Reduced Dimensional Models in Flow Problems} \label{chap:reduced}

% Mesh of Combined Dimensions
% Dirac sources models - 1d-2d, 1d-3d (Schwenk, Koeppl, Zunino, D'Andelo)
% xfem concept of Fumagalli, Scotti, D'Angelo ...

%%%%%%%%%%%%%%%%%%%%%%%%%%%%%%%%%%%%%%%%%%%%%%%%%%%%%%%%%%%%%%%%%%%%%%%%%%%%%%%%%%%%%%%%%%%%%%%%%%%%%%%%%%%%%%

\input th-reduced_dim.tex


%%%%%%%%%%%%%%%%%%%%%%%%%%%%%%%%%%%%%%%%%%%%%%%%%%%%%%%%%%%%%%%%%%%%%%%%%%%%%%%%%%%%%%%%%%%%%%%%%%%%%%%%%%%%%%

\chapter{XFEM in Pressure Model with Singularities} \label{chap:xfem_pressure}

% XFEM - state of the art
% well aquifer model
% singular enrichment
% numerical results
% aspects - enr radius, adaptive qudrature, conditioning

%%%%%%%%%%%%%%%%%%%%%%%%%%%%%%%%%%%%%%%%%%%%%%%%%%%%%%%%%%%%%%%%%%%%%%%%%%%%%%%%%%%%%%%%%%%%%%%%%%%%%%%%%%%%%%

\input th-xfem.tex

\input th-result-pressure.tex

%%%%%%%%%%%%%%%%%%%%%%%%%%%%%%%%%%%%%%%%%%%%%%%%%%%%%%%%%%%%%%%%%%%%%%%%%%%%%%%%%%%%%%%%%%%%%%%%%%%%%%%%%%%%%%

\chapter{Singular Enrichment for Velocity} \label{chap:xfem_mh}

% Flow123d concept
% Mixed-hybrid formulation
% current research on the problematics MH + XFEM

%%%%%%%%%%%%%%%%%%%%%%%%%%%%%%%%%%%%%%%%%%%%%%%%%%%%%%%%%%%%%%%%%%%%%%%%%%%%%%%%%%%%%%%%%%%%%%%%%%%%%%%%%%%%%%

\input th-model.tex

\input th-results-flow123d.tex

%%%%%%%%%%%%%%%%%%%%%%%%%%%%%%%%%%%%%%%%%%%%%%%%%%%%%%%%%%%%%%%%%%%%%%%%%%%%%%%%%%%%%%%%%%%%%%%%%%%%%%%%%%%%%%

% \chapter{Achieved Results} \label{chap:results}

%%%%%%%%%%%%%%%%%%%%%%%%%%%%%%%%%%%%%%%%%%%%%%%%%%%%%%%%%%%%%%%%%%%%%%%%%%%%%%%%%%%%%%%%%%%%%%%%%%%%%%%%%%%%%%

% In this chapter, we shall describe the work that has already been done. First part (Section \ref{sec:model_aquifer}) 
% is regarding a~model of a~well-aquifer system where the XFEM is applied to resolve a~point singularity in 2D domain. 
% The second part (Section \ref{sec:elements_intersections}) is aimed at a~geometrical problem of finding 
% intersections of incompatible meshes of different dimensions.




%%%%%%%%%%%%%%%%%%%%%%%%%%%%%%%%%%%%%%%%%%%%%%%%%%%%%%%%%%%%%%%%%%%%%%%%%%%%%%%%%%%%%%%%%%%%%%%%%%%%%%%%%%%%%%

\chapter{Mesh Intersection Algorithms} \label{chap:intersections}

%%%%%%%%%%%%%%%%%%%%%%%%%%%%%%%%%%%%%%%%%%%%%%%%%%%%%%%%%%%%%%%%%%%%%%%%%%%%%%%%%%%%%%%%%%%%%%%%%%%%%%%%%%%%%%

\input th-intersections.tex


%%%%%%%%%%%%%%%%%%%%%%%%%%%%%%%%%%%%%%%%%%%%%%%%%%%%%%%%%%%%%%%%%%%%%%%%%%%%%%%%%%%%%%%%%%%%%%%%%%%%%%%%%%%%%%

% %%%%%%%%%%%%%%%%%%%%%%%%%%%%%%%%%%%%%%%%%%%%%%%%%%%%%%%%%%%%%%%%%%%%%%%%%%%%%%%%%%%%%%%%%%%%%%%%%%%%%%%%%%%%%%
% 
% \chapter{Author's Publication Activity} \label{chap:publications}
% % presentations at conferences, sborniky z nich
% % article
% % Flow123d
% 
% %%%%%%%%%%%%%%%%%%%%%%%%%%%%%%%%%%%%%%%%%%%%%%%%%%%%%%%%%%%%%%%%%%%%%%%%%%%%%%%%%%%%%%%%%%%%%%%%%%%%%%%%%%%%%%
% \large\textsc{Publications}
% \begin{itemize}[label={}]
%
% \item
% M. Hokr, J. B{\v r}ezina, J. Kr{\' a}lovcov{\' a}, J. {\v R}{\' i}ha, P. Exner, I. Han{\v c}ilov{\' a}, A. Balv{\' i}n,
% \href{https://www.onepetro.org/conference-paper/ARMA-DFNE-18-1386}
% {Multidimensional Model of Flow and Transport in Fractured Rock for Support of Czech Deep Geological Repository Program},
% \emph{Proceedings of 2nd International Discrete Fracture Network Engineering Conference, Seattle}, ARMA-DFNE-18-1386,
% American Rock Mechanics Association. November 2018.
% \item
% J. B{\v r}ezina, P. Exner, \href{http://www.sciencedirect.com/science/article/pii/S0898122117300792}{Fast algorithms for intersection of non-matching grids using Plücker coordinates},
% \emph{Computers and Mathematics with Applications}, Volume 74, July 2017, pp. 174-187, ISSN 0898-1221, \href{https://doi.org/10.1016/j.camwa.2017.01.028}{doi.org/10.1016/j.camwa.2017.01.028}.
% 
% \item
% P. Exner, J. B{\v r}ezina, \href{http://www.sciencedirect.com/science/article/pii/S0096300315012862}{Partition of unity methods for approximation of point water sources in porous media},
% \emph{Applied Mathematics and Computation}, Volume 273, January 2016, pp. 21-32, ISSN 0096-3003, \href{http://dx.doi.org/10.1016/j.amc.2015.09.048}{doi:10.1016/j.amc.2015.09.048}.

% O. Sever{\' y}n, J. Stebel, P. Exner, Applied mathematics -- methodical handbook for AP course,
% project Education For Effective Transfer Of Technology And Knowledge In Science And Engineering (EE2.3.45.0011),
% Technical University in Liberec, 2015

% \item
% P. Exner, J. B{\v r}ezina, \href{http://www.ugn.cas.cz/actually/event/2015/sna/sna-sbornik.pdf}{Adaptive integration of singularity in partition of unity methods},
% in proceedings of \emph{Seminar on Numerical Analysis 2015}, Institute of Geonics AS CR, Ostrava, 2015, pp. 29-32, ISBN 978-80-86407-55-5, \\
% \href{http://www.ugn.cas.cz/actually/event/2015/sna/sna-sbornik.pdf}{http://www.ugn.cas.cz/actually/event/2015/sna/sna-sbornik.pdf}.
% 
% \item
% P. Exner, \href{http://www.cs.cas.cz/sna2014/sbornik.pdf}{Partition of unity methods for approximation of point water sources in porous media},
% in proceedings of \emph{Seminar on Numerical Analysis 2014}, Institute of Computer Science AS CR, Prague, 2014, pp. 29-32, ISBN 978-80-87136-16-4, \\
% \href{http://www.cs.cas.cz/sna2014/sbornik.pdf}{http://www.cs.cas.cz/sna2014/sbornik.pdf}.
% 
% 
% \end{itemize}
% %
% \vspace{0.5cm}
% %
% \large\textsc{Conferences}
% %
% \begin{itemize}[label={}]
% \item DFNE '18, Seattle (US) \\ Multidimensional model of flow and transport in fractured
% rock for support of Czech DGR program (presentation)
% \item CMWR '18, Saint Malo (FR) \\ Darcy Flow on Incompatible Meshes of
% Combined Dimensions (presentation)
% \item X-DMS '17, Ume{\o a} (SE) \\ Singular Enrichment of XFEM on Meshes of
% Combined Dimensions (presentation)
% \item FEM Symposium '16, Chemnitz (DE) \\ Extended Finite Element Methods Dealing with Singularities on Meshes of Combined Dimensions (presentation)
% \item X-DMS '15, Ferrara, (IT) \\ eXtended Discretization Methods (presentation, short course attendance)
% \item MAMERN VI '15, Pau, (FR) \\ The International Conference on Approximation Methods and Numerical Modelling in Environment and Natural Resources (presentation)
% \item SNA '15, Ostrava, (CZ) \\ Seminar on Numerical Analysis (presentation)
% \item ESCO '14, Plze{\v n}, (CZ) \\ European Seminar on Computing (presentation)
% \item Stanford, (US) \\ 40th Stanford Geothermal Workshop (attendance only)
% \item SNA '14, Nymburk, CZ \\ Seminar on Numerical Analysis (presentation)
% \item Student conference 2013, Liberec, CZ \\ Student conference at the Faculty of Mechatronics, Informatics and Interdisciplinary Studies, TUL (poster)
% \end{itemize}


%%%%%%%%%%%%%%%%%%%%%%%%%%%%%%%%%%%%%%%%%%%%%%%%%%%%%%%%%%%%%%%%%%%%%%%%%%%%%%%%%%%%%%%%%%%%%%%%%%%%%%%%%%%%%%

\chapter{Conclusion} \label{chap:summary}
% closure - what is done, further plan, mention traineeship

Based on the research of the related works and the experience gained at the conferences
(in particular MAMERN VI '15, X-DMS '15-17 and CMWR '18),
we are convinced that we dedicated our efforts to a~very interesting and hot topic with wide range of applications.
We are not aware of any closely connected work to this topic in the Czech Republic
which puts us in a pioneer position at least in the Czech scientific environment.

% The aims of the thesis are stated in the Chapter \ref{chap:aims} and these are both of theoretic and practical nature.
% There is a lot work to do in the code implementation part but some of the work has been done already 
% (Chapter \ref{chap:intersections}). The rest will be done, possibly with some support of the Flow123d development team. 


The work was also consulted during the traineeship at Technical University in Munich at the Department of Numerical Mathematics
lead by Prof. Barbara Wholmuth. Mainly the theoretical aspects of the work and new ideas were discussed.
We got also familiarized with a~different approach for modeling singularities without using XFEM, which is being developed there.



So far, we studied the well-aquifer model (Section \ref{sec:model_aquifer}). We compared several different XFEM methods, 
dealt with the integration accuracy and investigated some other aspects of the XFEM usage. The results
of our work were presented at several conferences and concluded in an article (list in Chapter \ref{chap:publications}).

A 6 months long traineeship is planned for the second half of the year 2016. A cooperation with the team lead
by professor Barbara Wohlmuth at the Technical University in Munich is to be established. They are experts
in the field of porous media modelling and they are also interested in models on meshes with combined dimensions
(see e.g. \cite{schwenck_2015} where fracture intersections in 2D is solved). The aim of the traineeship is to
work together both on the theory and implementation of the XFEM in 0D-2D and 1D-3D flow model.

%%%%%%%%%%%%%%%%%%%%%%%%%%%%%%%%%%%%%%%%%%%%%%%%%%%%%%%%%%%%%%%%%%%%%%%%%%%%%%%%%%%%%%%%%%%%%%%%%%%%%%%%%%%%%%


% \listoffigures
% \clearpage
% 
% \listoftables
% \clearpage


%   \nocite{gracie_modelling_2010, fries_corrected_2008, babuska_stable_2012, bangerth_deal.ii_2007, 
%           arnold_lecture_2009, craig_using_2011, sistek_bddc_2015, brezzi_mixed_1991, schwenck_xfem-based_2015,
%           fumagalli_efficient_2014, cattaneo_numerical_2015, koppl_tum_2015, maryska_mixed-hybrid_1995}
{\small
\bibliographystyle{mybibtex}
% \bibliographystyle{csplainnat}
\bibliography{citace,citace_xfem,citace_sgfem,citace_flow_my,citace_intersections}
}


\end{document}
