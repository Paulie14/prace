\documentclass[bibliography=totocnumbered,dvipsnames,FM,Dis,EN]{tulthesis}

% \setcounter{secnumdepth}{3}

\usepackage[czech, english]{babel}
\usepackage[utf8]{inputenc}

% Adds bibliography to tocbibind
% options:
% none - switch off everything
% numbib - adds number to bibliography
% https://ctan.org/pkg/tocbibind
\usepackage[nottoc]{tocbibind}

% \setlength{\parindent}{3em}
% \setlength{\parskip}{0.15cm}

\DeclareUnicodeCharacter{00A0}{~}

%% The lineno packages adds line numbers. Start line numbering with
%% \begin{linenumbers}, end it with \end{linenumbers}. Or switch it on
%% for the whole article with \linenumbers.
%% \usepackage{lineno}

% grahpics
\usepackage{graphicx}
\usepackage{subfig}
\newcommand{\fig}[1]{Figure \hyperref[#1]{\ref{#1}}}
\newcommand{\gref}[1]{Graph \hyperref[#1]{\ref{#1}}}
\newcommand{\figpath}{figures/}
\newcommand{\results}{results/}

\DeclareCaptionType[name={Graph}]{graph}
% \newcommand{\listofgarphs}{\tocfile{\listfigures}{graph}}

% živé odkazy v PDF
% \usepackage{hyperref}
% \hypersetup{colorlinks=true, linkcolor=tul, urlcolor=tul, citecolor=tul}
\hypersetup{colorlinks=false, linkcolor=tul, urlcolor=tul, citecolor=tul}
\hypersetup{pdftitle={Extended Finite Element Methods on Meshes of Combined Dimensions}}

%tables
\usepackage{booktabs}

% enumeration by alphabet
\usepackage[inline]{enumitem}

\usepackage{url}

% intersections chapter
% algorithms
\usepackage[vlined, linesnumbered, ruled, algochapter]{algorithm2e}
\newcommand{\listofalgorithmes}{\tocfile{\listalgorithmcfname}{loa}}
% \newcommand{\algo}[1]{\hyperref[#1]{Algorithm \ref{#1}}}
\newcommand{\algo}[1]{Algorithm \ref{#1}}
% \newcommand{\mykwsty}[1]{\textcolor{blue}{\emph{#1}}}
\newcommand{\myfuncsty}[1]{\textbf{\texttt{#1}}}
% \newcommand{\myvarsty}[1]{\textcolor{GoogleGreen}{\texttt{#1}}}
% \SetKwSty{mykwsty}
% \SetArgSty{textbf}
\SetFuncSty{myfuncsty}
% \SetDataSty{myvarsty}
\SetKwFunction{setlinks}{set links}
\SetKwFunction{vertexfaces}{vertex faces}
\SetKwFunction{edgefaces}{edge faces}
\SetKwFunction{intersectionA}{intersection12}
\SetKwFunction{intersectionB}{intersection13}
\SetKw{kwand}{and}
\SetKw{kwcontinue}{continue}
\SetKw{kwbreak}{break}
% \SetKwFunction{cost}{cost}
% \SetKw{Of}{of}
% \SetKwData{neighbors}{neighbors}
% \SetKwData{edge}{edge}


% for codes we use different labeling
\newenvironment{code}[1][htb]
  {\renewcommand{\algorithmcfname}{Code}% Update algorithm name
   \begin{algorithm}[#1]%
  }{\end{algorithm}}
  
%\usepackage{float}
%\newfloat{algorithm}{t}{lop}
\usepackage{array}
\newcommand{\plucker}{Pl\"{u}cker }
\newcommand{\nface}{$n$-face }
\newcommand{\nfaces}{$n$-faces }
\newcommand{\ngh}{NGH }
\newcommand{\figpathins}{figures/intersections/}

\pdfsuppresswarningpagegroup=1

% deklarace pro titulní stránku
\TULthesisType{Disertační práce}{Doctoral Thesis}
\TULtitle{Rozšířené metody konečných prvků pro aproximaci singularit}{Extended Finite Element Methods for Approximation of Singularities}
% Extended finite elements for mixed-hybrid model of Darcy flow
\TULprogramme{P3901}{Aplikované vědy v~inženýrství}{Applied Sciences in Engineering}
\TULbranch{3901V055}{Aplikované vědy v~inženýrství}{Applied Sciences in Engineering}
\TULauthor{Ing. Pavel Exner}
\TULsupervisor{doc. Mgr. Jan B{\v r}ezina, Ph.D.}
\TULyear{}


% just for our notes
% \usepackage[usenames,dvipsnames]{color}   %colors
\newcommand{\noteJB}[1]{{\color{Blue} \textbf{JB: } \textit{#1}}}
\newcommand{\notePE}[1]{{\color{Orange} \textbf{PE: } \textit{#1}}}

%
%*************************************************************************************************************
%
%                                                   DOCUMENT
%
%*************************************************************************************************************
%
% Title Page
% Copyright Page
% Abstract
% Dedication, Acknowledgements, and Preface (each optional)
% Table of Contents, with page numbers
% List of Tables, List of Figures, or List of Illustrations, with titles and page numbers (if applicable)
% List of Abbreviations (if applicable)
% List of Symbols (if applicable)
% Chapters, including:
% Introduction, if any
% Main body, with consistent subheadings as appropriate
% Appendices (if applicable)
% Endnotes (if applicable)
% References (see section on References for options)

\begin{document}


\ThesisStart{male}
% \ThesisTitle{CZ}
% \ThesisTitle{EN}

\input th-abstract.tex
\clearpage


\thispagestyle{empty}
% \vspace*{\fill}
\vspace*{2in}
\begin{center}
% \begin{flushright}
\emph{
% Dedicated to the memory of my late father, \\
% Zden{\v e}k Exner. \\
Dedicated to the memory of my late father, Zden{\v e}k Exner,\\
whom I still turn to for a guidance through my life,\\
even though we cannot share the days in this world anymore.
}
% \end{flushright}
\end{center}
\vspace*{\fill}
\clearpage


\begin{acknowledgement}
\thispagestyle{empty}
This work would not have been created without the great support of my colleagues, family and friends
and therefore I would like to express the deepest appreciation to all of them.

Especially, I would like to thank my supervisor, doc. Mgr. Jan B{\v r}ezina, Ph.D., for his leadership and all the support he provided me
during my doctorate and also before. His advice and his optimism helped me several times to get past a~problem
when the solution was too blurred for my eyes.

I also greatly appreciate the support from my colleague, Mgr. Jan Stebel, Ph.D.,
who never refused to answer any of my questions however silly or complicated they were.

I am grateful to Prof. Barbara Wohlmuth for inviting me to the Technical University of Munich,
for her hospitality during my research stay there and for the great opportunity to expand my horizons with the team at M2.

I owe a~huge debt to my family and friends for spending all the extra hours at work instead of being with them.
I would like to express my sincere gratitude to my mother, Hana Exnerov\'a, in particular, for always supporting me and believing in me.

Finally, I thank with love to my soul mate, Markéta Tůmová. She has been my best friend and great companion,
supported, encouraged, entertained and helped me through this exciting life period.


% I want to thank someone
% I would (particularly) like to thank someone
% I would like to express my gratitude to someone
% I would like to express the deepest appreciation to someone
% My deepest [heartfelt] appreciation goes to someone
% I would like to show my greatest appreciation to someone
% Special thanks (also) to someone
% I would like to offer my special thanks to someone
% I owe a very important debt to someone
% I owe my deepest gratitude to someone
% 
% Acknowledging comments and support
% Discussions with A and B have been illuminating [insightful]
% I appreciate the feedback offered by someone
% I have greatly benefited from someone
% I have had the support and encouragement of someone
% I received generous support from someone
% My intellectual debt is to someone
% someone gives insightful comments and suggestions
% someone has been (greatly, extraordinarily) tolerant and supportive ...
% someone gives me constructive comments and warm encouragement
% someone made enormous contribution to ...
% someone's support [encouragement, suggestions, comments] were invaluable
% someone's meticulous comments were an enormous help to me
% Advice and comments given by someone has been a great help in...
% I am particularly grateful for the assistance given by someone
% 
% Without his/her guidance and persistent help this paper [dissertation, thesis] would not have been possible.
% Without his/her encouragement [help, guidance], this paper [dissertation, project] would not have materialized.
% 
% I thank someone for permission to use [include] ...
% I would also like to express my gratitude to someone for their financial support.

\end{acknowledgement}

\tableofcontents
\clearpage

%% \linenumbers

\listoffigures
\clearpage

% \listofgraphs
% \clearpage
% 
% \listoftables
% \clearpage
% 
% \listofalgorithmes
% \clearpage

\listoftables
\clearpage


\begingroup
\let\clearpage\relax
\listofgraphs
\listofalgorithmes
\endgroup
\clearpage

\input th-list-of-symbols.tex
\clearpage


% TODO:
% Fig. vs Figure, ?? the ?? Figure
% use of subequations ??
% definition of seminorm in enrichment radius estimate section

%%%%%%%%%%%%%%%%%%%%%%%%%%%%%%%%%%%%%%%%%%%%%%%%%%%%%%%%%%%%%%%%%%%%%%%%%%%%%%%%%%%%%%%%%%%%%%%%%%%%%%%%%%%%%%

\chapter{Introduction}

% short introduction
% define problematics and open questions
% description of the document structure

%%%%%%%%%%%%%%%%%%%%%%%%%%%%%%%%%%%%%%%%%%%%%%%%%%%%%%%%%%%%%%%%%%%%%%%%%%%%%%%%%%%%%%%%%%%%%%%%%%%%%%%%%%%%%%

\input th-intro.tex


%%%%%%%%%%%%%%%%%%%%%%%%%%%%%%%%%%%%%%%%%%%%%%%%%%%%%%%%%%%%%%%%%%%%%%%%%%%%%%%%%%%%%%%%%%%%%%%%%%%%%%%%%%%%%%

\chapter{Reduced Dimensional Models} \label{chap:reduced}

% Mesh of Combined Dimensions
% Dirac sources models - 1d-2d, 1d-3d (Schwenk, Koeppl, Zunino, D'Andelo)
% xfem concept of Fumagalli, Scotti, D'Angelo ...

%%%%%%%%%%%%%%%%%%%%%%%%%%%%%%%%%%%%%%%%%%%%%%%%%%%%%%%%%%%%%%%%%%%%%%%%%%%%%%%%%%%%%%%%%%%%%%%%%%%%%%%%%%%%%%

\input th-reduced_dim.tex


%%%%%%%%%%%%%%%%%%%%%%%%%%%%%%%%%%%%%%%%%%%%%%%%%%%%%%%%%%%%%%%%%%%%%%%%%%%%%%%%%%%%%%%%%%%%%%%%%%%%%%%%%%%%%%

\chapter{Extended Finite Element Method} \label{chap:xfem_soa}

% XFEM - state of the art

%%%%%%%%%%%%%%%%%%%%%%%%%%%%%%%%%%%%%%%%%%%%%%%%%%%%%%%%%%%%%%%%%%%%%%%%%%%%%%%%%%%%%%%%%%%%%%%%%%%%%%%%%%%%%%

\input th-xfem.tex

%%%%%%%%%%%%%%%%%%%%%%%%%%%%%%%%%%%%%%%%%%%%%%%%%%%%%%%%%%%%%%%%%%%%%%%%%%%%%%%%%%%%%%%%%%%%%%%%%%%%%%%%%%%%%%

\chapter{Pressure Model with Singularities} \label{chap:xfem_pressure}

% well aquifer model
% singular enrichment
% numerical results
% aspects - enr radius, adaptive qudrature, conditioning

%%%%%%%%%%%%%%%%%%%%%%%%%%%%%%%%%%%%%%%%%%%%%%%%%%%%%%%%%%%%%%%%%%%%%%%%%%%%%%%%%%%%%%%%%%%%%%%%%%%%%%%%%%%%%%

\input th-pressure-model.tex

\input th-pressure-results.tex

%%%%%%%%%%%%%%%%%%%%%%%%%%%%%%%%%%%%%%%%%%%%%%%%%%%%%%%%%%%%%%%%%%%%%%%%%%%%%%%%%%%%%%%%%%%%%%%%%%%%%%%%%%%%%%

\chapter{Mixed Model with Singularities} \label{chap:xfem_mh}

% Flow123d concept
% Mixed-hybrid formulation
% current research on the problematics MH + XFEM

%%%%%%%%%%%%%%%%%%%%%%%%%%%%%%%%%%%%%%%%%%%%%%%%%%%%%%%%%%%%%%%%%%%%%%%%%%%%%%%%%%%%%%%%%%%%%%%%%%%%%%%%%%%%%%

\input th-mh-model.tex

\input th-mh-results.tex

%%%%%%%%%%%%%%%%%%%%%%%%%%%%%%%%%%%%%%%%%%%%%%%%%%%%%%%%%%%%%%%%%%%%%%%%%%%%%%%%%%%%%%%%%%%%%%%%%%%%%%%%%%%%%%

% \chapter{Achieved Results} \label{chap:results}

%%%%%%%%%%%%%%%%%%%%%%%%%%%%%%%%%%%%%%%%%%%%%%%%%%%%%%%%%%%%%%%%%%%%%%%%%%%%%%%%%%%%%%%%%%%%%%%%%%%%%%%%%%%%%%

% In this chapter, we shall describe the work that has already been done. First part (Section \ref{sec:model_aquifer}) 
% is regarding a~model of a~well-aquifer system where the XFEM is applied to resolve a~point singularity in 2D domain. 
% The second part (Section \ref{sec:elements_intersections}) is aimed at a~geometrical problem of finding 
% intersections of incompatible meshes of different dimensions.




%%%%%%%%%%%%%%%%%%%%%%%%%%%%%%%%%%%%%%%%%%%%%%%%%%%%%%%%%%%%%%%%%%%%%%%%%%%%%%%%%%%%%%%%%%%%%%%%%%%%%%%%%%%%%%

\chapter{Mesh Intersection Algorithms} \label{chap:intersections}

%%%%%%%%%%%%%%%%%%%%%%%%%%%%%%%%%%%%%%%%%%%%%%%%%%%%%%%%%%%%%%%%%%%%%%%%%%%%%%%%%%%%%%%%%%%%%%%%%%%%%%%%%%%%%%

\input th-intersections.tex



%%%%%%%%%%%%%%%%%%%%%%%%%%%%%%%%%%%%%%%%%%%%%%%%%%%%%%%%%%%%%%%%%%%%%%%%%%%%%%%%%%%%%%%%%%%%%%%%%%%%%%%%%%%%%%

\chapter{Conclusion} \label{chap:summary}
% closure - what is done, further plan, mention traineeship

%%%%%%%%%%%%%%%%%%%%%%%%%%%%%%%%%%%%%%%%%%%%%%%%%%%%%%%%%%%%%%%%%%%%%%%%%%%%%%%%%%%%%%%%%%%%%%%%%%%%%%%%%%%%%%

Based on the research of the related works and experience gained at the conferences
(in particular MAMERN VI '15, X-DMS '15-17 and CMWR '18),
we are convinced that we dedicated our efforts to a~very interesting and hot topic with wide range of applications.
We are also not aware of any closely connected work to this topic in the Czech Republic
which puts us in a pioneer position, in the Czech scientific environment at least.

In the first part of this work, a~reduced dimension concept was described and a~model for coupling
groundwater flow in non-planar 1d-2d and 1d-3d domains intersecting each other was suggested. 
The drawbacks of several approaches in FE approximation in such models were discussed leading us to
our solution: incompatible meshing of the domains and using the XFEM to couple them back together and to improve
FE approximation of arisen singularities by proper discrete space enrichments.

We provided an~extensive study of the currently available XFEM \cite{fries_xfem_overview_2010, babuska_stable_2012} in Chapter \ref{chap:xfem_soa}
and we addressed the singular enrichments in particular. 
Then in Chapter \ref{chap:xfem_pressure} we created a~model simulating pressure in a~well-aquifer system, inspired by \cite{gracie_modelling_2010, craig_using_2011}.
We studied different types of enrichments and compared them in terms of
convergence rate, linear system conditioning and sensitivity to mesh -- singularity alignment.
In the view of our numerical results, we found the SGFEM to be the most promising method
for singularity approximation in our model.
Apart from that we focused on implementation aspects of the XFEM and we improved adaptive quadrature rules
for an accurate integration on enriched elements. We also investigated the optimal enrichment zone
of singular enrichments and we verified our model on a~set of numerical test cases.
Our early work was summarized in \cite{exner_2016}.


Chapter \ref{chap:xfem_mh} was dedicated to the XFEM application in a~mixed problem
in order to extend the possibilities of the groundwater model in \cite{sistek_bddc_2015, flow123d}.
The mixed-hybrid form was carefully derived for both non-planar 1d-2d and 1d-3d case.
We suggested a~new singular SGFEM like enrichment of the standard Raviart-Thomas finite elements
and we applied it to the velocity discrete space.
We implemented the model as an~experimental part of the software Flow123d
and we provided a~set of numerical tests.
Since velocity is important in the attached processes (e.g. transport),
we put a~stress on the velocity precision and we traced the velocity convergence rate.
The optimal order of convergence was observed in both 1d-2d and 1d-3d tests,
which included single and multiple wells, overlapping enrichment zones and non-zero source prescribed.

The difficulty of the suggested vector enrichment is that it shares a~single degree of freedom per singularity
over its whole enrichment zone. This is a~source of two problems. First, the system matrix has some non-sparse rows,
which can then lead to a~loss of sparsity when applying a~preconditioner based on elimination.
Second, any heterogeneity, e.g. in conductivity, inside the enrichment zone cannot be captured by
a~single singular enrichment function.
We tried to find proper elementwise enrichment functions, similarly to the ones used in the pressure model,
however we struggled with the hybridized form, where such enrichment functions must be accompanied
by some corresponding Lagrange multipliers.
So far we were unsuccessful in performing a~numerical test of the inf-sup stability
of the mixed-hybrid form, this problem had to be left open.
We intend to further study the suggested elementwise enrichment functions and the inf-sup test in the mixed problem without hybridization,
however such model is not available in Flow123d yet.


A~necessary prerequisite for computations on incompatible meshes is the ability to determine
the intersections of the different meshed domains. In Chapter \ref{chap:intersections} we developed
a~fast and robust algorithm to compute such intersections on simplicial meshes.
Although we do need to solve only the non-planar 1d-2d and 1d-3d cases in our singular models,
we extended the algorithms also to higher dimensions, namely 2d-2d, 2d-3d.
New models in these cases are in focus of our future work and further development of the software Flow123d.
We exploited the properties of the \plucker coordinates \cite{platis_fast_2003, joswig_plucker_2013}
in the element-element intersection algorithms which provide not only the coordinates but also additional topological information.
This was then used in the global mesh intersection algorithms together with other modern techniques
such as BIH of axes aligned bounding boxes and advancing front tracing.
The suggested algorithms were shown to be competitive to other works \cite{moller_fast_1997, haines_fast_1991}.
The global mesh intersection algorithms were tested in Flow123d on an artificial and a~real case benchmarks
and they exhibit linear time complexity. The results were summarized in \cite{brezina_2017}.
Possible further improvements include a~deeper study on the precision of the used geometric predicates,
e.g. regarding the adaptivity in \cite{shewchuk_adaptive_1997}, and thorough code optimization in Flow123d.


The work was also consulted during the author's traineeship at the Technical University of Munich at the Department of Numerical Mathematics
lead by Prof. Barbara Wohlmuth. Mainly the theoretical aspects of the work and new ideas were discussed.
We got also familiarized with a~different approach for problems with Dirac delta sources \cite{koppl_tum_2015, koppl_vidotto_2018}
as a~coupling method for inclusions.


The goals of this thesis as set in the introduction were fulfilled to a great extent.
We studied the XFEM intensively and researched its usage in singular problems.
Apart from the created pressure model, we managed to suggest a~new velocity enrichment in the mixed-hybrid form
and implement a~working model in Flow123d. The model was derived and formulated in detail.
A~lot of technical work was done while preparing all the building blocks for the XFEM in the software.
Eventually, we left open several issues which were addressed above.

As we already pointed out, the future work may concern a~study of the vector enrichments in the mixed form.
Extensions of the discretization for 1d objects that are not straight might be of interest.
A~specialized iterative method can be suggested in order to solve the linear algebraic system efficiently,
including a~proper preconditioner.
Finally, some processes attached to the groundwater model may be considered using the velocity solution.
These processes, namely transport of substances, poroelasticity or heat transfer, then may require
similar kind of enrichment for scalar/vector quantities of interest. 



\chapter*{Author's Publications} \label{chap:publications}
% \addcontentsline{toc}{chapter}{Author's Publications}
% % presentations at conferences, sborniky z nich
% % article
% % Flow123d
% 
% %%%%%%%%%%%%%%%%%%%%%%%%%%%%%%%%%%%%%%%%%%%%%%%%%%%%%%%%%%%%%%%%%%%%%%%%%%%%%%%%%%%%%%%%%%%%%%%%%%%%%%%%%%%%%%
{\large\textsc{Publications}}
\begin{itemize}[label={}, leftmargin=*]

{\small
\item
M. Hokr, J. B{\v r}ezina, J. Kr{\' a}lovcov{\' a}, J. {\v R}{\' i}ha, P. Exner, I. Han{\v c}ilov{\' a}, A. Balv{\' i}n,
\href{https://www.onepetro.org/conference-paper/ARMA-DFNE-18-1386}
{Multidimensional Model of Flow and Transport in Fractured Rock for Support of Czech Deep Geological Repository Program},
\emph{Proceedings of 2nd International Discrete Fracture Network Engineering Conference, Seattle} (2018), ARMA-DFNE-18-1386,
American Rock Mechanics Association.\\
\url{https://www.onepetro.org/conference-paper/ARMA-DFNE-18-1386}.

\item
J. B{\v r}ezina, P. Exner, \href{http://www.sciencedirect.com/science/article/pii/S0898122117300792}{Fast algorithms for intersection of non-matching grids using Plücker coordinates},
\emph{Computers and Mathematics with Applications} 74 (2017) pp. 174-187. ISSN 0898-1221.
\href{https://doi.org/10.1016/j.camwa.2017.01.028}{\texttt{doi.org/10.1016/j.camwa.2017.01.028}}.

\item
P. Exner, J. B{\v r}ezina, \href{http://www.sciencedirect.com/science/article/pii/S0096300315012862}{Partition of unity methods for approximation of point water sources in porous media},
\emph{Applied Mathematics and Computation} 273 (2016) pp. 21-32. ISSN 0096-3003.
\href{http://dx.doi.org/10.1016/j.amc.2015.09.048}{\texttt{doi:10.1016/j.amc.2015.09.048}}.

\item
P. Exner, J. B{\v r}ezina, \href{http://www.ugn.cas.cz/actually/event/2015/sna/sna-sbornik.pdf}{Adaptive integration of singularity in partition of unity methods},
in proceedings of \emph{Seminar on Numerical Analysis 2015}, Institute of Geonics AS CR, Ostrava, (2015) pp. 29-32. ISBN 978-80-86407-55-5. \\
\url{http://www.ugn.cas.cz/actually/event/2015/sna/sna-sbornik.pdf}.

\item
P. Exner, \href{http://www.cs.cas.cz/sna2014/sbornik.pdf}{Partition of unity methods for approximation of point water sources in porous media},
in proceedings of \emph{Seminar on Numerical Analysis 2014}, Institute of Computer Science AS CR, Prague, (2014) pp. 29-32. ISBN 978-80-87136-16-4.\\
\url{http://www.cs.cas.cz/sna2014/sbornik.pdf}.
}
\end{itemize}

\vspace{0.5cm}
%
{\noindent\large\textsc{Software}}
\begin{itemize}[label={}, leftmargin=*]
{\small
\item
J. B{\v r}ezina, J. Stebel, D. Flanderka, P. Exner, J. Hyb{\v s}, software Flow123d, 
Technical University of Liberec, (2013--2019).
\url{http://flow123d.github.com}.
}
\end{itemize}
%


% Multi-level Monte Carlo methods - computation on coarse meshes - can include non-planar 1d-2d or 1d-3d intersections 
% on coarse meshes using the XFEM.

%%%%%%%%%%%%%%%%%%%%%%%%%%%%%%%%%%%%%%%%%%%%%%%%%%%%%%%%%%%%%%%%%%%%%%%%%%%%%%%%%%%%%%%%%%%%%%%%%%%%%%%%%%%%%%


%   \nocite{gracie_modelling_2010, fries_corrected_2008, babuska_stable_2012, bangerth_deal.ii_2007, 
%           arnold_lecture_2009, craig_using_2011, sistek_bddc_2015, brezzi_mixed_1991, schwenck_xfem-based_2015,
%           fumagalli_efficient_2014, cattaneo_numerical_2015, koppl_tum_2015, maryska_mixed-hybrid_1995}
{\small
\bibliographystyle{mybibtex}
% \bibliographystyle{csplainnat}
\bibliography{citace,citace_xfem,citace_sgfem,citace_flow_my,citace_intersections}
}


\end{document}
